\NeedsTeXFormat{LaTeX2e}

% \documentclass{igs}
% \documentclass[twocolumn]{igs}
% \documentclass[twocolumn,letterpaper]{igs}
  \documentclass[review,letterpaper]{igs}

  \usepackage{igsnatbib}
  %\usepackage{stfloats}

% check if we are compiling under latex or pdflatex
  \ifx\pdftexversion\undefined
    \usepackage[dvips]{graphicx}
  \else
    \usepackage[pdftex]{graphicx}
  \fi


\usepackage{amsmath}

% only include this in review mode:
\include{linenopatch.tex}


\begin{document}

\title[Extending the Bartholomaus hydrology model]{Correspondence \\ Extending the lumped subglacial-englacial \\ hydrology model of Bartholomaus and others (2011)}

\author{Ed Bueler}

\affiliation{Department of Mathematics and Statistics and Geophysical Institute, University of Alaska Fairbanks, USA \\
E-mail: \emph{\texttt{elbueler\@@alaska.edu}}}

\abstract{[\emph{Not needed for Correspondence.}]}

\maketitle

The model in \cite{Bartholomausetal2011} is carefully constructed to describe the hydrology of the Kennicott Glacier in Alaska.  It uses observed input flux and a highly-simplified hydrology to reproduce the hydrograph of the Kennicott River, other hydrographs, and the glacier motion (sliding) observed during the flood.  The goal of their model is to understand both the drainage through a subglacial and englacial network, and the glacier response, from a lateral lake outburst flood which occurs each summer, that is, a j\"okulhlaup.

Their model would seem to be significantly-different from recent distributed models, specifically \cite{Schoofetal2012} and \cite{Hewittetal2012}.  Some obvious similarities exist among the models, however.  They all describe the evolution of water-filled linked-cavity systems, with additional morphologies according to the different models, and they all include physical cavity opening and closing processes for cavities.  On the other hand the just-cited distributed theories are entirely subglacial, while the \cite{Bartholomausetal2011} model (from now on, the ``Bartholomaus model'') has both subglacial and englacial water storage.  Most essentially, however, the Bartholomaus model is ``lumped''.  That is, the entire glacier hydrological system is represented by one cell, with the major advantage that this allows parameter identification given the observed, also spatially-lumped, input and output fluxes.

The equations of \cite{Bartholomausetal2011} actually imply an unstated pressure equation, which we will derive below.  The form of this implicit pressure equation is suggested by equation (12) in \cite{Bartholomausetal2011}.  Its distributed version is a parabolic regularization of the elliptic pressure equation in \cite{Schoofetal2012}.  Said the other way, we can show that the pressure equation in \cite{Schoofetal2012} is the zero englacial porosity limit of the distributed version of the pressure equation implicit in the Bartholomaus model.  Exposing these connections is the major motivation for this short note.

A connection between the Bartholomaus theory and distributed models is also observed by \cite{Hewitt2013}.  However, the connection is limited to the addition of a englacial storage term in the mass conservation equation \cite[equation (7)]{Hewitt2013}.  Here we emphasize that, additionally, the elliptic pressure equation of \cite{Schoofetal2012}, which causes instantaneous pressure changes across distance, acquires a parabolic regularization through the addition of englacial storage (i.e.~equation \eqref{eq:barth:distpressure} below).

When restricted to steady state, the Bartholomaus model equations also imply an apparently-unnoticed functional relationship between pressure and cavity size, a relationship that might motivate the choices made in earlier hydrological theory \citep{FlowersClarke2002_theory}.
 
In the Bartholomaus model, the total volume of liquid water stored in the glacier is $S(t)$.  This is split into englacial $S_{en}(t)$ and subglacial $S_{sub}(t)$ portions:
\begin{equation}
S = S_{en} + S_{sub}.  \label{eq:barth:kinematics}
\end{equation}
The subglacial cavities have geometry determined by bedrock bumps which have horizontal spacing $\lambda_x,\lambda_y$, height $h$, and width $w_c$.  These combine to give a dimensionless capacity parameter $f=h w_c/(\lambda_x \lambda_y)$; the value $f=0.05$ is used for the Kennicott glacier.  Each cavity has cross-sectional area $A_c(t)$ and thus volume $w_c A_c$.  The glacier occupies a rectangle of dimensions $L\times W$ in the map-plane so that the number of cavities is $\nu = (LW)/(\lambda_x\lambda_y)$.  It follows that the subglacial storage volume is $S_{sub} = (w_c A_c) \nu = (f L W/h) A_c$.  Englacial water is assumed to fill a well-connected system of crevasses and moulins up to a level $z_w(t)$ above the bedrock, in a system which has macroporosity $\phi$, so the englacial storage is $S_{en}=L W \phi z_w$.  Mass conservation in the model is the simple statement \citep{Bartholomausetal2008}
\begin{equation}
\frac{dS}{dt} = Q_{in}(t) - Q_{out}(t). \label{eq:barth:massconserve}
\end{equation}
In the Kennicott glacier application, fluxes $Q_{in}$ and $Q_{out}$ are observed.

Denote the subglacial water pressure by $P(t)$.  In the Bartholomaus model, knowledge of $P$ is equivalent to knowledge of englacial storage because of the assumed efficient connection of to the subglacial system.  Noting the relation $S_{en}=L W \phi z_w$ above, we have
\begin{equation}
P = \rho_w g z_w = \frac{\rho_w g}{LW\phi} S_{en}.  \label{eq:barth:englacialpressure}
\end{equation}

Now, in the Bartholomaus model the cavity cross-sectional area $A_c$ evolves by physical opening and closure processes.  The rate of production of subglacial water through wall melt is denoted here simply by $Z$, with sign so that $Z>0$ when cavities are enlarging.  In fact $Z$ is further parameterized in \cite{Bartholomausetal2011}, but the details are unimportant to our calculations.  Let $C_c = (2 A)/n^n$, where $A$ and $n$ are the usual parameters in the Glen ice flow law \citep{CuffeyPaterson}.  Also denote the sliding speed by $u_b$ and let $P_o=\rho_i g H$ be the overburden pressure.  In these terms the cavity evolution equation in \cite{Bartholomausetal2011} is
\begin{equation}
\frac{dA_c}{dt} = u_b h + Z - C_c A_c (P_o-P)^n.  \label{eq:barth:cavityevolution}
\end{equation}
The three terms on the right are opening by cavitation, opening by wall melt, and closure by creep, respectively; compare (4) in \cite{Bartholomausetal2011}.

Equations \eqref{eq:barth:kinematics}, \eqref{eq:barth:massconserve}, \eqref{eq:barth:englacialpressure}, and \eqref{eq:barth:cavityevolution} combine to give an evolution equation for the pressure.  From \eqref{eq:barth:englacialpressure} we can write the pressure rate of change in terms of the englacial storage rate of change:
\begin{equation}
\frac{dP}{dt} = \rho_w g \frac{dz_w}{dt} = \frac{\rho_w g}{L W \phi} \frac{d S_{en}}{dt}. \label{eq:barth:dPdt}
\end{equation}
By combining \eqref{eq:barth:dPdt} with  the time-derivative of \eqref{eq:barth:kinematics}, and using both \eqref{eq:barth:massconserve} and the proportionality between $S_{sub}$ and $A_c$, we can rewrite in terms of fluxes and cavity area:
\begin{align}
\frac{dP}{dt} &= \frac{\rho_w g}{L W \phi} \left(\frac{d S}{dt} - \frac{d S_{sub}}{dt}\right) \\
&= \frac{\rho_w g}{L W \phi} \left(Q_{in} - Q_{out} - \frac{f L }{h} \frac{d A_c}{dt}\right). \notag
\end{align}
Finally incorporate equation \eqref{eq:barth:cavityevolution} to eliminate $dA_c/dt$:
\begin{align}
&\frac{dP}{dt} = \frac{\rho_w g}{L W \phi} \bigg(Q_{in} - Q_{out} \label{eq:barth:fullpressure} \\
&\qquad \qquad \qquad - \frac{f L }{h} \left[u_b h + Z - C_c A_c (P_o-P)^n\right]\bigg) \notag
\end{align}
Thus, though it is not stated there, \eqref{eq:barth:fullpressure} follows from the equations in \cite{Bartholomausetal2011}.

Equation \eqref{eq:barth:fullpressure} suggests how to extend the Bartholomaus model from ``lumped'' to ``distributed.''  Consider a one-dimensional glacier under which the water is flowing in the positive $x$ direction.  Let the transverse width be $W$ and replace $L$ by $\Delta x$, the along-flow length of one cell in a finite difference or finite volume scheme.  Thus rewrite \eqref{eq:barth:fullpressure} as
\begin{align}
&\frac{\phi W}{\rho_w g}\frac{dP}{dt} = - \frac{Q_{out} - Q_{in}}{\Delta x}  \label{eq:barth:pressurerewrite} \\
&\qquad \qquad \qquad \quad - \frac{f}{h} \left[u_b h + Z - C_c A_c (P_o-P)^n\right], \notag
\end{align}
which describes the pressure in one cell.  Note that ``$Q_{in}$'' is the upstream input flux into the cell while ``$Q_{out}$'' is the downstream output.  The continuum limit of \eqref{eq:barth:pressurerewrite} is clear, with $(Q_{out} - Q_{in})/\Delta x \to \partial Q/\partial x$ as $\Delta x \to 0$.  Thus a distributed flowline form of the Bartholomaus model is the partial differential equation
\begin{equation}
\frac{\phi W}{\rho_w g} \frac{\partial P}{\partial t} = - \frac{\partial Q}{\partial x} - \frac{f}{h} \left[u_b h + Z - C_c A_c (P_o-P)^n\right]. \label{eq:barth:distpressure}
\end{equation}

Such a distributed extension of the Bartholomaus model is not viable without a Darcy or other expression for $Q$.  A flux relation was not needed in the Kennicott glacier case because the input and output flux data form complete boundary conditions for a one-cell model.  However, using any reasonable Darcy-type formulation for the flux $Q$, such as the power laws (2.10) of \cite{Schoofetal2012}, equation \eqref{eq:barth:distpressure} becomes a nonlinear parabolic equation for the pressure.  Furthermore, when using a Darcy flux expression the $\phi\to 0$ (singular) limit of \eqref{eq:barth:distpressure} is an elliptic equation for the water pressure.  Specifically, the $\phi\to 0$ limit of \eqref{eq:barth:distpressure} is equation (2.12) in \cite{Schoofetal2012}.

Note that relation \eqref{eq:barth:distpressure}, and equation \eqref{eq:barth:steadypressure} below, should be solved subject to inequalities $0 \le P \le P_o$.  This is because the emergence of water at the surface of the glacier, from the efficiently-connected englacial system, bounds the subglacial pressure;\footnote{\cite{Bartholomausetal2011} report a supraglacial geyser lifting water a few meters above the glacier surface.  We suppose this situation is rare.} compare the physical justification of these bounds in \cite{Schoofetal2012}.

A two-horizontal-dimension version of equation \eqref{eq:barth:distpressure} is numerically solved in the ``distributed'' hydrology model code which is part of the Parallel Ice Sheet Model in its 0.6 release (February 2014).  We have observed that implementing a parabolic equation like \eqref{eq:barth:distpressure} subject to bounds $0 \le P \le P_o$ is substantially simpler than numerically solving the variational inequality form of the corresponding elliptic pressure equation \citep{Schoofetal2012}. 

An additional implication of the Bartholomaus model, again not stated in \cite{Bartholomausetal2011}, concerns the pressure in steady state.  Specifically, if we assume $dP/dt=0$ in equation \eqref{eq:barth:cavityevolution} then we can solve to find this relationship between pressure $P$ and cavity area $A_c$:
\begin{equation}
P = P_o - \left(\frac{u_b h + Z}{C_c A_c}\right)^{1/n}. \label{eq:barth:steadypressure}
\end{equation}
In the Bartholomaus model the cavities described by $A_c$ are assumed to always be water-filled---contrast \citep{Schoofetal2012}.  Thus equation \eqref{eq:barth:steadypressure} is equivalent to saying that in steady state the pressure is a function $P(W)$ of the area-averaged thickness $W$ of the subglacial water layer.  This is perhaps evidence supporting the \emph{a priori} inclusion of such a functional relationship into a model.  For example, the use of a power-law, like $P = P_o (W/W_{crit})^{7/2}$ used by \cite{FlowersClarke2002_theory}, may thereby be justified in some non-steady circumstances also.  On the other hand, relation \eqref{eq:barth:steadypressure} is not actually a power-law, though it is at least increasing with $W$.

Finally, it is interesting to observe from \eqref{eq:barth:steadypressure} that the steady water pressure does not depend on the englacial macroporosity $\phi$.  Though the englacial pressure is parameterized by $P=\rho_w g z_w$ in \cite{Bartholomausetal2011}, its \emph{steady} value is entirely determined by the balance between sliding, wall melt, and creep closure in the subglacial system.  The englacial system is passive in determining subglacial steady state.


\subsection*{Acknowledgements}  This work was supported by NASA grant \#NNX13AM16G.  Conversations with Ward van Pelt and Tim Bartholomaus were appreciated.  Ward, Constantine Khroulev, and Andy Aschwanden helped with the implementation and testing of the PISM code.

%         References
\bibliography{ice-bib}
\bibliographystyle{igs}

\end{document}
