\NeedsTeXFormat{LaTeX2e}

% \documentclass{igs}
% \documentclass[twocolumn]{igs}
  \documentclass[twocolumn,letterpaper]{igs}
% \documentclass[review]{igs}

  \usepackage{igsnatbib}
  \usepackage{stfloats}
\usepackage{amsmath}

% check if we are compiling under latex or pdflatex
  \ifx\pdftexversion\undefined
    \usepackage[dvips]{graphicx}
  \else
    \usepackage[pdftex]{graphicx}
  \fi

% the default is for unnumbered section heads
% if you really must have numbered sections, remove
% the % from the beginning of the following command
% and insert the level of sections you wish to be
% numbered (up to 4):

% \setcounter{secnumdepth}{2}


\usepackage{amssymb,alltt,verbatim,xspace,fancyvrb,color}
\usepackage[T1]{fontenc}

% hyperref should be the last package we load
\usepackage[pdftex,
                colorlinks=true,
                plainpages=false, % only if colorlinks=true
                linkcolor=blue,   % only if colorlinks=true
                citecolor=black,   % only if colorlinks=true
                urlcolor=magenta     % only if colorlinks=true
]{hyperref}

\ifx\text\undefined
\newcommand{\text}{\textrm}
\else
\fi


\begin{document}

\title{Correspondence \\ Extensions of the lumped subglacial-englacial \\ hydrology model of Bartholomaus, et al.~(2011)}

\author{Ed Bueler}

\affiliation{Department of Mathematics and Statistics and Geophysical Institute, University of Alaska Fairbanks, USA \\
E-mail: \emph{\texttt{elbueler\@@alaska.edu}}}

\abstract{[\emph{Not needed for Correspondence.}]}
\maketitle

The equations in \cite{Bartholomausetal2011} are chosen carefully so as to describe the remarkable hydrology of the Kennicott glacier in Alaska.  Their work uses observed input flux and a highly-simplified hydrology to reproduce the hydrograph of the yearly flood of the Kennicott river, and other hydrographs.  The goal of their model is to understand the drainage, through subglacial and englacial network which evolves over the time-scale of a day, of a lake which forms each summer on the side of the glacier.

Their model would seem to be significantly different from recent distributed models, specifically \cite{Schoofetal2012} and \cite{Hewittetal2012}.  Some obvious similarities exist among the models: They all describe the evolution of water-filled linked-cavity systems, with additional morphologies according to the model, and they include physical cavity opening and closing processes.  On the other hand the just-cited distributed theories are entirely subglacial, while the \cite{Bartholomausetal2011} model (from now on, the ``Bartholomaus model'') has both subglacial and englacial water storage.  Most essentially, however, the Bartholomaus model is ``lumped''.  That is, the entire glacier hydrological system is represented by one cell, with the major advantage that this allows parameter identification given the observed, also spatially-lumped, input and output fluxes.

The equations of \cite{Bartholomausetal2011} actually imply an unstated (lumped) pressure equation, which we will derive below.  The form of the implicit pressure equation is suggested by equation (12) in \cite{Bartholomausetal2011}.  The obvious distributed version of this pressure equation is a parabolic version of the elliptic pressure equation in \cite{Schoofetal2012}.  Said the other way, the pressure equation in \cite{Schoofetal2012} is the zero englacial porosity limit of the distributed version of the pressure equation implicit in the Bartholomaus model.

While a connection between the Bartholomaus theory and distributed models is observed by \cite{Hewitt2013}, the connection is limited to the addition of a englacial storage term in the mass conservation equation \cite[equation (7)]{Hewitt2013}.  Here we emphasize that, additionally, the elliptic (i.e.~instantaneous propagation in space) pressure equation of \cite{Schoofetal2012} acquires a parabolic regularization through the addition of englacial storage (i.e.~equation \eqref{eq:barth:distpressure} below).

Also, when restricted to steady state, the Bartholomaus model equations imply an apparently-unnoticed functional relationship between pressure and cavity size, a relationship that might motivate the choices made in an earlier hydrological theory \citep{FlowersClarke2002_theory}.  The current note explains these extensions.
 
In the Bartholomaus model, the total volume of liquid water stored in the glacier is $S(t)$.  This is split into englacial $S_{en}(t)$ and subglacial $S_{sub}(t)$ portions.  The cavities have geometry determined by bedrock bumps which have horizontal spacing $\lambda_x,\lambda_y$, height $h$, and width $w_c$.  These combine to give a dimensionless capacity parameter $f=(h w_c)/(\lambda_x \lambda_y)$; the value $f=0.05$ is used for the Kennicott glacier.  Each cavity has cross-sectional area $A_c(t)$ and volume $w_c A_c$.  The glacier occupies a rectangle of dimensions $L\times W$ in the map-plane so that the number of cavities is $\nu = (LW)/(\lambda_x\lambda_y)$.  It follows that the subglacial storage volume is $S_{sub} = (w_c A_c) \nu = (f L W/h) A_c$.

Englacial water is assumed to fill crevasses and moulins up to a level $z_w(t)$ above the bedrock, in a system which has macroporosity $\phi$ (dimensionless).  Thus the englacial storage is $S_{en}=L W \phi z_w$.  In summary, the Bartholomaus model uses this essentially-kinematical equation:
\begin{equation}
S = S_{en} + S_{sub},  \label{eq:barth:kinematics}
\end{equation}
along with proportionality between $S_{en}$ and $z_w$, and proportionality between $S_{sub}$ and $A_c$, respectively.
 
Mass conservation in the model is the simple statement \citep{Bartholomausetal2008}
\begin{equation}
\frac{dS}{dt} = Q_{in}(t) - Q_{out}(t). \label{eq:barth:massconserve}
\end{equation}
In the Kennicott glacier application, fluxes $Q_{in}$ and $Q_{out}$ are observed.

Let $P(t)$ be the subglacial water pressure.  Knowledge of the water pressure $P$ is equivalent to knowledge of the amount of englacial storage because there is an assumed efficient connection of the macroporous glacier to the subglacial system; note the equation $S_{en}=L W \phi z_w$.  Thus
\begin{equation}
P = \rho_w g z_w.  \label{eq:barth:englacialpressure}
\end{equation}

In the Bartholomaus model the cavity cross-sectional area $A_c$ evolves by physical opening and closure processes.  A wall melt parameterization is also given but, in keeping with the cavity evolution in the rest of the current paper, we simply denote it as a melt term $\dot m$.  Denote the sliding speed by $u_b$ and let $C_c = (2 A)/n^n$, where $A$ and $n$ are the usual parameters in the Glen ice flow law \citep{CuffeyPaterson}.  Let $P_o=\rho_i g H$ be the overburden pressure.  Then the cavity area evolution equation in \cite{Bartholomausetal2011} is
\begin{equation}
\frac{dA_c}{dt} = \dot m + u_b h - C_c A_c (P_o-P)^n.  \label{eq:barth:cavityevolution}
\end{equation}
The three terms on the right are opening by melt, cavitation, and closure by creep, respectively.

Equations \eqref{eq:barth:kinematics}, \eqref{eq:barth:massconserve}, \eqref{eq:barth:englacialpressure}, and \eqref{eq:barth:cavityevolution} combine to give an evolution equation for the pressure.  From \eqref{eq:barth:kinematics} and \eqref{eq:barth:englacialpressure} we can write the pressure rate of change in terms of the englacial storage rate of change:
\begin{equation}
\frac{dP}{dt} = \rho_w g \frac{dz_w}{dt} = \frac{\rho_w g}{L W \phi} \frac{d S_{en}}{dt}. \label{eq:barth:dPdt}
\end{equation}
By combining \eqref{eq:barth:dPdt} with  the time-derivative of \eqref{eq:barth:kinematics}, and using both \eqref{eq:barth:massconserve} and the proportionality between $S_{sub}$ and $A_c$, we can rewrite in terms of fluxes and cavity area:
\begin{align}
\frac{dP}{dt} &= \frac{\rho_w g}{L W \phi} \left(\frac{d S}{dt} - \frac{d S_{sub}}{dt}\right) \\
&= \frac{\rho_w g}{L W \phi} \left(Q_{in} - Q_{out} - \frac{f L }{h} \frac{d A_c}{dt}\right). \notag
\end{align}
Finally incorporate equation \eqref{eq:barth:cavityevolution} to eliminate $dA_c/dt$:
\begin{align}
&\frac{dP}{dt} = \frac{\rho_w g}{L W \phi} \bigg(Q_{in} - Q_{out} \label{eq:barth:fullpressure} \\
&\qquad \qquad \qquad - \frac{f L }{h} \left[\dot m + u_b h - C_c A_c (P_o-P)^n\right]\bigg) \notag
\end{align}
Thus, though it is not stated there, \eqref{eq:barth:fullpressure} follows from the equations in \cite{Bartholomausetal2011}.

Equation \eqref{eq:barth:fullpressure} also suggests how to extend the \cite{Bartholomausetal2011} theory from ``lumped'' into ``distributed.''  Consider a one-dimensional glacier flowing in the positive $x$ direction.  Let the transverse width be $W$ and replace $L$ by $\Delta x$, the length of a cell.  Note that ``$Q_{in}$'' would, in a distributed theory, be the upstream input flux into the cell while ``$Q_{out}$'' would be the downstream output.  Thus we rewrite \eqref{eq:barth:fullpressure} as
\begin{align}
&\frac{\phi W}{\rho_w g}\frac{dP}{dt} = - \frac{Q_{out} - Q_{in}}{\Delta x} \\
&\qquad \qquad \qquad \quad - \frac{f}{h} \left[\dot m + u_b h - C_c A_c (P_o-P)^n\right]. \notag
\end{align}
The continuum limit is then clear, with $(Q_{out} - Q_{in})/\Delta x \to \partial Q/\partial x$ as $\Delta x \to 0$.  Thus a distributed flowline form of the \cite{Bartholomausetal2011} theory is the partial differential equation
\begin{equation}
\frac{\phi W}{\rho_w g} \frac{\partial P}{\partial t} = - \frac{\partial Q}{\partial x} - \frac{f}{h} \left[\dot m + u_b h - C_c A_c (P_o-P)^n\right]. \label{eq:barth:distpressure}
\end{equation}

Of course a distributed extension of the \cite{Bartholomausetal2011} theory is not viable without a Darcy or other flux expression for $Q$.  Note that such a relation was not needed in the Kennicott glacier case because the input and output flux data form complete boundary conditions for a one-cell model, which has no inter-cell boundaries.  However, under any reasonable Darcy-type formulation for the flux $Q$ such as the power laws (2.10) of \cite{Schoofetal2012}, the $\phi\to 0$ limit of \eqref{eq:barth:distpressure} is an elliptic equation for the water pressure.  Specifically, the $\phi\to 0$ limit of \eqref{eq:barth:distpressure} is equation (2.12) in \cite{Schoofetal2012}.

An additional implication of the Bartholomaus model, again not stated in \cite{Bartholomausetal2011}, regards the pressure in steady state.  Specifically, if we assume steady state in equation \eqref{eq:barth:cavityevolution} then we can solve for pressure to find this relationship between pressure $P$ and cavity area $A_c$:
\begin{equation}
P = P_o - \left(\frac{\dot m + u_b h}{C_c A_c}\right)^{1/n}. \label{eq:barth:steadypressure}
\end{equation}

Because in the Bartholomaus model the cavities described by $A_c$ are assumed to be water filled---contrast \citep{Schoofetal2012}---equation \eqref{eq:barth:steadypressure} says that in steady state the pressure is a function of the amount of water, $P=P(W)$.  Because such a functional relationship is thereby supported in steady state, the use of a power-law, like $P = P_o (W/W_{crit})^{7/2}$ used by \cite{FlowersClarke2002_theory}, may be justified in some non-steady circumstances also.  On the other hand, relation \eqref{eq:barth:steadypressure} is not qualitatively-close to a power-law, other than to be increasing with $W$.  Relation \eqref{eq:barth:steadypressure} should be used subject to inequalities $0 \le P \le P_o$ because the emergence of water at the surface of the glacier, from the efficiently connected englacial system, bounds the subglacial pressure.

Finally, it is interesting to observe from \eqref{eq:barth:steadypressure} that the steady water pressure does not depend on the englacial macroporosity $\phi$.  Though the englacial pressure is parameterized by $P=\rho_w g z_w$ in \cite{Bartholomausetal2011}, its \emph{steady} value is entirely determined by the balance between sliding, wall melt, and creep closure in the subglacial system.  The englacial system is passive in determining subglacial steady state.


\subsection*{Acknowledgements}  This work was supported by NASA grant \#NNX13AM16G.  Conversations with Ward van Pelt and Tim Bartholomaus were much appreciated.

%         References
\bibliography{ice-bib}
\bibliographystyle{igs}

\end{document}
