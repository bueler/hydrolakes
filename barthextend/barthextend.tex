\NeedsTeXFormat{LaTeX2e}

% \documentclass{igs}
% \documentclass[twocolumn]{igs}
% \documentclass[twocolumn,letterpaper]{igs}
 \documentclass[review,letterpaper]{igs}

  \usepackage{igsnatbib}
  %\usepackage{stfloats}

% check if we are compiling under latex or pdflatex
  \ifx\pdftexversion\undefined
    \usepackage[dvips]{graphicx}
  \else
    \usepackage[pdftex]{graphicx}
  \fi


\usepackage{amsmath}

% only include this in review mode:
% for "review" mode, fix lineno problem
\newcommand*\patchAmsMathEnvironmentForLineno[1]{%
  \expandafter\let\csname old#1\expandafter\endcsname\csname #1\endcsname
  \expandafter\let\csname oldend#1\expandafter\endcsname\csname end#1\endcsname
  \renewenvironment{#1}%
     {\linenomath\csname old#1\endcsname}%
     {\csname oldend#1\endcsname\endlinenomath}}% 
\newcommand*\patchBothAmsMathEnvironmentsForLineno[1]{%
  \patchAmsMathEnvironmentForLineno{#1}%
  \patchAmsMathEnvironmentForLineno{#1*}}%
\AtBeginDocument{%
\patchBothAmsMathEnvironmentsForLineno{equation}%
\patchBothAmsMathEnvironmentsForLineno{align}%
\patchBothAmsMathEnvironmentsForLineno{flalign}%
\patchBothAmsMathEnvironmentsForLineno{alignat}%
\patchBothAmsMathEnvironmentsForLineno{gather}%
\patchBothAmsMathEnvironmentsForLineno{multline}%
}



\begin{document}

\title[Correspondence: Extending the Bartholomaus hydrology model]{Correspondence: \\ Extending the lumped subglacial-englacial \\ hydrology model of Bartholomaus and others (2011)}

\author{Ed Bueler}

\affiliation{Department of Mathematics and Statistics and Geophysical Institute, University of Alaska Fairbanks, USA \\
E-mail: \emph{\texttt{elbueler\@@alaska.edu}}}

\abstract{[\emph{Not needed for Correspondence.}]}

\maketitle

\cite{Bartholomausetal2011} model the subglacial and englacial hydrology of the Kennicott Glacier in Alaska, and the glacier's response to a lateral lake outburst flood which occurs each summer, that is, a j\"okulhlaup.  The model uses observed input flux and highly-simplified subglacial and englacial morphology to reproduce the hydrograph of the Kennicott River, other hydrographs, and the glacier motion (sliding).  The model is ``lumped,'' that is, the entire hydrological system is represented by one cell, with the advantage that parameter identification is possible given the observed input and output fluxes.

The goal of my short note is to extend their model in three specific ways: (\emph{i}) to show a (lumped) pressure equation which follows from their equations, (\emph{ii}) to show how to extend their model to the distributed flow-line case, with attention to the form of the distributed pressure equation, and (\emph{iii}) to state a water amount versus pressure relation which applies in the steady-state case of their model.  Thus this note is largely deductive, showing what holds in the \cite{Bartholomausetal2011} theory, but with a distributed model as an extension.

Their model (from now on, the ``Bartholomaus model'') would seem to be significantly-different from more recent distributed models, specifically the theory in \citep{Hewitt2011,Schoofetal2012,Hewittetal2012}.  Some similarities exist among them, however.  All of these theories describe the evolution of subglacial linked-cavity systems and include physical opening and closing processes for these cavities.  On the other hand the just-cited distributed theories are entirely subglacial, while the Bartholomaus has both subglacial and englacial water storage.  

The distributed version of the lumped pressure equation derivable in the Bartholomaus model (i.e.~\eqref{eq:barth:distpressure} below) is a parabolic regularization of the elliptic pressure equation in \cite{Schoofetal2012}.  Said the other way, we show that the pressure equation in \cite{Schoofetal2012} is the zero englacial porosity limit of the distributed version of the pressure equation implicit in the Bartholomaus model.  A connection between the Bartholomaus model and distributed models is observed by \cite{Hewitt2013}, but this is limited to the addition of an englacial storage term in the mass conservation equation.  \cite{Werderetal2013} include a parabolic pressure equation, with englacial void ratio in the coefficient of the pressure rate term, similar to what is implicit in the Bartholomaus model.  Exposing these connections of the Bartholomaus model to more recent literature is a motivation for this short note.

In the Bartholomaus model, the total volume of liquid water stored in the glacier is $S(t)$, and this is split into englacial $S_{en}(t)$ and subglacial $S_{sub}(t)$ portions:
\begin{equation}
S = S_{en} + S_{sub}.  \label{eq:barth:kinematics}
\end{equation}
Mass conservation in the model is the simple statement \citep{Bartholomausetal2008}
\begin{equation}
\frac{dS}{dt} = Q_{in}(t) - Q_{out}(t). \label{eq:barth:massconserve}
\end{equation}
In the Kennicott glacier application, fluxes $Q_{in}$ and $Q_{out}$ are observed.

The subglacial cavities have geometry determined by bedrock bumps which have horizontal spacing $\lambda_x,\lambda_y$, height $h$, and width $w_c$.  These combine to give a dimensionless capacity parameter $f=h w_c/(\lambda_x \lambda_y)$; the value $f=0.05$ is used for the Kennicott glacier.  Each cavity has cross-sectional area $A_c(t)$ and thus volume $w_c A_c$.  The glacier occupies a rectangle of dimensions $L\times W$ in the map-plane so that the number of cavities is $\nu = (LW)/(\lambda_x\lambda_y)$.  It follows that the subglacial storage volume is $S_{sub} = (w_c A_c) \nu = f L W A_c / h$, proportional to $A_c$ in this lumped case.

Englacial water is assumed to fill a well-connected system of crevasses and moulins up to a level $z_w(t)$ above the bedrock, in a system which has macroporosity $\phi$, so the englacial storage is $S_{en}=L W \phi z_w$.  Denote the subglacial water pressure by $P(t)$.  In the Bartholomaus model, $P$ is hydrostatic so knowledge of $P$ is equivalent to knowledge of englacial storage, because of the assumed efficient connection to the subglacial system; the englacial system acts as a piezometer.  In fact, noting the relation $S_{en}=L W \phi z_w$ above, and denoting the density of fresh water by $\rho_w$ and gravity by $g$, we have
\begin{equation}
P = \rho_w g z_w = \frac{\rho_w g}{LW\phi} S_{en}.  \label{eq:barth:englacialpressure}
\end{equation}

Now, in the Bartholomaus model the cavity cross-sectional area $A_c$ evolves by physical opening and closure processes.  The rate of production of subglacial water through wall melt is denoted here simply by $Z$, with sign $Z>0$ when cavities are enlarging.  The melt rate $Z$ is further parameterized in the Bartholomaus model, but the details are unimportant to the derivation here.  Let $C_c = (2 A)/n^n$, where $A$ and $n$ are the ice softness parameter and power in the Glen ice flow law \citep{CuffeyPaterson}, respectively.  Denote the sliding speed by $u_b$ and let $P_o=\rho_i g H$ be the overburden pressure, where $\rho_i$ is the density of ice.  In these terms the cavity evolution equation is
\begin{equation}
\frac{dA_c}{dt} = u_b h + Z - C_c A_c (P_o-P)^n.  \label{eq:barth:cavityevolution}
\end{equation}
The three terms on the right are opening by cavitation, opening by wall melt, and closure by creep, respectively; compare (4) in \cite{Bartholomausetal2011}.

Equations \eqref{eq:barth:kinematics}--\eqref{eq:barth:cavityevolution} combine to give an evolution equation for the pressure, as follows.  From \eqref{eq:barth:englacialpressure} we can write the pressure rate of change $dP/dt$ in terms of the englacial storage rate of change $d S_{en}/dt$.  Then using the time-derivative of \eqref{eq:barth:kinematics}, and both \eqref{eq:barth:massconserve} and the proportionality between $S_{sub}$ and $A_c$, we can rewrite in terms of fluxes and cavity area:
\begin{align}
\frac{dP}{dt} &= \frac{\rho_w g}{L W \phi} \frac{d S_{en}}{dt} \\
&= \frac{\rho_w g}{L W \phi} \left(\frac{d S}{dt} - \frac{d S_{sub}}{dt}\right) \notag \\
&= \frac{\rho_w g}{L W \phi} \left(Q_{in} - Q_{out} - \frac{f L W}{h} \frac{d A_c}{dt}\right). \notag
\end{align}
Finally, equation \eqref{eq:barth:cavityevolution} eliminates $dA_c/dt$:
\begin{align}
&\frac{dP}{dt} = \frac{\rho_w g}{L W \phi} \bigg(Q_{in} - Q_{out} \label{eq:barth:fullpressure} \\
&\qquad \qquad \qquad - \frac{f L W}{h} \left[u_b h + Z - C_c A_c (P_o-P)^n\right]\bigg). \notag
\end{align}
Thus, though it is not stated there, \eqref{eq:barth:fullpressure} follows from the equations in \cite{Bartholomausetal2011}.  Equation (12) in \cite{Bartholomausetal2011} is a restriction of \eqref{eq:barth:fullpressure} to the case of no creep closure and no input/output fluxes, however.

Furthermore, equation \eqref{eq:barth:fullpressure} suggests how to extend the Bartholomaus model from ``lumped'' to ``distributed.''  Consider a one-dimensional glacier under which the water is flowing in the positive $x$ direction.  Recalling $W$ is the transverse width, replace $L$ by $\Delta x$, the along-flow length of one cell in a finite difference or finite volume scheme.  Then \eqref{eq:barth:fullpressure} becomes
\begin{align}
&\frac{\phi W}{\rho_w g}\frac{dP}{dt} = - \frac{Q_{out} - Q_{in}}{\Delta x}  \label{eq:barth:pressurerewrite} \\
&\qquad \qquad \qquad \quad - \frac{f W}{h} \left[u_b h + Z - C_c A_c (P_o-P)^n\right], \notag
\end{align}
which still describes the pressure in one cell.  Because ``$Q_{in}$'' is the upstream input flux into the cell while ``$Q_{out}$'' is the downstream output, the continuum limit of \eqref{eq:barth:pressurerewrite} is therefore clear, with $(Q_{out} - Q_{in})/\Delta x \to \partial Q/\partial x$ as $\Delta x \to 0$.  Thus a distributed flowline form of the Bartholomaus model includes the partial differential equation
\begin{equation}
\frac{\phi W}{\rho_w g} \frac{\partial P}{\partial t} = - \frac{\partial Q}{\partial x} - \frac{f W}{h} \left[u_b h + Z - C_c A_c (P_o-P)^n\right]. \label{eq:barth:distpressure}
\end{equation}

A distributed extension of the Bartholomaus model is not viable without a Darcy or other expression for $Q$.  Such a flux parameterization was not needed in the Kennicott glacier application because the input and output flux data form complete boundary conditions for a one-cell model.  However, using any reasonable Darcy-type formulation for the flux $Q$, such as the power laws (2.10) of \cite{Schoofetal2012}, equation \eqref{eq:barth:distpressure} becomes a nonlinear parabolic equation for the pressure.  Furthermore, when using a Darcy flux expression, the $\phi\to 0$ (singular) limit of \eqref{eq:barth:distpressure} is an elliptic equation for the water pressure, namely equation (2.12) in \cite{Schoofetal2012}.

Equation \eqref{eq:barth:distpressure} should be solved in a time-dependent, coupled system with \eqref{eq:barth:cavityevolution}.  The state space of the resulting model, i.e.~the values which must be given as initial conditions for these evolution equations, is the pair of variables $(A_c,P)$, where both $A_c$ (equivalently, the subglacial water layer thickness $\eta$) and $P$ depend on time $t$ and space $x$.

Note that equation \eqref{eq:barth:distpressure} should be solved subject to inequalities $0 \le P \le (\rho_w/\rho_i) P_o$.  This is because the emergence of water at the surface of the glacier, from the efficiently-connected englacial system, bounds the subglacial pressure.  (Though \cite{Bartholomausetal2011} report a supraglacial geyser lifting water a few meters above the glacier surface, which violates the given upper bound, the use of the bound in a model supposes reasonably that such overpressure geysers are short-lived and small in magnitude relative to overburden.)  Compare the physical justification of related bounds $0 \le P \le P_o$ in \cite{Schoofetal2012}, which apply in the $\phi\to 0$ limit at least.

An additional implication of the Bartholomaus model, again not stated in \cite{Bartholomausetal2011}, concerns the pressure in steady state.  Specifically, $dP/dt=0$ in equation \eqref{eq:barth:cavityevolution} gives this relationship between pressure $P$ and cavity area $A_c$:
\begin{equation}
P = P_o - \left(\frac{u_b h + Z}{C_c A_c}\right)^{1/n}. \label{eq:barth:steadypressure}
\end{equation}
In the Bartholomaus model the cavities described by $A_c$ are assumed to always be water-filled---contrast \citep{Schoofetal2012}.  Thus equation \eqref{eq:barth:steadypressure} is equivalent to saying that in steady state the pressure is a function $P(\eta)$ of the area-averaged thickness $\eta$ of the subglacial water layer. Equation \eqref{eq:barth:steadypressure} also suggests the behavior of coupled equations \eqref{eq:barth:cavityevolution} and \eqref{eq:barth:fullpressure} (or \eqref{eq:barth:cavityevolution} and \eqref{eq:barth:distpressure} in the distributed case) at large englacial porosity values.  These are perhaps reasons supporting the \emph{a priori} inclusion of a water-amount-versus-pressure functional relationship like \eqref{eq:barth:steadypressure} into a model, for use only on longer-timescale questions, but this is certainly not recommended for numerical models which are intended to follow the time-scales which drove the creation of the Bartholomaus model.

A power-law $P = P_o (\eta/\eta_{crit})^{7/2}$, with $\eta_{crit}$ a positive constant, is used by \cite{FlowersClarke2002_theory} in non-steady circumstances also.  Relation \eqref{eq:barth:steadypressure} is not a power-law, however, though it is at least increasing with $\eta$ like the \cite{FlowersClarke2002_theory} relation.

It is interesting to observe from \eqref{eq:barth:steadypressure} that the steady water pressure does not depend on the englacial macroporosity $\phi$.  Though the englacial pressure is parameterized by $P=\rho_w g z_w$ in \cite{Bartholomausetal2011}, its \emph{steady} value is entirely determined by the balance between sliding, wall melt, and creep closure in the subglacial system.  The englacial system is passive in determining subglacial steady state but obviously critical in determining timescales of the hydrological, and thus glacier sliding, response.  In particular, the englacial porosity parameter $\phi$ is critical in this role \citep{vanPeltthesis}.

A two-horizontal-dimension version of coupled equations \eqref{eq:barth:cavityevolution} and \eqref{eq:barth:distpressure}, using a full-cavity assumption as in the Bartholomaus model, and using the general Darcy flux relation (2.10) from \cite{Schoofetal2012}, is numerically solved in the ``distributed'' hydrology model code which is part of the Parallel Ice Sheet Model (PISM) in its 0.6 release (February 2014); see \cite{BuelervanPeltDRAFT}.  The steady state result \eqref{eq:barth:steadypressure} is not used in the PISM implementation, other than when a user supplies to PISM the initial values for water amount (i.e.~$A_c$ or $\eta$ in the above equations) \emph{without} supplying initial values for pressure; the result of \eqref{eq:barth:steadypressure} is as good an initial guess for pressure as any other in this data-deficient situation.  We have observed that implementing a parabolic pressure equation like \eqref{eq:barth:distpressure}, subject to either of the above bounds on pressure, is substantially simpler than numerically solving the variational inequality form of the corresponding elliptic pressure equation \citep{Schoofetal2012}.  In particular, our work shows that physical pressure bounds can be preserved, a part of the linked-cavities model dismissed as ``prohibitively expensive'' by \cite{Werderetal2013}, by explicit time-stepping of the parabolic pressure equation \citep{BuelervanPeltDRAFT}.


\subsection*{Acknowledgements}  This work was supported by NASA grant \#NNX13AM16G.  Conversations with Ward van Pelt and Tim Bartholomaus, and comments by two anonymous reviewers, were much appreciated.

%         References
\bibliography{ice-bib}
\bibliographystyle{igs}

\end{document}
