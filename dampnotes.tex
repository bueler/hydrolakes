\documentclass[11pt,final]{amsart}%default 10pt
%prepared in AMSLaTeX, under LaTeX2e

\usepackage[total={6.2in,9.0in},top=1.2in,left=1.1in]{geometry}

\usepackage{natbib}

\usepackage{amssymb,alltt,verbatim,xspace,fancyvrb,color,empheq}
\usepackage{palatino}
\usepackage[sc]{mathpazo}
\usepackage[T1]{fontenc}

% check if we are compiling under latex or pdflatex
\ifx\pdftexversion\undefined
  \usepackage[final,dvips]{graphicx}
\else
  \usepackage[final,pdftex]{graphicx}
\fi

% hyperref should be the last package we load
\usepackage[pdftex,
                colorlinks=true,
                plainpages=false, % only if colorlinks=true
                linkcolor=blue,   % only if colorlinks=true
                citecolor=black,   % only if colorlinks=true
                urlcolor=magenta     % only if colorlinks=true
]{hyperref}

\newcommand{\normalspacing}{\renewcommand{\baselinestretch}{1.05}\tiny\normalsize}
\newcommand{\tablespacing}{\renewcommand{\baselinestretch}{1.0}\tiny\normalsize}
\normalspacing

\definecolor{myblue}{rgb}{.8, .8, 1}

\newcommand*\mybluebox[1]{%
\colorbox{myblue}{\hspace{1em}#1\hspace{1em}}}

% math macros
\newcommand\bv{\mathbf{v}}
\newcommand\bV{\mathbf{V}}
\newcommand\bq{\mathbf{q}}
\newcommand\bQ{\mathbf{Q}}

\newcommand\CC{\mathbb{C}}
\newcommand{\DDt}[1]{\ensuremath{\frac{d #1}{d t}}}
\newcommand{\ddt}[1]{\ensuremath{\frac{\partial #1}{\partial t}}}
\newcommand{\ddx}[1]{\ensuremath{\frac{\partial #1}{\partial x}}}
\newcommand{\ddy}[1]{\ensuremath{\frac{\partial #1}{\partial y}}}
\newcommand{\ddxp}[1]{\ensuremath{\frac{\partial #1}{\partial x'}}}
\newcommand{\ddz}[1]{\ensuremath{\frac{\partial #1}{\partial z}}}
\newcommand{\ddxx}[1]{\ensuremath{\frac{\partial^2 #1}{\partial x^2}}}
\newcommand{\ddyy}[1]{\ensuremath{\frac{\partial^2 #1}{\partial y^2}}}
\newcommand{\ddxy}[1]{\ensuremath{\frac{\partial^2 #1}{\partial x \partial y}}}
\newcommand{\ddzz}[1]{\ensuremath{\frac{\partial^2 #1}{\partial z^2}}}
\newcommand{\Div}{\nabla\cdot}
\newcommand\eps{\epsilon}
\newcommand{\grad}{\nabla}
\newcommand{\ihat}{\mathbf{i}}
\newcommand{\ip}[2]{\ensuremath{\left<#1,#2\right>}}
\newcommand{\jhat}{\mathbf{j}}
\newcommand{\khat}{\mathbf{k}}
\newcommand{\nhat}{\mathbf{n}}
\newcommand\lam{\lambda}
\newcommand\lap{\triangle}
\newcommand\Matlab{\textsc{Matlab}\xspace}
\newcommand\RR{\mathbb{R}}
\newcommand\vf{\varphi}

\newcommand{\Wlij}{W^l_{i,j}}
\newcommand{\Wij}{W_{i,j}}
\newcommand{\Plij}{P^l_{i,j}}
\newcommand{\Pij}{P_{i,j}}
\newcommand{\Ylij}{Y^l_{i,j}}
\newcommand{\Yij}{Y_{i,j}}
\newcommand{\upp}[3]{\big<#1\big|_{#3}\,#2\big>}

\newcommand{\Nbreen}{Nordenski\"oldbreen\xspace}


\title[]{A diffusive-closure model of subglacial hydrology}

\author[]{Ward van Pelt and Ed Bueler}


\begin{document}
\scriptsize \hfill \today \normalsize
\vspace{0.5in}

\maketitle
\thispagestyle{empty}

\section{Introduction}

Any reasonable dynamical model of the subglacial aquifer (liquid water layer) under a glacier or ice sheet has at least these two elements: the mass of the water is conserved and the water flows from high to low hydraulic potential.  Physical processes also control the geometry of the aquifer, including the opening of cavities by motion (sliding) of the overlying ice, and the closure of cavities and channels by creep.  These physical processes should combine to determine the local relationship between water amount and water pressure.  This paper describes such a model for distributed systems of linked cavities. 

We model water pressure using a damped approximation of a full-cavity formulation of a distributed system \cite{Schoofetal2012}.  In our model pressure is determined non-locally over each connected component of the hydrological system.  However, unlike the \cite{Schoofetal2012} model in which an elliptic variational inequality for pressure must be solved at each time step, we avoid solving an instantaneous distributed balance by not exactly enforcing the full-cavity condition.  Our model replaces such an elliptic pressure description by a parabolic approximation.


\section{Elements of a subglacial hydrology continuum model}

We consider a layer of water with thickness $W(t,x,y)$.  This thickness is only meaningful compared to observations if it is regarded as an average \citep{FlowersClarke2002_theory} over a horizontal scale of tens to thousands of meters.  As the hydrologic system has fine spatial variation which one is unable to model with cells of kilometer size, we only attempt to model such spatially-averaged versions of water amount and water pressure.

We assume that water is incompressible and of constant density.  Thus the thickness of the water layer tells us its mass.  Choosing to model subglacial hydrology using a water thickness is not a significant restriction on the physics.  Specific aquifer physics comes from choosing a form for the water flux and a closure for pressure.


\subsection*{Mass conservation}  Suppose $\bq$ is the vector water flux (units $\text{m}^2\,\text{s}^{-1}$) and $\Phi$ is a source term ($\text{m}\,\text{s}^{-1}$).  In two spatial dimensions the mass conservation equation is \citep{Clarke05}
\begin{equation} \label{eq:conserve}
\frac{\partial W}{\partial t} + \Div \bq = \Phi
\end{equation}
We always assume the water thickness is nonnegative:
\begin{equation}
W \ge 0.
\end{equation}

We might separate the water sources between the melt on the lower surface of the glacier versus englacial and/or supraglacial drainage:
\begin{equation}
\Phi = \rho_w^{-1} \left(m + S\right).  \label{drainagesplit}
\end{equation}
Here $\rho_w$ is the density of fresh liquid water, $m$ is the rate at which basal melting (refreeze) of ice adds (removes) water, and $S$ is the rate at which surface runoff or englacial drainage adds water.  For the \Nbreen, Svalbard example below we will take $m=0$ so that in fact only supraglacial drainage input is modeled.


\subsection*{Hydraulic potential and Darcy flow}  The water flux $\bq$ in equation \eqref{eq:conserve} is related to the gradient of a hydraulic potential (head) $\psi(t,x,y)$.  This quantity combines the actual subglacial water pressure $P(t,x,y)$ and the gravitational potential of the top of the water layer,
\begin{equation} \label{eq:potential}
\psi = P + \rho_w g\, (b+W).
\end{equation}
Here $\rho_w$ is the water density ($\text{kg}\,\text{m}^{-3}$), $g$ is the acceleration of gravity ($\text{m}\,\text{s}^{-2}$), and $z=b(x,y)$ ($\text{m}$) is the bedrock elevation, which for simplicity is time-independent.

We have added the term ``$\rho_w g W$'' to the best-known hydraulic potential formulas \citep[e.g.]{Clarke05} for the reason that is given by \cite{Hewittetal2012}, namely that it is correct because differences in the potential at the top of the water layer determines the driving potential gradient.  The term is important when considering local minima of the hydraulic potential, where subglacial lakes of finite (not infinitesimal) extent and finite (not infinite) depth should form.  As we will see, this term makes the mass conservation equation diffusive, though the diffusion is only significant when the water depth becomes substantial ($W\gg 1$).

Water flows from high to low hydraulic potential.  The simplest applicable expression of this property is a Darcy flux model for a water sheet following \cite{Clarke05}, namely
\begin{equation}  \label{eq:flux}
\bq = - \frac{K \, W}{\rho_w g} \grad \psi
\end{equation}
Here $K$ is the effective hydraulic conductivity ($\text{m}\,\text{s}^{-1}$).  The system transmits more water for a given head gradient if either the ability of the subglacial material to conduct water is bigger (i.e.~$K$ is larger) or if the water sheet is thicker ($W$ is larger).

Alternatively, following \cite{Schoofetal2012} we could take
\begin{equation}  \label{eq:fluxalt}
\bq = - \,\tilde k\, W^\alpha\, |\grad \psi|^{\beta-2} \grad \psi
\end{equation}
for $\alpha> 1$, $\beta>1$, and some positive coefficient $\tilde k$.  This more nonlinear form is justified as an instance of a Manning or Darcy-Weisbach law.  Note that \eqref{eq:flux} is the $\alpha=1,\beta=2$ case of \eqref{eq:fluxalt}.  While the significance of ``$\alpha>1$'' as a restriction on the possibilities covered by \eqref{eq:fluxalt} is not clear to the current authors, we believe that nothing will be fundamentally different about our model, either its exact solutions, its explicit numerical schemes, or the degree to which parameters can be identified, if we were to adopt \eqref{eq:fluxalt} instead of \eqref{eq:flux}.

% FIXME:  we use beta=2, which is in the allowed range.  using \beta<2, like beta=3/2 used by Schoof et al 2012, might actually help because the diffusive pressure equation would have a larger diffusion coefficient for areas where the pressure was not varying rapidly already; on the other hand beta<2 might be trouble because the max diffusivity could be locally very large, thus limiting time step, even though its global average was not too large


\subsection*{Overburden pressure and pressure bounds}  Recall now that the ice is a fluid which has a pressure field of its own, with basal value $P_o$, the \emph{overburden pressure}.  In this paper we make the shallow approximation that the ice pressure is hydrostatic \citep{GreveBlatter2009}:
\begin{equation} \label{eq:hydrostatic}
  P_o = \rho_i g H = \rho_i g (h-b).
\end{equation}
Here $\rho_i$ is the density of ice ($\text{kg}\,\text{m}^{-3}$), $H$ is the ice thickness (m), and $h$ is the ice upper surface elevation (m).  Now define the \emph{effective pressure}
\begin{equation}
N = P_o - P\label{eq:effective}
\end{equation}
This quantity measures how much of the ice load is carried by the mineral (till or bedrock) base, as opposed to how much is carried by pressurized subglacial water.

Because $P$ is nonnegative, and because the condition $P>P_o$ is presumed to cause the ice to lift and thus quickly lower the pressure back to overburden $P=P_o$ \citep{Schoofetal2012}, the pressure solution is subject to inequalities
\begin{equation}
0 \le P \le P_o. \label{eq:bounds}
\end{equation}
Though extreme cases could potentially occur, wherein a vapor layer forms in the cavity, or the ice is forced upward by a negative effective pressure, respectively \citep{Schoofetal2012}, we do not model these.  We will enforce bounds \eqref{eq:bounds} on the water pressure.


\subsection*{Advection-diffusion decomposition}  Combining \eqref{eq:potential} and \eqref{eq:flux} above we get the flux expression
\begin{equation}
  \bq = - \frac{K\, W}{\rho_w g} \left(\grad P + \rho_w g b\right) - K W \grad W. \label{eq:firstfluxdecomp}
\end{equation}
Thereby we have identified a part of the flux, the second term $-K W \grad W$ in \eqref{eq:firstfluxdecomp}, which is proportional to the gradient of the water thickness.

The first flux term in \eqref{eq:firstfluxdecomp} will dominate under the condition $|\grad H| \gg |\grad W|$.  This condition supposes that subglacial water pressure scales with the ice overburden pressure, which we have assumed is approximately hydrostatic.  Because $W$ represents a spatial average, and not the detailed local water thickness which would vary rapidly from cavity to adjacent cavity, $|\grad H| \gg |\grad W|$ is common because generally $H\gg W$.  The first flux term in \eqref{eq:firstfluxdecomp} will also dominate when $|\grad b| \gg |\grad W|$. 

We conceive of the decomposed flux as a transport process which is partly driven by a velocity field which varies in space and time (the first term in \eqref{eq:firstfluxdecomp}) and the remainder of which acts diffusively (the second term).  We will construct our conservative numerical scheme based on this understanding of how the flux is decomposed.  Also, we provide a related time scales analysis in section \ref{sec:num} below.  We will see later, however, that in near-steady-state circumstances the transport velocity (the first term in \eqref{eq:firstfluxdecomp}) is actually \emph{diffusive} in the mass conservation equation.

Define the velocity field
\begin{equation} \label{eq:vexpression}
  \bV = - \frac{K}{\rho_w g} \grad P - K \grad b.
\end{equation}
Equations \eqref{eq:potential}, \eqref{eq:flux}, and \eqref{eq:vexpression} combine to give the following advection-diffusion description of the flux,
\begin{equation} \label{eq:qexpression}
  \bq = \bV\, W - K W \grad W.
\end{equation}
From equations \eqref{eq:conserve} and \eqref{eq:qexpression} we derive an advection-diffusion equation \citep{HundsdorferVerwer2010} for the evolution of the water layer thickness:
\begin{equation} \label{eq:adeqn}
  \frac{\partial W}{\partial t} + \Div\left(\bV\, W\right) = \Div \left(K W \grad W\right) + \Phi.
\end{equation}

There are different numerical schemes for the advection term $\Div\left(\bV\, W\right)$ and the diffusion term $\Div \left(K W \grad W\right)$ in \eqref{eq:adeqn}; see section \ref{sec:num}.  These different schemes impose time step restrictions of different magnitudes.  We will see in practice that equation \eqref{eq:adeqn} is ``advection-dominated'' in the sense that the term $\Div\left(\bV\, W\right)$ dominates.  However, when  $\bV$ is nearly proportional to $-\grad W$, which occurs in near-steady conditions (section \ref{sec:steadyverif}), the numerical scheme for advection must generate a reasonable diffusiion approximation.

As is well known \citep{Clarke05}, the $\bq$ depends significantly on the ice surface slope because the ice overburden pressure dominates the subglacial water pressure.  Therefore the gradient of the hydraulic potential frequently follows the ice surface gradient.  The pressure model which we construct and use in this paper generally generates pressure fields with this property.  It is not so obvious, however, in any model which depends on physical mechanisms for the opening and closing of cavities, in what circumstances will the potential gradient be close to the surface gradient.

Clearly the flux $\bq$ depends on the bedrock slope $\grad b$ because the velocity $\bV$ has such a term.  Because bedrock elevation is rough (irregular) data in practice, however, this part of the velocity $\bV$ will not be very smooth, and it may be large in magnitude.


\subsection*{Capacity of the distributed system}  Suppose $\bv_b$ is the ice basal velocity (i.e.~the sliding velocity).  The evolution of the area-averaged thickness $Y$ (m) of the cavities in a distributed linked-cavity system \citep{Schoofetal2012} can be described as the sum of opening by cavitation and closure by creep \citep{Hewitt2011}:
\begin{equation}
\frac{\partial Y}{\partial t} = \mathcal{O}(|\bv_b|,Y) - \mathcal{C}(N,Y). \label{eq:hewittcapacity}
\end{equation}
Exactly as in \cite{Schoofetal2012} we choose an opening term
\begin{equation}
 \mathcal{O}(|\bv_b|,Y) = c_1 |\bv_b| (W_r - Y)_+. \label{eq:openingform}
\end{equation}
Here $W_r$ is a maximum roughness scale of the basal topography and $c_1$ is a constant; both of these constants must be constrained by observations (see below).  Also, we denote $X_+= \max\{0,X\}$ for a real number $X$.  We choose a form for the closing term which represents a regularization of that in the literature \citep{Hewitt2011,Schoofmeltsupply,Schoofetal2012}:
\begin{equation}
\mathcal{C}(N,Y) = c_2 A N^3 (Y+Y_0). \label{eq:closingform}
\end{equation}
Here $A$ is the ice softness, $c_2$ is a constant which must be constrained by observations (see below), and we have used Glen exponent $n=3$ for concreteness.  The regularization constant $Y_0>0$ (m) is taken to be small relative to typical values of the layer thickness $Y$; the role of this constant is addressed below.

Equation \eqref{eq:hewittcapacity} describes the evolution of the upper surface of the subglacial cavity.  The first term (cavitation) is always nonnegative (i.e.~causes opening) if we use \eqref{eq:openingform}, but it is only positive where the capacity is less than the roughness scale ($Y<W_r$).  The second term always represents closing because our modeled pressure will satisfy bounds \eqref{eq:bounds} below so that $0\le N \le P_o$.  The opening and closing terms \eqref{eq:openingform} and \eqref{eq:closingform} satisfy the inequalities in \cite{Schoofetal2012}, namely equations (2.5)--(2.7).

The physical intuition behind a pressure model which combines \eqref{eq:hewittcapacity} with a Darcy flux relation like \eqref{eq:flux} or \eqref{eq:fluxalt} and mass conservation \eqref{eq:conserve} is as follows.  If the cavity is locally larger than connected water sources can locally fill then the pressure should be locally lowered.  This both encourages inflow into the area and, because of the higher effective pressure, encourages cavity closure.  Conversely, if local water sources exceed local capacity then the increased pressure should push water out of the area, and also the closure rate should be reduced by the lowered effective pressure.  This intuition requires a pressure closure, however, which somehow relates the water amount $W$ to the capacity $Y$.

To explain the role of regularization in the closing term \eqref{eq:closingform}, first consider steady states of any model using \eqref{eq:hewittcapacity}.  These steady systems have a functional relationship between thickness $Y$ and effective pressure $N$ which comes from solving the steady condition for $N$:
\begin{equation}
\mathcal{O}(|\bv_b|,Y) = \mathcal{C}(N,Y). \label{eq:hewittsteady}
\end{equation}
The implicit function theorem says that if $\partial\mathcal{C}/\partial N$ is nonzero for a given $Y\ge 0$ then the effective pressure is determined: $N=N(Y)$.  At $Y=0$ we have $\partial\mathcal{C}/\partial N=0$ in the unregularized case $Y_0=0$.

If $Y<W_r$ and $Y_0>0$ then the unique value $N(Y)>0$ determined by \eqref{eq:hewittsteady} may not, however, satisfy the bound $N(Y) \le P_o$ implied by \eqref{eq:bounds}.  That is, the sliding speed $|\bv_b|$ may be so large that the cavitation opening rate exceeds the closing rate for any effective pressure $N$ satisfying $N\le P_o$.  Therefore, for given distributed values of $P_o$ and $|\bv_b|$, steady state equation \eqref{eq:hewittsteady} can only hold if
\begin{equation}
c_1 |\bv_b| (W_r - Y)_+ \le c_2 A P_o^3 (Y+Y_0). \label{eq:steadyOCbound}
\end{equation}
Condition \eqref{eq:steadyOCbound} must hold for the steady values of $Y$ over the entire domain.  If inequality \eqref{eq:steadyOCbound} applies then equation \eqref{eq:hewittsteady} determines a valid steady state effective pressure $N(Y)$ satisfying $0\le N(Y) \le P_o$, over the whole domain, and thus a water pressure $P(Y)$ satisfying bounds \eqref{eq:bounds}.  Otherwise, steady states of systems including equations \eqref{eq:hewittcapacity}, \eqref{eq:openingform}, and \eqref{eq:closingform} do not exist.


\section{Closures to determine pressure} \label{sec:closures}

At this point we do not know how to compute the water pressure $P$ or the effective pressure $N$ given the data of the problem, namely $b$, $H$, $\Phi$, and $|\bv_b|$.  The apparent state variables of the model so far are $W$, $Y$, and $P$, but the above equations can be simplified only to two partial differential equations in these three unknowns.   A closure is needed.

\subsection*{Closures without cavity geometry evolution}  We first consider two closures which appear in the literature.  Their consequences emerge as limiting cases of our more complete theory, but only near steady state.  These two closures combine with mass conservation to give fundamentally-different mathematical forms for the mass conservation equation.

At the simplest level one may assume that the pressure is equal to the overburden pressure,
\begin{equation}
P = P_o.\label{eq:Pisoverburden}
\end{equation}
This model is sometimes used for routing subglacial water under ice sheets \citep[for example]{Siegertetal2009}.  Under this assumption, and accepting the commonly-used approximation that the water layer thickness is small so it can be ignored in defining the hydraulic potential, we have $\psi=P_o+\rho_w g b$.  Straightforward calculations \eqref{eq:conserve}, \eqref{eq:flux}, and \eqref{eq:Pisoverburden} show that mass conservation equation \eqref{eq:adeqn} becomes a pure advection with a source term,
\begin{equation}
  \frac{\partial W}{\partial t} + \Div\left(\tilde\bV\, W\right) = \Phi  \label{eq:PisoverConservation}
\end{equation}
where $\tilde\bV = - K \left[R \grad h + (1-R) \grad b\right]$ and $R=\rho_i/\rho_w$.  As a true advection, \eqref{eq:PisoverConservation} possesses characteristic curves \citep{Evans}, the trajectories for the velocity field $\tilde\bV$.  These give the direction of water flow while mass conservation gives the magnitude of the flux.

At an almost opposite extreme in terms of the mathematical form, one might hypothesize that the water pressure is determined by the amount of water.  This situation which would arise, among other situations, from a subglacial hydrology which is well-connected to an englacial system in a temperate glacier \citep{FlowersClarke2002_trapridge}.  Specifically, \cite{FlowersClarke2002_theory} propose the closure that
\begin{equation}
P_{FC}(W) = P_o \left(\frac{W}{W_{\text{crit}}}\right)^{7/2}. \label{eq:PofWFC}
\end{equation}
For Trapridge glacier \cite{FlowersClarke2002_trapridge} use $W_{\text{crit}}=0.1$ and they make the approximation that the water layer thickness is small so it can be ignored in the hydraulic potential.  For general bedrock, from \eqref{eq:conserve}, \eqref{eq:flux}, and \eqref{eq:PofWFC} we get
\begin{equation}
  \frac{\partial W}{\partial t} + \Div\left(\hat\bV\, W\right) = \Div \left(\frac{KW}{\rho_w g} \grad P_{FC}(W)\right) + \Phi \label{eq:PofWFCConservation}
\end{equation}
where $\hat\bV = - K \grad b$.  In the flat bedrock case the model that follows from \eqref{eq:PofWFC} is a non-constant-coefficient nonlinear diffusion.  Indeed, as noted by \cite{Schoofetal2012}, the model is a porous-medium equation \citep{VazquezPME} like
\begin{equation}
\frac{\partial W}{\partial t} = \Div \left(\grad (W^\gamma)\right).  \label{eq:literalPME}
\end{equation}
The main idea in such an equation is that the direction of the flux is always $-\grad W$.

In this paper we apply the evolution model \eqref{eq:hewittcapacity} for the capacity of the distributed system, and thus we do not use either of the above closures.  However, in steady state our theory, and also that of \cite{Schoofetal2012}, allow the recovery of a functional relation $P=P(W)$.  This functional form is not, however, of power form \eqref{eq:PofWFC}.  Under steady conditions where the sliding also goes to zero, our theory also recovers \eqref{eq:Pisoverburden} and  and thus \eqref{eq:PisoverConservation}.  These connections are exposed in section \ref{sec:steadyverif} below.

\subsection*{Full-cavity closure}  Requiring the aquifer to be full is a closure:
\begin{equation}
W = Y.\label{eq:strongclosure}
\end{equation}
The consequences of this ``strong'' closure are explored at some length by \cite{Schoofetal2012}; it is the ``normal pressure'' case (4.13) in that reference.  Equation \eqref{eq:strongclosure} allows us to eliminate $Y$, but we can also use the time-derivative equation $\partial W/\partial t = \partial Y/\partial t$.  From equations \eqref{eq:conserve}, \eqref{eq:potential}, \eqref{eq:flux}, \eqref{eq:hewittcapacity}, \eqref{eq:openingform}, and \eqref{eq:closingform}, the relation $\partial W/\partial t = \partial Y/\partial t$ implies this equation relating $P$ and $W$:
\begin{equation}
\Div \left(\frac{K\,W}{\rho_w g} \grad \psi \right) + \Phi = c_1 |\bv_b| (W_r - W)_+ - c_2 A (P_o - P)^3 (W+Y_0).\label{eq:ellipticpressure}
\end{equation}
If $Y_0=0$ then equation \eqref{eq:ellipticpressure} is the same as the $\alpha=1$, $\beta=2$ cases of equations (2.12) and/or (4.17) in \citep{Schoofetal2012}.

Though we will not solve \eqref{eq:ellipticpressure}, it provides understanding of the structure of our mathematical model below.  Specifically, in a region where a given water thickness field $W$ is positive, \eqref{eq:ellipticpressure} is an elliptic PDE for $P$.  \cite{Schoofetal2012} show that the time-independent mathematical problem encompassing \eqref{eq:ellipticpressure}, constraints \eqref{eq:bounds}, and appropriate pressure boundary conditions can be written, together, as an elliptic variational inequality \citep{KinderlehrerStampacchia}.  The numerical analysis of such problems, which appear in other glaciological free boundary contexts \citep{SchoofStream,JouvetBueler2012}, can be addressed with a finite element approach \citep{Ciarlet}.

However, the goal of the current work is the initial selection of a subglacial hydrology model which can be straightforwardly coupled to three-dimensional ice dynamics and parallelized.  We need to determine parameters for this model from observations at the surface of glaciers.  These goals suggest that we seek a more easily-implemented model which does not require the solution of an elliptic problem at each time step.  We find such a model by weakening closure \eqref{eq:strongclosure}, as follows.

\subsection*{Diffusive nearly-full closure}  PDE \eqref{eq:ellipticpressure} is a stress balance for the incompressible, essentially-inviscid water layer.  It describes how pressures of different areas in the layer are related to each other, and to the layer geometry, at each instant.  We now propose a damped version of the full-cavity relation $Y=W$, and this implies a diffusive, parabolic version of \eqref{eq:ellipticpressure}.

First we replace $Y$ by $W$ in the right side of \eqref{eq:hewittcapacity}, giving
\begin{equation}
\frac{\partial Y}{\partial t} = c_1 |\bv_b| (W_r - W)_+ - c_2 A (P_o - P)^3 (W+Y_0), \label{eq:capacity}
\end{equation}
using the opening and closing terms given by \eqref{eq:openingform} and \eqref{eq:closingform}.  For equation \eqref{eq:capacity} we have used the full closure $W=Y$ directly.

Now let $E_0>0$ be a fixed, small vertical distance.  It is intended to be a fraction of the typical aquifer thickness.  We approximate the full-closure time derivative identity $\partial W/\partial t = \partial Y/\partial t$ by this new damped version:
\begin{equation}
\frac{E_0}{P_o} \frac{\partial P}{\partial t} =  \frac{\partial W}{\partial t}  - \frac{\partial Y}{\partial t}.\label{eq:dampeddstrong}
\end{equation}
Intuitively, Equation \eqref{eq:dampeddstrong} represents a kind of penalty method \citep{NocedalWright} for enforcing constraint \eqref{eq:strongclosure}.  The $E_0\to 0$ (singular) limit of equation \eqref{eq:dampeddstrong} recovers $\partial W/\partial t = \partial Y/\partial t$.  Now use expressions \eqref{eq:conserve} and \eqref{eq:flux} for $\partial W/\partial t$ and \eqref{eq:capacity} for $\partial Y/\partial t$:
\begin{align}
\frac{E_0}{P_o} \frac{\partial P}{\partial t} &= \Div \left(\frac{K\,W}{\rho_w g} \grad \psi\right) + \Phi - c_1 |\bv_b| (W_r - W)_+  + c_2 A (P_o - P)^3 (W+Y_0). \label{eq:diffusionpressure}
\end{align}
In any region where $W>0$, equation \eqref{eq:diffusionpressure} is a diffusion, a parabolic equation for $P$.

Intuitively, because $E_0/P_o$ is positive in any glacier-covered area, by equation \eqref{eq:dampeddstrong} the pressure increases when $W$ is growing faster than $Y$ (e.g.~if the aquifer is being forced open by water).  Conversely, the pressure decreases when $Y$ is growing faster than $W$ (e.g.~if cavitation is opening space for water but water is being supplied more slowly).  Recombining equations \eqref{eq:diffusionpressure} and \eqref{eq:dampeddstrong} allows us to see that a difference between the rates of change $\partial W/\partial t$ and $\partial Y/\partial t$ is generated by an imbalance of the mechanisms that are exactly balanced when equation \eqref{eq:ellipticpressure} holds.  Of course, the steady state case converts \eqref{eq:diffusionpressure} to \eqref{eq:ellipticpressure}.

\subsection*{A new subglacial hydrology model}  At this point we have a new mathematical model, consisting of conservation equation \eqref{eq:adeqn} for evolution of $W$, equation \eqref{eq:diffusionpressure} for evolution of pressure $P$, and the bounds on $W$ and $P$.  Defining $c_0 = K / \rho_w g$, here is the model:
\begin{empheq}[box=\mybluebox]{align}
0 &\le W, \notag \\
0 &\le P \le P_o, \notag \\
\psi &= P + \rho_w g (b + W), \notag \\
\frac{E_0}{P_o} \frac{\partial P}{\partial t} &= \Div \left(c_0\,W\, \grad \psi \right) + c_2 A (P_o - P)^3 (W+Y_0) - c_1 |\bv_b| (W_r - W)_+ + \Phi, \label{eq:bluebox} \\
\bV &= - c_0 \grad P - K \grad b, \notag \\
\frac{\partial W}{\partial t} + &\Div\left(\bV\, W\right) = \Div \left(K W \grad W\right) + \Phi. \notag
\end{empheq}

Model equations \eqref{eq:bluebox} relate the three classes of symbols in Table \ref{tab:symbols}.  (Additional symbols $\psi$, $\bV$, and $c_0$ are derived from these.)  The state functions $W$ and $P$ evolve according to model \eqref{eq:bluebox}.  Only the state functions must be provided with initial values, and only they must be saved when stopping and restarting a time-dependent numerical model.  The scalar parameters are all constant (i.e.~time- and space-independent) in the current context, though they could be allowed to vary spatially if desired.
% FIXME: actually we want K to depend on W<W_r or W>W_r
Default values must be chosen for these parameters, but exploration of the parameter space is appropriate.  The data functions are, in practice, supplied by an ice sheet model.  

\begin{table}[ht]
\caption{Symbols used in subglacial hydrology model \eqref{eq:bluebox}.}
\begin{tabular}{r|c}
class & symbols \\ \hline
\emph{state functions} & $W$, $P$ \\
\emph{scalar parameters} & $A$, $c_1$, $c_2$, $E_0$, $g$, $K$, $\rho_i$, $\rho_w$, $W_r$, $Y_0$ \\
\emph{data functions} & $b$, $\Phi$, $P_o$, $|\bv_b|$ \\
\hline
\end{tabular}
\label{tab:symbols}
\end{table}


\section{Steady states}  \label{sec:steadyverif}

\subsection*{Processes become decoupled in steady state}  The steady states of mathematical model \eqref{eq:bluebox} are worth considering both because the physical subglacial system is close to steady state much of the time, and because in the steady state case we can more easily find an exact solution.

Recall that the flux has two expressions $\bq = - c_0 W \grad \psi = \bV W - K W \grad W$.  Thus, here is a steady form of model \eqref{eq:bluebox} which is written in terms of $\bV,\bq,W,P$:
\begin{align}
\bV &= - c_0 \grad P - K \grad b, \label{eq:Vsteady} \\
\bq &= \bV W - K W \grad W, \label{eq:qsteady} \\
0 &= - \Div \bq + \Phi, \label{eq:masscontsteady} \\
0 &= c_2 A (P_o - P)^3 (W+Y_0) - c_1 |\bv_b| (W_r - W)_+. \label{eq:openclosesteady}
\end{align}
In the last equation we have eliminated ``$- \Div \bq + \Phi$'' because it is zero.  Note that we also have bounds $W\ge 0$ and $0 \le P \le P_o$.

\begin{table}[ht]
  \centering
  \caption{Physical constants and major model parameters.}
  \begin{tabular}{lllp{3.0in}} 
    \textbf{Name} & \textbf{Value} & \textbf{Units} & \textbf{Description}\\
\hline
    $A$ & $3.1689\times 10^{-24}$ & $\text{Pa}^{-3}\,\text{s}^{-1}$ & ice softness \citep{EISMINT96} \phantom{$\Big|$} \\
    $c_0$ & $1.0194\times 10^{-6}$ & $\text{m}^{2}\,\text{Pa}^{-1}\,\text{s}^{-1}$ & constant in model \eqref{eq:bluebox}: $c_0=K/(\rho_w g)$ \phantom{$\Big|$} \\
    $c_1$ & $0.5$ & $\text{m}^{-1}$ & cavitation coefficient in \eqref{eq:capacity} \\
    $c_2$ & $0.04$ & & creep closure coefficient in \eqref{eq:capacity} \\
    $E_0$ & $1.0$ & m & regularization thickness in \eqref{eq:dampeddstrong} \\
    $g$ & $9.81$ & m $\text{s}^{-2}$ & acceleration of gravity \\
    $K$ & $0.01$ & m $\text{s}^{-1}$ & effective hydraulic conductivity in \eqref{eq:flux} \\ % FIXME: K_min and K_max according to whether W>W_r
    $\rho_i$ & $910$ & $\text{kg}\,\text{m}^{-3}$ & ice density \citep{GreveBlatter2009} \\
    $\rho_w$ & $1000$ & $\text{kg}\,\text{m}^{-3}$ & fresh water density \citep{GreveBlatter2009} \\
    $W_r$ & $1$ & $\text{m}$ & roughness scale in \eqref{eq:capacity} \\
    $Y_0$ & $0.001$ & m & regularization for layer thickness in \eqref{eq:closingform} \\
    \hline
  \end{tabular}
 \label{tab:constants}
\end{table}

Relative to the time-dependent form \eqref{eq:bluebox}, processes have become decoupled in the steady state equations \eqref{eq:Vsteady}--\eqref{eq:openclosesteady}.  There are separate balances between the divergence of the flux and the water input on the one hand (i.e.~equation \eqref{eq:masscontsteady}), and the opening and closing processes on the other hand (i.e.~equation \eqref{eq:openclosesteady}).  Steady state equations \eqref{eq:Vsteady}--\eqref{eq:openclosesteady} are identical to those of the \cite{Schoofetal2012} model, where the same decoupling is also noted.  Specifically, in the one-dimensional case the above equations reduce to (5.8) and (5.10) from \cite{Schoofetal2012}.


\subsection*{Functional relationship for pressure in steady state}  Equation \eqref{eq:openclosesteady} allow us to write the pressure $P=P(W)$ in steady state as a continuous function of the water amount $W$.  This fact was already pointed out in considering the steady states of equation \eqref{eq:hewittcapacity}.  In particular, steady state is only possible if \eqref{eq:steadyOCbound} holds, but here with $W$ for $Y$:
\begin{equation}
c_1 |\bv_b| (W_r - W)_+ \le c_2 A P_o^3 (W+Y_0) \qquad \text{ in steady state}. \label{eq:steadyboundfirst}
\end{equation}
In this and later formulas define
\begin{equation}
s_b =  \left(\frac{c_1 |\bv_b|}{c_2 A}\right)^{1/3},  \label{eq:definesb}
\end{equation}
a scaled basal sliding speed which has units of pressure.  (One may think of $s_b$ as a scale for the amount of pressure drop associated to cavitation in steady state.)  Then \eqref{eq:steadyboundfirst} is equivalent to
\begin{equation}
W \ge W_c := \frac{s_b^3 W_r - P_o^3 Y_0}{s_b^3 + P_o^3} \qquad \text{ in steady state}. \label{eq:steadyboundsecond}
\end{equation}
This condition says that the water amount is above a critical level that depends on the sliding and the overburden pressure.  Inequalities \eqref{eq:steadyboundfirst} and \eqref{eq:steadyboundsecond} are actually less restrictive in the regularized case $Y_0>0$ than in the unregularized case $Y_0=0$.

If \eqref{eq:steadyboundfirst} or \eqref{eq:steadyboundsecond} holds then
\begin{equation}
P(W) = P_o - s_b \left(\frac{(W_r - W)_+}{W+Y_0}\right)^{1/3} \qquad \text{ in steady state}.  \label{eq:PofWsteady}
\end{equation}
Note that in \eqref{eq:PofWsteady} we have $P(W_c)=0$.  Underpressure ($P=0$) with subcritical water amount ($W<W_c$) does not occur in steady state though it can and does occur in nonsteady conditions.  

Formula \eqref{eq:PofWsteady} may apply even if $W\ge W_r$, in which case the water pressure takes the overburden value $P = P_o$.  However, if $P_o=0$ then \eqref{eq:steadyboundfirst} implies that either $W\ge W_r$ or $|\bv_b|=0$.  This describes the values of $W$ and $|\bv_b|$ at ice margins where $H\to 0$ and therefore $P_o\to 0$.

The rest of this section uses the specific values in Table \ref{tab:constants}.

\newcommand{\upto}{ \!\!\nearrow\! }
\newcommand{\downto}{ \!\searrow\! }
Figure \ref{fig:psteady-vb} shows the function $P(W)$ from \eqref{eq:PofWsteady} for several cases of uniform sliding speed $|\bv_b|$.  Figure \ref{fig:psteady-Po} shows $P(W)$ for several cases of uniform overburden pressure $P_o$.  Replacing $Y_0=0$ makes no apparent difference in these figures; not shown.  We see that as the water amount reaches the roughness scale ($W\upto W_r$) the pressure rises rapidly to overburden ($P(W) \upto P_o$).  At the other extreme, we see that $P(W) \downto 0$ if $W \downto W_c$.  The curves $P(W)$ in Figures \ref{fig:psteady-vb} and \ref{fig:psteady-Po}, which describe steady state, do not include the interval $0\le W < W_c$ because such underpressure conditions are not achievable in steady state.

\begin{figure}[ht]
\includegraphics[width=3.5in,keepaspectratio=true]{figs/psteady-vb}
\medskip
\caption{The steady state function $P(W)$ defined by equation \eqref{eq:PofWsteady} depends on the sliding speed.  Four cases are shown using a fixed uniform ice thickness of $H=1000$ m: $|\bv_b|=0$ m/a (blue), $10$ m/a (green), $100$ m/a (red), and $1000$ m/a (cyan).  The values of $W_c$ for these cases are indicated by black dots with $P(W_c)=0$.  Relation \eqref{eq:PofWFC} is shown with $W_{\text{crit}}=W_r$ for comparison (dashed black).}
\label{fig:psteady-vb}
\end{figure}

\begin{figure}[ht]
\includegraphics[width=3.5in,keepaspectratio=true]{figs/psteady-Po}
\medskip
\caption{Function $P(W)$ defined by \eqref{eq:PofWsteady} also depends on overburden pressure $P_o=\rho_i g H$.  We fix $|\bv_b|=100$ m/a and consider four cases of uniform thickness $H=$ $2000$ m (blue), $1000$ m (green), $500$ m (red), and $200$ m (cyan).}
\label{fig:psteady-Po}
\end{figure}

Recall that \cite{FlowersClarke2002_theory} propose the functional relation \eqref{eq:PofWFC} for both steady and nonsteady circumstances.  On the one hand formula \eqref{eq:PofWFC} has features in common with equation \eqref{eq:PofWsteady}: both functions $P(W)$ and $P_{FC}(W)$ are increasing and both relate the water pressure to the overburden pressure $P_o$.  However, while in \eqref{eq:PofWsteady} the relation to $P_o$ is additive, in \eqref{eq:PofWFC} it is a multiplicative scaling.  The power law form \eqref{eq:PofWFC} is not justified by the physical reasoning which led to equation \eqref{eq:PofWsteady}, even in steady state.   It would appear that any functional relationship $P(W)$ should also depend on the sliding velocity if cavitation is an influence on the water pressure.  For example, in the small sliding cases (e.g.~$|\bv_b| \ll 10$ m/a) creep will force $P(W) \approx P_o$ for a modest amount of water ($W\ll W_r$).  Of course the $W>W_{\text{crit}}$ case gives $P_{FC}(W) > P_o$ in \eqref{eq:PofWFC}, a problematic condition already noted by \cite{Schoofetal2012}.  This condition $P>P_o$ does not arise in \eqref{eq:PofWsteady} nor in the \cite{Schoofetal2012} theory.  Perhaps the most important contrast between the \cite{FlowersClarke2002_theory} theory and the current paper is that we have no functional relationship $P=P(W)$ in nonsteady conditions.

\subsection*{Velocity in steady state}  We now consider how the steady state water velocity $\bV$ depends on other quantities.  Equation \eqref{eq:PofWsteady} defines $P=P(W,P_o,s_b)$ while $\bV$ depends on $\grad P$, and thus we need derivatives of $P$.  In particular, in steady state we have
\begin{equation}
\frac{\partial P}{\partial W} =
    \begin{cases}
      \text{undefined}, & W \le W_c, \\
      \frac{1}{3} s_b (W_r + Y_0) (W+Y_0)^{-4/3} (W_r - W)^{-2/3}, & W_c < W < W_r, \\
      \text{undefined}, & W = W_r, \\
      0, & W > W_r.
    \end{cases}  \label{eq:dPdWsteady}
\end{equation}
Figures \ref{fig:psteady-vb} and \ref{fig:psteady-Po} illustrate these cases.  Note that the condition $W_c < W < W_r$ is identical to the normal pressure condition $0 < P < P_o$ in steady state.  Formula \eqref{eq:dPdWsteady} and the Figures agree that $\partial P / \partial W \to \infty$ as $W \upto W_r$.  Equations \eqref{eq:Vsteady}, \eqref{eq:PofWsteady}, and \eqref{eq:dPdWsteady} yield this formula for the velocity in steady state which applies in the normal pressure cases:
\begin{align}
\bV &= - c_0 \grad P - K \grad b = - c_0 \left[\frac{\partial P}{\partial P_o} \grad P_o + \frac{\partial P}{\partial s_b} \grad s_b + \frac{\partial P}{\partial W} \grad W\right] - K \grad b  \notag \\
    &= - c_0 \left[\grad \psi_o - \left(\frac{W_r - W}{W+Y_0}\right)^{1/3} \grad s_b + \frac{s_b (W_r + Y_0)}{3  (W+Y_0)^{4/3} (W_r - W)^{2/3}} \grad W\right]. \label{eq:Vsteadyexpand}
\end{align}
We have simplified the last expression by combining terms using the purely-geometrical function $\psi_o = P_o + \rho_w g b = \rho_i g H + \rho_w g b$ which we call the \emph{overburden potential}.  It is the hydraulic potential associated to an infinitesimal amount of water in the subglacial layer if it is at zero effective pressure.

Equation \eqref{eq:Vsteadyexpand} can help us understand the advective flux $\bV W$ part of $\bq=\bV W - K W \grad W$.  First, the direction of water transport $\bV$ is determined in steady state by a combination of a geometric direction ($\grad \psi_o$), a direction derived from variations in the sliding speed ($\grad s_b$), and a diffusive direction.  The last category includes all terms proportional to $-\grad W$, thus both the known diffusive flux $-K W \grad W$ and also the third term in \eqref{eq:Vsteadyexpand}.

We see that in steady state we have
\begin{equation}
\bq = - \alpha(W) \grad \psi_0 - \beta(W) \grad s_b - \gamma(W,s_b) \grad W  \label{eq:qabstract}
\end{equation}
for some coefficients $\alpha,\beta,\gamma$ which we now consider.  Restricting to the $Y_0=0$ case for simplicity, the first two coefficients $\alpha(W)$ and $\beta(W)$ go to zero as $W\to 0$.  However $\gamma(W,s_b)$ remains large when $W\to 0$, even if $K$ is small, as long as sliding is sustained ($s_b > 0$).  Thus for low water amount we should think of the water as diffusing in the layer \citep[compare equation (11) in][]{BBssasliding}.  When the water thickness closely approximates the roughness scale ($W\approx W_r$) then the second sliding term (i.e.~proportional to $\grad s_b$) contributes little while the third diffusive term is again strong if there is significant sliding.

Thinking more generally, it is no surprise that when the ice thickness, bed elevation, sliding velocity, or water thickness are highly variable in space then we can expect a larger amount of flow, and formula \eqref{eq:Vsteadyexpand} illustrates this.  Because the magnitude of the velocity $|\bV|$ determines the CFL time step restriction \citep{MortonMayers} associated to numerically solving the mass conservation equation in the model \eqref{eq:bluebox}, large variations in the same fields will generally reduce the time steps taken by a numerical model.


\subsection*{The radial case}  The above steady equations will be the basis for building a useful two-dimensional nearly-exact solution for $W$ and $P$.  This solution will depend on a single numerical solution of a first-order ODE which can be solved to very high precision.  Exact solutions in one horizontal dimension appear in \cite{Schoofetal2012}.  Here, by contrast, we consider a two horizontal dimension ice sheet geometry on a flat bedrock but our solution will only depend on the radial coordinate $r = \sqrt{x^2+y^2}$.  We derive the ODE now, after which we proceed to specify a particular exact solution for our verification purpose.

Consider steady state equations \eqref{eq:Vsteady}--\eqref{eq:masscontsteady} and eliminate $\bV$.  In the flat bed case the resulting pair of equations is
\begin{align}
q &= - c_0 W\, \left(\frac{dP}{dr} + \rho_w g \frac{dW}{dr}\right), \label{eq:rsflux} \\
\frac{1}{r}\frac{d}{dr}\left(r\,q\right) &= \Phi. \label{eq:rsconserve}
\end{align}
In the case of constant water input $\Phi(r)=\Phi_0$, which we assume for the exact solution, from \eqref{eq:rsconserve} we can integrate from $0$ to $r$ and use symmetry ($q(0)=0$) to get
\begin{equation}
q(r) = \frac{1}{2} \Phi_0\, r. \label{eq:qradial}
\end{equation}

On the other hand, equation \eqref{eq:PofWsteady} gives $P$ as a function of $W$ in steady state.  Suppose $h(r)$ is given so that $P_o(r)$ is also determined.  Assume that the scaled sliding speed $s_b(r)$ has a bounded derivative and that the solution $W(r)$ satisfies the normal pressure conditions $W_c < W < W_r$; both of these properties must be verified later for the constructed solution.  Now, by combining \eqref{eq:rsflux}, \eqref{eq:qradial}, \eqref{eq:PofWsteady}, and \eqref{eq:dPdWsteady} we can eliminate $q$ and $P$ to find
\begin{equation}
\varphi_0\, r = - W\, \left(\frac{dP_o}{dr} - \frac{ds_b}{dr} \left(\frac{W_r - W}{W+Y_0}\right)^{1/3} + \left(\frac{s_b (W_r + Y_0)}{3 (W+Y_0)^{4/3} (W_r - W)^{2/3}} + \rho_w g\right) \frac{dW}{dr}\right)  \label{eq:ODEfirst}
\end{equation}
in cases where $W_c < W < W_r$ and where $\varphi_0 = \Phi_0 / (2 c_0)$. 

Equation \eqref{eq:ODEfirst} is a first-order ordinary differential equation (ODE) for $W(r)$.  To put it in the standard form expected by a numerical ODE solver, solve for $dW/dr$:
\begin{equation}
\frac{dW}{dr} = \frac{\frac{ds_b}{dr} (W+Y_0) (W_r - W) - \Big[\varphi_0\, r W^{-1} + \frac{dP_o}{dr}\Big] (W + Y_0)^{4/3} \left(W_r - W\right)^{2/3}}{\frac{1}{3} s_b (W_r + Y_0) + \rho_w g (W + Y_0)^{4/3} (W_r - W)^{2/3}}.
\label{eq:WradialODE}
\end{equation}
Note that ODE \eqref{eq:WradialODE} has a constant solution $W(r)=W_r$.  To determine a nontrivial nearly-exact solution of this ODE we must choose an ice thickness $H(r)$ and a sliding speed $|\bv_b|(r)$ to determine $dP_o/dr$ and $ds_b/dr$, respectively.

The numerical solution of ODE \eqref{eq:WradialODE} will proceed from the ice margin inward toward the center.  To generate a nontrivial solution we assume a positive thickness of ice at the margin so that $P_o(L^-)>0$.  We also assume that at the margin there is some sliding so that $s_b(L^-)>0$, and indeed we assume (and check) that $s_b(L^-) W_r > P_o(L^-)^3 Y_0$.  At the ice margin $r=L$ we have $P=0$ so $W(L)=W_c(L^-)$ is the boundary (initial) condition for the ODE.  By our assumptions the initial condition also satisfies $W(L) < W_r$.  Then we integrate \eqref{eq:WradialODE} from $r=L$ to $r=0$, and the central value $W(0)$ is determined as part of the solution.

It is useful to have an ice cap geometry $h(r)$ in which the surface gradient formula is simple so that $dP_o/dr$ in \eqref{eq:WradialODE} is also simple.  The plug flow, flat bed surface elevation solution of \cite{Bodvardsson} has this property.  Extending to the radial case, equations (23) and (24) of \citep{Bodvardsson} give
\begin{equation}
h(r) = h_0 \left(1 - \frac{r^2}{R_0^2} \right) \label{eq:choosebodvardssonh}
\end{equation}
where $h(0)=h_0$ is the height of the center of the ice cap.  It follows that $dP_o/dr = - C r$ where $C=2\rho_i g h_0 R_0^{-2}$.  Also we choose $L=0.9 R_0$ so that $P_o(L)>0$.

The sliding speed could be determined by a model for stresses at the ice base and within the ice \citep{GreveBlatter2009}, but this kind of coupled ice and water dynamics solution is too advanced for the current purpose of hydrology model verification.  We choose a well-behaved sliding speed function which has no sliding near the ice cap center, and which increases in the radial direction:
\begin{equation}
|\bv_b|(r) = \begin{cases} 0, & 0 \le r \le R_1, \\
                           v_0  \left(\frac{r-R_1}{R_0-R_1}\right)^5, & R_1 < r < R_0, \\
                           0, & R_0 \le r.
             \end{cases}  \label{eq:choosevb}
\end{equation}
It follows from \eqref{eq:definesb} and \eqref{eq:choosevb} that $ds_b/dr$ in \eqref{eq:WradialODE} is bounded and continuous on $0\le r < R_0$.

Nw we can solve ODE \eqref{eq:WradialODE} with these data functions, the initial condition $W(L)$, and the specific values in Table \ref{tab:verifconstants}.  We do this using an adaptive numerical ODE solver.  The result $W(r)$ is shown in Figure \ref{fig:Wexact}.  Note that from \eqref{eq:choosebodvardssonh} and \eqref{eq:choosevb} there is a unique pressure value $P=P(W,r)$ from \eqref{eq:PofWsteady}.  This allows us to show in Figure \ref{fig:Wexact} the regions of the $r,W$ plane which correspond to overpressure, normal pressure, and underpressure.  We see that $W(r)$ is in the normal pressure region as $r$ decreases from $r=L$ to $r=R_1$.

\begin{table}[ht]
  \centering
  \caption{Constants used in constructing the nearly-exact solution.}
  \begin{tabular}{lllp{3.0in}}
    \textbf{Name} & \textbf{Value} & \textbf{Units} & \textbf{Description}\\
\hline
    $\Phi_0$ & $20$ & $\text{cm}\,\text{a}^{-1}$ & water input rate \\
    $h_0$ & $500$ & m & ice cap center thickness \\
    $R_0$ & $25$  & km & ice cap radius \\
    $R_1$ & $5$   & km & radial location $r=R_1$ of onset of sliding \\
    $v_0$ & $100$ & $\text{m}\,\text{s}^{-1}$ & sliding speed scale \\
    \hline
  \end{tabular}
 \label{tab:verifconstants}
\end{table}

\begin{figure}[ht]
\includegraphics[width=3.5in,keepaspectratio=true]{figs/exact-W-plot-onu}
\caption{Nearly-exact radial, steady solution for water thickness $W(r)$ (dashed).  In $r$-versus-$W$ space the overpressure (O), normal pressure (N), and underpressure (U) regions are determined (solid curves).}
\label{fig:Wexact}
\end{figure}

To generate Figure \ref{fig:Wexact} we used both the standard, non-stiff Runge-Kutta 4(5) Dormand-Prince method, which is \Matlab's \texttt{ode45}, and the stiff, variable-order solver \texttt{ode15s}, with relative tolerance $10^{-12}$ and absolute tolerance $10^{-9}$, and with identical results.
% could cite Shampine,Gladwell,Thompson (2003) on Matlab ODE solvers
Modest stiffness \citep{AscherPetzold} of ODE \eqref{eq:WradialODE} is observed for $r\approx R_1$, however.  The reason is that as the sliding goes to zero, the cavitation also goes to zero ($|\bv_b|\to 0$).  Because creep closure balances cavitation in steady state, it also goes to zero ($P\to P_o$).  The remaining active mechanisms in the model are variation in overburden pressure and the rate of water input.  They must balance according to the steady mass conservation equation \eqref{eq:masscontsteady}.  In this case with no sliding ($s_b=0$), ODE \eqref{eq:WradialODE} reduces to the much simpler form
\begin{equation}
\frac{dW}{dr} = - \frac{\varphi_o r W^{-1} + \frac{dP_o}{dr}}{\rho_w g}. \label{eq:WradialODEnoslide}
\end{equation}
Equation \eqref{eq:WradialODEnoslide} is actually the steady radial form of the mass conservation equation under the ``$P=P_o$'' closure, namely equation \eqref{eq:PisoverConservation}.

In equation \eqref{eq:WradialODEnoslide} we see that $dW/dr=0$ if $W$ satisfies $W = - \varphi_0 r / (dP_o/dr)$.  In our case with geometry \eqref{eq:choosebodvardssonh} this reduces to a constant value $W=\tilde W= 0.21764$ m because $\Phi_0$ is constant and $dP_o/dr$ is linear in $r$.  Both numerical ODE solvers mentioned above confirm that $W(r)$ is asymptotic to this constant value $\tilde W$ as $r\to 0$, and that $W(r)\approx \tilde W$ within about 1\% on all of $0\le r \le R_1$.  This is seen in Figure \ref{fig:Wexact}.

We compute the pressure diagnostically from the steady state functional relationship \eqref{eq:PofWsteady}.  The result is shown in Figure \ref{fig:Pexact}.

\begin{figure}[ht]
\includegraphics[width=3.5in,keepaspectratio=true]{figs/exact-P-plot}
\caption{Nearly-exact radial, steady solution pressure $P(r)$ (dashed) and overburden pressure $P_o$ (solid).}
\label{fig:Pexact}
\end{figure}


\section{Numerical schemes}  \label{sec:num}

\subsection*{Discretization of the mass conservation equation}  The mass conservation equation \eqref{eq:adeqn} will be discretized by an explicit, conservative finite difference method.   A centered, second-order scheme will be applied to the nonlinear diffusion part.  A pair of schemes for the advection part will be compared, first-order upwinding and a higher-order flux-limited upwind-biased method.  We first consider stable time steps.  Then we state these schemes in detail.

The time step restriction for the advective part, i.e.~the CFL condition for any of the schemes under consideration, is much more restrictive than the time-step restriction for the diffusion.  In the Appendix it is shown that stability for the first-order upwind scheme occurs with a time step $\Delta t \le \Delta t_{\text{CFL}}$ where
\begin{equation}
\Delta t_{\text{CFL}} \left(\frac{\max |\alpha|}{\Delta x} + \frac{\max |\beta|}{\Delta y}\right) = \frac{1}{2}. \label{eq:dtCFL}
\end{equation}
Here $\bV=(\alpha,\beta)$ is the advection velocity.  The higher-order advection scheme has the same restriction.  %FIXME: check this
The time step should also satisfy $\Delta t \le \Delta t_{W}$ where
\begin{equation}
\Delta t_W\, (2 K \max W) \left(\frac{1}{\Delta x^2} + \frac{1}{\Delta y^2}\right) = \frac{1}{2}. \label{eq:dtDIFFW}
\end{equation}
This is for the diffusion scheme to be stable.  The condition $\Delta t \le \min\{\Delta t_{\text{CFL}}, \Delta t_W\}$ is a sufficient condition for stability and convergence of this scheme \citep{MortonMayers}

To understand the significance of these time step restrictions, we consider typical values of the parameters.  The maximum water speed $|\bV|$ is about $10^5$ m/a in trial runs of the model, so $\max |\alpha| = \max |\beta| \approx 0.002$ m/s.  We take $K=10^{-2}$ m/s from Table \ref{tab:constants} and $\max W=1$ m as representative values.  Then, for a $\Delta x = \Delta y = 500$ m grid, the advective restriction \eqref{eq:dtCFL} is $\Delta t_{\text{CFL}} \approx 0.001$ year while the diffusive restriction from \eqref{eq:dtDIFFW} is $\Delta t_W \approx 0.1$ year.  Thus, unless velocities are unusually slow, or unless deep subglacial lakes develop so that $KW$ is large and $\Delta t_W$ is correspondingly small, we should not worry so much about this diffusive time scale.

There is significant uncertainty in this estimated time step restriction because the velocity field is a solution to the model.  The time step restrictions are applied adaptively, however, so stability is assured even though the total computational time is uncertain.

\begin{figure}[ht]
\centering
\includegraphics[width=2.5in,keepaspectratio=true]{figs/diffstencil}
\bigskip
\caption{Numerical scheme \eqref{eq:Wfd} for Equation \eqref{eq:adeqn} uses a grid-centered cell (dashed line).  The velocities, diffusivities, and fluxes are evaluated at the staggered grid locations (triangles) which are denoted with compass direction ($e,w,n,s$).  The state functions $W,P$ live at the regular grid points (diamonds).}
\label{fig:stencil}
\end{figure}

To set notation, suppose our rectangular computational domain has $M_x \times M_y$ gridpoints $(x_i,y_j)$ with uniform spacing $\Delta x,\Delta y$.  Let $\Wlij \approx W(t_l,x_i,y_j)$ and $\Plij \approx P(t_l,x_i,y_j)$ be the approximations of the continuum solution at the grid points.  Recall that $\bV = - c_0 \grad P - K \grad b$ and let $\bQ=\bV W$ so that $\Div \bQ = (\alpha W)_x + (\beta W)_y$.  We will compute velocity components and flux components at the staggered (cell-face-centered) points shown in Figure \ref{fig:stencil}.  We compute these values based on centered finite difference approximations of Equation \eqref{eq:vexpression}.  We use ``compass'' notation like $\alpha_e = \alpha_{i+1/2,j}$, and so on, for the components:
\begin{align}
\alpha_e &= - c_0 \frac{P_{i+1,j}-P_{i,j}}{\Delta x} - K \frac{b_{i+1,j}-b_{i,j}}{\Delta x}, \qquad \alpha_w = - c_0 \frac{P_{i,j}-P_{i-1,j}}{\Delta x} - K \frac{b_{i,j}-b_{i-1,j}}{\Delta x}, \label{eq:velocitycomp} \\
\beta_n  &= - c_0 \frac{P_{i,j+1}-P_{i,j}}{\Delta y} - K \frac{b_{i,j+1}-b_{i,j}}{\Delta y}, \qquad \beta_s = - c_0 \frac{P_{i,j}-P_{i,j-1}}{\Delta y} - K \frac{b_{i,j}-b_{i,j-1}}{\Delta y}. \notag
\end{align}
The face-centered (staggered-grid) normal fluxes $Q_e(\alpha_e)$, $Q_w(\alpha_w)$, $Q_n(\beta_n)$, and $Q_s(\beta_s)$ are components of the advective flux $\bV W$.  These quantities are functions of several regular-grid values of $W$, as described in more detail below, but they use only the single, stated velocity component at these staggered locations.  Finally, for the diffusive term the staggered-grid values of $W$ are computed by averaging, and again we use compass notation:
\begin{align}
W_e &= (W_{i,j}^l + W_{i+1,j}^l)/2, & W_w &= (W_{i-1,j}^l + W_{i,j}^l)/2, \label{eq:stagW} \\
W_n &= (W_{i,j}^l + W_{i,j+1}^l)/2, & W_s &= (W_{i,j-1}^l + W_{i,j}^l)/2. \notag
\end{align}

Now we can state our scheme for equation \eqref{eq:adeqn}:
\begin{align}
 &\frac{W_{i,j}^{l+1} - \Wlij}{\Delta t} + \frac{Q_e(\alpha_e) - Q_w(\alpha_w)}{\Delta x} + \frac{Q_n(\beta_n) - Q_s(\beta_s)}{\Delta y}  \label{eq:Wfd} \\
      &\qquad = K \bigg[\frac{W_e \left(W_{i+1,j}^l - \Wlij\right) - W_w \left(\Wlij - W_{i-1,j}^l\right)}{\Delta x^2}  \notag \\
      &\qquad\qquad\qquad + \frac{W_n \left(W_{i,j+1}^l - \Wlij\right) - W_s \left(\Wlij - W_{i,j-1}^l\right)}{\Delta y^2}\bigg] + \Phi_{ij}. \notag
\end{align}
If the flux components $Q$ had their exact continuum values, instead of discrete approximations as they are in fact, the local truncation error \citep{MortonMayers} of scheme \eqref{eq:Wfd} would be $O(\Delta t^1 + \Delta x^2 + \Delta y^2)$.  The actual truncation error depends on the nature of the discrete fluxes, which we address next.


\subsection*{Flux discretization}  FIXME:  To explain the upwind method for \eqref{eq:adeqn}, consider the model equation
\begin{equation} \label{eq:modeladvect}
u_t + (v(x) u)_x = 0
\end{equation}
for some quantity $u(t,x)$ transported by a flux $q = v(x) u$.  A ``donor cell'' upwind scheme can be described as a finite volume scheme \citep{LeVeque} wherein a grid point $x_j$ is the center of a cell.  We consider the flux at the cell interfaces $x_{j-1/2}$ and $x_{j+1/2}$.  We decide which cell contributes the advected quantity $U$ based on the sign of the velocity $v(x)$ at these interfaces.  To display the scheme we define the following upwind notation,
\newcommand{\up}[2]{\big<#1\big|\,#2\big>}
	$$\up{v}{U_j} := v \begin{Bmatrix} U_j, & v \ge 0 \\ U_{j+1}, & v < 0 \end{Bmatrix}.$$
For the model equation \eqref{eq:modeladvect} on a space-time grid $(t_l,x_j)$ we set
\begin{equation}\label{eq:modelfdadvect}
\frac{U_j^{l+1} - U_j^l}{\Delta t} + \frac{\up{v_+}{U_j^l} - \up{v_-}{U_{j-1}^l}}{\Delta x} = 0
\end{equation}
where $v_+ = v(x_{j+1/2})$ and $v_-=v(x_{j-1/2})$.

FIXME: This scheme, which uses first-order upwinding, has $O(\Delta t^1 + \Delta x^1 + \Delta y^1)$ truncation error.

FIXME: Because of the positivity proven in the appendix for this scheme, if $\Phi\ge 0$ then the lower bound $W\ge 0$ is true for the updated values at the new time $t_{l+1}$ if it is true at time $t_l$.  However, because refreeze is possible, generally $\Phi$ has either sign and so we must enforce $W\ge 0$ on the updated values.



\subsection*{Discretization of the pressure evolution equation}  The pressure evolution equation \eqref{eq:diffusionpressure} is a nonlinear diffusion with additional ``reaction'' terms associated to opening and closing.  Unlike \eqref{eq:adeqn} there is no dominating advection term.  We discretize it using an explicit centered, second-order scheme.  Again, because this is an explicit scheme, we consider stable time steps immediately.

The time step restriction is comparable to \eqref{eq:dtDIFFW}, though the proof in the appendix does not suffice to \emph{prove} stability under this condition because of the additional reaction terms.  Noting that $P_o=\rho_i g H$, the time step must satisfy $\Delta t \le \Delta t_P$ where
\begin{equation}
\Delta t_P\, \left(\frac{2 K \rho_i \max H \max W}{\rho_w E_0}\right) \left(\frac{1}{\Delta x^2} + \frac{1}{\Delta y^2}\right) = \frac{1}{2} \label{eq:dtDIFFP}
\end{equation}
The resulting time step $\Delta t_P$ is a fraction of $\Delta t_W$ from \eqref{eq:dtDIFFW}:
\begin{equation}
\Delta t_P = \frac{\rho_w E_0}{\rho_i \max H}\, \Delta t_W.  \label{eq:dtDIFFPfromW}
\end{equation}
In fact, with the estimates $\rho_w/\rho_i \approx 1$, $E_0\approx 1$ m, and $\max H \approx 1000$ m we have $\Delta t_P$ which is about 1000 times smaller than $\Delta t_W$.  With these values and others used earlier (i.e.~$\Delta x = \Delta y = 500$ m, $\max |\bV|=10^5$ m/a, $K=10^{-2}$ m/s and $\max W=1$ m) we get
\begin{align*}
  \Delta t_{\text{CFL}} &\approx 0.001  \text{ year} &&\text{ from \eqref{eq:dtCFL}}, \\
  \Delta t_W            &\approx 0.1    \text{ year} &&\text{ from \eqref{eq:dtDIFFW}}, \\
  \Delta t_P            &\approx 0.0001 \text{ year} &&\text{ from \eqref{eq:dtDIFFP} or \eqref{eq:dtDIFFPfromW}.}
\end{align*}
Thus the numerical scheme for pressure diffusion, given next, has the shortest time step.  This analysis says if is only about 10 times shorter than the CFL restriction for the advection, however.  Furthermore, the precise size of the stable time step $\Delta t_P$ scales inversely with the adjustable small thickness $E_0$.  By choosing $E_0$ larger or smaller we can make the time step restriction on $\Delta t_P$ less or more severe, respectively.

Now, the scheme we use for \eqref{eq:diffusionpressure} is similar to \eqref{eq:Wfd} for \eqref{eq:adeqn} but without a need for approximating advection:
\begin{align}
\frac{E_0}{(P_o)_{i,j}} \frac{P_{i,j}^{l+1} - \Plij}{\Delta t} &= c_0 \bigg[\frac{W_e \left(\psi_{i+1,j}^l - \psi_{i,j}^l\right) - W_w \left(\psi_{i,j}^l - \psi_{i-1,j}^l\right)}{\Delta x^2}  \label{eq:Pfd} \\
      &\qquad\qquad + \frac{W_n \left(\psi_{i,j+1}^l - \psi_{i,j}^l\right) - W_s \left(\psi_{i,j}^l - \psi_{i,j-1}^l\right)}{\Delta y^2}\bigg] \notag \\
      &\qquad + c_2 A \left(\rho_i g H_{i,j}- \Plij\right)^3 \Wlij - c_1 |\bv_b|_{i,j} \left(W_r - \Wlij\right)_+ + \Phi_{i,j}. \notag
\end{align}
For implementation it is useful to restate \eqref{eq:Pfd} in explicit update form.  First define
	$$\omega_x = \frac{c_0 \Delta t}{\Delta x^2}, \qquad \omega_y = \frac{c_0 \Delta t}{\Delta y^2}.$$
Also let
	$$\mathcal{O}_{ij} = c_1 |\bv_b|_{i,j} \left(W_r - \Wlij\right)_+, \qquad \mathcal{C}_{ij} = c_2 A \left(\rho_i g H_{i,j} - \Plij\right)^3 \Wlij$$
be the gridded values of the cavitation-opening and creep-closure rates.  Then scheme \eqref{eq:Pfd} is equivalent to this form:
\begin{align}
P_{i,j}^{l+1} &= \Plij +  \frac{(P_o)_{i,j}}{E_0} \bigg[\omega_x W_e \left(\psi_{i+1,j}^l - \psi_{i,j}^l\right) - \omega_x W_w \left(\psi_{i,j}^l - \psi_{i-1,j}^l\right) \label{eq:Pfdupdate} \\
      &\qquad\qquad\qquad\quad + \omega_y W_n \left(\psi_{i,j+1}^l - \psi_{i,j}^l\right) - \omega_y W_s \left(\psi_{i,j}^l - \psi_{i,j-1}^l\right) \notag \\
      &\qquad\qquad\qquad\quad + \Delta t\, \left(\mathcal{C}_{ij} - \mathcal{O}_{ij} + \Phi_{i,j}\right)\bigg]. \notag
\end{align}

\subsection*{One time step of the model}  Mathematical model \eqref{eq:bluebox} evolves $W$ and $P$.  One time step of this evolution is described as follows.

This description assumes that the ice geometry and ice sliding speed are fixed so that $h_{i,j}$, $b_{i,j}$, $(P_o)_{i,j}$, and $|\bv_b|_{i,j}$ are all determined for the duration of the run.  The ice geometry may be quite general, with ice-free land and floating ice allowed.  In fact, the ice geometry determines true-false masks for grid cell state based on a sea level of elevation zero:
\begin{align*}
\text{\texttt{icefree}}_{i,j} &= (h_{i,j} > 0)\, \&\, (h_{i,j} = b_{i,j}), \\
\text{\texttt{float}}_{i,j}   &= (\rho_i (H_{\text{float}})_{i,j} < - \rho_{sw}\, b_{i,j}).
\end{align*}
Here $H_{\text{float}}=h_{i,j} / (1 - r)$ is the thickness of the ice if it is floating, $r=\rho_i / \rho_{sw}$, and we take a sea-water density $\rho_{sw}=1028.0$.  Note that $\text{\texttt{float}}_{i,j}$ is true in ice-free ocean.  While the subglacial layer we are attempting to model has both \texttt{icefree} and \texttt{float} masks false, the other cases provide boundary conditions.

Thus one time step follows this algorithm:
\renewcommand{\labelenumi}{\emph{(\roman{enumi})}}
\begin{enumerate}
\item Start with values $\Wlij$, $\Plij$ which satisfy the bounds $W\ge 0$ and $0 \le P \le P_o$.
\item Compute the current values of the hydraulic potential, $\psi_{i,j}^l = \Plij + \rho_w g(b_{i,j} + \Wlij)$, but with $\psi_{i,j}^l=(P_o)_{i,j}$ where $\text{\texttt{float}}_{i,j}$.
\item Compute velocity components at staggered grid locations from \eqref{eq:velocitycomp}.   %FIXME: however, dPdx and dPdy should not difference outside the ice domain?
\item Get $W$ values averaged onto the staggered grid from \eqref{eq:stagW}.
%FIXME: however, Wea and Wno should not average from outside the ice domain?
\item Get time step $\Delta t = \min\{\Delta t_{\text{CFL}}, \Delta t_W, \Delta t_P\}$ using criteria \eqref{eq:dtCFL}, \eqref{eq:dtDIFFW}, and \eqref{eq:dtDIFFPfromW}.  (As noted, $\Delta t = \Delta t_P$ is expected for most grids.)
\item Use \eqref{eq:Wfd} to compute preliminary values for $W_{i,j}^{l+1}$.  Don't use diffusive flux when a point or its neighbor is \texttt{icefree} or \texttt{float}.
\item If $\text{\texttt{icefree}}_{i,j}$ or $\text{\texttt{float}}_{i,j}$ then set $W_{i,j}^{l+1}=0$.  If $W_{i,j}^{l+1}<0$ then reset (project) $W_{i,j}^{l+1}=0$; this applies only when $\Phi<0$ because the rest of the method is positivity-preserving (see Appendix).
\item Use \eqref{eq:Pfdupdate} to compute preliminary values for $P_{i,j}^{l+1}$.  Don't compute divergence of diffusive flux when neighbors are \texttt{icefree} or \texttt{float}.
\item If $\text{\texttt{icefree}}_{i,j}$ set $P_{i,j}^{l+1}=0$.  If $\text{\texttt{float}}_{i,j}$ then set $P_{i,j}^{l+1} = (P_o)_{i,j}$; this is the pressure of sea water at the base of the ice.  If $P_{i,j}^{l+1}$ does not satisfy bounds $0 \le P \le P_o$ then reset (project) back into the correct range.
\item Update time $t_{l+1}=t_l+\Delta t$ and repeat at \emph{(i)}.
\end{enumerate}

The mass accounting scheme must keep track of the projections in step \emph{(vii)}.  In particular, that water which is lost at the margin where either the thickness goes to zero on land or the ice becomes floating (grounding line) may be part of the intended modeling goal.  On the other hand there is no expectation that pressure update \eqref{eq:Pfdupdate} will preserve bounds $0\le P \le P_o$.  The pressure is not a conserved quantity, so projection step \emph{(ix)} requires no accounting.



\section{Numerical results}  \label{sec:results}

\subsection*{Verification of the coupled model}  FIXME: some text on verification procedure

\begin{figure}[ht]
\includegraphics[width=4.0in,keepaspectratio=true]{figs/refineW}
\caption{Average (circle) water thickness error $|W-W_{exact}|$ decays with decreasing $\Delta x = \Delta y$.  The average error decays at $O(\Delta x^{0.74})$.  The maximum (star) error does not decay.  In fact the maximum $W$ error at the ice margin is approximately 1 m in the $\Delta x=375$ m case (off scale).}
\label{fig:refineW}
\end{figure}

\begin{figure}[ht]
\includegraphics[width=4.0in,keepaspectratio=true]{figs/refineP}
\caption{Average (circle) and maximum (star) pressure error $|P-P_{exact}|$ decays with decreasing $\Delta x = \Delta y$.  The average error decays at $O(\Delta x^{0.96})$.}
\label{fig:refineP}
\end{figure}

\subsection*{Steady results for a tidewater glacier}

FIXME: results from nbreen

\begin{figure}[ht]
\includegraphics[width=7.0in,keepaspectratio=true]{figs/bed-surf-250m}
\caption{FIXME}
%\label{fig:refineP}
\end{figure}

\begin{figure}[ht]
\includegraphics[width=7.0in,keepaspectratio=true]{figs/icethk-icefree-float-250m}
\caption{FIXME}
%\label{fig:refineP}
\end{figure}

\begin{figure}[ht]
\includegraphics[width=7.0in,keepaspectratio=true]{figs/outline-input-250m}
\caption{FIXME}
%\label{fig:refineP}
\end{figure}

\begin{figure}[ht]
\includegraphics[width=7.0in,keepaspectratio=true]{figs/W-Pmask-250m}
\caption{FIXME}
%\label{fig:refineP}
\end{figure}

\begin{figure}[ht]
\includegraphics[width=7.0in,keepaspectratio=true]{figs/Po-P-250m}
\caption{FIXME}
%\label{fig:refineP}
\end{figure}

FIXME:  text about F\&C equation \eqref{eq:PofWFC}

\begin{figure}[ht]
\includegraphics[width=3.0in,keepaspectratio=true]{figs/isPofW-250m} \,
\includegraphics[width=3.0in,keepaspectratio=true]{figs/isPofW-250m-month}
\caption{Left: Scatter plot of $(W,P)$ pairs for all cells at end of 5 year steady-input simulation on a 250 m grid.  Red dashed is equation \eqref{eq:PofWFC} with $W_{\text{crit}} = W_r = 1$ m.  Right: Same except at the end of a one month simulation, and with equation \eqref{eq:PofWFC} using $W_{\text{crit}} = W_r / 3$.}
\label{fig:isPofWnbreen}
\end{figure}

\clearpage\newpage

\small
\bibliography{ice_bib}  % generally requires link to pism/doc/ice_bib.bib
\bibliographystyle{agu}
\normalsize

\clearpage\newpage
\appendix

\section{Positivity and stability of the mass conservation scheme}

Explicit numerical scheme \eqref{eq:Wfd} for the mass conservation PDE \eqref{eq:adeqn} is sufficiently simple so that we can analyze its properties.  We now sketch a maximum principle argument \citep{MortonMayers} for the scheme.  This shows both stability and a positivity-preserving property: if the water input $\Phi$ is nonnegative and the discrete water thicknesses $\Wlij$ are nonnegative then, under the given stability conditions, the updated values $W_{i,j}^{l+1}$ are also nonnegative.    We consider only the case where all of the discrete velocities at the centers of cell interfaces are nonnegative: $\alpha_e\ge 0$, $\alpha_w\ge 0$, $\beta_n\ge 0$, $\beta_s\ge 0$, but the other upwinding cases, wherein these velocity components have various signs, can be handled by similar special-case arguments like the present one.

Let $\nu_x = \Delta t/\Delta x$, $\nu_y = \Delta t/\Delta y$, $\mu_x = K \Delta t / (\Delta x)^2$, and $\mu_y = K \Delta t / (\Delta y)^2$.  We rewrite \eqref{eq:Wfd} as a computation of the next value $W_{i,j}^{l+1}$, and collect terms:
\begin{align*}
 W_{i,j}^{l+1} &= \Wlij - \nu_x \left(\alpha_e \Wlij - \alpha_w W_{i-1,j}^l\right) - \nu_y \left(\beta_n \Wlij - \beta_s W_{i,j-1}^l\right)  \\
      &\qquad + \mu_x \left[W_e \left(W_{i+1,j}^l - \Wlij\right) - W_w \left(\Wlij - W_{i-1,j}^l\right)\right]  \\
      &\qquad + \mu_y \left[W_n \left(W_{i,j+1}^l - \Wlij\right) - W_s \left(\Wlij - W_{i,j-1}^l\right)\right] + \Delta t \Phi_{ij}.
\end{align*}
Rearranging a bit further we get
\begin{align*}
 W_{i,j}^{l+1} &= (\nu_x \alpha_w + \mu_x W_w) W_{i-1,j}^l + (\mu_x W_e) W_{i+1,j}^l + (\nu_y \beta_s + \mu_y W_s) W_{i,j-1}^l + (\mu_y W_n) W_{i,j+1}^l \\
      &\qquad + \Big[1 - \nu_x \alpha_e - \nu_y \beta_n - \mu_x (W_e + W_w) - \mu_y (W_n + W_s)\Big] \Wlij + \Delta t \Phi_{ij}
\end{align*}
so that the new value is a linear combination of the old values, plus a source term:
\begin{equation}
W_{i,j}^{l+1} = A W_{i-1,j}^l + B W_{i+1,j}^l + C W_{i,j-1}^l + D W_{i,j+1}^l + E \Wlij + \Delta t \Phi_{ij}. \label{eq:lincomb}
\end{equation}
Because of our assumption about nonnegative velocities, and assuming $\Wlij \ge 0$ for all $i,j$, we see that coefficients $A,B,C,D$ are all nonnegative, and that only $E$ could be negative.  Thus we can state a sufficient condition based on an equal split between advective and diffusive parts.  First assume a CFL-type time step restriction for the advection term in  \eqref{eq:adeqn}:
\begin{equation}
\nu_x \alpha_e + \nu_y \beta_n = \Delta t \left(\frac{\alpha_e}{\Delta x} + \frac{\beta_n}{\Delta y}\right) \le \frac{1}{2}. \label{eq:adstabcond}
\end{equation}
Also assume a second time-step restriction on the diffusion:
\begin{equation}
\mu_x (W_e + W_w) + \mu_y (W_n + W_s) = \Delta t \left(\frac{K(W_e + W_w)}{\Delta x^2} + \frac{K(W_n + W_s)}{\Delta y^2}\right) \le \frac{1}{2}. \label{eq:diffstabcond}
\end{equation}
The right-hand sides of these inequalities are each $1/2$, and of course $1-(1/2)-(1/2)=0$.  Thus the coefficient $E$ in \eqref{eq:lincomb} is nonnegative:
	$$E = 1 - \nu_x \alpha_e - \nu_y \beta_n - \mu_x (W_e + W_w) - \mu_y (W_n + W_s) \ge 0.$$
It follows from \eqref{eq:lincomb} that if $\Wlij\ge 0$ and $\Phi_{ij}\ge 0$ for all $i,j$ then $W_{ij}^{l+1}\ge 0$.

We see that all coefficients in linear combination \eqref{eq:lincomb} are nonnegative if the time step restrictions are satisfied.  Furthermore the coefficients add to one.  It follows \citep{MortonMayers} that scheme \eqref{eq:Wfd} is positivity-preserving when $\Phi\ge 0$.  It also follows that the scheme is stable under conditions \eqref{eq:dtCFL} and \eqref{eq:dtDIFFW}; these are the all-upwinding-cases generalizations of inequalities \eqref{eq:adstabcond} and \eqref{eq:diffstabcond}, respectively.


\end{document}
