\documentclass[12pt,final]{amsart}%default 10pt
%prepared in AMSLaTeX, under LaTeX2e

\usepackage[margin=1in]{geometry}

\usepackage{natbib}

\usepackage{amssymb,alltt,verbatim,xspace,fancyvrb}
\usepackage{palatino}

% check if we are compiling under latex or pdflatex
\ifx\pdftexversion\undefined
  \usepackage[final,dvips]{graphicx}
\else
  \usepackage[final,pdftex]{graphicx}
\fi

% hyperref should be the last package we load
\usepackage[pdftex,
                colorlinks=true,
                plainpages=false, % only if colorlinks=true
                linkcolor=blue,   % only if colorlinks=true
                citecolor=black,   % only if colorlinks=true
                urlcolor=magenta     % only if colorlinks=true
]{hyperref}

\newcommand{\normalspacing}{\renewcommand{\baselinestretch}{1.1}\tiny\normalsize}
\newcommand{\tablespacing}{\renewcommand{\baselinestretch}{1.0}\tiny\normalsize}
\normalspacing

% math macros
\newcommand\bv{\mathbf{v}}
\newcommand\bV{\mathbf{V}}
\newcommand\bq{\mathbf{q}}

\newcommand\CC{\mathbb{C}}
\newcommand{\DDt}[1]{\ensuremath{\frac{d #1}{d t}}}
\newcommand{\ddt}[1]{\ensuremath{\frac{\partial #1}{\partial t}}}
\newcommand{\ddx}[1]{\ensuremath{\frac{\partial #1}{\partial x}}}
\newcommand{\ddy}[1]{\ensuremath{\frac{\partial #1}{\partial y}}}
\newcommand{\ddxp}[1]{\ensuremath{\frac{\partial #1}{\partial x'}}}
\newcommand{\ddz}[1]{\ensuremath{\frac{\partial #1}{\partial z}}}
\newcommand{\ddxx}[1]{\ensuremath{\frac{\partial^2 #1}{\partial x^2}}}
\newcommand{\ddyy}[1]{\ensuremath{\frac{\partial^2 #1}{\partial y^2}}}
\newcommand{\ddxy}[1]{\ensuremath{\frac{\partial^2 #1}{\partial x \partial y}}}
\newcommand{\ddzz}[1]{\ensuremath{\frac{\partial^2 #1}{\partial z^2}}}
\newcommand{\Div}{\nabla\cdot}
\newcommand\eps{\epsilon}
\newcommand{\grad}{\nabla}
\newcommand{\ihat}{\mathbf{i}}
\newcommand{\ip}[2]{\ensuremath{\left<#1,#2\right>}}
\newcommand{\jhat}{\mathbf{j}}
\newcommand{\khat}{\mathbf{k}}
\newcommand{\nhat}{\mathbf{n}}
\newcommand\lam{\lambda}
\newcommand\lap{\triangle}
\newcommand\Matlab{\textsc{Matlab}\xspace}
\newcommand\RR{\mathbb{R}}
\newcommand\vf{\varphi}

\newcommand{\Wlij}{W^l_{i,j}}
\newcommand{\Wij}{W_{i,j}}
\newcommand{\Ylij}{Y^l_{i,j}}
\newcommand{\Yij}{Y_{i,j}}
\newcommand{\upp}[3]{\big<#1\big|_{#3}\,#2\big>}



\title[]{A less-minimal model of subglacial hydrology}

\author[]{Ward van Pelt and Ed Bueler}


\begin{document}

\maketitle

\thispagestyle{empty}

%\setcounter{tocdepth}{1}
%\tableofcontents

\section{Introduction}

Any reasonable model of the subglacial aquifer (liquid water layer) has at least these two elements: liquid water is conserved and water flows from high to low hydraulic potential  (``head'').  Additionally, physical processes control the geometry of the layer, including the opening of cavities by sliding and the closure of cavities (and channels) by creep.  Additionally, channels open by melting, sediment moves, and so on, but we do not model these here.

We model water pressure by a damped form of the full-cavity formulation.  In the exactly-full cavity formulation the result is an elliptic variational inequality \citep{Schoofetal2012} for pressure that causes the computation of pressure to be nonlocal over each connected component of the hydrological system.  Here we avoid this instantaneous distributed balance by not exactly enforcing the full-cavity condition.


\section{Continuum model}

\subsection*{Elements}  We consider a layer of water with thickness $W(t,x,y)$.  This thickness is only likely to be meaningful compared to observations, however, if it is regarded as an average over a horizontal scale of tens to thousands of meters.  As the hydrologic system has fine spatial variation which one is unlikely to be able to model, we will attempt only to model spatially-averaged versions of water amount and water pressure.

We assume that water is incompressible so that the thickness of the water layer tells us the mass of water.  Choosing to model subglacial hydrology using a water thickness is not a significant restriction on the physics.

Water is conserved.  In two spatial dimensions this is the equation \citep{Clarke05}
\begin{equation} \label{eq:conserve}
\frac{\partial W}{\partial t} + \Div \bq = \Phi
\end{equation}
where $\bq$ is the vector water flux (units $\text{m}^2\,\text{s}^{-1}$) and $\Phi$ is a source term ($\text{m}\,\text{s}^{-1}$).

We might separate the water sources between the melt on the lower surface of the glacier and the en- or supra-glacial drainage origin,
  $$\Phi = \rho_w^{-1} \left(m + S\right)$$
where $\rho_w$ is the density of fresh liquid water, $m$ is the rate at which basal melting (refreeze) of ice adds (removes) water, and $S$ is the rate at which surface runoff or englacial drainage adds water.  Note $m$ and $S$ have units $\text{kg}\,\text{m}^{-2}\,\text{s}^{-1}$.

\newcommand{\Nbreen}{Nordenski\"oldbreen\xspace}
For the \Nbreen, Svalbard example below we will take $m=0$ so that only supraglacial input is modeled.

The water flux $\bq$ in equation \eqref{eq:conserve} is related to the gradient of a hydraulic potential $\psi(t,x,y)$ which combines the actual subglacial water pressure $P(t,x,y)$ and the gravitational potential corresponding to top of the layer of water at the location on the bed of the glacier,
\begin{equation} \label{eq:potential}
\psi = P + \rho_w g\, (b+W).
\end{equation}
Here $z=b(x,y)$ is the time-independent bedrock elevation, which we assume is given by time-independent data.

Water flows from high to low hydraulic potential.  The simplest model is for a water sheet \citep{Clarke05}
\begin{equation}  \label{eq:flux}
\bq = - \frac{K \, W}{\rho_w g} \grad \psi
\end{equation}
Here, $\rho_w$ is the water density ($\text{kg}\,\text{m}^{-3}$), $g$ the gravitational acceleration ($\text{m}\,\text{s}^{-2}$) and $K$ is the effective hydraulic conductivity ($\text{m}\,\text{s}^{-1}$).  Notice that the system transmits more water for a given head gradient if either the ability of the subglacial material to conduct water is bigger (i.e.~$K$ is larger) or if the water sheet is thicker ($W$ is larger).

\subsection*{Advection-diffusion form}  Combining \eqref{eq:potential} and \eqref{eq:flux} above we have this expression which identifies a part of the flux as proportional to the gradient of the water thickness:
	$$\bq = - \frac{K\, W}{\rho_w g} \left(\grad P + \rho_w g b\right) - K W \grad W.$$
The flux which is down the gradient of the conserved quantity will act diffusively.  The remaining flux is likely to be dominant.  It is, to our knowledge, not particularly diffusive.  We conceive of it as transport driven by a velocity field which varies in space and time.  We will construct a conservative numerical scheme based on this understand of how the flux is decomposed.

Let
\begin{equation} \label{eq:vexpression}
  \bV = - \frac{K}{\rho_w g} \grad P - K \grad b
\end{equation}
be the velocity field.  Then equations \eqref{eq:potential}, \eqref{eq:flux}, and \eqref{eq:vexpression} combine to this simpler description of the flux,
\begin{equation} \label{eq:qexpression}
  \bq = \bV\, W - K W \grad W.
\end{equation}

Generally the water flow $\bq$ depends significantly on the ice surface slope because the gradient of the overburden pressure follows that slope.  The pressure model below may generate pressure fields with that property, but this is not obvious.  More obviously the flux depends on the bedrock slope because the velocity $\bV$ has such a term.  Because the bedrock elevation comes from rough data in practice, this part of the velocity $\bV$ will not be very smooth.  The part of the flux from the pressure gradient may not be very smooth either, depending on the solution of the pressure model below.

In any case, from equations \eqref{eq:conserve} and \eqref{eq:qexpression} we derive an advection-diffusion equation \citep{HundsdorferVerwer2010,MortonMayers} for the evolution of the water layer thickness:
\begin{equation} \label{eq:adeqn}
  \frac{\partial W}{\partial t} + \Div\left(\bV\, W\right) = \Div \left(K W \grad W\right) + \Phi.
\end{equation}
As we will see, the significance of this form is that there are numerical differences in how the advection term $\Div\left(\bV\, W\right)$ is handled relative to the diffusion term $\Div \left(K W \grad W\right)$.

\subsection*{Capacity of the system}  First recall that the ice is a fluid which has a pressure field of its own, with basal value $P_o$, the \emph{overburden pressure}.  We make the shallow approximation that the ice pressure is hydrostatic \citep{GreveBlatter2009}:
\begin{equation} \label{eq:hydrostatic}
  P_o = \rho_i g H = \rho_i g (h-b).
\end{equation}
Here $\rho_i$ is the density of ice ($\text{kg}\,\text{m}^{-3}$), $H$ is the ice thickness (m), and $h$ is the ice upper surface elevation (m).  Now let
\begin{equation}
N = P_o - P\label{eq:effective}
\end{equation}
be the effective pressure.  The effective pressure is high if the subglacial water is unpressurized and low in the high pressure water case.  The effective pressure is a measure of much of the ice load is carried by the mineral (till or bedrock) base.

The evolution of the capacity thickness $Y$ (m) of a linked-cavity system is described as the sum of opening by cavitation and closure by creep:
\begin{equation}
\frac{\partial Y}{\partial t} = c_1 \left(W_r - Y\right) |\bv_b| - c_2 A |N|^3 Y, \label{eq:capacity}
\end{equation}
where $W_r$ is a maximum roughness scale of the basal topography, $\bv_b$ is the ice basal velocity (i.e.~sliding velocity), $A$ is the ice softness, and $c_1,c_2$ are constants to be fit by data (see below).  We have used Glen exponent $n=3$ for concreteness.

Equation \eqref{eq:capacity} describes the evolution of the upper (ice) surface of the subglacial cavity.  The creep term is sensitive to the difference of ice and water pressure, but ice creep is relatively slow.  On the other hand there are presumably-strong stresses (pressures) within the liquid water associated to keeping the cavity full because liquid water is an incompressible fluid with low viscosity.  For example, if the cavity is locally larger than local water sources can easily fill then the pressure field should cause inflow into this local area, assuming water is available in connected cavities.  Conversely, if local water sources exceed capacity then the pressure field should force water out of the local area.  Only if there is insufficient connection to neighboring capacity should the extreme cases occur, wherein a vapor (e.g. air) layer forms in the cavity, or the ice is forced upward by a negative effective pressure, respectively.

\subsection*{Full-cavity closure for pressure}  At this point we do not know how to compute pressure.  The apparent state variables of the model so far are $W$,$Y$,$P$, in the sense that for a fixed ice geometry and ice velocity, the above equations can be written as two partial differential equations in these three variables (along with a number of additional parameters, of course).  Two equations are not enough to determine these variables.  A closure is needed.

A strong closure is to require the aquifer to be full at all locations and times:
\begin{equation}
W = Y.\label{eq:strongclosure}
\end{equation}
The consequences of this strong closure are explored by \cite{Schoofetal2012} in some detail in a broader context.  In our simpler circumstances, equation \eqref{eq:strongclosure} allows us to eliminate ``$Y$'' and equate expressions for $\partial W/\partial t = \partial Y/\partial t$ from equations \eqref{eq:conserve}, \eqref{eq:potential}, \eqref{eq:flux} with that from \eqref{eq:capacity} to get this equation:
\begin{equation}
\Div \left(\frac{K\,W}{\rho_w g} \grad \left(P + \rho_w g (b+W)\right) \right) = c_1 \left(W_r - W\right) |\bv_b| - c_2 A |P_o - P|^3 W.\label{eq:ellipticpressure}
\end{equation}

Though we will avoid actually solving equation \eqref{eq:ellipticpressure}, it is significant to our theory.  If we have an instantaneous water thickness field $W$ then, assuming adequate pressure boundary conditions, \eqref{eq:ellipticpressure} is an elliptic PDE for the unknown pressure field.  Of course the pressure solution is actually subject to inequalities
\begin{equation}
0 \le P \le P_o. \label{eq:bounds}
\end{equation}
Thus the mathematical problem is an elliptic variational inequality explored by \cite{Schoofetal2012}.  The numerical analysis of such problems can be approached within a finite element \citep{SchoofStream,JouvetBueler2012} numerical paradigm.

However, the goal of this work is the selection of suitable subglacial hydrology models for coupling to ice dynamics, and potential for parameter evaluation from observations, and these goals suggest the need for rapid implementation.  Therefore, in these notes, we seek a more easily-implemented finite difference alternative which allows us to explore the space of physical parameters using real data.

\subsection*{Nearly-full-cavity closure}  Therefore we replace rigid constraint \eqref{eq:strongclosure}, namely $0 = Y - W$, with a damped version.  Choose a characteristic time scale $\tau$ which should be short (hours to months) compared to ice flow change time scales (months to centuries).  Then consider this equation
\begin{equation}
- \tau \frac{\partial Y}{\partial t} = Y - W. \label{eq:dampedclosure}
\end{equation}
As $\tau \to 0$ this equation recovers the full-cavity closure $Y=W$.  It says that if the notional cavity upper surface ($Y$) is locally above the water level ($Y>W$) then it should lower ($\partial Y/\partial t < 0$), and, vice versa, if the notional cavity upper surface is locally below the water ($Y<W$) then the cavity top should rise ($\partial Y/\partial t > 0$).   Said a different way, equation \eqref{eq:dampedclosure} represents a penalty method for enforcing constraint \eqref{eq:strongclosure}.

Equation \eqref{eq:dampedclosure} does not allow us to eliminate $Y$ as we may when using \eqref{eq:strongclosure}.  Thus \eqref{eq:dampedclosure} requires us to keep the separate state variable $Y$ in a time-dependent way.

On the other hand, equation \eqref{eq:dampedclosure} allows us to express pressure in terms of the geometric state variables $W$ and $Y$, as follows.  Solve \eqref{eq:capacity} for $N$, thus $P$, and substitute the expression for $\partial Y/\partial t$ from \eqref{eq:dampedclosure}:
\begin{equation}
P = P_o - \left(\frac{\tau^{-1} (Y-W) + c_1 \left(W_r - Y\right) |\bv_b|}{c_2 A Y}\right)^{1/3}.  \label{eq:pressurenormal}
\end{equation}

The above expression for $P$ applies only in certain ``normal'' cases, however.  In particular it assumes positive effective pressure ($N\ge 0$).  Furthermore we must address how the bounds \eqref{eq:bounds} are maintained.  Note that the denominator of the fraction in \eqref{eq:pressurenormal} is always positive but the denominator can change sign.  Let
\begin{equation}
\mathcal{Z} = \tau^{-1} (Y-W) + c_1 \left(W_r - Y\right) |\bv_b|, \label{eq:Znumerator}
\end{equation}
the numerator appearing in Equation \eqref{eq:pressurenormal}.

We have the following conditional scheme for pressure:
\begin{equation}
P = \begin{cases}
0, & \mathcal{Z} \ge c_2 A Y P_o^3\,\, (\ge 0) , \\
P_o, & \mathcal{Z} \le 0, \\
P_o - \mathcal{Z}^{1/3} (c_2 A Y)^{-1/3}, & \text{all other cases}.
\end{cases} \label{eq:pressureWY}
\end{equation}
The last ``normal pressure'' case occurs exactly when the given formula computes a pressure satisfying \eqref{eq:bounds}.  As a result of this scheme we have a function which computes pressure in terms of $Y,W$, namely $P = f(Y,W)$.  Of course $f$ is also a function of model parameters (i.e.~$W_r,c_1,c_2,\dots$) and of ice flow model variables (i.e.~$P_o,|\bv_b|$).

\subsection*{Solvable mathematical model}  The above equations have now yielded a solvable mathematical model, but the reader deserves a clarified version.  We essentially have four major definitions, equations \eqref{eq:hydrostatic}, \eqref{eq:Znumerator}, \eqref{eq:pressureWY}, and \eqref{eq:vexpression} which we can apply in the following order at each time step:
\begin{align}
P_o &= \rho_i g (h-b), \label{eq:AGAINhydrostatic} \\
\mathcal{Z} &= \tau^{-1} (Y-W) + c_1 \left(W_r - Y\right) |\bv_b|, \label{eq:AGAINZnumerator} \\
P &= \begin{cases}
0, & \mathcal{Z} \ge c_2 A Y P_o^3\,\, (\ge 0) , \\
P_o, & \mathcal{Z} \le 0, \\
P_o - \mathcal{Z}^{1/3} (c_2 A Y)^{-1/3}, & \text{all other cases},
\end{cases} \label{eq:AGAINpressureWY} \\
\bV &= - \frac{K}{\rho_w g} \grad P - K \grad b. \label{eq:AGAINvexpression}
\end{align}
The time step itself is for these two evolution equations, which are equations \eqref{eq:adeqn} and \eqref{eq:dampedclosure}, respectively,
\begin{align}
\frac{\partial W}{\partial t} + \Div\left(\bV\, W\right) &= \Div \left(K W \grad W\right) + \Phi, \label{eq:AGAINadeqn} \\
\frac{\partial Y}{\partial t} &= \frac{Y - W}{-\tau}. \label{eq:AGAINdampedclosure}
\end{align}

The above equations relate three classes of symbols, \emph{parameters}, \emph{data functions}, and \emph{state functions}, as summarized in the following table:

\begin{table}
\begin{tabular}{r|c}
 & symbols \\ \hline
\emph{parameters (scalar)} & $g$, $\rho_i$, $\rho_w$, $A$, $K$, $W_r$, $c_1$, $c_2$, $\tau$ \\
\emph{data functions} & $\Phi$, $b$, $h$, $|\bv_b|$ \\
\emph{state functions} & $W$, $Y$
\end{tabular}
\end{table}

\noindent The parameters are time- and space-independent.  Default values must be chosen for these parameters, even as the informed model user will seek to change these values and explore the parameter space.  The data functions are, essentially, supplied by the rest of the ice sheet model.

The two state functions $W$ and $Y$ are the key evolving quantities in the numerical model's memory.  They must be saved by the current model at any stopping and/or restarting point.  Initial values, and boundary values, of these quantities must somehow be supplied.  Note that we always assume these thicknesses are positive:
\begin{equation}
W \ge 0, \qquad Y \ge 0.
\end{equation}

The defined quantities $P_o$, $\mathcal{Z}$, $P$, and $\bV$ in equations \eqref{eq:AGAINhydrostatic}, \dots, \eqref{eq:AGAINvexpression} could be thought of as a fourth class of symbols, \emph{derived functions}, but of course they can be eliminated from any expression according to taste.


\section{Numerical scheme}

Equation \eqref{eq:AGAINadeqn} will be discretized in this section by an explicit\footnote{This is a first draft.  I expect to find that ``explicit'' is not good enough because on fine grids the diffusive time-step restriction is too severe.  The most natural ``good enough'' guess for a scheme is a semi-implicit solution with implicit treatment of the diffusive part.} conservative first-order upwind method for the advection part and a centered, second-order scheme for the nonlinear diffusion part.  Equation \eqref{eq:AGAINdampedclosure} will be discretized by a semi-implicit first-order Euler method (implicit for $Y$ and explicit for $W$).

To set notation, suppose our rectangular computational domain has $M_x \times M_y$ gridpoints $(x_i,y_j)$ with uniform spacing $\Delta x,\Delta y$.  Let $\Wlij \approx W(t_l,x_i,y_j)$ and $\Ylij \approx Y(t_l,x_i,y_j)$ be the approximation of the continuum solution for the state functions at the grid points.

To explain the upwind method for \eqref{eq:AGAINadeqn}, consider the model equation
\begin{equation} \label{eq:modeladvect}
u_t + (v(x) u)_x = 0
\end{equation}
for some quantity $u(t,x)$ transported by a flux $q = v(x) u$.  A ``donor cell'' upwind scheme can be described as a finite volume scheme \citep{LeVeque} wherein a grid point $x_j$ is the center of a cell.  We consider the flux at the cell interfaces $x_{j-1/2}$ and $x_{j+1/2}$.  We decide which spatial finite difference to compute based on the sign of the velocity $v(x)$ at these interfaces.

The scheme is easier to display if we define the following upwind notation,
\newcommand{\up}[2]{\big<#1\big|\,#2\big>}
	$$\up{v}{U_j} := v \begin{Bmatrix} U_j, & v \ge 0 \\ U_{j+1}, & v < 0 \end{Bmatrix}.$$
For the model equation \eqref{eq:modeladvect} on a space-time grid $(t_l,x_j)$ we set
\begin{equation}\label{eq:modelfdadvect}
\frac{U_j^{l+1} - U_j^l}{\Delta t} + \frac{\up{v_+}{U_j^l} - \up{v_-}{U_{j-1}^l}}{\Delta x} = 0
\end{equation}
where $v_+ = v(x_{j+1/2})$ and $v_-=v(x_{j-1/2})$.

\begin{figure}[ht]
\centering
\includegraphics[width=2.5in,keepaspectratio=true]{figs/diffstencil}
\bigskip
\caption{The numerical scheme for Equation \eqref{eq:AGAINadeqn} is a finite volume scheme for the grid-centered cell (dashed line).  The velocity and diffusivities are evaluated at the staggered grid locations (triangles).  The state functions $W,Y$ are needed at the regular grid points (diamonds).}
\label{fig:stencil}
\end{figure}

Now we can state our scheme for equation \eqref{eq:AGAINadeqn}, starting with the coefficients.  Suppose the velocity has components $\bV = (\alpha,\beta)$ and recall that $\Div \left(\bV W\right) = (\alpha W)_x + (\beta W)_y$.  We will compute velocity components at staggered (cell-face-centered) points, shown with triangle markers in Figure \ref{fig:stencil}.  We compute these values based on centered finite difference approximations of Equation \eqref{eq:AGAINvexpression}.

Let $c_0=K/(\rho_w g)$, a derived parameter used for convenience, which allows us to write a simplest description of the velocity:  $\bV = - c_0 \grad P - K \grad b$.  We use ``compass'' notation for the components:
\begin{align*}
\alpha_e &= \alpha_{i+1/2,j} = - c_0 \frac{P_{i+1,j}-P_{i,j}}{\Delta x} - K \frac{b_{i+1,j}-b_{i,j}}{\Delta x}, \\
\alpha_w &= \alpha_{i-1/2,j} = - c_0 \frac{P_{i,j}-P_{i-1,j}}{\Delta x} - K \frac{b_{i,j}-b_{i-1,j}}{\Delta x}, \\
\beta_n  &= \beta_{i,j+1/2} = - c_0 \frac{P_{i,j+1}-P_{i,j}}{\Delta y} - K \frac{b_{i,j+1}-b_{i,j}}{\Delta y}, \\
\beta_s  &= \beta_{i,j-1/2} = - c_0 \frac{P_{i,j}-P_{i,j-1}}{\Delta y} - K \frac{b_{i,j}-b_{i,j-1}}{\Delta y}.
\end{align*}

Similarly for the diffusive term, the staggered-grid values of the current water thicknesses are computed by averaging: $W_e = (W_{i,j}^l + W_{i+1,j}^l)/2$, $W_w = (W_{i-1,j}^l + W_{i,j}^l)/2$, $W_n = (W_{i,j}^l + W_{i,j+1}^l)/2$, $W_s = (W_{i,j-1}^l + W_{i,j}^l)/2$.


We apply the conservative upwind scheme in each variable, indicating the active index (either $i$ or $j$) in our upwind notation:
\begin{align}
 &\frac{W_{i,j}^{l+1} - \Wlij}{\Delta t} + \frac{\upp{\alpha_e}{\Wlij}{i} - \upp{\alpha_w}{W_{i-1,j}^l}{i}}{\Delta x} + \frac{\upp{\beta_n}{\Wlij}{j} - \upp{\beta_s}{W_{i,j-1}^l}{j}}{\Delta y}  \label{eq:Wfd} \\
      &\qquad = K \bigg[\frac{W_e \left(W_{i+1,j}^l - \Wlij\right) - W_w \left(\Wlij - W_{i-1,j}^l\right)}{\Delta x^2}  \notag \\
      &\qquad\qquad\qquad + \frac{W_n \left(W_{i,j+1}^l - \Wlij\right) - W_s \left(\Wlij - W_{i,j-1}^l\right)}{\Delta y^2}\bigg] + \Phi_{ij}. \notag
\end{align}
Because of the first-order upwinding this scheme has $O(\Delta t^1 + \Delta x^1 + \Delta y^1)$ truncation error.




\small
\bibliography{ice_bib}  % generally requires link to pism/doc/ice_bib.bib
\bibliographystyle{agu}

\appendix

\section{Positivity and stability of the numerical scheme}

The explicit numerical scheme \eqref{eq:Wfd} for the model PDE \eqref{eq:adeqn} is sufficiently simple so that we can analyze its properties.  First, the scheme is stable in the sense that if sufficient conditions relating to the advective and diffusive time scales are satisfied then wiggles cannot grow.  Specifically, in this appendix we sketch a maximum principle argument for stability \citep{MortonMayers}.  On the other hand, as a closely-related statement, we note that the scheme positivity-preserving: if the water input $\Phi$ is nonnegative and the discrete water thicknesses $\Wlij$ are also nonnegative at step $t_l$ then, under the same sufficient stability conditions, the discrete water thicknesses at the next time step $W_{i,j}^{l+1}$ are nonnegative.

Let $\nu_x = \Delta t/\Delta x$, $\nu_y = \Delta t/\Delta y$, $\mu_x = K \Delta t / (\Delta x)^2$, and $\mu_y = K \Delta t / (\Delta y)^2$.  In what follows we consider only the case where all of the discrete velocities at the middle of the cell edges are nonnegative: $\alpha_e\ge 0$, $\alpha_w\ge 0$, $\beta_n\ge 0$, $\beta_s\ge 0$.  The many other cases, where these velocity components have various signs, can be handled by similar special-case arguments like the present one, and these other cases are left as a standard exercise for the reader \citep{MortonMayers}.

Now we rewrite \eqref{eq:Wfd} as a computation of the next value $W_{i,j}^{l+1}$, and collect terms:
\begin{align*}
 W_{i,j}^{l+1} &= \Wlij - \nu_x \left(\alpha_e \Wlij - \alpha_w W_{i-1,j}^l\right) - \nu_y \left(\beta_n \Wlij - \beta_s W_{i,j-1}^l\right)  \\
      &\qquad + \mu_x \left[W_e \left(W_{i+1,j}^l - \Wlij\right) - W_w \left(\Wlij - W_{i-1,j}^l\right)\right]  \\
      &\qquad + \mu_y \left[W_n \left(W_{i,j+1}^l - \Wlij\right) - W_s \left(\Wlij - W_{i,j-1}^l\right)\right] + \Delta t \Phi_{ij} \\
      &= (\nu_x \alpha_w + \mu_x W_w) W_{i-1,j}^l + (\mu_x W_e) W_{i+1,j}^l + (\nu_y \beta_s + \mu_y W_s) W_{i,j-1}^l + (\mu_y W_n) W_{i,j+1}^l \\
      &\qquad + \Big[1 - \nu_x \alpha_e - \nu_y \beta_n - \mu_x (W_e + W_w) - \mu_y (W_n + W_s)\Big] \Wlij + \Delta t \Phi_{ij}.
\end{align*}
The new value is a linear combination of the old values, plus a source term:
   $$W_{i,j}^{l+1} = A W_{i-1,j}^l + B W_{i+1,j}^l + C W_{i,j-1}^l + D W_{i,j+1}^l + E \Wlij + \Delta t \Phi_{ij}.$$
Because of our assumption about nonnegative velocities, and assuming $\Wlij \ge 0$ for all $i,j$, coefficients $A,B,C,D$ are all nonnegative, and only $E$ could be negative, depending on the time step.  In fact, let us state for simplicity a sufficient condition based on an equal split between advective and diffusive parts; let us assume
	$$\nu_x \alpha_e + \nu_y \beta_n = \Delta t \left(\frac{\alpha_e}{\Delta x} + \frac{\beta_n}{\Delta y}\right) \le \frac{1}{2},$$
and
    $$\mu_x (W_e + W_w) + \mu_y (W_n + W_s) = \Delta t \left(\frac{K(W_e + W_w)}{\Delta x^2} + \frac{K(W_n + W_s)}{\Delta y^2}\right) \le \frac{1}{2}.$$
FIXME: to complete

\end{document}
