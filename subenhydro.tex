\documentclass[11pt,final]{amsart}
%prepared in AMSLaTeX, under LaTeX2e

\usepackage[total={6.2in,9.0in},top=1.2in,left=1.1in]{geometry}

\usepackage{natbib}

\usepackage{amssymb,alltt,verbatim,xspace,fancyvrb,color,empheq}
\usepackage{palatino}
\usepackage[sc]{mathpazo}
\usepackage[T1]{fontenc}

% check if we are compiling under latex or pdflatex
\ifx\pdftexversion\undefined
  \usepackage[final,dvips]{graphicx}
\else
  \usepackage[final,pdftex]{graphicx}
\fi

% hyperref should be the last package we load
\usepackage[pdftex,
                colorlinks=true,
                plainpages=false, % only if colorlinks=true
                linkcolor=blue,   % only if colorlinks=true
                citecolor=black,   % only if colorlinks=true
                urlcolor=magenta     % only if colorlinks=true
]{hyperref}

\newcommand{\normalspacing}{\renewcommand{\baselinestretch}{1.05}\tiny\normalsize}
\newcommand{\tablespacing}{\renewcommand{\baselinestretch}{1.0}\tiny\normalsize}
\normalspacing

\definecolor{myblue}{rgb}{.8, .8, 1}

\newcommand*\mybluebox[1]{%
\colorbox{myblue}{\hspace{1em}#1\hspace{1em}}}

\newcommand*\myredbox[1]{%
\colorbox{red}{\hspace{1em}#1\hspace{1em}}}

% math macros
\newcommand\bv{\mathbf{v}}
\newcommand\bV{\mathbf{V}}
\newcommand\bq{\mathbf{q}}
\newcommand\bQ{\mathbf{Q}}

\newcommand\CC{\mathbb{C}}
\newcommand{\DDt}[1]{\ensuremath{\frac{d #1}{d t}}}
\newcommand{\ddt}[1]{\ensuremath{\frac{\partial #1}{\partial t}}}
\newcommand{\ddx}[1]{\ensuremath{\frac{\partial #1}{\partial x}}}
\newcommand{\ddy}[1]{\ensuremath{\frac{\partial #1}{\partial y}}}
\newcommand{\ddxp}[1]{\ensuremath{\frac{\partial #1}{\partial x'}}}
\newcommand{\ddz}[1]{\ensuremath{\frac{\partial #1}{\partial z}}}
\newcommand{\ddxx}[1]{\ensuremath{\frac{\partial^2 #1}{\partial x^2}}}
\newcommand{\ddyy}[1]{\ensuremath{\frac{\partial^2 #1}{\partial y^2}}}
\newcommand{\ddxy}[1]{\ensuremath{\frac{\partial^2 #1}{\partial x \partial y}}}
\newcommand{\ddzz}[1]{\ensuremath{\frac{\partial^2 #1}{\partial z^2}}}
\newcommand{\Div}{\nabla\cdot}
\newcommand\eps{\epsilon}
\newcommand{\grad}{\nabla}
\newcommand{\ihat}{\mathbf{i}}
\newcommand{\ip}[2]{\ensuremath{\left<#1,#2\right>}}
\newcommand{\jhat}{\mathbf{j}}
\newcommand{\khat}{\mathbf{k}}
\newcommand{\nhat}{\mathbf{n}}
\newcommand\lam{\lambda}
\newcommand\lap{\triangle}
\newcommand\Matlab{\textsc{Matlab}\xspace}
\newcommand\RR{\mathbb{R}}
\newcommand\vf{\varphi}

\newcommand{\Wlij}{W^l_{i,j}}
\newcommand{\Wij}{W_{i,j}}
\newcommand{\Plij}{P^l_{i,j}}
\newcommand{\Pij}{P_{i,j}}
\newcommand{\Ylij}{Y^l_{i,j}}
\newcommand{\Yij}{Y_{i,j}}
\newcommand{\upp}[3]{\big<#1\big|_{#3}\,#2\big>}

\newcommand{\Nbreen}{Nordenski\"oldbreen\xspace}

\newcommand{\citeapos}[1]{\citeauthor{#1}'s [\citeyear{#1}]}


\title[]{A distributed model of subglacial and englacial hydrology \\ in tidewater glaciers and ice sheets}

\author[]{Ed Bueler and Ward van Pelt}


\begin{document}
\graphicspath{{figs/}}

\scriptsize \hfill \today \normalsize
\vspace{0.5in}

\maketitle
\thispagestyle{empty}

\setcounter{tocdepth}{1}
\tableofcontents

\section{Introduction}

Any reasonable dynamical model of the liquid water underneath and within a glacier or ice sheet has at least these two elements: the mass of the water is conserved and the water flows from high to low hydraulic potential \citep{Clarke05}.  Beyond that there are many variations considered in the existing literature.  Physical processes may control the geometry of the linked cavities \citep{Kamb1987} or conduits (channels) \citep{Nye1976} in which the water moves.  These processes may include the opening of cavities by sliding of the overlying ice past bedrock bumps (i.e.~cavitation), melt on the walls of cavities and conduits, and the closure of cavities and conduits by creep \citep{Hewitt2011}.  Water could be exchanged with a porous englacial system \citep{Bartholomausetal2011} or it could be stored in a porous till \citep{Tulaczyketal2000b}.  While the pressure and amount of water in conduits could evolve by physical processes, the existing theory of conduits apparently requires their locations to be fixed a priori \citep{PimentelFlowers2011,Schoofmeltsupply}.

This paper describes a model for distributed systems of linked subglacial cavities, with connection to additional englacial storage.  The cavities open by sliding of the ice and wall melt.  They close by ice creep.  These physical processes combine to determine the relationship between water amount and water pressure \citep{Schoofetal2012}.  Conduits are not, however, included in the model.
%FIXME:?  Wall melt is included as a contribution to the mass conservation equation but not in the cavity evolution equation.

Pressure is determined non-locally over each connected component of the hydrological system, and no prior functional relation between subglacial water amount and pressure is assumed in our model \citep[compare][]{FlowersClarke2002_theory}.  The connection to a macroporous englacial system implies that the subglacial water pressure solves an equation which is a parabolic approximation of the distributed pressure equation given in elliptic variational inequality form by \cite{Schoofetal2012}.  Unlike that model, in which a variational inequality for pressure must be solved at each time step, in this paper the connection to englacial storage allows us to avoid solving an instantaneous distributed balance, thus easing implementation and parallelization.

We start by considering basic physical principles.  We derive a fundamental advection-diffusion form of the mass conservation equation.  Then we consider cavity evolution and alternative closures which can determine the pressure.  The equations which apply in steady state are stated, and we see that pressure is function of subglacial water amount in steady state.  In the steady radial case we compute an exact solution for subglacial water amount and pressure, a useful tool for verification.  Then we present a parallel numerical implementation within the Parallel Ice Sheet Model \citep{pism-user-manual}, with particular attention to time step restrictions and the treatment of advection.  The first numerical results are for the verification case; we show convergence under grid refinement.  After that we show results for the Nordenskioldbreen tidewater glacier in Svalbard.


\section{Elements of a sub- and en-glacial hydrology continuum model}

\subsection*{Mass conservation}  We assume that liquid water is incompressible and of constant density.  Thus the thickness of a water layer, denoted by $W(t,x,y)$, tells us its mass.  Our statement of mass conservation (below) will describe the evolution of $W$.  Choosing to model subglacial hydrology using a water thickness as the state variable is not a significant restriction on the physics because specific subglacial physics comes from choosing a form for the water flux and a closure for the pressure.  This thickness is only meaningful compared to observations if it is regarded as an average over a horizontal scale of tens to thousands of meters \citep{FlowersClarke2002_theory}.

Additionally we allow water to be stored englacially in a system of crevasses, cracks, veins, and/or moulins of unspecified morphology but with overall macro porosity $\phi$ \citep{Bartholomausetal2011}.  Throughout this paper the case $\phi=0$ which may apply to cold glaciers and the interior of ice sheets will also be considered.  (In this case there is no englacial storage.  The extension that $\phi$ varies in time and map-plane location ($\phi(t,x,y)$) is natural but is not considered here.)  The mass of englacial water can be tracked with the elevation above the bed $z_{en}(t,x,y)$ for the top of the (notional) ``water table'' in the porous glacier.  In a column of the glacier with cross-sectional area $\Delta x \times \Delta y = \Delta a$ the volume of water is then $\phi\, \Delta a \, z_{en}$ \citep{Bartholomausetal2011}.  We can just as easily track the water by an effective thickness for the englacial water, however.  We define
\begin{equation}
W_{en}(t,x,y) = \frac{\phi\, \Delta a\, z_{en}}{\Delta a} = \phi\, z_{en}. \label{eq:definezen}
\end{equation}
The total effective thickness of the water at map-plane location $(x,y)$ and time $t$ is $W+W_{en}$.  We always assume the water thicknesses are nonnegative: $W \ge 0$, $W_{en} \ge 0$.

We separate the water sources between the melt on the lower surface of the glacier versus supraglacial drainage.  Let $m_{\text{wall}}$ be the rate at which the cavity walls melt or refreeze.  Let $m_{\text{drain}}$ be the rate at which surface runoff drains to the subglacier layer.  The total input to the subglacial layer is denoted
\begin{equation}
m = m_{\text{wall}} + m_{\text{drain}}. \label{eq:totalinput}
\end{equation}

Suppose $\bq$ is the (vector) water flux (units $\text{m}^2\,\text{s}^{-1}$).  We will not consider the possibility of lateral englacial transport; the water flux $\bq$ is concentrated within a subglacial layer.  In two spatial dimensions the mass conservation equation is \citep{Clarke05}
\begin{equation} \label{eq:conserve}
\frac{\partial (W + W_{en})}{\partial t} + \Div \bq = \frac{m}{\rho_w}
\end{equation}
where $\rho_w$ is the density of fresh liquid water.  Coupling between the subglacial and englacial systems is addressed in Section \ref{sec:closures}.

\subsection*{Hydraulic potential and Darcy flow}  The water flux $\bq$ in equation \eqref{eq:conserve} is related to the gradient of a hydraulic potential $\psi(t,x,y)$ for the subglacial water.  Water moves only laterally in the subglacial layer, and there is no lateral transport in the englacial network.  The flux $\bq$ is the vertically-integrated fluid flux in the subglacial layer.  It combines the actual subglacial water pressure $P(t,x,y)$ and the gravitational potential of the top of the water layer,
\begin{equation} \label{eq:potential}
\psi = P + \rho_w g\, (b+W).
\end{equation}
Here $\rho_w$ is the water density ($\text{kg}\,\text{m}^{-3}$), $g$ is the acceleration of gravity ($\text{m}\,\text{s}^{-2}$), and $z=b(x,y)$ ($\text{m}$) is the bedrock elevation, which for simplicity is time-independent.

We have added the term ``$\rho_w g W$'' to the standard hydraulic potential formula $\psi_0 = P + \rho_w g b$ \citep[for example]{Clarke05} for the reason that is given by \cite{Hewittetal2012}, namely that it is correct.  Differences in the potential at the \emph{top} of the subglacial water layer determine the driving potential gradient, though of course such differences may be small, in which case ignoring them may do no harm.  The  ``$\rho_w g W$'' term is most important when considering local minima of the hydraulic potential, where subglacial lakes of finite (not infinitesimal) extent and finite (not infinite) depth should form if water is available.  As we will see, this term makes the mass conservation equation diffusive when the water depth becomes substantial ($W\gg 1$), as it would be in a subglacial lake, and this term keeps the modeled lakes from being singularities of the water thickness field.

Water flows from high to low hydraulic potential.  The simplest applicable expression of this property is a Darcy flux model for a water sheet following \cite{Clarke05}, namely
\begin{equation}  \label{eq:fluxearly}
\bq = - \frac{k}{\rho_w g}\,W \grad \psi
\end{equation}
Here $k \ge 0$ is the effective hydraulic conductivity ($\text{m}\,\text{s}^{-1}$).  More generally \cite{Schoofetal2012} suggests
\begin{equation}  \label{eq:flux}
\bq = - \, \frac{k}{\rho_w g}\, W^\alpha\, |\grad \psi|^{\beta-2} \grad \psi
\end{equation}
for $\alpha \ge 1$, $\beta>1$, and some coefficient $k>0$ with units that depend on $\alpha$ and $\beta$.

Power-law form \eqref{eq:flux} is justified as an instance of a Manning or Darcy-Weisbach law \citep{Schoofetal2012}.  Among others, \cite{Clarke05} suggests $\alpha=1$ and $\beta=2$, to give \eqref{eq:fluxearly} above, \cite{Hewitt2011} suggests $\alpha=3$ and $\beta = 2$, and \cite{Hewittetal2012} suggest $\alpha=5/4$ and $\beta=3/2$.  The current paper will implement law \eqref{eq:flux} generally.

\subsection*{Overburden pressure and pressure bounds}  The ice is a viscous fluid which has a stress field of its own.  The basal value of the downward normal stress is traditionally called the \emph{overburden pressure}, which we denote by $P_o$.  In this paper we make the shallow approximation that it is hydrostatic \citep{GreveBlatter2009}:
\begin{equation} \label{eq:hydrostatic}
  P_o = \rho_i g H = \rho_i g (h-b).
\end{equation}
Here $\rho_i$ is the density of ice ($\text{kg}\,\text{m}^{-3}$), $H$ is the ice thickness (m), and $h$ is the ice upper surface elevation (m).  Now define the \emph{effective pressure}
\begin{equation}
N = P_o - P\label{eq:effective}
\end{equation}
This quantity measures how much of the ice load is carried by the mineral (till or bedrock) base (as opposed to how much is carried by pressurized subglacial water).

Because $P$ is nonnegative, and because the condition $P>P_o$ is presumed to cause the ice to lift and thus quickly lower the pressure back to overburden $P=P_o$ \citep{Schoofetal2012}, it follows that the pressure solution is subject to inequalities
\begin{equation}
0 \le P \le P_o. \label{eq:bounds}
\end{equation}
Though extreme cases can occur, such as where the ice is forced upward by a negative effective pressure \citep{Schoofetal2012}, our theory does not model cases violating bounds \eqref{eq:bounds}.

\subsection*{Advection-diffusion decomposition}  Combining \eqref{eq:potential} and \eqref{eq:flux}, and separating the term proportional to $\grad W$, we get the flux expression
\begin{align}
  \bq &= - \frac{k}{\rho_w g}  W^\alpha \left|\grad \left(P + \rho_w g (b+W) \right)\right|^{b-2} \grad \left(P + \rho_w g b\right)  \label{eq:firstfluxdecomp} \\
      &\qquad - k W^\alpha \left|\grad \left(P + \rho_w g (b+W) \right)\right|^{b-2} \grad W. \notag
\end{align}
This separation identifies a part of the flux, the second term, which is proportional to the gradient of the water thickness $\grad W$.  It acts diffusively in the mass conservation equation.  Because the subglacial water pressure generally scales with the ice overburden pressure $\rho_i g H$, the first flux term in \eqref{eq:firstfluxdecomp} will, however, dominate under the condition $|\grad H| \gg |\grad W|$ (or when $|\grad b| \gg |\grad W|$).

Decomposed flux \eqref{eq:firstfluxdecomp} describes a transport process which combines a velocity field which varies in space and time (defined carefully below) and a second term which acts diffusively.  We will construct our conservative numerical scheme based on this understanding of how the flux is decomposed.  We will see later that in near-steady-state circumstances the part of the transport velocity which is proportional to $\grad P$ is also actually \emph{diffusive} in the mass conservation equation.  In conditions far from steady state, however, the direction of $\grad P$ is different from from the direction $\grad W$.

To simplify the model slightly a small water thickness approximation is made
\begin{equation}
\left|\grad \left(P + \rho_w g (b+W) \right)\right| \approx \left|\grad \left(P + \rho_w g b \right)\right|.  \label{eq:Wsmall}
\end{equation}
Thus $W\approx 0$ in the nonlinear scalar coefficients in \eqref{eq:firstfluxdecomp}.  This makes no change in the $\beta=2$ case.  Now define the nonlinear effective hydraulic conductivity
\begin{equation}
K = K(W,\grad P,\grad b) = k W^{\alpha-1} \left|\grad(P+\rho_w g b)\right|^{\beta - 2}. \label{eq:Kdefine}
\end{equation}
Note $K = k$ in the $\alpha=1$ and $\beta=2$ case covered by \eqref{eq:fluxearly}.  In terms of this nonlinear conductivity we write the transporting velocity field
\begin{equation} \label{eq:vexpression}
  \bV = - \frac{K}{\rho_w g}\, \grad \left(P + \rho_w g b\right).
\end{equation}
The nonlinear diffusivity is $D = K\,W$.  With this notation, equation \eqref{eq:firstfluxdecomp} is replaced by a cleaner decomposition of the flux,
\begin{equation} \label{eq:qexpression}
  \bq = \bV\, W - D \grad W = \bV\, W - K\,W \grad W.
\end{equation}
From equations \eqref{eq:conserve} and \eqref{eq:qexpression} we derive an advection-diffusion equation \citep{HundsdorferVerwer2010} for the evolution of the water layer thickness:
\begin{equation} \label{eq:adeqn}
  \frac{\partial (W+W_{en})}{\partial t} = - \Div\left(\bV\, W\right) + \Div \left(K\,W \grad W\right) + \frac{m}{\rho_w}.
\end{equation}

There are distinct numerical schemes (section \ref{sec:num}) for the advection term $\Div\left(\bV\, W\right)$ and the diffusion term $\Div \left(K\,W \grad W\right)$ in \eqref{eq:adeqn}.  These different schemes impose time step restrictions of different magnitudes.  We will see in practice that equation \eqref{eq:adeqn} is advection-dominated in the sense that $|\bV W| \gg |K\,W \grad W|$.  However, in near-steady conditions the velocity $\bV$ can be almost proportional to $-\grad W$ so there is no longer a clean separation of advection versus diffusion.  This implies that the numerical schemes for advection and diffusion must be tested in combination.  We measure the convergence behavior of the combined schemes in section \ref{sec:results}.

As is well known \citep{Clarke05}, the flux $\bq$ depends significantly on the ice surface slope because the ice overburden pressure dominates the subglacial water pressure.  Therefore the gradient of the hydraulic potential frequently follows the ice surface gradient.  The pressure model which we construct and use in this paper also generates pressure fields with this property in some circumstances.  This is not such an obvious property, however, in a model like ours which depends on physical mechanisms for the opening and closing of cavities.  Clearly the flux $\bq$ also depends on the bedrock slope $\grad b$.  Because bedrock elevation is rough (irregular) data in practice, this part of the velocity $\bV$ may not be very smooth, and it may be large in magnitude.

In the rest of this paper we use both of the equivalent forms \eqref{eq:flux} and \eqref{eq:qexpression} for the flux $\bq$.  The former emphasizes the relation between flow and pressure while the latter is essential in numerical considerations for the mass continuity equation.

%FIXME:  revive text on enthalpy and wall melt and dissipation heating


\subsection*{Capacity of the distributed system}  Suppose $\bv_b$ is the ice basal velocity (i.e.~the sliding velocity).  The evolution of the area-averaged thickness, also called the bed separation \citep{Bartholomausetal2011}, of the cavities in a distributed linked-cavity system \citep{Schoofetal2012} can be described as the sum of opening by cavitation and closure by creep \citep{Hewitt2011}.  Let $Y$ (m) denote that bed separation, so that
\begin{equation}
\frac{\partial Y}{\partial t} = \mathcal{O}(|\bv_b|,Y) - \mathcal{C}(N,Y). \label{eq:hewittcapacity}
\end{equation}
Exactly as in \cite{Schoofetal2012} we choose an opening term
\begin{equation}
 \mathcal{O}(|\bv_b|,Y) = c_1 |\bv_b| (W_r - Y)_+. \label{eq:openingform}
\end{equation}
Here $W_r$ is a maximum roughness scale of the basal topography and $c_1$ is a constant; both of these constants must be indirectly constrained by observations in practice (see Section \ref{sec:results}).  Also, we denote $X_+= \max\{0,X\}$ for a real number $X$.  We choose a form for the closing term which represents a regularization of that in the literature \citep{Hewitt2011,Schoofmeltsupply,Schoofetal2012}:
\begin{equation}
\mathcal{C}(N,Y) = c_2 A N^3 (Y+Y_0). \label{eq:closingform}
\end{equation}
Here $A$ is the ice softness, $c_2$ is a constant which must be constrained by observations (see below), and we have used Glen exponent $n=3$ for concreteness.  The regularization constant $Y_0>0$ (m) is taken to be small relative to typical values of the bed separation $Y$; the role of this constant is addressed below.

Equation \eqref{eq:hewittcapacity} describes the evolution of the upper surface of the subglacial cavity.  The first term (cavitation) is always nonnegative (i.e.~causes opening) if we use \eqref{eq:openingform}, but it is only positive where the bed separation is less than the roughness scale ($Y<W_r$).  The second term always represents closing because our modeled pressure will satisfy bounds \eqref{eq:bounds} so that $0\le N \le P_o$.  The opening and closing terms \eqref{eq:openingform} and \eqref{eq:closingform} satisfy the inequalities in \cite{Schoofetal2012}, namely equations (2.5)--(2.7).

The physical intuition behind a pressure model which combines \eqref{eq:hewittcapacity} with a Darcy flux relation like \eqref{eq:flux} and mass conservation \eqref{eq:conserve} is as follows.  If the cavity is larger than connected water sources can fill then the pressure should be lowered.  This pressure drop encourages inflow and it by \eqref{eq:closingform} accelerates cavity closure.  Conversely, if local water sources exceed capacity then the increased pressure should push water out of the area and creep closure should be reduced.  This intuition requires a pressure closure, addressed in the next section.

To explain the role of regularization in the closing term \eqref{eq:closingform}, consider steady states of any model using \eqref{eq:hewittcapacity}.  These steady systems have a functional relationship between thickness $Y$ and effective pressure $N$ which comes from solving the steady condition for $N$:
\begin{equation}
\mathcal{O}(|\bv_b|,Y) = \mathcal{C}(N,Y). \label{eq:hewittsteady}
\end{equation}
The implicit function theorem says that if $\partial\mathcal{C}/\partial N$ is nonzero for a given $Y\ge 0$ then the effective pressure is determined: $N=N(Y)$.  Such applies for all $Y\ge 0$ because $Y_0>0$.  By contrast, in the unregularized case $Y_0=0$ we have $\partial\mathcal{C}/\partial N=0$ at $Y=0$ so that the steady effective pressure is indeterminant.  The numerical model we implement runs fine if $Y_0=0$ but its steady state is better understood if $Y_0>0$.

If $Y<W_r$ and $Y_0>0$ then a unique value $N(Y)>0$ is determined by \eqref{eq:hewittsteady}.  It may not, however, satisfy the bound $N(Y) \le P_o$ implied by \eqref{eq:bounds}.  That is, the sliding speed $|\bv_b|$ may be sufficiently large, and the bed separation $Y$ may be sufficiently below $W_r$, so that the cavitation opening rate exceeds the closing rate for any effective pressure $N$ satisfying $N\le P_o$.  In fact, for given values of $P_o$ and $|\bv_b|$, steady state equation \eqref{eq:hewittsteady} can only hold for a given $Y$ if
\begin{equation}
c_1 |\bv_b| (W_r - Y)_+ \le c_2 A P_o^3 (Y+Y_0). \label{eq:steadyOCbound}
\end{equation}
Condition \eqref{eq:steadyOCbound} must hold for the steady values of $Y$ over the entire domain.  If inequality \eqref{eq:steadyOCbound} applies then equation \eqref{eq:hewittsteady} determines a valid steady state effective pressure $N(Y)$ satisfying $0\le N(Y) \le P_o$, over the whole domain, and thus a water pressure $P(Y)$ satisfying bounds \eqref{eq:bounds}.


\section{Closures to determine pressure} \label{sec:closures}

At this point we do not know how to compute the water pressure $P$ or the effective pressure $N$ given values of the other data of the problem, namely $b$, $H$, $\Phi$, $|\bv_b|$, $W$, and $Y$.  The apparent state variables of the model so far are $W$, $Y$, and $P$, but the above equations can be simplified only to two partial differential equations in these three unknowns.   A closure is needed.

\subsection*{Closures without cavity evolution}  To start we briefly consider three simple closures which appear in the literature but which do not use cavity evolution equation \eqref{eq:hewittcapacity} or similar physics.  The resulting simplified models emerge as limiting cases of our more complete theory in or near steady state conditions.  These closures are distinguished both by their physical motivation and by the mathematical form they imply for the mass conservation equation.  For simplicity we present each of these simplified closures as subglacial-only, thus setting $W_{en}=0$ in equation \eqref{eq:conserve}, and we state only the constant conductivity case ($\alpha=1$ and $\beta=2$ in equation \eqref{eq:flux}).

\renewcommand{\labelenumi}{\textbf{\Roman{enumi}.}}
\begin{enumerate}
\item The assumption that the pressure is equal to the overburden pressure is a simple closure:
\begin{equation}
P = P_o.\label{eq:Pisoverburden}
\end{equation}
This model is sometimes used for routing subglacial water under ice sheets so as to identify subglacial lake locations \citep[for example]{Siegertetal2009}.  Straightforward calculations using equations \eqref{eq:conserve}, \eqref{eq:flux}, and \eqref{eq:Pisoverburden} show that the advection-diffusion form \eqref{eq:adeqn} has an ice-geometry-determined velocity.  Specifically,
\begin{gather}
  \frac{\partial W}{\partial t} = - \Div\left(\tilde\bV\, W\right) + \Div\left(k \,W\, \grad W\right) + \frac{m}{\rho_w}   \label{eq:PisoverConservation} \\
  \phantom{sdklj ljsdf} \text{where} \qquad \tilde\bV = - k \left[\frac{\rho_i}{\rho_w} \grad h + \left(1-\frac{\rho_i}{\rho_w}\right) \grad b\right]. \notag
\end{gather}

Because the approximation $W\ll H$ is usually accepted, so that the hydraulic potential is defined in a way that is insensitive to the water layer thickness, i.e.~$\psi = P_o + \rho_w g b$ \citep{Siegertetal2009}, in fact the diffusion term ``$\Div\left(k \,W\, \grad W\right)$'' on the right of \eqref{eq:PisoverConservation} is usually not included.  With this common simplification, \eqref{eq:PisoverConservation} becomes a pure advection with a velocity $\tilde\bV$ which is independent of $W$.  Therefore equation \eqref{eq:PisoverConservation} possesses characteristic curves \citep{Evans} which are \emph{a priori} known trajectories of the water flow.  The more complete models we consider in this paper do not have such characteristic curves.  They have a flux which depends directly and indirectly on the gradient of the water amount $W$, the quantity which is being advected.

The closure \textbf{I} continuum model in the form stated above, equation \eqref{eq:PisoverConservation} \emph{with} the diffusion term, is well-posed for positive initial and boundary values on $W$ \citep[compare][]{Hewittetal2012}.  Continuum solutions have finite water layer thickness at all times.  By contrast, equation \eqref{eq:PisoverConservation} without the diffusion term, which is the way it usually appears in the literature, will exhibit continuum solutions with infinite concentration at every location where the simplified potential $\psi = P_o + \rho_w g b$ has a minimum.  Such a model is not well-posed and its numerical implementation is badly-behaved under grid refinement.

\medskip

\item At an almost opposite extreme in terms of the mathematical form, one might close the model by assuming that the water pressure is locally determined by the amount of water.  \cite{FlowersClarke2002_theory} propose
\begin{equation}
P_{FC}(W) = P_o \left(\frac{W}{W_{\text{crit}}}\right)^{7/2}. \label{eq:PofWFC}
\end{equation}
For Trapridge glacier \cite{FlowersClarke2002_trapridge} use $W_{\text{crit}}=0.1$.  (Figure \ref{fig:psteady-vb} below illustrates this function.)  One obvious concern with form \eqref{eq:PofWFC} is that $P_{FC}(W)$ can be arbitrarily larger than overburden pressure for large amounts of water ($W \gg W_{\text{crit}}$), but \eqref{eq:PofWFC} is not intended to apply when $W$ is actually large, such as in subglacial lakes.  For general bedrock, from \eqref{eq:conserve}, \eqref{eq:flux}, and \eqref{eq:PofWFC} we get this equation with $\hat\bV = - k \grad b$:
\begin{equation}
  \frac{\partial W}{\partial t} = - \Div\left(\hat\bV\, W\right) + \Div \left(\frac{k\,W}{\rho_w g} \grad P_{FC}(W)\right) + \frac{m}{\rho_w}. \label{eq:PofWFCConservation}
\end{equation}
In the flat bedrock case we see that $\hat\bV=0$ so \eqref{eq:PofWFCConservation} is a nonlinear diffusion.  \cite{Schoofetal2012} observe that \eqref{eq:PofWFCConservation} generalizes the porous-medium equation $\partial W/\partial t = \Div \left(\grad (W^\gamma)\right)$ \citep{VazquezPME}.  The main idea in such a nonlinear diffusion, and the main concern when using closure \eqref{eq:PofWFC}, is that the direction of the flux is $-\grad W$.

\medskip

\item Another simple closure by \cite{BBssasliding} uses a pressure function comparable to \eqref{eq:PofWFC} but different in detail.  Pressure is proportional to water amount but it is capped at a fixed fraction of overburden:
\begin{equation}
P_{BB}(W) = \lambda\,P_o \min\left\{1,W/W_{\text{crit}}\right\}. \label{eq:PofWBB}
\end{equation}
(Figure \ref{fig:psteady-vb} below also illustrates this function.)  Values $\lambda=0.95$ and $W_{\text{crit}}=2$ m are used in \citep{BBssasliding}.  While this functional form is only motivated as a mechanism to model basal shear stress from a saturated till layer \citep{Tulaczyketal2000b}, the ``obvious concern'' mentioned above is absent.  The current paper seeks to improve the Parallel Ice Sheet Model (PISM) by replacing closure \eqref{eq:PofWBB} with one that is more physical, as we add mass conservation to the PISM hydrology submodel.
\end{enumerate}

In this paper we apply evolution model \eqref{eq:hewittcapacity} for the capacity of the distributed system, and thus we do not use any of the above closures \eqref{eq:Pisoverburden}, \eqref{eq:PofWFC}, or \eqref{eq:PofWBB}.  However, in steady state conditions our theory recovers a functional relation $P=P(W)$, as shown in Figure \ref{fig:psteady-vb}.  (The same can be said for the \cite{Schoofetal2012} theory.)  This relation is in a form which is neither a power-law like \eqref{eq:PofWFC} nor piecewise-linear like \eqref{eq:PofWBB}.  Under steady conditions where the ice sliding velocity is zero, our theory also recovers \eqref{eq:Pisoverburden} and \eqref{eq:PisoverConservation}.  Our theory uses advection-diffusion decomposition \eqref{eq:adeqn} and thus in that sense it also extends the ``routing'' model \eqref{eq:PisoverConservation}.  These connections are further exposed in section \ref{sec:steadyverif} below.

\subsection*{A closure relating subglacial pressure to amount of englacial water}  Recall that the englacial water amount is parameterized by an effective thickness $W_{en}=\phi z_{en}$ where $\phi$ is the macroporosity and $z_{en}$ is the height above the bed of the top of the englacial water table.

There is an obvious choice, which we adopt, for a relationship between the amount of englacial water (i.e.~$W_{en}$ or $z_{en}$) and the subglacial pressure $P$.  Namely, we assume that the englacial water pressure is hydrostatic \citep{Bartholomausetal2011}.  Furthermore we suppose that at the bottom of the glacier the englacial system is efficiently connected to the subglacial system, so that there is no pressure jump.  It follows that the subglacial pressure $P$ is equal to the hydrostatic pressure of the lowest englacial water:
\begin{equation}
P = \rho_w g z_{en} = \frac{\rho_w g}{\phi} W_{en}. \label{eq:Penhydrostatic}
\end{equation}
Thus the englacial system is modeled as a frictionless pressure gauge, a piezometer in the groundwater sense, on the subglacial system.  Of course, this ``gauge'' is distributed over the glacier area: the englacial system stores $W_{en}\, dA$ volume of water over each element $dA$ of map-plane area.

Equation \eqref{eq:Penhydrostatic} determines $W_{en}$ from the subglacial pressure $P$ in cases where the local total water amount (i.e.~$W+W_{en}$) is large; specifically $W_{en} = \phi P / (\rho_w g)$.  But we must be careful, for mass conservation, to only keep/put water in englacial storage when there is sufficient available water at the subglacial connection.  In practice we will update the combined quantity $W+W_{en}$ at time step using equation \eqref{eq:adeqn}.  Then we will decide at the new time how much of the water should be stored englacially.  In terms of the updated total amount $W_{tot}=W+W_{en}$ we choose the following continuous parameterization using the value $\tilde W_{en} = \phi P / (\rho_w g)$ from \eqref{eq:Penhydrostatic}:
\begin{equation}
W_{en} = F(W_{tot}) = \begin{cases}
             0,                       & 0 \le W_{tot} \le \tilde W_{en}, \\
             W_{tot} - \tilde W_{en}, & \tilde W_{en} < W_{tot} < 2 \tilde W_{en}, \\
             \tilde W_{en},           & W_{tot} \ge 2 \tilde W_{en}. \end{cases}
    \label{eq:WenFunctionWtot}
\end{equation}
This modifies \eqref{eq:Penhydrostatic} only in the small-water case.  For large water amounts equation \eqref{eq:Penhydrostatic} directly determines $W_{en}$.  It follows from equation \eqref{eq:WenFunctionWtot} that a positive amount of englacial water only exists in locations where there is subglacial water:
      $$\{W_{en}>0\} \subset \{W>0\}.$$


\subsection*{Full-cavity closure}  Equations  \eqref{eq:adeqn}, \eqref{eq:hewittcapacity}, and \eqref{eq:Penhydrostatic} do not yet determine the evolution of the major distributed variables $W$, $W_{en}$, $Y$ and $P$.  However, \emph{requiring the subglacial layer to be full of water} is a closure for the subglacial pressure $P$.  We will adopt this ``full-cavity closure'' in our model:
\begin{equation}
W = Y.\label{eq:strongclosure}
\end{equation}
The consequences of this closure are explored at some length by \cite{Schoofetal2012}, who describe the case where cavities are full as the ``normal pressure'' condition (e.g.~equation (4.13)).  The \cite{Schoofetal2012} model, which does not include englacial storage, has the model of the current paper as its $\phi\to 0$ limit.

Equation \eqref{eq:strongclosure} allows us to eliminate $Y$ as a state variable.  In fact, if $W=Y$ then equations \eqref{eq:conserve}, \eqref{eq:hewittcapacity}, and \eqref{eq:Penhydrostatic} combine to give
\begin{equation}
\mathcal{O}(|\bv_b|,W) - \mathcal{C}(N,W) + \frac{\phi}{\rho_w g}\frac{\partial P}{\partial t} + \Div \bq = \frac{m}{\rho_w}. \label{eq:initialformpressure}
\end{equation}
We can write this as an evolution equation for pressure $P$, with the flux $\bq$ written in terms of the hydraulic potential $\psi$ (see equation \eqref{eq:flux}):
\begin{equation}
\frac{\phi}{\rho_w g}\frac{\partial P}{\partial t} = \Div\left(\frac{k}{\rho_w g} W^\alpha |\grad \psi|^{\beta-2} \grad \psi\right) + \frac{m}{\rho_w} + \mathcal{C}(P_o-P,W) - \mathcal{O}(|\bv_b|,W). \label{eq:pressureequation}
\end{equation}

Because of the close relation between $P$ and $\psi$, we regard \eqref{eq:pressureequation} as a nonlinear parabolic equation for $P$.  Roughly-speaking, it is a diffusion for $P$ which is coupled to the advection-diffusion equation \eqref{eq:adeqn} for $W$.  In fact, now that $Y$ is eliminated, equations \eqref{eq:adeqn}, \eqref{eq:Penhydrostatic}, and \eqref{eq:pressureequation} can be identified as our major model equations for the model state variables $W$, $W_{en}$, and $P$.  (See the final model statement \eqref{eq:bluebox} below.)

Because the pressure solution $P$ is subject to bounds \eqref{eq:bounds}, pressure can not be regarded as a conserved quantity in this model.  Though equation \eqref{eq:pressureequation} is a kind of stress balance, a form of conservation of momentum, there is no obvious way to guarantee that we actually conserve momentum because the enforcement of bounds \eqref{eq:bounds} involves large and un-accounted forces \citep{Schoofetal2012}.  However, verifiable conservation of mass is important in the applications of a subglacial hydrology model.  For this reason we keep variable $W_{en}$, which otherwise could be eliminated using equation \eqref{eq:Penhydrostatic}.  This allows us to carefully consider the numerical conservation of the total water mass $W+W_{en}$ in section \ref{sec:num}.

Consider the $\phi=0$ case of equation \eqref{eq:pressureequation},
\begin{equation}
0 = \Div\left(\frac{k}{\rho_w g} W^\alpha |\grad \psi|^{\beta-2} \grad \psi\right) + \frac{m}{\rho_w} + \mathcal{C}(P_o-P,W) - \mathcal{O}(|\bv_b|,W). \label{eq:ellipticpressure}
\end{equation}
The connection to englacial storage has been removed in this \emph{elliptic} pressure equation.  It describes how pressures of different areas in the layer are related to each other, and to the layer geometry, at each instant by processes that act instantaneously.  Equation \eqref{eq:ellipticpressure} is the major pressure equation in \citep{Schoofetal2012}.  Specifically, the $\phi=0$ and $Y_0=0$ case of equation \eqref{eq:pressureequation} is the same as equation (2.12) or (4.17a) in \citep{Schoofetal2012}.

Though we do not use the $\phi\to 0$ limit of \eqref{eq:pressureequation}, namely \eqref{eq:ellipticpressure}, it does provide understanding of the structure of our mathematical model.  Specifically, \cite{Schoofetal2012} show that the time-independent mathematical problem encompassing \eqref{eq:ellipticpressure}, constraints \eqref{eq:bounds}, and appropriate pressure boundary conditions can be written as an elliptic variational inequality \citep{KinderlehrerStampacchia}.  This same kind of variational inequality problem would need to be solved at each step of an implicit time-stepping numerical implementation of the model we develop here, because we do require bounds \eqref{eq:bounds}.  The numerical analysis of such problems, which also appear in other glaciological free boundary contexts \citep{SchoofStream,JouvetBueler2012}, can be addressed with a finite element approach \citep{Ciarlet}.

\subsection*{Regularization to reduce stiffness}  As has been noted by other authors \citep{Clarke2003,Schoofetal2012}, the differential equations for subglacial hydrology are \emph{stiff} \citep{AscherPetzold}.  Primarily  this means that the timescales for pressure evolution are short compared to the timescales of water movement.  This contrast can be stated in terms of the relative speeds of pressure waves and water transport \citep[Appendix A, for example]{Clarke2003}.

This stiffness is not mysterious---it follows from the incompressibility of water and the non-distensibility (i.e.~hardness) of the ice and bedrock.  \cite{Clarke2003} addresses it by including in his subglacial water equation a relaxation (damping) parameter  ``$\beta$'' which is the (small) compressibility of water.  He uses a value for $\beta$ which is more than two orders of magnitude larger than the physical value.  Comparable relaxation can be achieved through modeled distensibility of the ice surrounding the subglacial water \citep[Appendix A]{Clarke2003}.  By contrast, the \cite{Schoofetal2012} theory is the ``infinitely stiff'' limit where the pressure equation \eqref{eq:ellipticpressure} has differential-algebraic character because it includes no time derivative \citep{AscherPetzold}.

\citeapos{Clarke2003} parameter $\beta$ appears in his equation exactly as the englacial porosity $\phi$ appears in equation \eqref{eq:pressureequation} here, multiplying the pressure time derivative.  The effective physical porosity $\phi$ may be small, however, say $\sim 10^{-3}$ for temperate glaciers but orders of magnitude smaller for cold ice sheets, so \eqref{eq:pressureequation} is still quite stiff.  For many modeling purposes the time-evolution of pressure can, however, be slowed below the rates given by $\phi$ without loss of usefulness of the model results.  Thus we propose that for computation a regularization $\phi \to \phi+\phi_0$ be allowed, even with $\phi_0$ larger than $\phi$, as desired by the model user and as appropriate to the model purpose.  (See numerical results in section \ref{sec:results}.)  Thus we write the finalized pressure equation in our model:
\begin{equation}
\frac{\phi+\phi_0}{\rho_w g} \frac{\partial P}{\partial t} = \Div \left( \frac{k}{\rho_w g} W^\alpha |\grad \psi|^{\beta-2} \grad \psi \right) + \frac{m}{\rho_w} + \mathcal{C}(P_o-P,W) - \mathcal{O}(|\bv_b|,W). \label{eq:regpressureequation}
\end{equation}
The model runs, though more slowly, with $\phi_0=0$.  See section \ref{sec:num} for a quantitative analysis of stiffness on the numerical implementation.


\section{New subglacial hydrology model} \label{sec:newmodel}

\subsection*{Summary of equations and symbols}  The goal of the current work is the selection and testing of an easily-implementable subglacial hydrology model which can be coupled to an existing three-dimensional ice dynamics model, and which can be parallelized.  The remainder of the paper will demonstrate that we have succeeded in these goals for the model which combines equations \eqref{eq:adeqn}, \eqref{eq:Penhydrostatic}, and \eqref{eq:regpressureequation}.  As will be described in section \ref{sec:num}, we have implemented an explicit, adaptive time-stepping numerical scheme for these equations in the Parallel Ice Sheet Model.  The complete set of equations for this new mathematical model also includes the important bounds on $W$ and $P$:
\begin{empheq}[box=\mybluebox]{align}
W &\ge 0, \notag \\
0 &\le P \le P_o, \notag \\
\psi &= P + \rho_w g (b + W), \notag \\
K &= k W^{\alpha-1} \left|\grad(P+\rho_w g b)\right|^{\beta-2}, \notag \\
\bV &= - \frac{K}{\rho_w g} \grad\left(P + \rho_w g b\right), \notag \\
\frac{\phi+\phi_0}{\rho_w g} \frac{\partial P}{\partial t} &= \Div \left(  \frac{k}{\rho_w g} W^\alpha |\grad \psi|^{\beta-2} \grad \psi \right) + \frac{m}{\rho_w} \label{eq:bluebox} \\
  &\qquad \qquad + c_2 A (P_o - P)^3 (W+Y_0) - c_1 |\bv_b| (W_r - W)_+, \notag \\
 \frac{\partial (W + W_{en})}{\partial t} &= - \Div\left(\bV\, W\right) + \Div \left(K\,W \grad W\right) + \frac{m}{\rho_w}, \notag \\
 W_{en} &= \frac{\phi}{\rho_w g} P. \notag
\end{empheq}
Formula \eqref{eq:WenFunctionWtot} replaces the last equation if the total water amount $W + W_{en}$ is small.  The total water input $m$ is the sum given by \eqref{eq:totalinput}.

\begin{table}[ht]
\caption{Symbols for functions used in sub- and en-glacial hydrology model \eqref{eq:bluebox}.}
\begin{tabular}{l|l}
\hline
\emph{state functions} & \begin{tabular}{ll}
        $W$ & subglacial water layer thickness \\
        $W_{en}$ & effective englacial water layer thickness \\
        $P$ & subglacial water pressure \\
        \end{tabular} \\ \hline
\emph{data functions} &  \begin{tabular}{ll}
        $b$ & bedrock elevation \\
        $m$ & total melt water input; $=m_{\text{wall}}+m_{\text{drain}}$ \\
        $P_o$ & overburden pressure; $= \rho_i g H$ if $H=$ ice thickness \\
        $|\bv_b|$ & ice sliding speed \\
        \end{tabular} \\ \hline
\end{tabular}
\label{tab:symbols}
\end{table}

\begin{table}[ht]
  \centering
  \caption{Physical constants, model parameters, and regularization parameters.  ``Default'' values may be overriden in experiments.  The model runs, though more slowly, with the regularization parameters set to zero.}
  \begin{tabular}{lllp{3.0in}} 
    \textbf{Name} & \textbf{Default Value} & \textbf{Units} & \textbf{Description}\\
\hline
    $g$ & $9.81$ & m $\text{s}^{-2}$ & acceleration of gravity \\
    $\rho_i$ & $910$ & $\text{kg}\,\text{m}^{-3}$ & ice density \citep{GreveBlatter2009} \\
    $\rho_w$ & $1000$ & $\text{kg}\,\text{m}^{-3}$ & fresh water density \citep{GreveBlatter2009} \\
    \hline
    $A$ & $3.1689\times 10^{-24}$ & $\text{Pa}^{-3}\,\text{s}^{-1}$ & ice softness \citep{EISMINT96} \phantom{$\Big|$} \\
    $\alpha$ & 1 & & power in Darcy-Weisbach flux formula \eqref{eq:flux} \\
    $\beta$ & 2 & & power in flux formula \eqref{eq:flux} \\
    $c_1$ & $0.5$ & $\text{m}^{-1}$ & cavitation coefficient \\
    $c_2$ & $0.04$ & & creep closure coefficient \\
    $\phi$ & $0.001$ & & physical macroporosity of englacial system \\
    $k$ & $0.01$ & $\text{m}^{2-\alpha} \text{s}^{-1}$ & constant hydraulic conductivity  \\
    $W_r$ & $1$ & $\text{m}$ & roughness scale \\
    \hline
    $\phi_0$ & $0.01$ & & regularizing porosity in pressure equation \\
    $Y_0$ & $0.001$ & m & regularization for layer thickness \\
    \hline
  \end{tabular}
 \label{tab:constants}
\end{table}

Model equations \eqref{eq:bluebox} relate five classes of symbols, namely the state functions and data functions listed in Table \ref{tab:symbols} and the scalar physical constants, model parameters, and regularization parameters listed in Table \ref{tab:constants}.  (Additional symbols $\psi$, $K$, and $\bV$ are derived from these.)  The state functions $W$, $W_{en}$, $P$ evolve according to model \eqref{eq:bluebox}.  Only the state functions must be provided with initial values, and only they must be saved when stopping and restarting a time-dependent numerical model.  The data functions are, in practice, supplied by an ice sheet model.  The scalar parameters are all constant (i.e.~time- and space-independent) in the current paper but they could be allowed to vary spatially if desired.  Although default values must be chosen for the parameters, as is done in Table \ref{tab:constants}, exploration of the parameter space is essential in using the model.  Such exploration occurs in section \ref{sec:results}.

\subsection*{Reduction to existing models}  Three limits of model \eqref{eq:bluebox} can now be stated precisely:
\begin{itemize}

\item The zero englacial storage ($\phi,\phi_0\to 0$) limit of \eqref{eq:bluebox} is the model described by \cite{Schoofetal2012}.  The equations in this case are straightforward to write down; we set $Y_0=0$ and we make the small water approximation ($W\approx 0$) in the hydraulic potential to get complete agreement:
\begin{empheq}[box=\mybluebox]{align}
W &\ge 0, \notag \\
0 &\le P \le P_o, \notag \\
\psi &= P + \rho_w g b, \notag \\
\bV &= - \frac{k}{\rho_w g} W^{\alpha-1} |\grad \psi|^{\beta-2} \grad \psi,  \label{eq:schoofsmodel} \\
0 &= \Div \left( \frac{k}{\rho_w g} W^\alpha |\grad \psi|^{\beta-2} \grad \psi \right) + \frac{m}{\rho_w} + c_2 A (P_o - P)^3 W - c_1 |\bv_b| (W_r - W)_+, \phantom{dsaf} \notag \\
\phantom{dsaf} \frac{\partial W}{\partial t} &= - \Div\left(\bV\, W\right) + \frac{m}{\rho_w}. \notag
\end{empheq}
Mathematically, model \eqref{eq:bluebox} is a parabolic regularization of \eqref{eq:schoofsmodel} based on an efficient connection to porous englacial storage with a small porosity parameter.

\item The $P \to P_o$ limit of \eqref{eq:bluebox}, in which physical processes for the evolution of pressure are ignored and the pressure reverts to overburden, is the standard model for routing water to, and for locating, subglacial lakes under cold ice sheets \cite{Siegertetal2009}.  The englacial storage is removed here ($\phi\to 0$); there is no stiffness issue.  The pressure is not an unknown and the parameterization of cavity evolution is inactive.  We can state the model for general Darcy-Weisbach flux from \eqref{eq:flux}:
\begin{empheq}[box=\mybluebox]{align}
W &\ge 0, \notag \\
\tilde\bV &= - \frac{k}{\rho_w g} W^{\alpha-1} \left|\grad (P_o + \rho_w g b)\right|^{\beta-2} \grad \left(P_o + \rho_w g b\right), \label{eq:lakesmodel} \\
\phantom{ldsfj} \frac{\partial W}{\partial t} &= - \Div\left(\tilde\bV\, W\right) + \Div \left(k W^\alpha \left|\grad (P_o + \rho_w g b)\right|^{\beta-2} \grad W\right) + \frac{m}{\rho_w}. \phantom{ldsfj} \notag
\end{empheq}
As noted in section \ref{sec:closures}, in the $\alpha=1$ and $\beta=2$ case of this model routes water with a velocity $\tilde\bV$ which is determined entirely by ice and bedrock geometry.  Because of the way we have including $W$ into the hydraulic potential (i.e.~$\psi = P_o + \rho_w g(b+W)$ in this case), so that large solution $W$ implies some diffusion, the model is well-posed and has continuous solutions.

\item The non-distributed or lumped form of \eqref{eq:bluebox}, in which $\Div \bq = (q_{out} - q_{in})/L$ where $L$ is the length of the glacier, is essentially the porous glacier model of \cite{Bartholomausetal2011}.  The precise correspondence is explained in Appendix \ref{app:barth}.
\end{itemize}


\section{Steady states}  \label{sec:steadyverif}

\subsection*{Processes become decoupled in steady state}  The steady states of mathematical model \eqref{eq:bluebox} are worth considering because the physical subglacial system is close to steady state much of the time, because we can evaluate the nature of the water flux, and because we can more easily find an exact solution.  We will see that a function which relates the water pressure $P$ directly to the amount $W$ arises in steady state, although no such function exists generally.  In this section we address only the $\alpha=1$ and $\beta=2$ case but the major conclusions apply for general powers $\alpha,\beta$.

Recall that the flux has two expressions $\bq = - k (\rho_w g)^{-1} W \grad \psi = \bV W - k W \grad W$.  Here is the steady form of model \eqref{eq:bluebox} which is written in terms of $\bV,\bq,W,P$:
\begin{align}
\bV &= - \frac{k}{\rho_w g} \grad \left(P + \rho_w g b\right), \label{eq:Vsteady} \\
\bq &= \bV W - k W \grad W, \label{eq:qsteady} \\
0 &= - \Div \bq + \frac{m}{\rho_w}, \label{eq:masscontsteady} \\
0 &= c_2 A (P_o - P)^3 (W+Y_0) - c_1 |\bv_b| (W_r - W)_+. \label{eq:openclosesteady}
\end{align}
(To derive the last equation one eliminates ``$- \Div \bq + m/\rho_w$'' because it is zero.)  Note that we also have bounds $W\ge 0$ and $0 \le P \le P_o$.

Relative to the time-dependent form \eqref{eq:bluebox}, processes have become decoupled in the steady state equations \eqref{eq:Vsteady}--\eqref{eq:openclosesteady}.  There are separate balances between the divergence of the flux and the water input on the one hand (i.e.~equation \eqref{eq:masscontsteady}), and the opening and closing processes on the other hand (i.e.~equation \eqref{eq:openclosesteady}).  Steady state equations \eqref{eq:Vsteady}--\eqref{eq:openclosesteady} are also stated by \cite{Schoofetal2012} model, where the decoupling is also noted.  Specifically, in the one-dimensional case the above equations reduce to (5.8) and (5.10) from \cite{Schoofetal2012}.


\subsection*{Functional relationship for pressure in steady state}  Equation \eqref{eq:openclosesteady} allows us to write the pressure $P=P(W)$ in steady state as a continuous function of the water amount $W$.  (This fundamental fact was already pointed out in considering the steady states of equation \eqref{eq:hewittcapacity}.)  Steady state is only possible if condition \eqref{eq:steadyOCbound} holds, but here with $W$ for $Y$:
\begin{equation}
c_1 |\bv_b| (W_r - W)_+ \le c_2 A P_o^3 (W+Y_0) \qquad \text{ in steady state}. \label{eq:steadyboundfirst}
\end{equation}
In this and later formulas define the following scaled basal sliding speed which has units of pressure:
\begin{equation}
s_b =  \left(\frac{c_1 |\bv_b|}{c_2 A}\right)^{1/3}.  \label{eq:definesb}
\end{equation}
(One may think of $s_b$ as a scale for the pressure drop associated to cavitation in steady state.)  Then \eqref{eq:steadyboundfirst} is equivalent to
\begin{equation}
W \ge W_c := \frac{s_b^3 W_r - P_o^3 Y_0}{s_b^3 + P_o^3} \qquad \text{ in steady state}. \label{eq:steadyboundsecond}
\end{equation}
This condition says that the water amount is above a critical level that depends on the sliding and the overburden pressure.  If the sliding is large ($s_b \gg P_o$) then $W_c\approx W_r$.

If \eqref{eq:steadyboundfirst} or \eqref{eq:steadyboundsecond} holds then
\begin{equation}
P(W) = P_o - s_b \left(\frac{(W_r - W)_+}{W+Y_0}\right)^{1/3} \qquad \text{ in steady state}.  \label{eq:PofWsteady}
\end{equation}
Note that in \eqref{eq:PofWsteady} we have $P(W_c)=0$.  Underpressure ($P=0$) with subcritical water amount ($W<W_c$) does not occur in steady state though it can occur in nonsteady conditions.  Formula \eqref{eq:PofWsteady} may apply even if $W\ge W_r$, in which case the water pressure takes the overburden value $P = P_o$.  However, if $P_o=0$ then \eqref{eq:steadyboundfirst} implies that either $W\ge W_r$ or $|\bv_b|=0$.  This describes the values of $W$ and $|\bv_b|$ at ice margins where $H\to 0$ and therefore $P_o\to 0$.

\newcommand{\upto}{ \!\!\nearrow\! }
\newcommand{\downto}{ \!\searrow\! }
Figure \ref{fig:psteady-vb} shows the function $P(W)$ from \eqref{eq:PofWsteady} for several cases of sliding speed $|\bv_b|$.  Figure \ref{fig:psteady-Po} shows $P(W)$ for several cases of overburden pressure $P_o$.  We see that as the water amount reaches the roughness scale ($W\upto W_r$) the pressure rises rapidly to overburden ($P(W) \upto P_o$).  At the other extreme, we see that $P(W) \downto 0$ if $W \downto W_c$.  The curves $P(W)$ in Figures \ref{fig:psteady-vb} and \ref{fig:psteady-Po}, which describe steady state, do not include the interval $0\le W < W_c$ because such underpressure conditions are not achievable in steady state.

\begin{figure}[ht]
\includegraphics[width=3.5in,keepaspectratio=true]{psteady-vb}
\medskip
\caption{The steady state function $P(W)$ defined by equation \eqref{eq:PofWsteady} depends on the sliding speed.  Four cases are shown (in color) using a fixed uniform ice thickness of $H=1000$ m: $|\bv_b|=0$ m/a (blue), $10$ m/a (green), $100$ m/a (red), and $1000$ m/a (cyan).  The values of $W_c$ for these cases are indicated by black dots at $P=0$.  Relations \eqref{eq:PofWFC} (dashed black) and \eqref{eq:PofWBB} (dash-dot black) are shown with $W_{\text{crit}}=W_r$ for comparison.}
\label{fig:psteady-vb}
\end{figure}

\begin{figure}[ht]
\includegraphics[width=3.5in,keepaspectratio=true]{psteady-Po}
\medskip
\caption{Function $P(W)$ defined by \eqref{eq:PofWsteady} also depends on overburden pressure $P_o=\rho_i g H$.  We fix $|\bv_b|=100$ m/a and consider four cases of uniform thickness $H=$ $2000$ m (blue), $1000$ m (green), $500$ m (red), and $200$ m (cyan).}
\label{fig:psteady-Po}
\end{figure}

Recall that \cite{FlowersClarke2002_theory} propose function $P_{FC}(W)$ (equation \eqref{eq:PofWFC}) for both steady and nonsteady circumstances.  Both functions $P(W)$ in \eqref{eq:PofWsteady} and $P_{FC}(W)$ are increasing and both relate the water pressure to the overburden pressure $P_o$.  However, while in \eqref{eq:PofWsteady} the relation to $P_o$ is additive, in \eqref{eq:PofWFC} it is a multiplicative scaling.  The power law form \eqref{eq:PofWFC} is not justified by the physical reasoning which led to equation \eqref{eq:PofWsteady}, even in steady state.   It would appear that any functional relationship $P(W)$ should also depend on the sliding velocity, as it does here, if cavitation is to influence the water pressure.  Of course the $W>W_{\text{crit}}$ case gives $P_{FC}(W) > P_o$ in \eqref{eq:PofWFC}, a problematic condition already noted by \cite{Schoofetal2012}, but this condition does not arise in \eqref{eq:PofWsteady}.  The most important contrast between the \cite{FlowersClarke2002_theory} theory and the current paper is that we have not assumed a relationship $P=P(W)$ in nonsteady conditions.

\subsection*{Water velocity in steady state}  We now consider how the steady state water velocity $\bV$ depends on other quantities.  Equation \eqref{eq:PofWsteady} defines $P=P(W,P_o,s_b)$ while $\bV$ depends on $\grad P$, and thus we need derivatives of $P$.  In particular, in steady state we have
\begin{equation}
\frac{\partial P}{\partial W} =
    \begin{cases}
      \text{undefined}, & W \le W_c, \\
      \frac{1}{3} s_b (W_r + Y_0) (W+Y_0)^{-4/3} (W_r - W)^{-2/3}, & W_c < W < W_r, \\
      \text{undefined}, & W = W_r, \\
      0, & W > W_r.
    \end{cases}  \label{eq:dPdWsteady}
\end{equation}
(Note that the condition $W_c < W < W_r$ is identical to the normal pressure condition $0 < P < P_o$ in steady state.)  Formula \eqref{eq:dPdWsteady} and Figures \ref{fig:psteady-vb} and \ref{fig:psteady-Po} agree that $\partial P / \partial W \to \infty$ as $W \upto W_r$.  Equations \eqref{eq:Vsteady}, \eqref{eq:PofWsteady}, and \eqref{eq:dPdWsteady} yield this formula for the velocity in steady state which applies in the normal pressure cases:
\begin{align}
\bV &= - \frac{k}{\rho_w g} \grad P - k \grad b = - \frac{k}{\rho_w g} \left[\frac{\partial P}{\partial P_o} \grad P_o + \frac{\partial P}{\partial s_b} \grad s_b + \frac{\partial P}{\partial W} \grad W\right] - k \grad b  \notag \\
    &= - \frac{k}{\rho_w g} \left[\grad \left(P_o + \rho_w g b\right) - \left(\frac{W_r - W}{W+Y_0}\right)^{1/3} \grad s_b + \frac{s_b (W_r + Y_0)}{3  (W+Y_0)^{4/3} (W_r - W)^{2/3}} \grad W\right]. \label{eq:Vsteadyexpand}
\end{align}

Equation \eqref{eq:Vsteadyexpand} helps us understand the advective flux ``$\bV W$'' in $\bq=\bV W - k W \grad W$ in steady state.  First, the direction of water transport $\bV$ is determined by a combination of a geometric direction ($\grad \left(P_o + \rho_w g b\right)$), a direction derived from variations in the sliding speed ($\grad s_b$), and a diffusive direction.  The last category includes all terms proportional to $-\grad W$, thus both the \emph{a priori} diffusive flux $-k W \grad W$ and also the third term in \eqref{eq:Vsteadyexpand}.  A significant fraction of $\bV W$ is diffusive in steady state.  (To our knowledge, the identification of the diffusiveness of this steady velocity is new to this paper.)

Let $\psi_o = P_o + \rho_w g b$ and consider the $Y_0=0$ case for simplicity.  We see that in steady state we can write the flux as a linear combination of gradients,
\begin{equation}
\bq = \bV W - k W \grad W = - \alpha(W) \grad \psi_o + \beta(W) \grad s_b - \gamma(W) \grad W,  \label{eq:qabstract}
\end{equation}
with coefficients
\begin{align*}
\alpha(W) &= \frac{k}{\rho_w g} W, \\
\beta(W) &= \frac{k}{\rho_w g} \left(W_r - W\right)^{1/3} W^{2/3}, \\
\gamma(W) &= \frac{k s_b W_r}{3 \rho_w g (W_r - W)^{2/3}}\, W^{-1/3} + k W.
\end{align*}
The first two coefficients $\alpha(W)$ and $\beta(W)$ go to zero as $W\to 0$.  However $\gamma(W)$ remains large when $W\to 0$, even if $k$ is small, as long as sliding is sustained ($s_b \gg 0$).  Thus for low water amount and sustained sliding we should think of the water as diffusing in the layer; the apparently advective term ``$\bV W$'' is acting diffusively.  When the water thickness closely approximates the roughness scale ($W\approx W_r$) then $\beta(W)$ is also small, while in that circumstance $\gamma(W)$ is again strong as long as sliding is sustained.

Thinking more generally, it is no surprise that when one of the ice thickness, bed elevation, sliding velocity, or water thickness fields is highly-variable in space then we can expect larger flows in steady state.  Formula \eqref{eq:qabstract} reflects this intuition.  Because the magnitude of the velocity $|\bV|$ determines the CFL time step restriction \citep{MortonMayers} associated to numerically solving the mass conservation equation, large variations in these spatial fields will generally reduce the time steps taken by a numerical model.


\section{A nearly-exact steady solution}

\subsection*{The radial steady-state equations}  The above steady equations are the basis on which we now build a nearly-exact solution for $W$ and $P$ in the map-plane.  This solution, which is useful for verifying the numerical schemes even in the non-steady case, will depend on the high-precision numerical solution of a scalar first-order ODE initial value problem.  Exact solutions in one horizontal dimension (waves) also appear in \cite{Schoofetal2012}.

Consider steady state equations \eqref{eq:Vsteady}--\eqref{eq:masscontsteady}, and assume all quantities only depend on the radial coordinate $r = \sqrt{x^2+y^2}$.  Eliminate $\bV$.  In the flat bed case the resulting pair of equations is
\begin{align}
q &= - \frac{k}{\rho_w g} W\, \left(\frac{dP}{dr} + \rho_w g \frac{dW}{dr}\right), \label{eq:rsflux} \\
\frac{1}{r}\frac{d}{dr}\left(r\,q\right) &= \frac{m}{\rho_w}. \label{eq:rsconserve}
\end{align}

In the case of constant water input where $m/\rho_w=\Phi_0 > 0$, which we assume for the exact solution, we can integrate \eqref{eq:rsconserve} from $0$ to $r$ and use symmetry ($q(0)=0$) to get
\begin{equation}
q(r) = \frac{1}{2} \Phi_0\, r. \label{eq:qradial}
\end{equation}
On the other hand, equation \eqref{eq:PofWsteady} gives $P$ as a function of $W$ in steady state.  Suppose $h(r)$ is given so that $P_o(r)$ is also determined.  Assume that the scaled sliding speed $s_b(r)$ has a bounded derivative and that the solution $W(r)$ satisfies the normal pressure conditions $W_c < W < W_r$; both of these properties must be verified later for the constructed solution.  Now, by combining \eqref{eq:PofWsteady}, \eqref{eq:dPdWsteady}, \eqref{eq:rsflux}, and \eqref{eq:qradial} we can eliminate $q$ and $P$ to find
\begin{equation}
\omega_0\, r = - W\, \left(\frac{dP_o}{dr} - \frac{ds_b}{dr} \left(\frac{W_r - W}{W+Y_0}\right)^{1/3} + \left(\frac{s_b (W_r + Y_0)}{3 (W+Y_0)^{4/3} (W_r - W)^{2/3}} + \rho_w g\right) \frac{dW}{dr}\right)  \label{eq:ODEfirst}
\end{equation}
where $\omega_0 = \rho_w g \Phi_0 (2 k)^{-1}$.  This equation applies to cases where $W_c < W < W_r$.

Equation \eqref{eq:ODEfirst} is a first-order ordinary differential equation (ODE) for $W(r)$.  To put it in the standard form expected by a numerical ODE solver, solve for $dW/dr$:
\begin{equation}
\frac{dW}{dr} = \frac{\frac{ds_b}{dr} (W+Y_0) (W_r - W) - \Big[\omega_0\, r W^{-1} + \frac{dP_o}{dr}\Big] (W + Y_0)^{4/3} \left(W_r - W\right)^{2/3}}{\frac{1}{3} s_b (W_r + Y_0) + \rho_w g (W + Y_0)^{4/3} (W_r - W)^{2/3}}.
\label{eq:WradialODE}
\end{equation}
Equation \eqref{eq:WradialODE} has a constant solution $W(r)=W_r$.

\subsection*{Chosen surface elevation and sliding velocity}  To generate a nontrivial exact solution we will have a positive thickness of ice at the margin so that $P_o(L^-)>0$; Figure \ref{fig:Pexact} shows such a small cliff at the margin.  We also assume that at the margin there is some sliding so that $s_b(L^-)>0$, and we require that $s_b(L^-) W_r > P_o(L^-)^3 Y_0$.  At the ice margin $r=L$ we have water pressure $P=0$ so $W(L)=W_c(L^-)$ is the boundary (initial) condition for the ODE.  The initial condition at $r=L$ also satisfies $W(L) < W_r$.  Then we integrate \eqref{eq:WradialODE} from $r=L$ to $r=0$.  The central water thickness value $W(0)$ is determined as part of the solution.

It is useful to have an ice cap geometry $h(r)$ in which the surface gradient formula is simple so that $dP_o/dr$ in \eqref{eq:WradialODE} is also simple.  The plug flow, flat bed surface elevation solution of \cite{Bodvardsson} has this property.  Extending to the radial case, equations (23) and (24) of \citep{Bodvardsson} give
\begin{equation}
h(r) = h_0 \left(1 - \frac{r^2}{R_0^2} \right) \label{eq:choosebodvardssonh}
\end{equation}
where $h(0)=h_0$ is the height of the center of the ice cap.  It follows that $dP_o/dr = - C r$ where $C=2\rho_i g h_0 R_0^{-2}$.  We choose $L=0.9 R_0$ and we note that $h(L)=0.19 h_0$ in \eqref{eq:choosebodvardssonh}.

The sliding speed could be determined by a model for stresses at the ice base and within the ice \citep{GreveBlatter2009}, but a coupled ice and water dynamics solution is too advanced for initial model verification.  Instead we choose a well-behaved sliding speed function which has no sliding near the ice cap center, and which increases in the radial direction:
\begin{equation}
|\bv_b|(r) = \begin{cases} 0, & 0 \le r \le R_1, \\
                           v_0  \left(\frac{r-R_1}{L-R_1}\right)^5, & R_1 < r \le L.
             \end{cases}  \label{eq:choosevb}
\end{equation}
It follows from \eqref{eq:definesb} and \eqref{eq:choosevb} that $ds_b/dr$ in \eqref{eq:WradialODE} is bounded and continuous on $0\le r \le L$.

\subsection*{Numerical solution of radial ODE to give nearly-exact $W(r)$}  Now we solve ODE \eqref{eq:WradialODE} with initial condition $W(L)=W_c(L)$ and the specific values in Table \ref{tab:verifconstants} by using an adaptive numerical ODE solver.  The result $W(r)$ is shown in Figure \ref{fig:Wexact}.  Because equations \eqref{eq:choosebodvardssonh} and \eqref{eq:choosevb} imply a pressure functional relation $P=P(W,r)$ from \eqref{eq:PofWsteady}, we can show in Figure \ref{fig:Wexact} the regions of the $r,W$ plane which correspond to overpressure, normal pressure, and underpressure (solid curves).  We see that $W(r)$ is in the normal pressure region as $r$ decreases from $r=L$ to $r=R_1$.  At $r=R_1$ the function $W(r)$ switches into the overpressure case because there is no sliding.  Figure \ref{fig:Pexact} shows the corresponding pressure solution $P(r)=P(W(r))$ from \eqref{eq:PofWsteady}.

\begin{table}[ht]
  \centering
  \caption{Constants used in constructing the nearly-exact solution.}
  \begin{tabular}{lllp{3.0in}}
    \textbf{Name} & \textbf{Value} & \textbf{Units} & \textbf{Description}\\
\hline
    $\Phi_0$ & $0.2$ & $\text{m}\,\text{a}^{-1}$ & $=m/\rho_w$; constant water input rate \\
    $h_0$ & $500$ & m & ice cap center thickness \\
    $L$   & $22.5$& km & $=0.9 R_0$; actual ice cap margin \\
    $R_0$ & $25$  & km & ideal ice cap radius \\
    $R_1$ & $5$   & km & radial location $r=R_1$ of onset of sliding \\
    $v_0$ & $100$ & $\text{m}\,\text{a}^{-1}$ & sliding speed scale \\
    \hline
  \end{tabular}
 \label{tab:verifconstants}
\end{table}

\begin{figure}[ht]
\includegraphics[width=3.5in,keepaspectratio=true]{exact-W-plot-onu}
\caption{Nearly-exact radial, steady solution for water thickness $W(r)$ (dashed).  In $r$-versus-$W$ space the overpressure (O), normal pressure (N), and underpressure (U) regions are determined by ice geometry and sliding velocity (solid curves; see text).}
\label{fig:Wexact}
\end{figure}

To generate $W(r)$ in Figure \ref{fig:Wexact} we used both a Runge-Kutta 4(5) Dormand-Prince method and a variable-order stiff solver, with relative tolerance $10^{-12}$ and absolute tolerance $10^{-9}$, and with essentially identical results.  Modest stiffness \citep{AscherPetzold} of ODE \eqref{eq:WradialODE} is observed at $r\approx R_1$, however.  The reason is that as the sliding goes to zero, the cavitation also goes to zero ($|\bv_b|\to 0$).  Because creep closure balances cavitation in steady state, it also goes to zero ($P\to P_o$).  The remaining active mechanisms in the model are the variable overburden pressure and the rate of water input.  They must exactly balance.  In fact, in this case \eqref{eq:WradialODE} reduces to the much simpler form
\begin{equation}
\frac{dW}{dr} = - \frac{\varphi_o r W^{-1} + \frac{dP_o}{dr}}{\rho_w g}. \label{eq:WradialODEnoslide}
\end{equation}
Though we have not derived it this way, Equation \eqref{eq:WradialODEnoslide} is the steady radial form of the mass conservation equation under the ``$P=P_o$'' closure, namely equation \eqref{eq:PisoverConservation}.

In equation \eqref{eq:WradialODEnoslide} we see that $dW/dr=0$ if $W$ satisfies $W = - \varphi_0 r / (dP_o/dr)$.  In our case with geometry \eqref{eq:choosebodvardssonh} this reduces to a constant value $W=\tilde W= 0.21764$ m because $\Phi_0$ is constant and $dP_o/dr$ is linear in $r$.  Both numerical ODE solvers mentioned above confirm that $W(r)$ is asymptotic to this constant value $\tilde W$ as $r\to 0$, and that $W(r)\approx \tilde W$ within about 1\% on all of $0\le r \le R_1$.  This is seen in Figure \ref{fig:Wexact}.

\begin{figure}[ht]
\includegraphics[width=3.5in,keepaspectratio=true]{exact-P-plot}
\caption{Nearly-exact radial, steady solution pressure $P(r)$ (dashed) and overburden pressure $P_o$ (solid).}
\label{fig:Pexact}
\end{figure}


\section{Numerical schemes}  \label{sec:num}

\subsection*{Discretization of the mass conservation equation}  Mass conservation equation \eqref{eq:adeqn}, which is part of the combined mathematical model \eqref{eq:bluebox}, will be discretized by an explicit, conservative finite difference method.   A centered, second-order scheme will be applied to the diffusion part.  A pair of schemes for the advection part will be compared, namely first-order upwinding and a higher-order flux-limited upwind-biased method.

We first consider stable time steps.  The time step restriction for the advective part, namely the CFL condition for either of the schemes under consideration, is much more restrictive in this case than the time-step restriction for the diffusion.  Stability for these schemes occurs with a time step $\Delta t \le \Delta t_{\text{CFL}}$ where
\begin{equation}
\Delta t_{\text{CFL}} \left(\frac{\max |u|}{\Delta x} + \frac{\max |v|}{\Delta y}\right) = \frac{1}{2}. \label{eq:dtCFL}
\end{equation}
Here $\bV=(u,v)$ is the advection velocity.  Because of the additional diffusion process, the time step should also satisfy $\Delta t \le \Delta t_{W}$  where
\begin{equation}
\Delta t_W\, 2 \max(K W) \left(\frac{1}{\Delta x^2} + \frac{1}{\Delta y^2}\right) = \frac{1}{2} \label{eq:dtDIFFW}
\end{equation}
\citep{MortonMayers}.  The condition $\Delta t \le \min\{\Delta t_{\text{CFL}}, \Delta t_W\}$ is sufficient for stability and convergence of the overall scheme for \eqref{eq:adeqn}.  Indeed, Appendix \ref{app:positivestable} shows that this is sufficient for the first-order upwind scheme in the $\phi=0$ and $\alpha = 0$ case.  Standard theory suggests the higher-order flux-limited advection scheme \citep{HundsdorferVerwer2010}, and the general $\phi,\alpha$ cases, are stable under these same sufficient conditions because we have \eqref{eq:adeqn} in advection-diffusion form.

To understand these restrictions quantitatively, we consider typical values of the parameters in the $\alpha=1$ and $\beta=2$ case.  The maximum water speed $|\bV|$ is about $10^5$ m/a in trial runs of the model, so $\max |u| = \max |v| \approx 0.002$ m/s.  We take $k=10^{-2}$ m/s and $\max W\approx 10 W_r=10$ m  from Table \ref{tab:constants} as representative values.  Then, for a $\Delta x = \Delta y = 500$ m grid, the advective restriction \eqref{eq:dtCFL} is $\Delta t_{\text{CFL}} \approx 0.001$ year while the diffusive restriction from \eqref{eq:dtDIFFW} is $\Delta t_W \approx 0.01$ year.  Thus, unless the global peak velocity is unusually slow, or unless deep subglacial lakes develop so that $K W$ is large and $\Delta t_W$ is correspondingly small, this diffusive time scale is longer than the CFL time scale.  There is significant uncertainty in these estimated values because the velocity field is part of the coupled solution to the model.  Also the $\alpha > 1$ cases make the time step restrictions more volatile.  The time step restrictions are applied adaptively, however, so stability is assured even though the total computational time is uncertain.

\begin{figure}[ht]
\centering
\includegraphics[width=2.9in,keepaspectratio=true]{diffstencil}
\bigskip
\caption{Numerical scheme \eqref{eq:Wfd} for Equation \eqref{eq:adeqn} uses a grid-centered cell (dashed line).  The velocities, diffusivities, and fluxes are evaluated at the staggered grid locations (triangles) which are denoted with compass direction ($e,w,n,s$).  Scheme \eqref{eq:Pfd} has the same stencil.  The state functions $W,P$ live at the regular grid points (diamonds).}
\label{fig:stencil}
\end{figure}

To set notation, suppose our rectangular computational domain has $M_x \times M_y$ gridpoints $(x_i,y_j)$ with uniform spacing $\Delta x,\Delta y$.  Let $\Wlij \approx W(t_l,x_i,y_j)$ and $\Plij \approx P(t_l,x_i,y_j)$ be the numerical approximations.  Recall that $\bV$ is determined from pressure and bed elevation; see \eqref{eq:bluebox}.  We will compute velocity components and flux components at the staggered (cell-face-centered) points shown in Figure \ref{fig:stencil}.  We compute these values based on centered finite difference approximations of equations \eqref{eq:vexpression} and \eqref{eq:qexpression}.

We use ``compass'' indices such as $u_e = u_{i+1/2,j}$ for the ``east'' staggered component.  Similarly we use compass indices for staggered grid values of the water layer thickness, and these are computed by averaging regular grid values:
\begin{equation}
W_e = (W_{i,j}^l + W_{i+1,j}^l)/2, \qquad W_n = (W_{i,j}^l + W_{i,j+1}^l)/2. \label{eq:stagW}
\end{equation}
In this case we can compute the ``west'' and ``south'' values by shifting indices: $W_w = W_e\big|_{(i-1,j)}$ and $W_s = W_n\big|_{(i,j-1)}$.  Thus there are only two distinct staggered grid values (e.g.~east and north) to compute per regular grid location $(x_i,y_j)$.
The nonlinear effective conductivity $K=K(W,\grad P,\grad b)$ from \eqref{eq:Kdefine} is also needed at staggered locations.  As a notational convenience define $R=P+\rho_w g b$ and define these staggered approximations of $|\grad(P+\rho_w g b)|^2$ \citep[compare][]{Mahaffy}:
\begin{align*}
R_e &= \left|\frac{R_{i+1,j}-R_{i,j}}{\Delta x}\right|^2 + \left|\frac{R_{i+1,j+1}+R_{i,j+1} - R_{i+1,j-1}-R_{i,j-1}}{4\Delta y}\right|^2, \\
R_n &= \left|\frac{R_{i+1,j+1}+R_{i+1,j} - R_{i-1,j+1}-R_{i-1,j}}{4\Delta x}\right|^2 + \left|\frac{R_{i,j+1}-R_{i,j}}{\Delta y}\right|^2.
\end{align*}
Thereby define
\begin{equation}
K_e = k W_e^{\alpha-1} R_e^{(\beta-2)/2}, \qquad K_n = k W_n^{\alpha-1} R_n^{(\beta-2)/2}.  \label{eq:stagK}
\end{equation}
Now we can find the velocity components needed at staggered locations by differencing:
\begin{align}
u_e &= - K_e \left(\frac{P_{i+1,j}-P_{i,j}}{\rho_w g \Delta x} + \frac{b_{i+1,j}-b_{i,j}}{\Delta x}\right), \quad v_n = - K_n \left(\frac{P_{i,j+1}-P_{i,j}}{\rho_w g \Delta y} + \frac{b_{i,j+1}-b_{i,j}}{\Delta y}\right). \label{eq:velocitycomp}
\end{align}
Also $u_w = u_e\big|_{(i-1,j)}$, $v_s = v_n\big|_{(i,j-1)}$, $K_w = K_e\big|_{(i-1,j)}$, and $K_s = K_n\big|_{(i,j-1)}$.

Define $Q_e(u_e)$, $Q_w(u_w)$, $Q_n(v_n)$, and $Q_s(v_s)$ as the face-centered (staggered-grid) normal components of the advective flux $\bV W$.  These quantities are described in more detail in the next subsection.  They use only the staggered velocity component but there is upwinding to determine which $W$ value(s) are used.  Here is the scheme for equation \eqref{eq:adeqn} using \eqref{eq:stagK} and \eqref{eq:velocitycomp} above:
\begin{align}
 &\frac{(W+W_{en})_{i,j}^{l+1} - (W+W_{en})_{i,j}^l}{\Delta t} = - \frac{Q_e(u_e) - Q_w(u_w)}{\Delta x} - \frac{Q_n(v_n) - Q_s(v_s)}{\Delta y} \label{eq:Wfd} \\
      &\qquad\qquad + \frac{K_e W_e \left(W_{i+1,j}^l - \Wlij\right) - K_w W_w \left(\Wlij - W_{i-1,j}^l\right)}{\Delta x^2}  \notag \\
      &\qquad\qquad\qquad + \frac{K_n W_n \left(W_{i,j+1}^l - \Wlij\right) - K_s W_s \left(\Wlij - W_{i,j-1}^l\right)}{\Delta y^2}  + \frac{m_{ij}}{\rho_w}. \notag
\end{align}
Assuming no error in the flux components $Q$, the local truncation error \citep{MortonMayers} of scheme \eqref{eq:Wfd} would be $O(\Delta t^1 + \Delta x^2 + \Delta y^2)$ as an approximation of \eqref{eq:adeqn}.  The actual truncation error depends on the nature of the discrete fluxes, which we address next.

It is useful to rewrite scheme \eqref{eq:Wfd} in ``update'' form, and we should address the englacial storage update.  Let $\nu_x = \Delta t/\Delta x$, $\nu_y = \Delta t/\Delta y$, $\mu_x = \Delta t/\Delta x^2$, and $\mu_y = \Delta t/\Delta y^2$.  The following is equivalent to \eqref{eq:Wfd}:
\begin{align}
 (W+W_{en})_{i,j}^{l+1} &= (W_{en})_{i,j}^l + W_{i,j}^l - \nu_x \left(Q_e(u_e) - Q_w(u_w)\right) - \nu_y \left(Q_n(v_n) - Q_s(v_s)\right) \label{eq:Wupdate} \\
  &\qquad + \mu_x \left(K_e W_e \left(W_{i+1,j}^l - \Wlij\right) - K_w W_w \left(\Wlij - W_{i-1,j}^l\right)\right) \notag \\
  &\qquad + \mu_y \left(K_n W_n \left(W_{i,j+1}^l - \Wlij\right) - K_s W_s \left(\Wlij - W_{i,j-1}^l\right)\right) + \Delta t \frac{m_{ij}}{\rho_w}. \notag
\end{align}
In particular, at time $t_l$ we assume $W^l$ and $(W_{en})^l$ are known fields.  Then \eqref{eq:Wupdate} gives new values for the total water amount $(W+W_{en})^{l+1}$.  If this total water amount comes out negative then it is projected back to zero.  Then the total water amount is split by first computing $W_{en}$ by using the function in \eqref{eq:WenFunctionWtot} and then computing $W$:
\begin{equation} (W_{en})_{i,j}^{l+1} = F\left((W+W_{en})_{i,j}^{l+1}\right), \qquad  W_{i,j}^{l+1} = (W+W_{en})_{i,j}^{l+1} - (W_{en})_{i,j}^{l+1}.  \label{eq:Wenupdate}
\end{equation}
This conserves mass in all cases and makes \eqref{eq:Penhydrostatic} true if there is sufficient water to ``charge'' the englacial system and still have water at the subglacier.

\subsection*{Fluxes in the mass conservation equation}  Appendix \ref{app:fluxlimiters} reviews some of the now-standard theory of flux-limiters for transport schemes.  Here we state the two flux-discretization alternatives which we actually test as flux discretizations for scheme \eqref{eq:Wupdate}.  These choices are choices of functions $\Psi$ from Table \ref{tab:fluxlimiters} in Appendix \ref{app:fluxlimiters}.  The methods we try are first-order upwind and the Koren flux-limiter \citep{HundsdorferVerwer2010}.

Recall that the velocity has components $\bV=(u,v)$ which are evaluated at these staggered locations following \eqref{eq:velocitycomp}.  Also recall that the flux at the staggered grid location $(x_{i+1/2},y_j)$ is denoted ``$Q_e(u_e)$'' and that the flux at $(x_i,y_{j+1/2})$ is denoted ``$Q_n(v_n)$.''  To ensure conservation we must have a single formula for $Q_{i+1/2,j}$ whether this flux is ``$Q_e$'' for $(x_i,y_j)$ or ``$Q_w$'' for $(x_{i+1},y_j)$; similar comments apply to ``$Q_n$'' versus ``$Q_s$'':
\begin{align}
Q_e(u_e) &= \begin{cases} u_e \left[W_{i,j} + \Psi(\theta_{i}) (W_{i+1,j} - W_{i,j})\right], & u_e \ge 0, \\ u_e \left[W_{i+1,j} + \Psi\left((\theta_{i+1})^{-1}\right) (W_{i,j} - W_{i+1,j})\right], & u_e < 0, \end{cases} \label{eq:adfluxes} \\
Q_n(v_n) &= \begin{cases} v_n \left[W_{i,j} + \Psi(\theta_{j}) (W_{i,j+1} - W_{i,j})\right], & v_n \ge 0, \\ v_n \left[W_{i,j+1} + \Psi\left((\theta_{j+1})^{-1}\right) (W_{i,j} - W_{i,j+1})\right], & v_n < 0. \end{cases} \notag
\end{align}
The subscripted $\theta$ quotients are as follows:
\begin{align*}
\theta_i &= \frac{W_{i,j}-W_{i-1,j}}{W_{i+1,j} - W_{i,j}}, & (\theta_{i+1})^{-1} &= \frac{W_{i+2,j}-W_{i+1,j}}{W_{i+1,j} - W_{i,j}}, \\
\theta_j &= \frac{W_{i,j}-W_{i,j-1}}{W_{i,j+1} - W_{i,j}}, & (\theta_{j+1})^{-1} &= \frac{W_{i,j+2}-W_{i,j+1}}{W_{i,j+1} - W_{i,j}}.
\end{align*}
One does not even compute these ``$\theta$s'' when using first-order upwind.  On the other hand, when using the Koren flux-limiter the stencil in Figure \ref{fig:stencil} is extended because regular grid neighbors $W_{i+2,j}$, $W_{i-2,j}$, $W_{i,j+2}$, $W_{i,j-2}$ are potentially involved in updating $W_{i,j}$.

For either the first-order or Koren schemes, if the water input $m$ is negative then we must actively enforce the positivity of the water thickness $W$.  That is, positivity of the advection-diffusion scheme is a desirable property but it does not ensure positivity of the solution if there is actual water removal ($m < 0$).  Therefore we project (reset) $W$ to be nonnegative at the end of each time step, as further explained in \eqref{eq:Wenupdate}.

\subsection*{Discretization of the pressure evolution equation}  The pressure evolution equation \eqref{eq:regpressureequation} is a nonlinear diffusion with additional ``reaction'' terms associated to opening and closing.  Unlike solving \eqref{eq:adeqn} for $W$, when solving \eqref{eq:regpressureequation} for $P$ there is no dominating advection term.  Therefore we discretize it using a centered second-order scheme.  Because this is also an explicit scheme, we consider stable time steps immediately.

The time step restriction we identify is comparable to \eqref{eq:dtDIFFW}, though the proof technique in Appendix \ref{app:positivestable} does not suffice to \emph{prove} stability under this condition because of the additional reaction terms.  The time step must satisfy $\Delta t \le \Delta t_P$ where
\begin{equation}
\Delta t_P\, \left(\frac{2 \rho_i\, \max(K W)}{\rho_w (\phi+\phi_0)}\right) \left(\frac{1}{\Delta x^2} + \frac{1}{\Delta y^2}\right) = \frac{1}{2} \label{eq:dtDIFFP}
\end{equation}
The resulting time step $\Delta t_P$ is a fraction of $\Delta t_W$ from \eqref{eq:dtDIFFW}:
\begin{equation}
\Delta t_P = \frac{\rho_w (\phi+\phi_0)}{\rho_i}\, \Delta t_W.  \label{eq:dtDIFFPfromW}
\end{equation}
In fact, with the estimates $\rho_w/\rho_i \approx 1$ and $\phi+\phi_0 \approx 0.01$, we have $\Delta t_P$ which is about 100 times smaller than $\Delta t_W$.  With these values and others used earlier (i.e.~$\alpha=1$, $\beta=2$, $\Delta x = \Delta y = 500$ m, $\max |\bV|=10^5$ m/a, $k=10^{-2}$ m/s and $\max W=10$ m) we get
\begin{align*}
  \Delta t_{\text{CFL}} &\approx 0.001  \text{ year} &&\text{ from \eqref{eq:dtCFL}}, \\
  \Delta t_W            &\approx 0.01   \text{ year} &&\text{ from \eqref{eq:dtDIFFW}}, \\
  \Delta t_P            &\approx 0.0001 \text{ year} &&\text{ from \eqref{eq:dtDIFFPfromW}.}
\end{align*}

This analysis suggests that the numerical scheme for pressure diffusion, given next, has the shortest time step.  Note that $\Delta t_{\text{CFL}}=O(\Delta x)$ while $\Delta t_W$ and $\Delta t_P$ are $O(\Delta x^2)$.  In actual computation with $\alpha \ne 1$ and $\beta \ne 2$ it seems to be common for $\Delta t_p$ to be $10$--$300$ times shorter than the CFL restriction for the advection.  In any case, the size of the stable time step $\Delta t_P$ scales with the adjustable regularized porosity $\phi+\phi_0$.  By choosing $\phi_0$ larger or smaller we can make the time step restriction on $\Delta t_P$ less or more severe, respectively.

The scheme we use for \eqref{eq:regpressureequation} is similar to \eqref{eq:Wfd} for \eqref{eq:adeqn} but without a need for approximating advection.  Denote $\psi_{i,j}^l = P_{i,j}^l + \rho_w g (b_{i,j}^l + W_{i,j}^l)$.  Let
	$$\mathcal{O}_{ij} = c_1 |\bv_b|_{i,j} \left(W_r - \Wlij\right)_+, \qquad \mathcal{C}_{ij} = c_2 A \left(\rho_i g H_{i,j} - \Plij\right)^3 \Wlij$$
be the gridded values of the cavitation-opening and creep-closure rates.  The scheme is
\begin{align}
\frac{\phi+\phi_0}{\rho_w g} \frac{P_{i,j}^{l+1} - \Plij}{\Delta t} &= \frac{1}{\rho_w g} \bigg[\frac{K_e W_e \left(\psi_{i+1,j}^l - \psi_{i,j}^l\right) - K_w W_w \left(\psi_{i,j}^l - \psi_{i-1,j}^l\right)}{\Delta x^2}  \label{eq:Pfd} \\
      &\qquad\qquad + \frac{K_n W_n \left(\psi_{i,j+1}^l - \psi_{i,j}^l\right) - K_s W_s \left(\psi_{i,j}^l - \psi_{i,j-1}^l\right)}{\Delta y^2}\bigg] \notag \\
      &\qquad + \mathcal{C}_{ij} - \mathcal{O}_{ij} + \frac{m_{ij}}{\rho_w}. \notag
\end{align}
Again it is useful to restate \eqref{eq:Pfd} in explicit update form.  First define $\omega_x = 1/\Delta x^2$, $\omega_y = 1/\Delta y^2$, and   Then scheme \eqref{eq:Pfd} is equivalent to this form:
\begin{align}
P_{i,j}^{l+1} = \Plij +  \frac{\,\Delta t}{\phi+\phi_0} &\bigg[\omega_x K_e W_e \left(\psi_{i+1,j}^l - \psi_{i,j}^l\right) - \omega_x K_w W_w \left(\psi_{i,j}^l - \psi_{i-1,j}^l\right) \label{eq:Pfdupdate} \\
      &\quad + \omega_y K_n W_n \left(\psi_{i,j+1}^l - \psi_{i,j}^l\right) - \omega_y K_s W_s \left(\psi_{i,j}^l - \psi_{i,j-1}^l\right)\bigg] \notag \\
      &+ \frac{\rho_w g\,\Delta t}{\phi+\phi_0} \left(\mathcal{C}_{ij} - \mathcal{O}_{ij} + \frac{m_{ij}}{\rho_w}\right). \notag
\end{align}

There are special cases at the boundaries of the active subglacial layer: (\emph{i}) where there is no ice $H_{i,j}=0$ and land ($b_{i,j}>0$) we set $P_{i,j}^{l+1}=0$, (\emph{ii}) where the ice is floating we set $P_{i,j}^{l+1}=(P_o)_{i,j}$, and (\emph{iii}) where there is grounded ice ($H_{i,j}>0$) and no water ($W_{i,j}^l=0$) we again set $P_{i,j}^{l+1}=(P_o)_{i,j}$. 

\subsection*{One time step of the model}  Mathematical model \eqref{eq:bluebox} evolves $W$, $W_{en}$, and $P$.  One time step of the fully-discretized evolution is described next as a ``recipe''.  We treat the ice and bedrock geometry, and the ice sliding speed, as fixed so that $h_{i,j}$, $b_{i,j}$, $(P_o)_{i,j}$, and $|\bv_b|_{i,j}$ are denoted as time-independent.

The ice geometry may be quite general, with ice-free land and floating ice allowed.  In fact, the ice geometry determines boolean masks for grid cell state based on a sea level of elevation zero:
\begin{align*}
\text{\texttt{icefree}}_{i,j} &= (h_{i,j} > 0)\, \&\, (h_{i,j} = b_{i,j}), \\
\text{\texttt{float}}_{i,j}   &= (\rho_i (H_{\text{float}})_{i,j} < - \rho_{sw}\, b_{i,j}).
\end{align*}
Here we take a sea-water density $\rho_{sw}=1028.0$ and $H_{\text{float}}=h_{i,j} / (1 - r)$  is the thickness of the ice if it is floating, where $r=\rho_i / \rho_{sw}$.  Note that $\text{\texttt{float}}_{i,j}$ is true in ice-free ocean.  The subglacial layer we are attempting to model exists only for grounded ice, that is, only if both \texttt{icefree} and \texttt{float} masks are false.  The other mask cases provide boundary conditions when they are neighbors to grounded ice cells.

One time step follows this algorithm:

\bigskip\medskip
\renewcommand{\labelenumi}{\emph{(\roman{enumi})}}
\begin{enumerate}
\item Start with values $\Wlij$, $(W_{en})_{i,j}^l$, $\Plij$ which satisfy the bounds $W\ge 0$ and $0 \le P \le P_o$.
\item Compute the current values of the hydraulic potential $\psi_{i,j}^l$, but with $\psi_{i,j}^l=(P_o)_{i,j}$ where $\text{\texttt{float}}_{i,j}$.
\item Compute velocity components $u_e$, $v_n$ at staggered grid locations from \eqref{eq:velocitycomp}.
\item Get $W$ values averaged onto the staggered grid from \eqref{eq:stagW}.
%FIXME: however, Wea and Wno should not average from outside the ice domain?
\item Get time step $\Delta t = \min\{\Delta t_{\text{CFL}}, \Delta t_W, \Delta t_P\}$ using criteria \eqref{eq:dtCFL}, \eqref{eq:dtDIFFW}, and \eqref{eq:dtDIFFPfromW}, but based on the actual gridded values of $\bV$ and $W$.
\item If $\text{\texttt{icefree}}_{i,j}$ set $P_{i,j}^{l+1}=0$.  If $\text{\texttt{float}}_{i,j}$ then set $P_{i,j}^{l+1} = (P_o)_{i,j}$; this is the pressure of sea water at the base of the ice.  If $\Wlij=0$ and $\text{\texttt{icefree}}_{i,j}$ and $\text{\texttt{float}}_{i,j}$ are both false, then set $P_{i,j}^{l+1} = (P_o)_{i,j}$.  Otherwise use \eqref{eq:Pfdupdate} to compute preliminary values for $P_{i,j}^{l+1}$ at the remaining locations, but do not compute the $x$-($y$-)direction divided-difference contribution to the flux divergence in \eqref{eq:Pfdupdate} when either $x$-($y$-)neighbor is \texttt{icefree} or \texttt{float}.
\item If $P_{i,j}^{l+1}$ does not satisfy bounds $0 \le P \le P_o$ then reset (project) into this range.
\item Using \eqref{eq:adfluxes} and a particular flux-limiter, or first-order upwinding without flux limiter, compute the advective fluxes $Q_e(\alpha_e)$ at all staggered-grid points $(i+1/2,j)$ and $Q_n(\beta_n)$ at all staggered-grid points $(i,j+1/2)$.  
\item If $\text{\texttt{icefree}}_{i,j}$ or $\text{\texttt{float}}_{i,j}$ then set $W_{i,j}^{l+1}=0$ and $(W_{en})_{i,j}^{l+1}=0$.  Otherwise use \eqref{eq:Wupdate} to compute preliminary values for the total water $(W+W_{en})_{i,j}^{l+1}$.
\item Apply rule \eqref{eq:Wenupdate} to update $(W_{en})_{i,j}^{l+1}$ and then $W_{i,j}^{l+1}$ from the preliminary values for total water.
\item If $W_{i,j}^{l+1}<0$ then reset (project) $W_{i,j}^{l+1}=0$.
\item Update time $t_{l+1}=t_l+\Delta t$ and repeat at \emph{(i)}.
\end{enumerate}

\medskip
This recipe goes with a bookkeeping and reporting scheme for mass.  Note that water is lost or gained at the margin where either the thickness goes to zero on land (margins), or at locations where the ice becomes floating (grounding lines), according to the hydraulic potential differences which cause transport.  Because such loss/gain may be the modeling goal---users want hydrological discharge---these amounts are reported.  This reporting scheme also tracks the projections in step \emph{(xi)}, which represent a mass conservation error which goes to zero under grid refinement.

\subsection*{Verification of the coupled model}  By using the coupled steady-state nearly-exact solution constructed in section \ref{sec:steadyverif} we can verify the numerical schemes described above.  Verification is the process of actually measuring the errors made by the numerical scheme, especially as the numerical grid is refined.

Our exact solution is for the steady-state.  Therefore we initialize our time-stepping numerical scheme with the exact steady solution and we measure the error relative to the steady exact values after some period of time integration.  The continuum time-dependent model \eqref{eq:bluebox} would cause no drift but we can measure the drift away from the exact steady solution generated by the numerical approximation to \eqref{eq:bluebox}, that is, by the scheme which is the ``recipe'' above.  The magnitude of the drift depends on the spatial grid size.  The rate of convergence under grid refinement is primarily determined by the quality of our spatial discretizations.

For the verification runs we use the default values in Table \ref{tab:constants}.  The exact solution is shown in Figures \ref{fig:Wexact} and \ref{fig:Pexact}, and it uses constants from Table \ref{tab:verifconstants}.  We do a one model-month run on grids with spacing decreasing by factors of two from $5$ km to $156$ m.  Figure \ref{fig:refineWPpism} shows the results based on first-order upwinding for the fluxes.

\begin{figure}[ht]
\includegraphics[width=3.0in,keepaspectratio=true]{refineWpism} \quad \includegraphics[width=3.0in,keepaspectratio=true]{refinePpism}
\caption{Left: Average water thickness error $|W-W_{exact}|$ decays as $O(\Delta x^{0.87})$.  Right: Average pressure error $|P-P_{exact}|$ decays as $O(\Delta x^{1.20})$}
\label{fig:refineWPpism}
\end{figure}

Because they give evidence for numerical convergence, these results suggest that our numerical solution method for these coupled advection-diffusion (for $W$) and diffusion-reaction (for $P$) equations are being solved correctly.  The rate of convergence is not very good, however.  The location of the large errors is entirely in the neighborhood of the ice margin $r=L$ (not shown).  The method for handling boundary conditions is critical to determining the magnitude of the error and its rate of decay.  Further research is needed to decide how to handle this boundary in a more-nearly optimal manner.

The rates of convergence for average errors are nearly identical for the higher resolution flux-limited (Koren) scheme and for the first-order upwinding scheme (not shown).  Our problem is a combined advection-diffusion problem in which both the advection velocity and the diffusivity are solution-dependent, and thus it is difficult to separate the errors arising from the numerical treatments of advection and diffusion.  The first-order upwinding scheme for the advection has much larger numerical diffusivity but this diffusivity may be compatible with the continuum model (i.e.~desirable) diffusivity.  Based on this verification evidence it is reasonable to choose first-order upwinding for applications.  This scheme is simpler to implement and it requires fewer floating point operations.  It also uses a smaller stencil for less communication in a parallel implementation.



\section{Numerical results}  \label{sec:results}

\subsection*{Steady results for a tidewater glacier}

FIXME: decide on where this paper is going

FIXME: redo results from nbreen to use PISM and make better figures

\begin{comment}

\begin{figure}[ht]
\includegraphics[width=7.0in,keepaspectratio=true]{icethk-icefree-float-250m}
\caption{FIXME}
%\label{fig:X}
\end{figure}

\begin{figure}[ht]
\includegraphics[width=7.0in,keepaspectratio=true]{outline-input-250m}
\caption{FIXME}
%\label{fig:X}
\end{figure}

\begin{figure}[ht]
\includegraphics[width=7.0in,keepaspectratio=true]{W-Pmask-250m}
\caption{FIXME}
%\label{fig:X}
\end{figure}

\begin{figure}[ht]
\includegraphics[width=7.0in,keepaspectratio=true]{Po-P-250m}
\caption{FIXME}
%\label{fig:X}
\end{figure}

\end{comment}

FIXME:  text about F\&C equation \eqref{eq:PofWFC}

\begin{figure}[ht]
\includegraphics[width=3.0in,keepaspectratio=true]{isPofW-250m} \,
\includegraphics[width=3.0in,keepaspectratio=true]{isPofW-250m-month}
\caption{Left: Scatter plot of $(W,P)$ pairs for all cells at end of 5 year steady-input simulation on a 250 m grid.  Red dashed is equation \eqref{eq:PofWFC} with $W_{\text{crit}} = W_r = 1$ m.  Right: Same except at the end of a one month simulation, and with equation \eqref{eq:PofWFC} using $W_{\text{crit}} = W_r / 3$.}
\label{fig:isPofWnbreen}
\end{figure}


%\clearpage\newpage

\small
\bibliography{ice_bib}  % generally requires link to pism/doc/ice_bib.bib
\bibliographystyle{agu}
\normalsize


%\clearpage\newpage
\appendix
\small

\section{Relation to the Bartholomaus et al.~(2011) model}  \label{app:barth}

The model in \cite{Bartholomausetal2011} describes the evolution of the intriguing hydrology of the Kennicott glacier in Alaska.  It is a significantly different model from the distributed one of \cite{Schoofetal2012}, which we focus on in the main text, but similarities exist.  Both are interested in the relationship between sliding and the evolution of linked-cavity systems, and both consider physical cavity opening and closing processes.  On the other hand the \cite{Schoofetal2012} theory is distributed and entirely subglacial while the \cite{Bartholomausetal2011} is ``lumped'' (i.e.~the entire glacier is represented by one cell) and there is both subglacial and englacial storage.

In this Appendix we restate the published equations of the \cite{Bartholomausetal2011} model and then derive a pressure evolution equation which applies in that model.  The form of this pressure equation is suggested by equation (12) in \cite{Bartholomausetal2011} but its complete form is not stated there.  By extending it to the distributed case, this pressure equation can be recognized as a diffusive (parabolic) version of the elliptic pressure equation in \cite{Schoofetal2012}.

We start by describing the variables and equations of the \cite{Bartholomausetal2011} model.  The total volume of liquid water stored in the glacier is $S(t)$ (m$^3$), and this is split into englacial $S_{en}(t)$ and subglacial $S_{sub}(t)$ portions, thus $S=S_{en}+S_{sub}$.  The cavities have geometry partially-determined by bedrock bumps with cartesian spacing $\lambda_x,\lambda_y$ (m), height $h$ (m), and width $w_c$ (m).  These combine to give a dimensionless capacity parameter $f=(h w_c)/(\lambda_x \lambda_y)$; the value $f=0.05$ is used for the Kennicott glacier application.  Each cavity has cross-sectional area $A_c(t)$ (m$^2$) and volume $w_c A_c$ (m$^3$).  The glacier occupies a rectangle of dimensions $L\times W$ in the map-plane so that the number of cavities is $\nu = (LW)/(\lambda_x\lambda_y)$.  It follows that the subglacial storage volume is $S_{sub} = (w_c A_c) \nu = (f L W/h) A_c$.  Englacial water is assumed to fill crevasses and moulins up to a level $z_w(t)$ (m) above the bedrock, and it has macroporosity $\phi$ (dimensionless).  Thus the englacial storage is $S_{en}=L W \phi z_w$.  In summary, the ``kinematics'' of the \cite{Bartholomausetal2011} model are these equations:
\begin{equation}
S = S_{en} + S_{sub}, \qquad S_{en} = L W \phi z_{en}, \qquad S_{sub} = \frac{f L W}{h} A_c.  \label{eq:barth:kinematics}
\end{equation}

Mass conservation in the model is the simple statement  \citep{Bartholomausetal2008}
\begin{equation}
\frac{dS}{dt} = Q_{in}(t) - Q_{out}(t). \label{eq:barth:massconserve}
\end{equation}
In the Kennicott glacier application, fluxes $Q_{in}$ and $Q_{out}$ are given by observations.

As usual in the current paper, the overburden pressure is denoted $P_o=\rho_i g H$ and the water pressure $P(t)$ so that $N=P_o-P$ is the effective pressure applied by the glacier to its bed.  Knowledge of the water pressure $P$ is equivalent to knowledge of the amount of englacial storage because there is an assumed efficient connection of the macroporous glacier to the subglacial system.  Englacial water therefore applies the hydrostatic pressure to the subglacier, and thus
\begin{equation}
P = \rho_w g z_w.  \label{eq:barth:englacialpressure}
\end{equation}

In the \cite{Bartholomausetal2011} model the cavity cross-sectional area evolves by physical opening and closure processes.  A wall melt term is also given but, in keeping with the cavity evolution in the rest of the current paper, we simply denote it as a melt term $\dot m$.  Denote the sliding speed by $u_b$ and let $C_c = (2 A)/n^n$.  Then the cavity area evolution equation (4) in \cite{Bartholomausetal2011} is
\begin{equation}
\frac{dA_c}{dt} = u_b h + \dot m - C_c A_c (P_o-P)^n.  \label{eq:barth:cavityevolution}
\end{equation}
The three terms on the right are opening by cavitation, melt, and closure by creep, respectively.

Equations \eqref{eq:barth:kinematics}, \eqref{eq:barth:massconserve}, \eqref{eq:barth:englacialpressure}, and \eqref{eq:barth:cavityevolution} combine to give an evolution equation for the pressure.  From \eqref{eq:barth:kinematics} and \eqref{eq:barth:englacialpressure} we can write the pressure rate of change in terms of the englacial storage rate of change:
	$$\frac{dP}{dt} = \rho_w g \frac{dz_w}{dt} = \frac{\rho_w g}{L W \phi} \frac{d S_{en}}{dt}.$$
But \eqref{eq:barth:kinematics} and \eqref{eq:barth:massconserve}  allow us to rewrite in terms of fluxes and cavity area:
    $$\frac{dP}{dt} = \frac{\rho_w g}{L W \phi} \left(\frac{d S}{dt} - \frac{d S_{sub}}{dt}\right) = \frac{\rho_w g}{L W \phi} \left(Q_{in} - Q_{out} - \frac{d S_{sub}}{dt}\right).$$
Now use \eqref{eq:barth:kinematics} to write the evolution in terms of the rate of change of $A_c$:
    $$\frac{dP}{dt} = \frac{\rho_w g}{L W \phi} \left(Q_{in} - Q_{out} - \frac{f L }{h} \frac{d A_c}{dt}\right).$$
Finally incorporate equation \eqref{eq:barth:cavityevolution} to eliminate $dA_c/dt$:
\begin{equation}
\frac{dP}{dt} = \frac{\rho_w g}{L W \phi} \left(Q_{in} - Q_{out} - \frac{f L }{h} \left[u_b h + \dot m - C_c A_c (P_o-P)^n\right]\right)  \label{eq:barth:fullpressure}
\end{equation}
We emphasize that, though it is not stated there, the equation \eqref{eq:barth:fullpressure} follows from the model equations as stated in \cite{Bartholomausetal2011}.

Equation \eqref{eq:barth:fullpressure} suggests how to extend the \cite{Bartholomausetal2011} theory from ``lumped'' into ``distributed.''  Consider a one-dimensional glacier flowing in the positive $x$ direction.  Let the transverse width be $W$ and replace $L$ by $\Delta x$.  Note that ``$Q_{in}$'' would, in a distributed theory, be the upstream flux while ``$Q_{out}$'' would be downstream.  Thus we rewrite \eqref{eq:barth:fullpressure} as
\begin{equation*}
\frac{\phi}{\rho_w g}\frac{dP}{dt} = \frac{1}{W} \left(- \frac{Q_{out} - Q_{in}}{\Delta x} - \frac{f}{h} \left[u_b h + \dot m - C_c A_c (P_o-P)^n\right]\right).
\end{equation*}
The continuum limit is then clear, with $(Q_{out} - Q_{in})/\Delta x \to \partial Q/\partial x$.  Thus a distributed flowline form of the \cite{Bartholomausetal2011} theory is the partial differential equation
\begin{equation}
\frac{\phi}{\rho_w g} \frac{\partial P}{\partial t} = \frac{1}{W} \left(- \frac{\partial Q}{\partial x} - \frac{f}{h} \left[u_b h + \dot m - C_c A_c (P_o-P)^n\right]\right).  \label{eq:barth:distpressure}
\end{equation}

A distributed extension of the \cite{Bartholomausetal2011} theory is not, however, viable without a Darcy or other flux expression for $Q$.  Of course one must also use a distributed mass conservation equation like \eqref{eq:conserve}.  A flux expression, Darcy or otherwise, was not needed in the Kennicott glacier case because the lumped input and output from the hydrological system were available as data.

An implication of the model in \cite{Bartholomausetal2011}, again not stated there, regards the pressure in steady state.  Steady state ($d/dt = 0$) applied to \eqref{eq:barth:cavityevolution} gives
\begin{equation*}
0 = u_b h + \dot m - C_c A_c (P_o-P)^n.
\end{equation*}
Thus there is a relationship between pressure $P$ and cavity area $A_c$ in steady state:
\begin{equation}
P = P_o - \left(\frac{u_b h + \dot m}{C_c A_c}\right)^{1/n} = P(A_c). \label{eq:barth:steadypressure}
\end{equation}
This equation is just like equation \eqref{eq:PofWsteady} in the main text; again, in steady state the pressure is a function of the amount of water.  It is interesting to also observe, however, that \eqref{eq:barth:steadypressure} shows that the steady water pressure does not depend on the englacial macroporosity $\phi$.  Though the pressure is parameterized englacially by $P=\rho_w g z_w$ in \cite{Bartholomausetal2011}, its \emph{steady} value is entirely determined by the balance between sliding, wall melt, and creep closure in the subglacial system.  The englacial system is passive in determining steady state.

Returning to equation \eqref{eq:barth:distpressure} we can now connect the theory outlined in this Appendix with the main theory in the paper.  Namely, if we extend \eqref{eq:barth:distpressure} to two horizontal dimensions ($\partial Q/\partial t \to \Div \bq$ and so on) and we add Darcy relation \eqref{eq:flux} then we essentially get \eqref{eq:pressureequation}.  Note that under any reasonable Darcy-type formulation for the flux $Q$ in \eqref{eq:barth:distpressure}, the $\phi\to 0$ limit of \eqref{eq:barth:distpressure} is an elliptic equation for the water pressure.  In fact, the $\phi\to 0$ and distributed version of the subglacial-plus-englacial theory of \cite{Bartholomausetal2011} is the theory in \cite{Schoofetal2012}.  Therefore the main text simultaneously describes a distributed extension of the \cite{Bartholomausetal2011} theory and an englacial storage extension of the \cite{Schoofetal2012} theory.


\section{Flux-limiter methods} \label{app:fluxlimiters}

Because \eqref{eq:adeqn} in the main text is an advection-dominated PDE, the well-known goals for discretizing the fluxes include non-oscillation and positivity \citep{HundsdorferVerwer2010} in addition to reduced truncation error.  Schemes addressing these goals, which are often described as finite volume discretizations \citep{LeVeque}.  We now introduce a few such schemes using an advection-only model equation in which an abstract quantity $u(t,x)$ is transported by an abstract velocity $v(x)$:
\begin{equation} \label{eq:modeladvect}
u_t + (v(x) u)_x = 0
\end{equation}

Suppose that the grid points $x_i$ are the centers of equally-spaced cells with $x_{i+1}-x_i=\Delta x$.  Cell interfaces are at $x_{i-1/2}=x_w$ and $x_{i+1/2}=x_e$.  Suppose we do not discretize time and instead we denote $u_i=u_i(t)$.  We say a spatial discretization of \eqref{eq:modeladvect} is \emph{positive} if $u(0,x)\ge 0$ for all $x$ implies $u(t,x)\ge 0$ for all $t\ge 0$ and all $x$.  (Such a property is desirable for any conservation scheme for the water thickness $W$ because thicknesses are intrinsically nonnegative.)  The basic discretization of \eqref{eq:modeladvect} uses the fluxes $Q=v(x) u$ at the cell interfaces $x_w$ and $x_e$, namely
\begin{equation}
\frac{du_i}{dt} + \frac{Q_e - Q_w}{\Delta x} = 0. \label{eq:basicmodelFD}
\end{equation}

But we must choose a flux parameterization.  The simplest positive scheme for the fluxes is first-order upwinding in conservative form \citep[section I.4.3]{HundsdorferVerwer2010}.  We use the upwind value $u_i$ for the flux at $x_e$ because $v_e = v(x_e) \ge 0$:
\begin{equation}
Q_e = v_e u_i \label{eq:upwindfluxfirst}
\end{equation}
Also $Q_w = v_w x_{i-1}$ because $v_w = v(x_w) \ge 0$.  Scheme \eqref{eq:upwindfluxfirst} is also called the ``donor cell'' upwind method \citep{LeVeque}.

A second-order truncation error scheme which is \emph{not} positive uses a centered average in computing the flux.  We can regard this as a correction to the first-order upwind form:
\begin{equation}
Q_e = v_e \frac{u_i+u_{i+1}}{2} = v_e \left[u_i + \frac{1}{2} (u_{i+1} - u_i)\right]. \label{eq:centerfluxfirst}
\end{equation}
A yet higher-resolution scheme is third-order upwind-biased fluxes which can again be written as a correction to first-order upwinding:
\begin{equation}
Q_e = v_e \frac{-u_{j-1} + 5 u_i + 2 u_{i+1}}{6} = v_e \left[u_i + \left(\frac{1}{3}+\frac{1}{6} \theta_i \right) (u_{i+1} - u_i)\right] \label{eq:thirdfluxfirst}
\end{equation}
where
\begin{equation}
\theta_i = \frac{u_{i} - u_{i-1}}{u_{i+1} - u_i}.  \label{eq:thetadefine}
\end{equation}
Unfortunately, despite the upwind-biasing in the third-order scheme, both \eqref{eq:centerfluxfirst} and \eqref{eq:thirdfluxfirst} cause oscillations and are not positive.

The ``flux-limiting'' approach regards the corrections in \eqref{eq:centerfluxfirst} and \eqref{eq:thirdfluxfirst} as too large to allow positivity, specifically near local extrema of $u$ \citep[section III.1.1]{HundsdorferVerwer2010}.  Godunov's barrier theorem \citep[section I.7.1]{HundsdorferVerwer2010} says that we cannot get truncation error better than first-order with a discretization that is both linear and positive.  However, there exist nonlinear ``flux-limited'' (correction-limited) formulas which restore these properties, and, because the schemes under consideration are explicit, the nonlinearity does not represent a significant computational cost.

These flux-limited schemes can be written in the same general form as above, with a correction to the first-order upwinded flux:
\begin{equation}
Q_e = v_e \left[u_i + \Psi(\theta_i) (u_{i+1} - u_i)\right], \qquad v_e \ge 0. \label{eq:fluxlimiterform}
\end{equation}
If we have $v(x)<0$ in \eqref{eq:modeladvect} then the same function $\Psi$ should be used but with the direction reversed \citep[section III.1.1]{HundsdorferVerwer2010}:
\begin{equation}
Q_e = v_e \left[u_{i+1} + \Psi\left((\theta_{i+1})^{-1}\right) (u_i - u_{i+1})\right], \qquad v_e < 0. \label{eq:fluxlimiterformreversed}
\end{equation}
The forms for $Q_w$ in \eqref{eq:basicmodelFD} simply replace $v_e \to v_w$, $i\to i-1$, and $i+1\to i$.  Table \ref{tab:fluxlimiters} shows several cases for $\Psi$ including those already considered.

The difference ratio $\theta_i$ in \eqref{eq:thetadefine} can take any real value.  For the time-discretizations we will consider, sufficient conditions on $\Psi(\theta)$ to give a positive advection scheme for model equation \eqref{eq:modeladvect} are
\begin{equation}
0 \le \Psi(\theta) \le 1, \qquad 0 \le \frac{\Psi(\theta)}{\theta} \le 1 \qquad \text{ for all } \theta \in \RR.
\end{equation}
The schemes in Table \ref{tab:fluxlimiters} which satisfy these conditions are marked with ``$\ast$''.

For smooth functions and fine grids we have $\theta_i\approx 1$ except near extrema where $u_{i+1} - u_i$ and $u_i - u_{i-1}$ are both near zero.  Note that in every case in Table \ref{tab:fluxlimiters} where there is a correction ($\Psi(\theta)\ne 0$) we have $\Psi(1)=1/2$.  The Koren flux-limiter in Table \ref{tab:fluxlimiters} \citep{HundsdorferVerwer2010} has $\Psi(\theta) = \frac{1}{3}+\frac{1}{6} \theta$ for $\frac{2}{5} \le \theta \le 4$.  We can think of the Koren scheme as a positive form of the third-order upwind-biased scheme.

\begin{table}[ht]
  \centering
  \caption{Flux schemes written in limiter form \eqref{eq:fluxlimiterform} using $\theta=\theta_i$ from \eqref{eq:thetadefine}.}
  \begin{tabular}{ll}
    \textbf{scheme ($\ast=$ positive)} & \textbf{formula} \\
\hline
    $\ast$ first-order upwinding               & $\phantom{\Big|}\Psi(\theta) = 0$ \\
    \phantom{$\ast$} second-order centered     & $\phantom{\Big|}\Psi(\theta) = \frac{1}{2}$  \\
    \phantom{$\ast$} third-order upwind-biased & $\phantom{\Big|}\Psi(\theta) = \frac{1}{3}+\frac{1}{6} \theta$  \\
    $\ast$ van Leer 1974                       & $\phantom{\Big|}\Psi(\theta) = \frac{1}{2} \frac{\theta + |\theta|}{1+\theta}$  \\
    $\ast$ Koren 1993                          & $\phantom{\Big|}\Psi(\theta) = \max\left\{0,\min\{1,\theta,\frac{1}{3}+\frac{1}{6} \theta\}\right\}$  \\
    \hline
  \end{tabular}
 \label{tab:fluxlimiters}
\end{table}

To summarize these flux discretizations for model equation \eqref{eq:modeladvect}, first-order upwinding \eqref{eq:upwindfluxfirst} gives positivity but only $O(\Delta x^1)$ truncation error while second-order centered differencing \eqref{eq:centerfluxfirst} and third-order upwind-biased differencing \eqref{eq:thirdfluxfirst} give better truncation error (i.e.~$O(\Delta x^2)$ and $O(\Delta x^3)$, respectively), but not positivity.  The flux-limited Koren and van Leer schemes in Table \ref{tab:fluxlimiters} give positivity and also better truncation error away from difficult areas (i.e.~near extrema and non-smooth areas where $\theta_j$ is not near one), but they revert to first-order in these difficult areas.

The above discussion was limited to the spatial discretization; we have applied ``method of lines'' only.  A time discretization is also required, and for this we simply choose forward Euler.  The resulting fully discrete system is positive if
\begin{equation}
\max_x \frac{|v(x)|\Delta t}{\Delta x} \le \frac{1}{2} \label{eq:CFL}
\end{equation}
\citep[section III.1.1]{HundsdorferVerwer2010}.  One can show that these positive schemes are also total variation diminishing (TVD) if the velocity is constant ($v(x)=v_0$).  We identify condition \eqref{eq:CFL} as simply ``CFL'' even though it is more strict than the CFL condition that suffices for stability \citep{MortonMayers}.


\section{Positivity and stability of the mass conservation scheme} \label{app:positivestable}

Explicit numerical scheme \eqref{eq:Wfd} for the mass conservation PDE \eqref{eq:adeqn}, combined with the first-order upwind case of formulas \eqref{eq:adfluxes}, is sufficiently simple so that we can analyze its properties.  For this scheme we sketch a maximum principle argument which shows stability \citep{MortonMayers}.  The argument also shows positivity \citep{HundsdorferVerwer2010} as long as the total water input is nonnegative, though only the case $m_{ij} = 0$ is shown.  We assume no englacial storage ($W_{en}=0$ and $\phi=\phi_0=0$) and we use $\alpha=1$ for simplicity.  We consider only the upwinding case where the discrete velocities at cell interfaces are nonnegative: $u_e\ge 0$, $u_w\ge 0$, $v_n\ge 0$, $v_s\ge 0$.  The other upwinding cases can be handled by similar special-case arguments like this one.

We start by restating equation \eqref{eq:Wupdate} with the above simplifications:
\begin{align*}
 W_{i,j}^{l+1} &= \Wlij - \nu_x \left(u_e \Wlij - u_w W_{i-1,j}^l\right) - \nu_y \left(v_n \Wlij - v_s W_{i,j-1}^l\right)  \\
      &\qquad + \mu_x \left[W_e \left(W_{i+1,j}^l - \Wlij\right) - W_w \left(\Wlij - W_{i-1,j}^l\right)\right]  \\
      &\qquad + \mu_y \left[W_n \left(W_{i,j+1}^l - \Wlij\right) - W_s \left(\Wlij - W_{i,j-1}^l\right)\right].
\end{align*}
Rearranging terms we get
\begin{align*}
 W_{i,j}^{l+1} &= (\nu_x u_w + \mu_x W_w) W_{i-1,j}^l + (\mu_x W_e) W_{i+1,j}^l + (\nu_y v_s + \mu_y W_s) W_{i,j-1}^l + (\mu_y W_n) W_{i,j+1}^l \\
      &\qquad + \Big[1 - \nu_x u_e - \nu_y v_n - \mu_x (W_e + W_w) - \mu_y (W_n + W_s)\Big] \Wlij.
\end{align*}
Thus the new value is a linear combination of old values:
\begin{equation}
W_{i,j}^{l+1} = A W_{i-1,j}^l + B W_{i+1,j}^l + C W_{i,j-1}^l + D W_{i,j+1}^l + E \Wlij. \label{eq:lincomb}
\end{equation}
Because of our assumption about nonnegative velocities, and assuming $\Wlij \ge 0$ for all $i,j$, we see that coefficients $A,B,C,D$ are all nonnegative.  Only $E$ could be negative; requiring it to be nonnegative is a sufficient stability condition.

We state such a condition based on an equal split between advective and diffusive parts.  First there is a CFL restriction for the advection terms; compare \eqref{eq:dtCFL}:
\begin{equation}
\nu_x \alpha_e + \nu_y \beta_n = \Delta t \left(\frac{u_e}{\Delta x} + \frac{u_n}{\Delta y}\right) \le \frac{1}{2}. \label{eq:adstabcond}
\end{equation}
The second is a time-step restriction on the diffusion; compare \eqref{eq:dtDIFFW}:
\begin{equation}
\mu_x (W_e + W_w) + \mu_y (W_n + W_s) = \Delta t \left(\frac{k (W_e + W_w)}{\Delta x^2} + \frac{k (W_n + W_s)}{\Delta y^2}\right) \le \frac{1}{2}. \label{eq:diffstabcond}
\end{equation}
If \eqref{eq:adstabcond} and \eqref{eq:diffstabcond} hold then the coefficient $E$ in \eqref{eq:lincomb} is nonnegative:
	$$E = 1 - \nu_x u_e - \nu_y v_n - \mu_x (W_e + W_w) - \mu_y (W_n + W_s) \ge 0.$$

It also follows from \eqref{eq:lincomb} that if $\Wlij\ge 0$ for all $i,j$ then $W_{ij}^{l+1}\ge 0$.  This is the positivity-preserving assertion.  Because the coefficients in linear combination \eqref{eq:lincomb} also add to one, it follows  from \eqref{eq:adstabcond} and \eqref{eq:diffstabcond} that the scheme is stable \citep{MortonMayers}.  More generally, under conditions \eqref{eq:dtCFL} and \eqref{eq:dtDIFFW}, the conditions which describe all of the upwinding cases, the scheme is stable and positivity-preserving.

\end{document}
