\documentclass[11pt,reqno]{amsart}
%prepared in AMSLaTeX, under LaTeX2e
\addtolength{\oddsidemargin}{-.65in}
\addtolength{\evensidemargin}{-.65in}
\addtolength{\topmargin}{-.3in}
\addtolength{\textwidth}{1.5in}
\addtolength{\textheight}{.6in}

\renewcommand{\baselinestretch}{1.1}

\usepackage{verbatim} % for "comment" environment

\usepackage[pdftex, colorlinks=true, plainpages=false, linkcolor=blue, citecolor=red, urlcolor=blue]{hyperref}

\newtheorem*{thm}{Theorem}
\newtheorem*{defn}{Definition}
\newtheorem*{example}{Example}
\newtheorem*{problem}{Problem}
\newtheorem*{remark}{Remark}

\newcommand{\mtt}{\texttt}
\usepackage{alltt,xspace}
\usepackage[normalem]{ulem}
\newcommand{\mfile}[1]
{\medskip\begin{quote}\scriptsize \begin{alltt}\input{#1.m}\end{alltt} \normalsize\end{quote}\medskip}

\usepackage[final]{graphicx}
\newcommand{\mfigure}[1]{\includegraphics[height=2.5in,
width=3.5in]{#1.eps}}
\newcommand{\regfigure}[2]{\includegraphics[height=#2in,
keepaspectratio=true]{#1.eps}}
\newcommand{\widefigure}[3]{\includegraphics[height=#2in,
width=#3in]{#1.eps}}

% macros
\usepackage{amssymb}

\usepackage[T1, OT1]{fontenc}
\renewcommand{\dh}{\fontencoding{T1}\selectfont{\symbol{240}}}

\newcommand{\bod}{B\"o\dh varsson\xspace}
\newcommand{\bods}{B\"o\dh varsson's}
\newcommand{\citebod}{B\"o\dh varsson (1955)\xspace}
\newcommand{\citepbod}{(B\"o\dh varsson, 1955)\xspace}

\newcommand{\bA}{\mathbf{A}}
\newcommand{\bB}{\mathbf{B}}
\newcommand{\bE}{\mathbf{E}}
\newcommand{\bF}{\mathbf{F}}
\newcommand{\bJ}{\mathbf{J}}
\newcommand{\br}{\mathbf{r}}
\newcommand{\bx}{\mathbf{x}}
\newcommand{\hbi}{\mathbf{\hat i}}
\newcommand{\hbj}{\mathbf{\hat j}}
\newcommand{\hbk}{\mathbf{\hat k}}
\newcommand{\hbn}{\mathbf{\hat n}}
\newcommand{\hbr}{\mathbf{\hat r}}
\newcommand{\hbt}{\mathbf{\hat t}}
\newcommand{\hbx}{\mathbf{\hat x}}
\newcommand{\hby}{\mathbf{\hat y}}
\newcommand{\hbz}{\mathbf{\hat z}}
\newcommand{\hbphi}{\mathbf{\hat \phi}}
\newcommand{\hbtheta}{\mathbf{\hat \theta}}
\newcommand{\complex}{\mathbb{C}}
\newcommand{\ppr}[1]{\frac{\partial #1}{\partial r}}
\newcommand{\ppt}[1]{\frac{\partial #1}{\partial t}}
\newcommand{\ppx}[1]{\frac{\partial #1}{\partial x}}
\newcommand{\ppy}[1]{\frac{\partial #1}{\partial y}}
\newcommand{\ppz}[1]{\frac{\partial #1}{\partial z}}
\newcommand{\pptheta}[1]{\frac{\partial #1}{\partial \theta}}
\newcommand{\ppphi}[1]{\frac{\partial #1}{\partial \phi}}
\newcommand{\pp}[2]{\frac{\partial #1}{\partial #2}}
\newcommand{\ppp}[2]{\frac{\partial^2 #1}{\partial^2 #2}}
\newcommand{\pppp}[3]{\frac{\partial^2 #1}{\partial #2 \partial #3}}
\newcommand{\Div}{\ensuremath{\nabla\cdot}}
\newcommand{\Curl}{\ensuremath{\nabla\times}}
\newcommand{\curl}[3]{\ensuremath{\begin{vmatrix} \hbi & \hbj & \hbk \\ \partial_x & \partial_y & \partial_z \\ #1 & #2 & #3 \end{vmatrix}}}
\newcommand{\cross}[6]{\ensuremath{\begin{vmatrix} \hbi & \hbj & \hbk \\ #1 & #2 & #3 \\ #4 & #5 & #6 \end{vmatrix}}}
\newcommand{\eps}{\epsilon}
\newcommand{\grad}{\nabla}
\newcommand{\image}{\operatorname{im}}
\newcommand{\integers}{\mathbb{Z}}
\newcommand{\ip}[2]{\ensuremath{\left<#1,#2\right>}}
\newcommand{\lam}{\lambda}
\newcommand{\lap}{\triangle}
\newcommand{\Matlab}{\textsc{Matlab}\xspace}
\newcommand{\exers}[1]{\bigskip\noindent\textbf{Exercises} #1}
\newcommand{\fexer}[2]{\bigskip\noindent\textbf{Lesson #1, \##2}\quad }
\newcommand{\prob}[1]{\bigskip\noindent\textbf{#1} }
\newcommand{\pts}[1]{(\emph{#1 pts}) }
\newcommand{\epart}[1]{\medskip\noindent\textbf{(#1)}\quad }
\newcommand{\ppart}[1]{\,\textbf{(#1)}\quad }
\newcommand{\note}[1]{[\scriptsize #1 \normalsize]}
\newcommand{\MatIN}[1]{\mtt{>> #1}}
\newcommand{\onull}{\operatorname{null}}
\newcommand{\rank}{\operatorname{rank}}
\newcommand{\range}{\operatorname{range}}
\renewcommand{\P}{\mathcal{P}}
\newcommand{\real}{\mathbb{R}}
\newcommand{\trace}{\operatorname{tr}}
\renewcommand{\Re}{\operatorname{Re}}
\renewcommand{\Im}{\operatorname{Im}}
\newcommand{\Arg}{\operatorname{Arg}}

\newcommand{\comm}[2]{\item \emph{#1}:\, #2}

\renewcommand{\ln}[2]{\comm{line #1}{#2}}
\newcommand{\lnpage}[3]{\comm{line #1 \underline{on page #2}}{#3}}
\newcommand{\lns}[2]{\comm{lines #1}{#2}}
\newcommand{\lnspage}[3]{\comm{lines #1 \underline{on page #2}}{#3}}
\newcommand{\fg}[2]{\comm{Figure #1}{#2}}
\newcommand{\eqn}[2]{\comm{equation #1}{#2}}

\newcommand{\reply}[2]{
\medskip\medskip
\item  \begin{quote}
\emph{#1}
\end{quote}

\medskip
\noindent #2}


\title[Replies to Editor Goldberg's comments]{Replies to Editor Goldberg's initial comments on \\ \emph{Mass-conserving subglacial hydrology in PISM}}

\author{Ed Bueler}

\date{\today}

\begin{document}
\maketitle

\thispagestyle{empty}



\subsection*{Editor's comments and instructions}  \begin{quote}
\emph{This is an interesting manuscript and right for this journal. The graphics are easy to follow and it is for the most part well written.}

\emph{I have a few preliminary comments -- aside from typos that i point out, they should be regarded as suggestions, as the manuscript has not yet gone to review, but bear in mind i may return to them if not addressed, especially if echoed by referees. The length kept me from looking closely enough so there may be other typos.}
\end{quote}

\medskip
\noindent Thanks for the quick comments.  I can address them point-by-point.  Most of my effort went into revisions motivated by these comments; the rebuttals below are brief.   The line and equation numbering below refer to the original submitted manuscript.

\subsection*{Detailed replies}  
\begin{itemize}
\reply{Overall: This paper is very long, and I believe unduly so. This could impact the review process in that it may scare off potential referees, and the reviews themselves may take longer than usual. In the equation descriptions, there is a fair amount of discussion that could be considered review (e.g. 5.1), and the descriptions of the equations actually used are in some places long-winded (e.g. 5.3). Many of the details of the steady state analysis, while new, should be in an appendix.}{As we have a single model with few parameters which generalizes four major, apparently-disparate published models (i.e.~basically Tulaczyck, LeBrocq, Bartholomaus, and Schoof), there is no way to avoid length.

\medskip
But I agree.  I have followed your advice to put the steady state analysis in the appendix, and additional reductions make the total length two pages less.  Significantly, the reader gets to the end of the conclusion on page 20 now, as opposed to page 23 before.

\medskip
A reader I strongly trust on this, Martin Truffer, managed to get through the whole thing.  Yes, this shows his endurance, but his final comment was ``I am learning a lot reading this, especially a lot of context between these different subglacial drainage models. This is not just a good paper for PISM, it is really educational for any glaciologist.''  So I am going to trust that some readers, and perhaps even some reviewers, get all the way through, and appreciate the comprehensive aspects, the ``fair amount of discussion'' needed to make sense of the nutty state of the current literature.}

\reply{Also, something i am curious about -- could a more realistic treatment of till hydrology (with a closure for pore pressure) serve as a better-constrained regularization of the thickness equation?}{It would be nice to have a closure for pore pressure that relates it to cavity/conduit pressure.  Do you have one?  I.e.~do you know how to change the pressure of nearly-zero-conductivity saturated till 10 meters away from a conduit or a cavity of a known pressure?

\medskip
It must have a lot to do with motion of the till, not just motion of the water through immobile till.  As we want a model which does minimal harm to ice dynamics in PISM, it is more important that it have few parameters than that it have this particular process speculation, in my opinion.  People like Truffer and Tulaczyck are so far not speculating about a meaningfully-area-averaged rule, that I know of.  So I am sticking with a simpler idea, as in Tulaczyck: assume all the water goes in the till.  Then add a simple idea: decide on a capacity for the till; once the till is full the water goes in the transport network.  And the latter has a closure or two; see the paper.

\medskip
Regarding the role of till in regularizing the transportable water pressure equation:  There isn't such a role as far as I can tell.  The idea about englacial porosity as a regularization is that gravitational potential energy can be had by having the pressure do the work of pushing water up in the network.  For till you'd have to speculate that it is really viscously compressible or something, which is hard to believe (and hard to parameterize, though Clarke (1983) sort of did such a thing).

\medskip
Regarding the role of till in regularizing the water layer \emph{thickness} equation:  It is so diffusive that it doesn't really need it.  Not sure I am responding to your curiosity, so moving on \dots}

\reply{Detailed comments:
l65: i'm not sure, but i think ``drained'' refers to the case where the till is well-hydraulically connected to some aquifer of imposed pressure.}{Not sure about ``imposed pressure,'' but if you mean ``modeled pressure,'' then yes that is what we all mean by ``drained.''  That is exactly the meaning I want, and I am now using it throughout.  (Previously ``dumped'' was used in some places.)}

\reply{l117: the placing of the conduits can be quite general, down to the grid scale in hewitt 2013 i believe.}{No doubt, but the key phrases for the reader of this part are that the conduit models have ``no known continuum limit'' and the parameters don't have ``grid-spacing-independent meaning''.

\medskip
The current conduit models are not ``lattice models'' because they have a grid---that's just numerical PDEs.  They're ``lattice models'' because when you change the spacing between conduits you have to refit the coefficients by some essentially unknown procedure to get the same effect (but on the finer grid \dots).  Said a possibly better way: If Schoof and company knew the continuum limit they would write down a PDE (with spatial derivatives in the directions perpendicular to the conduits, not just along them).  Instead they are o.k.~with readers figuring that someday they will model all the conduits in detail.  As this won't happen for millenia, everyone is happy being hopeful.

\medskip
But for PISM we need the model to not blow up when the user decides to refine or coarsen by several factors of two.  We need a PDE or system thereof.  This drives the paper.}

\reply{l205: switch infinitesimal and infinite}{I take this as an indication of my bad writing.  I did mean these words the way they are used.  But I rewrote this to (hopefully) reduce confusion: ``Indeed, subglacial lakes of infinitesimal extent and infinite depth form at local minima of the hydraulic potential if [the diffusion term] is absent \dots''}

\reply{l242: *the* power-law?}{Yup.}

\reply{sentence beginning l371: awkward grammar}{Yes.  Simplified.}

\reply{l470: these seem important enough to state here}{I guess it is a judgement call.  I decided to remove the whole reference to those inequalities as not sufficiently important.  (They are still true, of course.)}

\reply{l833: one *might* regard?}{Yup.}

\reply{eq (57) -- are you being consistent with lower/uppercase for ice thickness?}{In this case the bed is flat and $H=h$.  As ``$H$'' appears in the relevant formulas (i.e.~only for overburden pressure), I've kept it the way it was.}

\reply{l1072: I believe this is true for conservation laws and TVD schemes -- but since the continuity equation is coupled to other physics (opening/closure), i'm not sure you can make this claim definitively}{That's why the relevant sentence only ``suggests'' the same conclusion for the higher-order schemes.}

\reply{eq (61): i don't know the exact result up to numerical constants and i'm not going to look it up, but i just want to make sure there is not a typo, i.e. the "2" looks out of place}{FIXME}

\reply{l1320-1326: how do these boundary conditions compare to those used in schoof et al 2012?}{FIXME}

\reply{comment on section 11: not clear on the convergence test. was each simulation actually allowed to equilibrate? if so, or if not, what was the stopping criterion? did you examine whether the model equilibrates to this state from a more distant point?}{FIXME}

\end{itemize}

\end{document}