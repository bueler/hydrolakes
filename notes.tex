\documentclass[12pt,final]{amsart}%default 10pt
%prepared in AMSLaTeX, under LaTeX2e
%\addtolength{\topmargin}{-0.15in}
%\addtolength{\textheight}{0.7in}
%\addtolength{\oddsidemargin}{-0.5in}
%\addtolength{\evensidemargin}{-0.5in}
%\addtolength{\textwidth}{1.2in}

\usepackage[margin=1in]{geometry}

\usepackage{natbib}

% hyperref should be the last package we load
\usepackage[pdftex,
                colorlinks=true,
                plainpages=false, % only if colorlinks=true
                linkcolor=blue,   % only if colorlinks=true
                citecolor=black,   % only if colorlinks=true
                urlcolor=magenta     % only if colorlinks=true
]{hyperref}

\usepackage{amssymb,alltt,verbatim,xspace,fancyvrb}
\usepackage{palatino}

% check if we are compiling under latex or pdflatex
\ifx\pdftexversion\undefined
  \usepackage[final,dvips]{graphicx}
\else
  \usepackage[final,pdftex]{graphicx}
\fi

\newcommand{\normalspacing}{\renewcommand{\baselinestretch}{1.1}\tiny\normalsize}
\newcommand{\tablespacing}{\renewcommand{\baselinestretch}{1.0}\tiny\normalsize}
\normalspacing

% math macros
\newcommand\CC{\mathbb{C}}
\newcommand{\DDt}[1]{\ensuremath{\frac{d #1}{d t}}}
\newcommand{\ddt}[1]{\ensuremath{\frac{\partial #1}{\partial t}}}
\newcommand{\ddx}[1]{\ensuremath{\frac{\partial #1}{\partial x}}}
\newcommand{\ddy}[1]{\ensuremath{\frac{\partial #1}{\partial y}}}
\newcommand{\ddxp}[1]{\ensuremath{\frac{\partial #1}{\partial x'}}}
\newcommand{\ddz}[1]{\ensuremath{\frac{\partial #1}{\partial z}}}
\newcommand{\ddxx}[1]{\ensuremath{\frac{\partial^2 #1}{\partial x^2}}}
\newcommand{\ddyy}[1]{\ensuremath{\frac{\partial^2 #1}{\partial y^2}}}
\newcommand{\ddxy}[1]{\ensuremath{\frac{\partial^2 #1}{\partial x \partial y}}}
\newcommand{\ddzz}[1]{\ensuremath{\frac{\partial^2 #1}{\partial z^2}}}
\newcommand{\Div}{\nabla\cdot}
\newcommand\eps{\epsilon}
\newcommand{\grad}{\nabla}
\newcommand{\ihat}{\mathbf{i}}
\newcommand{\ip}[2]{\ensuremath{\left<#1,#2\right>}}
\newcommand{\jhat}{\mathbf{j}}
\newcommand{\khat}{\mathbf{k}}
\newcommand{\nhat}{\mathbf{n}}
\newcommand\lam{\lambda}
\newcommand\lap{\triangle}
\newcommand\Matlab{\textsc{Matlab}\xspace}
\newcommand\RR{\mathbb{R}}
\newcommand{\Up}{\ensuremath{\operatorname{Up}}}
\newcommand\vf{\varphi}



\title[]{HYDROLAKES:  a minimal model of subglacial hydrology}

\author[]{Ed Bueler}


\begin{document}

\maketitle

\thispagestyle{empty}

%\setcounter{tocdepth}{1}
%\tableofcontents

\section{Introduction} 

Any reasonable model of the aquifer has at least these two elements: liquid water is conserved and water flows from high to low hydraulic potential (``head'').  Additionally, physical processes control the geometry of the aquifer/layer (e.g.~cavities open by sliding, cavities/channels close by creep, channels open by melting, sediment moves, \dots), but we do not model these here.  Instead we model water pressure by the highly-simplified assumption that the water pressure is equal to the overburden pressure, or is a fixed multiple thereof.\footnote{This project is a fork from the project that Sarah Child and Brad Booch did at McCarthy in 2012.  Don't want to forget that \dots}

The assumption water pressure is equal to the overburden pressure is justified if creep closure dominates over sliding or wall melt.

Water is conserved.  In fact, if we regard the water as being in a layer \citep{Clarke05}, whether thick or thin, we can keep track of its thickness $W(t,x)$ in one spatial dimension by the equation
\begin{equation} \label{eq:conserve}
W_t + Q_x = \Phi
\end{equation}
where $Q$ is the water flux, with SI units $\text{m}^2\,\text{s}^{-1}$, and $\Phi$ is a source term with units $\text{m}\,\text{s}^{-1}$.  The layer thickness $W$ here is only likely to be meaningful, however, if it is regarded as an average over a horizontal scale of tens to thousands of meters, because of the fine spatial variation which one is unlikely to be able to model.  We will attempt only to model spatially-averaged versions of water amount and water pressure.

Equation \eqref{eq:conserve} is a conservation statement.  Solve equation \eqref{eq:conserve} in the case $Q=0$ and $\Phi(x)$ is known.  (That is: no lateral flux of water.)  On the other hand, describe the steady state case where $W_t=0$; how does a ``flux boundary condition'' come in?

We might separate the water sources between the melt on the lower surface of the glacier and the en- or supra-glacial drainage origin,
  $$\Phi = \rho_w^{-1} \left(m + S\right)$$
where $\rho_w$ is the density of fresh liquid water, $m$ is the rate at which basal melting (refreeze) of ice adds (removes) water, and $S$ is the rate at which surface runoff or englacial drainage adds water.  But this split is not really critical; for the first part of this project let's take $\Phi$ to be constant, $\Phi=\Phi_0$.  Note $m$ and $S$ have units $\text{kg}\,\text{m}^{-2}\,\text{s}^{-1}$ while $\Phi$ has units $\text{m}\,\text{s}^{-1}$.  (Here the density is mass-per-volume and the fluxes $m,S$ are mass rates per area.  This gives $\Phi$ in the right units whether or not we have flow-line geometry.)

There is a hydraulic potential.  We will relate the water flux $Q$ in equation \eqref{eq:conserve} to the gradient of the hydraulic potential $\psi(t,x)$.  By definition, $\psi$ combines the actual water pressure $P(t,x)$ and the gravitational potential corresponding to that mass of water at the location on the bed of the glacier,
\begin{equation} \label{eq:oldpotential}
\psi = P + \rho_w g\, b.
\end{equation}
But really it should be
\begin{equation} \label{eq:potential}
\psi = P + \rho_w g\, (b+W).
\end{equation}
Here $z=b(x)$ is the bedrock elevation.  We will assume that both the ice thickness $H(x)$ and the bed elevation $b(x)$ are given, time-independent data.

Water flows from high to low hydraulic potential.  The simplest such is to assume a water sheet, which gives equation
\begin{equation}
Q = - \frac{K \, W}{\rho_w g} \psi_x
\label{eq:flux}
\end{equation}
(Clarke 2005).  Here, $\rho_w$ is the water density, $g$ the gravitational acceleration and $K$ the effective hydraulic conductivity.  The transmissivity of the system is really the product $KW$.  Notice that the system transmits more water for a given head gradient if either the holes are bigger ($K$ is larger) or the water sheet is thicker ($W$ is larger).  The main point here: By \eqref{eq:flux}, water flows from high to low fluid potential.  

From equations \eqref{eq:potential} and \eqref{eq:flux}, flow depends on both horizontal gradients in the water pressure and on the bedrock slope.  However, because the bedrock elevation comes from rough data in practice, the hydraulic potential is not actually very smooth.  The gradient $\nabla \psi$ will be seen to have very large spatial variability in practice.

Combine equations (1), (2), and (3) in a single equation, eliminating symbols $Q$ and $\psi$, to get a new equation $W_t + \dots = \dots$.  Get
  $$W_t - \frac{K}{\rho_w g} \left(W \left(P + \rho_w g\, b\right)_x\right)_x = \Phi$$

Water pressure $P$ equals the overburden pressure.  The ice is a fluid which has a pressure field of its own.  At the base of the ice we denote this as $P_i$, the \emph{overburden pressure}.  We make, without worrying further\footnote{You should rethink this issue if you are solving the Stokes equations in highly-variable ice flows.}, that this ice pressure is hydrostatic (``cryostatic''):
	$$P_i = \rho_i g H.$$
Here $\rho_i$ is the density of ice and $H$ is the ice thickness.

The extreme model is
\begin{equation}
P=P_i \label{extremeone}
\end{equation}
It has been used at least once in modeling Antarctica \citep{LeBrocqetal2009}.

Combine equations (1), (2), (3), and (4) in a single equation, eliminating symbols $Q$, $P$ and $\psi$.  Get  
  $$W_t - \frac{K}{\rho_w g} \left(W \left(\rho_i g H + \rho_w g\, b\right)_x\right)_x = \Phi$$
What are the correct boundary conditions in a one-dimensional version of the new equation?

In fact, let's now choose some data for the the ice sheet surface and bed elevation.  For East Antarctica I have written a mfile \texttt{getEAIStransect.m}, which calls modest data processing scripts \texttt{buildant.m,netcdf.m} to extract and plot surface elevation and bed elevation.  You will also need a 100Mb NetCDF file called \texttt{Antarctica\_5km\_dev1.0.nc}, which I can supply.  As an admittedly-modest alteration, modify these to plot overburden pressure for these geometry data.  Compute the hydraulic potential assuming extreme model \eqref{extremeone}, and then show $Q$ in some manner (arrows?).  If the flow is down the overburden pressure gradient, where will lakes occur?


Build a numerical model of the equation from Extreme Model 1.  It might be very helpful to consider the abstract form
	$$u_t + (v u)_x = f.$$
for $v(x)$ and $f(x)$ given.  How would you test your numerical scheme?  (\emph{Here we actually get our hands dirty in numerical methods.  Talk to me about finite difference schemes and boundary conditions.})




\small
\bibliography{ice_bib}  % generally requires link to pism/doc/ice_bib.bib
\bibliographystyle{agu}


\end{document}
