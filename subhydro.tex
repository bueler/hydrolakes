\documentclass[11pt,final]{amsart}
%prepared in AMSLaTeX, under LaTeX2e

\usepackage[total={6.2in,9.0in},top=1.2in,left=1.1in]{geometry}

\usepackage{natbib}

\usepackage{amssymb,alltt,verbatim,xspace,fancyvrb,color,empheq}
\usepackage{palatino}
\usepackage[sc]{mathpazo}
\usepackage[T1]{fontenc}

% check if we are compiling under latex or pdflatex
\ifx\pdftexversion\undefined
  \usepackage[final,dvips]{graphicx}
\else
  \usepackage[final,pdftex]{graphicx}
\fi

% hyperref should be the last package we load
\usepackage[pdftex,
                colorlinks=true,
                plainpages=false, % only if colorlinks=true
                linkcolor=blue,   % only if colorlinks=true
                citecolor=black,   % only if colorlinks=true
                urlcolor=magenta     % only if colorlinks=true
]{hyperref}

\newcommand{\normalspacing}{\renewcommand{\baselinestretch}{1.05}\tiny\normalsize}
\newcommand{\tablespacing}{\renewcommand{\baselinestretch}{1.0}\tiny\normalsize}
\normalspacing

\definecolor{myblue}{rgb}{.8, .8, 1}

\newcommand*\mybluebox[1]{%
\colorbox{myblue}{\hspace{1em}#1\hspace{1em}}}

\newcommand*\myredbox[1]{%
\colorbox{red}{\hspace{1em}#1\hspace{1em}}}

% math macros
\newcommand\bv{\mathbf{v}}
\newcommand\bV{\mathbf{V}}
\newcommand\bq{\mathbf{q}}
\newcommand\bQ{\mathbf{Q}}

\newcommand\CC{\mathbb{C}}
\newcommand{\DDt}[1]{\ensuremath{\frac{d #1}{d t}}}
\newcommand{\ddt}[1]{\ensuremath{\frac{\partial #1}{\partial t}}}
\newcommand{\ddx}[1]{\ensuremath{\frac{\partial #1}{\partial x}}}
\newcommand{\ddy}[1]{\ensuremath{\frac{\partial #1}{\partial y}}}
\newcommand{\ddxp}[1]{\ensuremath{\frac{\partial #1}{\partial x'}}}
\newcommand{\ddz}[1]{\ensuremath{\frac{\partial #1}{\partial z}}}
\newcommand{\ddxx}[1]{\ensuremath{\frac{\partial^2 #1}{\partial x^2}}}
\newcommand{\ddyy}[1]{\ensuremath{\frac{\partial^2 #1}{\partial y^2}}}
\newcommand{\ddxy}[1]{\ensuremath{\frac{\partial^2 #1}{\partial x \partial y}}}
\newcommand{\ddzz}[1]{\ensuremath{\frac{\partial^2 #1}{\partial z^2}}}
\newcommand{\Div}{\nabla\cdot}
\newcommand\eps{\epsilon}
\newcommand{\grad}{\nabla}
\newcommand{\ihat}{\mathbf{i}}
\newcommand{\ip}[2]{\ensuremath{\left<#1,#2\right>}}
\newcommand{\jhat}{\mathbf{j}}
\newcommand{\khat}{\mathbf{k}}
\newcommand{\nhat}{\mathbf{n}}
\newcommand\lam{\lambda}
\newcommand\lap{\triangle}
\newcommand\Matlab{\textsc{Matlab}\xspace}
\newcommand\RR{\mathbb{R}}
\newcommand\vf{\varphi}

\newcommand{\Wtil}{W_{\text{til}}}
\newcommand{\Wtilmax}{W_{\text{til}}^{\text{max}}}
\newcommand{\Wen}{W_{\text{en}}}
\newcommand{\zen}{z_{\text{en}}}
\newcommand{\Wtot}{W_{\text{tot}}}

\newcommand{\Wlij}{W^l_{i,j}}
\newcommand{\Wij}{W_{i,j}}
\newcommand{\Plij}{P^l_{i,j}}
\newcommand{\Pij}{P_{i,j}}
\newcommand{\Ylij}{Y^l_{i,j}}
\newcommand{\Yij}{Y_{i,j}}
\newcommand{\upp}[3]{\big<#1\big|_{#3}\,#2\big>}

\newcommand{\Nbreen}{Nordenski\"oldbreen\xspace}

\newcommand{\citeapos}[1]{\citeauthor{#1}'s [\citeyear{#1}]}


\title[]{A distributed numerical model of subglacial hydrology \\ in tidewater glaciers and ice sheets}

\author[]{Ed Bueler and Ward van Pelt}


\begin{document}
\graphicspath{{figs/}}

\scriptsize \hfill \today \normalsize
\vspace{0.5in}

\maketitle
\thispagestyle{empty}

%\setcounter{tocdepth}{1}
%\tableofcontents

\section{Introduction}

Any reasonable dynamical model of the liquid water underneath and within a glacier or ice sheet has at least these two elements: the mass of the water is conserved and the water flows from high to low hydraulic potential \citep{Clarke05}.  Beyond that there are many variations considered in the literature.  Physical processes may control the geometry of linked cavities \citep{Kamb1987} or conduits (channels) \citep{Nye1976} in which the water moves.  These processes may include the opening of cavities by sliding of the overlying ice past bedrock bumps (cavitation), melt on the walls of cavities and conduits, or the closure of cavities and conduits by creep \citep{Hewitt2011}.  Water could be exchanged with a macro-porous englacial system \citep{Bartholomausetal2011} or it could be stored in a porous till \citep{Tulaczyketal2000b}.

This paper describes a class of models for distributed systems of linked subglacial cavities, with additional storage of water in the pore spaces of subglacial till.  The cavities open by sliding of the ice over bedrock roughness and they close by ice creep, two physical processes which combine to determine the relationship between water amount and pressure.  Conduits are not included in the model, but, within the broad class considered here, we observe conduit-like behavior in large-scale cases where the pressure gradient approximates the overburden gradient.

Pressure is determined non-locally over each connected component of the hydrological system.  In particular, no functional relation between subglacial water amount and pressure is assumed in our model \citep[compare][]{FlowersClarke2002_theory}.  The presence of a macro-porous englacial system would implie that the subglacial water pressure solves an equation which is a parabolic approximation of the distributed pressure equation given in elliptic variational inequality form by \cite{Schoofetal2012}.  We adopt that approximation as a desirable regularization.  Thus, unlike the \cite{Schoofetal2012} model, in which a variational inequality for pressure must be solved at each time step, in this paper an incomplete model for a connection to englacial storage allows us to avoid solving an instantaneous distributed balance, thus easing implementation.

All water movement, whether stored in cavities or till, is carefully implemented in a mass-conserving manner.  FIXME: this is nontrivial because of transfer to/from till and because of free-boundaries

Wall melt in cavities is calculated diagnostically from the modeled flux and hydraulic gradient.  If included as a contribution to the mass conservation equation, however, as is well-known, the addition of wall melt generates an unstable distributed system.  While the pressure and amount of water in conduits could evolve by physical processes, the existing theory of conduits apparently requires their locations to be fixed a priori \citep{Hewittetal2012,PimentelFlowers2011,Schoofmeltsupply}.  We do not adopt such a mesh based, non-continuum model \citep[compare][]{Hewittetal2012}.  FIXME: we can adopt a dual system, but this is not yet implemented

The next section considers basic physical principles.  We generate a fundamental advection-diffusion form of the mass conservation equation.  In section \ref{sec:capacity} we consider cavity evolution and the till storage and transer mechanism.  Section \ref{sec:closures} reviews closures which determine the subglacial water pressure.  With these components laid out, in section \ref{sec:newmodel} we state the combined model and we identify its major parameters.  In section \ref{sec:steadyverif} the simplified equations which apply in steady state are given.  In steady state the pressure is a function of subglacial water amount.  We then compute an exact steady solution for subglacial water amount and pressure in the distributed (linked cavities) model, a useful tool for verification.  In section \ref{sec:num} we present numerical schemes for the general model.  These schemes are implemented in parallel in the Parallel Ice Sheet Model \citep{pism-user-manual}, with particular attention to time step restrictions and the treatment of advection.  The first numerical results, in which we show convergence under grid refinement, are for the verification case.  After that we show results for the Nordenskioldbreen tidewater glacier in Svalbard in section \ref{sec:results}.


\section{Elements of a continuum hydrology model} \label{sec:elements}

\subsection*{Mass conservation}  We assume that liquid water is incompressible and of constant density.  Thus the thickness of the layer of transportable water, denoted by $W(t,x,y)$, determines its mass.  Our statement of mass conservation (below) will describe the evolution of this thickness $W$.  Choosing to model subglacial hydrology using a water thickness as the state variable is not, by itself, a significant restriction on the physics.  Such a thickness, however, is only meaningful compared to observations if it is regarded as an average over a horizontal scale of tens to thousands of meters \citep{FlowersClarke2002_theory}.

In addition there is water stored locally in the pore spaces of till \citep{Tulaczyketal2000b} which is described by another effective thickness $\Wtil(t,x,y)$.  The limited storage capacity of the till gives an inequality $\Wtil \le \Wtilmax$.  Though the unknown parameter $\Wtilmax$ could vary in time and space according to a sediment evolution model, in the current paper it is a fixed nonnegative constant of order one to ten meters \citep{BBssasliding,TrufferEchelmeyerHarrison}.  Our theory allows setting $\Wtilmax=0$ to turn off the till storage mechanism.

While the thickness $W$ describes the amount of water in subglacial cavities and the connections between cavities, the system through which it can travel, the water in till pore spaces is much less mobile.  Our model necessarily includes, however, a parameterization for transfer $W \leftrightarrow \Wtil$.

The total effective thickness of the water at map-plane location $(x,y)$ and time $t$ is
\begin{equation}
\Wtot = W + \Wtil.  \label{eq:totalwater}
\end{equation}
We always assume the water thicknesses are nonnegative: $W \ge 0$ and $\Wtil \ge 0$.  The total thickness is conserved in our model.  In two map-plane dimensions the mass conservation equation is \citep{Clarke05}
\begin{equation} \label{eq:conserve}
\frac{\partial \Wtot}{\partial t} + \Div \bq = \frac{m}{\rho_w}
\end{equation}
where $\Div = (\partial/\partial x) + (\partial/\partial y)$, $\bq$ is the (vector) water flux (units $\text{m}^2\,\text{s}^{-1}$), $m$ is the total input to the subglacial layer (units $\text{kg}\,\text{m}^{-2}\,\text{s}^{-1}$) and $\rho_w$ is the density of fresh liquid water (units $\text{kg}\,\text{m}^{-3}$).  In our theory the water flux $\bq$ is concentrated within the subglacial layer, and we do not consider the possibility of lateral englacial transport.

We separate the water sources between melt on the lower surface of the glacier and supraglacial drainage.  Let $m_{\text{wall}}$ be the rate at which the cavity walls melt or refreeze.  Let $m_{\text{drain}}$ be the rate at which surface runoff drains to the subglacier layer.  The total input to the subglacial layer is
\begin{equation}
m = m_{\text{wall}} + m_{\text{drain}}. \label{eq:totalinput}
\end{equation}
The wall melt term is further parameterized in section FIXME below.

\subsection*{Hydraulic potential, Darcy flow, and effective pressure}  The water flux $\bq$ in equation \eqref{eq:conserve} is related to the gradient of a hydraulic potential $\psi(t,x,y)$ for the subglacial water.  This potential combines the actual subglacial water pressure $P(t,x,y)$ and the gravitational potential of the top of the water layer \citep{Goelleretal2013,Hewittetal2012}
\begin{equation} \label{eq:potential}
\psi = P + \rho_w g\, (b+W).
\end{equation}
Here $\rho_w$ is the water density ($\text{kg}\,\text{m}^{-3}$), $g$ is the acceleration of gravity ($\text{m}\,\text{s}^{-2}$), and $z=b(x,y)$ ($\text{m}$) is the bedrock elevation, which for simplicity is time-independent.

We have added the term ``$\rho_w g W$'' to the standard hydraulic potential formula $\psi_0 = P + \rho_w g b$ \citep[for example]{Clarke05} because differences in the potential at the \emph{top} of the subglacial water layer determine the driving potential gradient for a fluid layer.  Of course such differences may be small, in which case ignoring them may do no harm.  However, the  ``$\rho_w g W$'' term is most important when considering local minima of the hydraulic potential, where subglacial lakes of finite (not infinitesimal) extent and finite (not infinite) depth should form.  As we will see, this term makes the mass conservation equation diffusive.  When the water depth becomes substantial ($W\gg 1$), as it would be in a subglacial lake, this term keeps the modeled lakes from being singularities of the water thickness field.  Note that only the transportable water $W$, and not the total water $\Wtot$, is used to determine the potential.

Water flows from high to low hydraulic potential.  The simplest applicable expression of this property is a Darcy flux model for a water sheet following \cite{Clarke05}, namely
\begin{equation}  \label{eq:fluxearly}
\bq = - k \,W\, \grad \psi
\end{equation}
Here $k$ is the effective hydraulic conductivity.  More generally \cite{Schoofetal2012} suggests
\begin{equation}  \label{eq:flux}
\bq = - k\, W^\alpha\, |\grad \psi|^{\beta-2} \grad \psi
\end{equation}
for $\alpha \ge 1$, $\beta>1$, and a coefficient $k>0$ with units that depend on $\alpha$ and $\beta$.  Power-law form \eqref{eq:flux} is justified as an instance of a Manning or Darcy-Weisbach law \citep{Schoofetal2012}.  \cite{Clarke05} suggests $\alpha=1$ and $\beta=2$, to give \eqref{eq:fluxearly} above, \cite{CreytsSchoof2009} use $\alpha=3/2$ and $\beta=3/2$, \cite{Hewitt2011} uses $\alpha=3$ and $\beta = 2$, and \cite{Hewittetal2012} suggest $\alpha=5/4$ and $\beta=3/2$.  The current paper implements law \eqref{eq:flux} generally but uses the \cite{Clarke05} or \cite{Hewittetal2012} exponents in numerical experiments.

Ice in a glacier is a viscous fluid which has a stress field of its own.  The basal value of the downward normal stress is traditionally called the \emph{overburden pressure}, which we denote by $P_o$.  We make the shallow approximation that it is hydrostatic \citep{GreveBlatter2009}:
\begin{equation} \label{eq:hydrostatic}
  P_o = \rho_i g H.
\end{equation}
Here $\rho_i$ is the density of ice ($\text{kg}\,\text{m}^{-3}$) and $H$ is the ice thickness (m).  Because $P$ is nonnegative, and because the condition $P>P_o$ is presumed to cause the ice to lift and thus quickly lower the pressure back to overburden $P=P_o$ \citep{Schoofetal2012}, it follows that the pressure solution is subject to inequalities
\begin{equation}
0 \le P \le P_o. \label{eq:bounds}
\end{equation}
Though extreme cases can occur, such as where the ice is forced upward by a negative effective pressure \citep{Schoofetal2012}, our theory does not model any cases which  violate bounds \eqref{eq:bounds}.  We also define the \emph{effective pressure}
\begin{equation}
N = P_o - P  \label{eq:effective}
\end{equation}
which measures how much of the ice load is carried by the mineral (till or bedrock) base, as opposed to how much is carried by pressurized subglacial water.  Bounds \eqref{eq:bounds} imply $0 \le N \le P_o$.

\subsection*{Advection-diffusion decomposition}  Combining \eqref{eq:potential} and \eqref{eq:flux}, and separating the term proportional to $\grad W$, we get the flux expression
\begin{align}
  \bq &= - k  W^\alpha \left|\grad \left(P + \rho_w g (b+W) \right)\right|^{\beta-2} \grad \left(P + \rho_w g b\right)  \label{eq:firstfluxdecomp} \\
      &\qquad - \rho_w g k W^\alpha \left|\grad \left(P + \rho_w g (b+W) \right)\right|^{\beta-2} \grad W. \notag
\end{align}
The second term is proportional to the gradient of the water thickness $\grad W$.  This part of the flux acts diffusively in the mass conservation equation \eqref{eq:conserve}.  Because the subglacial water pressure generally scales with the ice overburden pressure $\rho_i g H$, the first flux term in \eqref{eq:firstfluxdecomp} will dominate in the common situation $|\grad H| \gg |\grad W|$.  In any case, decomposed flux \eqref{eq:firstfluxdecomp} describes a transport process in equation \eqref{eq:conserve} which has both a velocity field which varies in space and time and a second term which acts diffusively.  We will construct our conservative numerical scheme based on this understanding of how the flux is decomposed.  We will see later that in near-steady-state circumstances the part of the transport velocity which is proportional to $\grad P$ is also actually \emph{diffusive} in the mass conservation equation.  In conditions far from steady state, however, the direction of $\grad P$ is different from the direction $\grad W$, and the separation of velocity from diffusion is more clear.

To simplify the model slightly, the small thickness approximation $W\approx 0$ is made inside the absolute value signs in \eqref{eq:firstfluxdecomp}, namely
\begin{equation}
\left|\grad \left(P + \rho_w g (b+W) \right)\right| \approx \left|\grad \left(P + \rho_w g b \right)\right|.  \label{eq:Wsmall}
\end{equation}
This makes no change in the $\beta=2$ case.  Define the nonlinear effective hydraulic conductivity
\begin{equation}
K = k W^{\alpha-1} \left|\grad(P+\rho_w g b)\right|^{\beta - 2}. \label{eq:Kdefine}
\end{equation}
Note $K = k$ in the $\alpha=1$ and $\beta=2$ case covered by \eqref{eq:fluxearly}.  In terms of this nonlinear conductivity $K$ we write the velocity field and nonlinear diffusivity:
\begin{equation} \label{eq:vexpression}
  \bV = - K\, \grad \left(P + \rho_w g b\right), \qquad D = \rho_w g K W.
\end{equation}
Now expression \eqref{eq:firstfluxdecomp} is a cleaner advection-diffusion decomposition,
\begin{equation} \label{eq:qexpression}
  \bq = - K W \grad \psi = \bV\, W - D \grad W.
\end{equation}

In the rest of this paper we use either of the equivalent forms in \eqref{eq:qexpression} for the flux $\bq$.  The former with $\grad \psi$ emphasizes the relation between flow and pressure, while the latter with $\bV$ and $D$ is natural in numerical considerations for the mass conservation equation.

From equations \eqref{eq:conserve} and \eqref{eq:qexpression} we can now derive an advection-diffusion equation \citep{HundsdorferVerwer2010} for the evolution of the water amount:
\begin{equation} \label{eq:adeqn}
  \frac{\partial \Wtot}{\partial t} = - \Div\left(\bV\, W\right) + \Div \left(D \grad W\right) + \frac{m}{\rho_w}.
\end{equation}
There are distinct numerical schemes (section \ref{sec:num}) for the advection term $\Div\left(\bV\, W\right)$ and the diffusion term $\Div \left(D \grad W\right)$.  These different schemes impose time step restrictions of different magnitudes.  We will see in practice that equation \eqref{eq:adeqn} is advection-dominated in the sense that $|\bV W| \gg |D \grad W|$.  However, in near-steady conditions where the velocity $\bV$ can be almost proportional to $-\grad W$ there may be no clean separation of advection and diffusion.  Certainly the numerical schemes for advection and diffusion must be tested in combination.  This explains why we measure convergence behavior of the \emph{combined} numerical schemes in section \ref{sec:results}.

As is well known \citep{Clarke05}, the flux $\bq$ depends significantly on the ice surface slope because the ice overburden pressure dominates the subglacial water pressure \citep{Shreve1972}.  Therefore the gradient of the hydraulic potential frequently follows the ice surface gradient.  The pressure model in this paper also generates pressure fields with this property in some circumstances.  However, in a model like ours which depends on physical mechanisms for the opening and closing of cavities, the connection between $\bq$ and the surface slope is much less direct.

%FIXME:  revive text on enthalpy and wall melt and dissipation heating?  see phoenix.tex


\section{Capacity and geometry of the hydrologic system} \label{sec:capacity}

\subsection*{Capacity of the linked-cavity (distributed) system}  The evolution of the area-averaged thickness, also called the bed separation \citep{Bartholomausetal2011}, of the cavities in a distributed linked-cavity system \citep{Schoofetal2012} can be described as the sum of opening and closing processes \citep{Hewitt2011}.  Let $Y$ denote that bed separation, so that
\begin{equation}
\frac{\partial Y}{\partial t} = \mathcal{O}(|\bv_b|,Y) - \mathcal{C}(N,Y) \label{eq:hewittcapacity}
\end{equation}
where $\bv_b$ is the ice base (sliding) velocity.  Exactly as in \cite{Schoofetal2012} we choose an opening term based on cavitation
\begin{equation}
 \mathcal{O}(|\bv_b|,Y) = c_1 |\bv_b| (W_r - Y)_+. \label{eq:openingform}
\end{equation}
Here $W_r$ is a maximum roughness scale of the basal topography and $c_1$ is a constant; both of these constants must be indirectly constrained by observations in practice (see Section \ref{sec:results}).  Also, we denote $X_+= \max\{0,X\}$ for a real number $X$.  We choose a form for the closing term based on creep which is used by  \cite{Hewitt2011,Schoofmeltsupply,Schoofetal2012}:
\begin{equation}
\mathcal{C}(N,Y) = c_2 A N^3 Y. \label{eq:closingform}
\end{equation}
Here $A$ is the ice softness, $c_2$ is a constant which must be constrained by observations (see below), and we have used Glen exponent $n=3$ for concreteness.

Equation \eqref{eq:hewittcapacity} describes the evolution of the upper surface of the subglacial cavities.  The first term (cavitation) is always nonnegative if we use \eqref{eq:openingform}, but it is only positive where the bed separation is less than the roughness scale ($Y<W_r$).  The second term always represents closing because our modeled pressure will satisfy bounds \eqref{eq:bounds} so that $0\le N \le P_o$.  The opening and closing terms \eqref{eq:openingform} and \eqref{eq:closingform} satisfy the inequalities (2.5)--(2.7) in \cite{Schoofetal2012}.

The physical intuition behind a pressure model which combines \eqref{eq:hewittcapacity} with a Darcy flux relation like \eqref{eq:flux} and mass conservation \eqref{eq:conserve} is as follows.  If the cavity is larger than connected water sources can fill then the pressure should be lowered.  This pressure drop encourages inflow and, by \eqref{eq:closingform}, it also accelerates cavity closure.  Conversely, if local water sources exceed capacity then the increased pressure should push water out of the area and creep closure should be reduced.  This ``intuition'' requires a pressure closure, however, which is addressed in the next section.

Consider steady states of any model using \eqref{eq:hewittcapacity}.  These steady systems have a functional relationship between thickness $Y$ and effective pressure $N$ which comes from solving the steady condition for $N$:
\begin{equation}
\phantom{foo} \qquad \qquad \mathcal{O}(|\bv_b|,Y) = \mathcal{C}(N,Y) \qquad \qquad \text{in steady state}. \label{eq:hewittsteady}
\end{equation}
The implicit function theorem says that if $\partial\mathcal{C}/\partial N$ is nonzero then the effective pressure is determined from $Y$: $N=N(Y)$.  Such applies for all $Y> 0$.  In steady state with $Y=0$, however, the effective pressure is indeterminant according to \eqref{eq:hewittsteady}.

If $0<Y<W_r$ then a unique value $N(Y)>0$ is determined by \eqref{eq:hewittsteady}.  It may not, however, satisfy the bound $N(Y) \le P_o$ implied by \eqref{eq:bounds}.  That is, the sliding speed $|\bv_b|$ may be sufficiently large, and the bed separation $Y$ may be sufficiently below $W_r$, so that the cavitation opening rate exceeds the closing rate for any effective pressure $N$ satisfying $N\le P_o$.  In fact, for given values of $P_o$ and $|\bv_b|$, steady state equation \eqref{eq:hewittsteady} can only hold for a given $Y$ if
\begin{equation}
c_1 |\bv_b| (W_r - Y)_+ \le c_2 A P_o^3 Y. \label{eq:steadyOCbound}
\end{equation}
Condition \eqref{eq:steadyOCbound} must hold for the steady values of $Y$ over the entire domain.  If inequality \eqref{eq:steadyOCbound} applies then equation \eqref{eq:hewittsteady} determines a valid steady state effective pressure $N(Y)$ satisfying $0\le N(Y) \le P_o$ and thus a water pressure $P(Y)$ satisfying bounds \eqref{eq:bounds}.

\subsection*{Local till storage}  As noted already, some portion of the conserved water may occupy the pore spaces of a modeled layer of subglacial till.  On the one hand, evidence for such a till layer is strong: holes drilled to bed commonly find till under both glaciers and ice sheets \citep{Paterson}.  On the other hand will couple subglacial hydrology to the dynamics of glacier ice through a Mohr-Coulomb model of till deformation \citep{BBssasliding,SchoofTill,TrufferEchelmeyerHarrison}.

When water is transported laterally by the distributed system and it reaches unsaturated till then it also occupies the pore spaces of the till over/through/beside which it passes, and, when water is added to till that is already saturated, then this excess water becomes transportable.  These considerations imply that $\Wtil$ should evolve in a manner that allows transfer to and from the transportable thickness $W$, while also being constrained by the finite capacity, here given by $\Wtilmax$, of the till storage.

The simplest ordinary differential equation (in time) for $\Wtil$ which has the above properties is this equation with a constant rate parameter $\mu\ge 0$ and a transfer ratio $\tau\ge 0$:
\begin{equation}
\frac{\partial \Wtil}{\partial t} = \mu \left(\min\left\{\tau W,\Wtilmax\right\} - \Wtil\right) \label{eq:tilldynamics}
\end{equation}
As illustrated schematically in Figure \ref{fig:tillschema}, if $\tau W \ge \Wtilmax$ then \eqref{eq:tilldynamics} says that $\Wtil$ simply increases to $\Wtilmax$ over time, while if transport takes away water so that $\tau W < \Wtilmax$ then the till drains and $\Wtil$ decreases.  If $\tau=0$ then the model reduces to pure decay $\partial \Wtil/\partial t = - \mu \Wtil$.  Thus $0\le \Wtil \le \Wtilmax$ at all times for the solution of \eqref{eq:tilldynamics}.  Also the conservation statement \eqref{eq:conserve} or \eqref{eq:adeqn} for the total water amount $\Wtot=W+\Wtil$ implies that water drained from till is transportable in the distributed system.

\begin{figure}[ht]
\bigskip
\includegraphics[width=4.5in,keepaspectratio=true]{tillschema}
\bigskip
\caption{The till-stored water amount $\Wtil$ evolves by \eqref{eq:tilldynamics}.  The bounds $0\le \Wtil \le \Wtilmax$ are maintained in all situations.  If $\tau W < \Wtilmax$ then $\Wtil$ approaches $\tau W$, from either side.  If $\tau W \ge \Wtilmax$ then $\Wtil$ approaches $\Wtilmax$ from below.}
\label{fig:tillschema}
\end{figure}

FIXME: In the \texttt{-hydrology routing} model a different evolution is needed because the input water $m$ must go directly into till storage, as there is no ``$W$'' variable.  \cite{BBssasliding} includes a specific parameterization already, namely
\begin{equation}
\frac{\partial \Wtil}{\partial t} = \frac{m}{\rho_w} - C  \label{eq:tilldynamicsrouting}
\end{equation}
where $C>0$ is a decay factor that means that the base will drain without ongoing (small) input.  Actually the right-hand-side is only used to compute $\partial \Wtil/\partial t$ if $\Wtil < \Wtilmax$ or the right-hand-side is negative.  If $\Wtil = \Wtilmax$ and the right-hand-side is positive then we set $\partial \Wtil/\partial t=0$.

In section \ref{sec:num} we propose a stable numerical scheme for the combination of \eqref{eq:conserve} and \eqref{eq:tilldynamics} which ensures both discrete conservation of $\Wtot$ and bounds $0\le \Wtil \le \Wtilmax$ for the discretized values of $\Wtil$.  Appendix \ref{app:transportstorage} describes an abstract view of the transport-plus-storage model we consider here.


\section{Closures to determine pressure} \label{sec:closures}

At this point we do not know how to compute the water pressure $P$, or equivalently the effective pressure $N$, given the data of the problem, namely $b$, $H$, $m$, $|\bv_b|$, $W$, $\Wtil$, and $Y$.  The (apparent) state variables of the model so far are $W$, $\Wtil$, $Y$, and $P$, but the evolution equations listed so far, namely \eqref{eq:adeqn}, \eqref{eq:hewittcapacity}, and \eqref{eq:tilldynamics}, can only be simplified to three equations in these four unknowns.   A closure is needed.

\subsection*{Closures without cavity evolution}  We consider three simple closures which appear in the literature but which do not use cavity evolution equation \eqref{eq:hewittcapacity} or similar physics.  The resulting simplified models emerge as limiting cases of our more complete theory in steady state conditions.  These closures differ in their physical motivation and the mathematical form they imply for the mass conservation equation.  For simplicity we present each of these simplified closures without till storage, that is, with $\Wtil=0$ in previous equations.  (Equivalently with $\Wtilmax=0$.)  Also we state only the constant conductivity case, that is, we assume $\alpha=1$ and $\beta=2$ in equation \eqref{eq:flux}.

\renewcommand{\labelenumi}{\textbf{\Roman{enumi}.}}
\begin{enumerate}
\item Setting the pressure equal to the overburden pressure is the simplest closure \citep{Shreve1972}:
\begin{equation}
P = P_o.\label{eq:Pisoverburden}
\end{equation}
This model is sometimes used for ``routing'' subglacial water under ice sheets so as to identify subglacial lake locations \citep{Livingstoneetal2013TCD,Siegertetal2009}.  Straightforward calculations using equations \eqref{eq:conserve}, \eqref{eq:flux}, and \eqref{eq:Pisoverburden} show that the advection-diffusion form \eqref{eq:adeqn} has an ice-geometry-determined velocity,
\begin{equation}
  \frac{\partial W}{\partial t} = - \Div\left(\tilde\bV\, W\right) + \Div\left(\rho_w g k \,W\, \grad W\right) + \frac{m}{\rho_w}   \label{eq:PisoverConservation}
\end{equation}
where
\begin{equation}
\tilde\bV = - \rho_w g k \left[\frac{\rho_i}{\rho_w} \grad h + \left(1-\frac{\rho_i}{\rho_w}\right) \grad b\right].
\end{equation}

Because the approximation $W\ll H$ is usually accepted, so that the hydraulic potential is insensitive to the water layer thickness, i.e.~$\psi = P_o + \rho_w g b$ \citep{Siegertetal2009}, the diffusion term on the right of \eqref{eq:PisoverConservation} is usually not included.  With this common simplification, equation \eqref{eq:PisoverConservation} becomes a pure advection with a velocity $\tilde\bV$ which is independent of $W$.  It therefore possesses characteristic curves \citep{Evans} which, given ice geometry $b$ and $H$, are \emph{a priori} known trajectories of the water flow.  The more complete models we consider in this paper do not have such characteristic curves.  They have a flux which depends on the gradient of $W$, the quantity which is being advected.

Equation \eqref{eq:PisoverConservation} as stated, \emph{with} the diffusion term, is well-posed for positive initial and boundary values on $W$ \citep[compare][]{Hewittetal2012}.  Continuum solutions have finite water layer thickness at all times.  By contrast, equation \eqref{eq:PisoverConservation} without the diffusion term, as it usually appears in the literature, will exhibit continuum solutions with infinite concentration at every location where the simplified potential $\psi = P_o + \rho_w g b$ has a minimum.  In fact, applications using the simplified potential essentially only compute the characteristic curves \citep[i.e.~``pathways'',][]{Livingstoneetal2013TCD}.  Because the simplified model is not well-posed, numerical implementations are badly-behaved under grid refinement.

\medskip

\item At an almost opposite extreme in terms of the mathematical form, one might close the model by assuming that the water pressure is locally determined by the amount of water.  \cite{FlowersClarke2002_theory} propose
\begin{equation}
P_{FC}(W) = P_o \left(\frac{W}{W_{\text{crit}}}\right)^{7/2}. \label{eq:PofWFC}
\end{equation}
For Trapridge glacier \cite{FlowersClarke2002_trapridge} use $W_{\text{crit}}=0.1$.  (Figure \ref{fig:psteady-vb} below illustrates this function.)  One obvious concern with form \eqref{eq:PofWFC} is that $P_{FC}(W)$ can be arbitrarily larger than overburden pressure for large amounts of water ($W \gg W_{\text{crit}}$).  Equation \eqref{eq:PofWFC} is not, however, intended to apply when $W$ is actually large, such as in subglacial lakes.  From \eqref{eq:conserve}, \eqref{eq:flux}, and \eqref{eq:PofWFC} we get the equation:
\begin{equation}
  \frac{\partial W}{\partial t} = \Div\left((\rho_w g k \grad b)\, W\right) + \Div \left(k\,W \grad P_{FC}(W)\right) + \frac{m}{\rho_w}. \label{eq:PofWFCConservation}
\end{equation}

In the flat bedrock case $\grad b=0$ we see that \eqref{eq:PofWFCConservation} is a nonlinear diffusion.  Indeed, \cite{Schoofetal2012} observe that \eqref{eq:PofWFCConservation} generalizes the porous-medium equation $\partial W/\partial t = \grad^2 (W^\gamma)$ \citep{VazquezPME}.  The main idea in such a nonlinear diffusion, and a significant concern when considering applications of \eqref{eq:PofWFC}, is that the direction of the flux is $-\grad W$, while it would seem that the more physical model with $\bq \sim -\grad \psi$ would give flux directions from $-\grad W$ in many cases.

\medskip

\item Another simple closure by \cite{BBssasliding} uses a pressure function comparable to \eqref{eq:PofWFC} but different in detail.  Pressure is proportional to water amount but it is capped at a fixed fraction of overburden:
\begin{equation}
P_{BB}(W) = \lambda\,P_o \min\left\{1,\frac{W}{\Wtilmax}\right\}. \label{eq:PofWBB}
\end{equation}
(Figure \ref{fig:psteady-vb} below also illustrates this function.)  Values $\lambda=0.95$ and $W_{\text{crit}}=2$ m are used in \citep{BBssasliding}.  While this functional form is only motivated as a mechanism to model the yield stress of a till layer using the Mohr-Coulomb criterion \citep{TrufferEchelmeyerHarrison,Tulaczyketal2000b}, the ``obvious concern'' mentioned above is absent.  The current paper seeks to improve the Parallel Ice Sheet Model (PISM) by replacing closure \eqref{eq:PofWBB} with a mass conserving hydrology submodel which has a more physical mechanism for determining water pressure.
\end{enumerate}

\medskip
By constrast with these simple closures, in this paper we apply evolution model \eqref{eq:hewittcapacity} for the capacity of the distributed system.  However, in steady state conditions our theory\footnote{The same can be said for the \cite{Schoofetal2012} theory.} recovers a functional relation $P=P(W)$, as shown in Figure \ref{fig:psteady-vb}, but this relation is neither a power-law like \eqref{eq:PofWFC} nor piecewise-linear like \eqref{eq:PofWBB}.  Under steady conditions where the ice sliding velocity is zero, our theory also recovers \eqref{eq:Pisoverburden} and \eqref{eq:PisoverConservation}.  Our theory uses advection-diffusion decomposition \eqref{eq:adeqn} and thus in that mathematical sense it extends the ``routing'' model \eqref{eq:PisoverConservation}.  These connections are further exposed in section \ref{sec:steadyverif} below.


\subsection*{Full-cavity closure}  As already noted, equations  \eqref{eq:adeqn}, \eqref{eq:hewittcapacity}, and \eqref{eq:tilldynamics} do not yet determine the evolution of the state variables $W$, $\Wtil$, $Y$ and $P$.  However, \emph{requiring the subglacial layer to be full of water} is a closure for the subglacial pressure $P$.  We will adopt this ``full-cavity closure'' in our model:
\begin{equation}
W = Y.\label{eq:strongclosure}
\end{equation}

The consequences of this closure are explored at some length by \cite{Schoofetal2012}.  They describe the case where cavities are full as the ``normal pressure'' condition (e.g.~equation (4.13)).  The \cite{Schoofetal2012} model, which does not include till storage, is the model which comes from taking $\Wtilmax=0$ in the current paper.

AAA FIXME START

Equation \eqref{eq:strongclosure} allows us to eliminate $Y$ as a state variable.  In fact, if $W=Y$ then equations \eqref{eq:conserve} and \eqref{eq:hewittcapacity} combine to give
\begin{equation}
\mathcal{O}(|\bv_b|,W) - \mathcal{C}(N,W) + \frac{\phi}{\rho_w g}\frac{\partial P}{\partial t} + \Div \bq = \frac{m}{\rho_w}. \label{eq:initialformpressure}
\end{equation}
We can write this as an evolution equation for pressure $P$.  We write the flux $\bq$ in terms of the hydraulic potential $\psi$ and we use the nonlinear conductivity $K$ defined in equation \eqref{eq:Kdefine}:
\begin{equation}
\frac{\phi}{\rho_w g}\frac{\partial P}{\partial t} = \Div\left(K W \grad \psi\right) + \frac{m}{\rho_w} + \mathcal{C}(P_o-P,W) - \mathcal{O}(|\bv_b|,W). \label{eq:pressureequation}
\end{equation}

Because of the close relation between $P$ and $\psi$, we regard \eqref{eq:pressureequation} as a nonlinear parabolic equation for $P$.  Roughly-speaking, it is a diffusion for $P$ which is coupled to the advection-diffusion equation \eqref{eq:adeqn} for $W$.  In fact, now that $Y$ is eliminated, equations \eqref{eq:adeqn}, \eqref{eq:tilldynamics}, and \eqref{eq:pressureequation} can be identified as our major model equations for the model state variables $W$, $\Wtil$, $\Wen$, and $P$.  (See the final model statement \eqref{eq:bluebox} below.)

Though equation \eqref{eq:pressureequation} is a kind of stress balance, a form of conservation of momentum, there is no obvious way to guarantee that we actually conserve momentum because the enforcement of bounds \eqref{eq:bounds} on $P$ involves large and un-accounted forces \citep{Schoofetal2012}.  However, verifiable conservation of mass is essential in the applications of a subglacial hydrology model.  We carefully consider the numerical conservation of the total water mass $\Wtot$ in section \ref{sec:num}.

Consider the $\phi=0$ case of equation \eqref{eq:pressureequation},
\begin{equation}
0 = \Div\left(K W \grad \psi\right) + \frac{m}{\rho_w} + \mathcal{C}(P_o-P,W) - \mathcal{O}(|\bv_b|,W). \label{eq:ellipticpressure}
\end{equation}
The connection to englacial storage has been removed, which generates an \emph{elliptic} pressure equation.  It describes how pressures of different areas in the layer are related to each other, and to the layer geometry, at each instant by processes that act instantaneously.  Equation \eqref{eq:ellipticpressure} is the major pressure equation in \citet[equations (2.12), (4.17a)]{Schoofetal2012}.  Though we do not use equation \eqref{eq:ellipticpressure}, it provides understanding of the structure of our mathematical model.  Specifically, \cite{Schoofetal2012} show that the time-independent mathematical problem encompassing \eqref{eq:ellipticpressure}, constraints \eqref{eq:bounds}, and appropriate pressure boundary conditions can be written as an elliptic variational inequality \citep{KinderlehrerStampacchia}.  This same kind of variational inequality problem would be solved at each step of an \emph{implicit} time-stepping numerical implementation of the model we develop here, because we require bounds \eqref{eq:bounds} (as do \citep{Schoofetal2012}).  The numerical analysis of such problems, which also appear in other glaciological free boundary contexts \citep{SchoofStream,JouvetBueler2012}, can be addressed with a finite element approach \citep{Ciarlet}.

\subsection*{Regularization to reduce stiffness}  As has been noted by other authors \citep{Clarke2003,Schoofetal2012}, the differential equations for subglacial hydrology are \emph{stiff} \citep{AscherPetzold}.  Primarily  this means that the timescales for pressure evolution are short compared to the timescales of water movement.  This contrast can be stated in terms of the relative speeds of pressure waves and water transport \citep[Appendix A]{Clarke2003}.

This stiffness follows from the incompressibility of water and the non-distensibility (i.e.~hardness) of the ice and bedrock.  \cite{Clarke2003} addresses it by including in his subglacial water equation a relaxation (damping) parameter  ``$\beta$'' which is based on the small compressibility of water, but which is more than two orders of magnitude larger than the physical value.  \citeapos{Clarke2003} parameter $\beta$ appears in his equation exactly as the englacial porosity $\phi$ appears in equation \eqref{eq:pressureequation}, multiplying the pressure time derivative.  The effective physical porosity $\phi$ may be small, however, say $\sim 10^{-3}$ for temperate glaciers but orders of magnitude smaller for cold ice sheets, so \eqref{eq:pressureequation} remains stiff.  At the other extreme the \cite{Schoofetal2012} theory is ``infinitely stiff''; a model using pressure equation \eqref{eq:ellipticpressure} is differential-algebraic because it includes no time derivative \citep{AscherPetzold}.

AAA FIXME END

Now we can write the finalized pressure equation in our model:
\begin{align}
\frac{\phi_0}{\rho_w g} \frac{\partial P}{\partial t} = &\Div \left(K W \grad \psi \right) + \frac{m}{\rho_w} + \mathcal{C}(P_o-P,W) - \mathcal{O}(|\bv_b|,W) \label{eq:regpressureequation} \\
  &\quad - \mu \left(\min\left\{\tau W,\Wtilmax\right\} - \Wtil\right) \notag
\end{align}
Section \ref{sec:num} gives a quantitative analysis of the stiffness of \eqref{eq:regpressureequation} and its effect on the numerical implementation.


\section{A new subglacial hydrology model} \label{sec:newmodel}

\subsection*{Summary of equations and symbols}  The goals of the current work are the selection, implementation, verification, and demonstration of a subglacial hydrology model which is coupled to an existing three-dimensional ice dynamics model, and which is effectively parallelized.  The remainder of the paper will demonstrate that we have succeeded in the stated goals.

The complete set of evolution equations for the model is the following:
\begin{empheq}[box=\mybluebox]{align}
\frac{\partial W}{\partial t} &= - \Div\left(\bV\, W\right) + \Div \left(D \grad W\right) + \frac{m}{\rho_w} - T, \label{eq:bluebox} \\
\frac{\partial \Wtil}{\partial t} &= T, \notag \\
\frac{\phi_0}{\rho_w g} \frac{\partial P}{\partial t} &= \Div \left(K W \grad \psi \right) + \frac{m}{\rho_w} - T + c_2 A (P_o - P)^3 W - c_1 |\bv_b| (W_r - W)_+. \notag
\end{empheq}
The model also includes bounds on the state variables, namely $0\le W$, $0\le \Wtil \le \Wtilmax$, and $0 \le P \le P_o$.  We have already defined several derived functions:
\begin{align*}
K &= k W^{\alpha-1} \left|\grad(P+\rho_w g b)\right|^{\beta-2}, \\
\bV   &= - K \grad\left(P + \rho_w g b\right), \\
D     &= \rho_w g K W, \\
\psi &= P + \rho_w g (b + W), \\
T    &= \mu \left(\min\left\{W,\Wtilmax\right\} - \Wtil\right).
\end{align*}
In the last of these definitions we have named the right side of \eqref{eq:tilldynamics} as ``$T$'' for the ``transfer'' model between cavities and till.  As will be described in section \ref{sec:num}, we have implemented an explicit, adaptive time-stepping numerical scheme for equations \eqref{eq:bluebox} in the Parallel Ice Sheet Model.

\begin{table}[ht]
\caption{Functions used in hydrology model \eqref{eq:bluebox}, including symbol, units, and meaning.}
\begin{tabular}{l|l}
\hline
\emph{state functions} & \begin{tabular}{lll}
        $W$ & m \phantom{llllllllllll\,} & transportable subglacial water layer thickness \\
        $\Wtil$ & m & effective thickness of water stored in till \\
        $P$ & Pa & subglacial water pressure \\
        \end{tabular} \\ \hline
\emph{data functions} &  \begin{tabular}{lll}
        $b$ & m & bedrock elevation \\
        $H$ & m & ice thickness \\
        $m$ & $\text{kg}\,\text{m}^{-2}\,\text{s}^{-1}$ & total melt water input; $=m_{\text{wall}}+m_{\text{drain}}$ \\
        $P_o$ & Pa & overburden pressure; $= \rho_i g H$ \\
        $|\bv_b|$ & $\text{m}\,\text{s}^{-1}$ & ice sliding speed \\
        \end{tabular} \\ \hline
\end{tabular}
\label{tab:symbols}
\end{table}

\begin{table}[ht]
  \centering
  \caption{Physical constants and model parameters.  Default values are overridden in some experiments.}
  \begin{tabular}{lllp{3.0in}} 
    \textbf{Name} & \textbf{Default Value} & \textbf{Units} & \textbf{Description}\\
\hline
    $g$ & $9.81$ & m $\text{s}^{-2}$ & acceleration of gravity \\
    $\rho_i$ & $910$ & $\text{kg}\,\text{m}^{-3}$ & ice density \citep{GreveBlatter2009} \\
    $\rho_w$ & $1000$ & $\text{kg}\,\text{m}^{-3}$ & fresh water density \citep{GreveBlatter2009} \\
    \hline
    $A$ & $3.1689\times 10^{-24}$ & $\text{Pa}^{-3}\,\text{s}^{-1}$ & ice softness \citep{EISMINT96} \phantom{$\Big|$} \\
    $\alpha$ & $5/4$ & & power in flux formula  \citep{Hewittetal2012} \\
    $\beta$ & $3/2$ & & power in flux formula  \citep{Hewittetal2012} \\
    $c_1$ & $0.5$ & $\text{m}^{-1}$ & cavitation coefficient \\
    $c_2$ & $0.04$ & & creep closure coefficient \\
    $\phi_0$ & $0.005$ & & englacial porosity \citep{Bartholomausetal2011} \\
    $k$ & $0.01$ & $\text{m}^{2\beta-\alpha} \text{s}^{2\beta-3} \text{kg}^{1-\beta}$ & conductivity coefficient  \citep{Hewittetal2012} \\
    $\mu$ & $10^{-6}$ & $\text{s}^{-1}$ & rate constant for evolution of till storage \\
    % \mu = 1/week (about)
    $\tau$ & 10 & & transfer ratio for till storage model \\
    $W_r$ & $0.1$ & $\text{m}$ & roughness scale \citep{Hewittetal2012} \\
    $\Wtilmax$ & $2\phantom{\Big|}$ & $\text{m}$ & max.~water in till \citep{BBssasliding} \\
    \hline
  \end{tabular}
 \label{tab:constants}
\end{table}

Model equations \eqref{eq:bluebox} relate four kinds of symbols, namely the state functions and data (or coupling) functions listed in Table \ref{tab:symbols}, and the physical constants and model parameters in Table \ref{tab:constants}.  Only the state functions must be provided with initial values, and only they must be saved when stopping and restarting a time-dependent numerical model.  The data functions are either supplied by true observations or they are provided by other components an ice sheet model (e.g.~the stress balance in an ice dynamics model could provide $\bv_b$).  The model parameters in Table \ref{tab:constants} are all constant (i.e.~time- and space-independent) in the current paper but they could be allowed to vary spatially if desired.  Although default values must be chosen for the parameters, as is done in Table \ref{tab:constants}, exploration of the parameter space is essential; see section \ref{sec:results}.

\subsection*{Reduction to existing models}  Four reductions of model \eqref{eq:bluebox} can now be stated precisely:
\begin{itemize}

\item The zero till storage ($\Wtilmax=0$) and zero regularizing porosity ($\phi_0=0$) case of \eqref{eq:bluebox} is the model described by \cite{Schoofetal2012}.  The equations in this case are straightforward to write down:
\begin{empheq}[box=\mybluebox]{align}
\phantom{dsaf} \frac{\partial W}{\partial t} &= - \Div\left( K W \grad \psi \right) + \frac{m}{\rho_w}, \label{eq:schoofsmodel} \\
0 &= \Div \left( K W \grad \psi \right) + \frac{m}{\rho_w} + c_2 A (P_o - P)^3 W - c_1 |\bv_b| (W_r - W)_+.\phantom{dsaf}  \notag
\end{empheq}
The bounds $W \ge 0$ and $0 \le P \le P_o$ are unchanged.  Model \eqref{eq:bluebox} is a parabolic regularization of \eqref{eq:schoofsmodel} based on a notional connection to porous englacial storage, with a small porosity parameter $\phi_0$, and with additional till storage.

\item The $P=P_o$ limit of \eqref{eq:bluebox}, in which physical processes for the evolution of pressure are ignored and the pressure reverts to overburden, is the standard model for ``routing'' water to subglacial lakes under cold ice sheets \citep{Livingstoneetal2013TCD,Siegertetal2009}.  Assuming again that till storage is removed ($\Wtilmax=0$) then the model has only $W$ as a state variable.  The pressure is not an unknown, the parameterization of cavity evolution is inactive, and the pressure evolution equation is not needed.  There is now less stiffness because the model is an advection using a mostly-geometrically-determined velocity.    The single evolution equation is
\begin{empheq}[box=\mybluebox]{align}
\phantom{ldsfj} \frac{\partial W}{\partial t} &= - \Div\left(\bV\, W\right) + \Div \left(D \grad W\right) + \frac{m}{\rho_w}. \phantom{ldsfj} \label{eq:lakesmodel}
\end{empheq}
along with the bound $W \ge 0$ and additional definitions $\psi = P_o + \rho_w g (b + W)$, $K = k W^{\alpha-1} \left|\grad \psi\right|^{\beta-2}$, $\bV = - K \grad \left(P_o + \rho_w g b\right)$, and $D = \rho_w g K W$.  As noted in section \ref{sec:closures}, the $\alpha=1$ and $\beta=2$ case of this model routes water with a velocity which is determined entirely by ice and bedrock geometry.  Because of the way we have including $W$ into the hydraulic potential, so that large $W$ implies some diffusion, the model is well-posed and has continuous solutions.  If we restore till storage to this model then we use \eqref{eq:tilldynamicsrouting} for its evolution.

\item If transport is removed ($\bq=0$), notional englacial storage is removed ($\phi_0=0$), and if the approximation $P=\lambda P_o$ is assumed (for a constant $\lambda$ with typical value $0.98$), then the resulting model is essentially the nonconserving saturated till model described by \cite{BBssasliding}.  More precisely, we have $W=0$ so $\Wtot=\Wtil$.  We then determine the pressure by equation \eqref{eq:PofWBB}, and $\Wtil$ by equation \eqref{eq:tilldynamicsrouting}.

\item The non-distributed or lumped form of \eqref{eq:bluebox}, in which $\Div \bq = (q_{out} - q_{in})/L$ where $L$ is the length of the glacier, is essentially the porous glacier model of \cite{Bartholomausetal2011}.  The precise correspondence is explained in Appendix \ref{app:barth}.
\end{itemize}


\section{Steady states}  \label{sec:steadyverif}

\subsection*{Processes become decoupled in steady state}  The steady states of mathematical model \eqref{eq:bluebox} are worth considering for at least three reasons: (\emph{i}) the physical subglacial system is close to steady state much of the time, (\emph{ii}) we can more clearly identify vaious contributions to the water flux in steady state (equation \eqref{eq:qabstract} below), and (\emph{iii}) we can find an exact solution in two spatial variables.  We will also see that in steady state there is a function which relates the water pressure $P$ directly (locally) to the water amount $W$, though no such function exists in the time-dependent theory.

In this section we address only the $\alpha=1$ and $\beta=2$ case but the major conclusions apply for general powers $\alpha,\beta$.  Furthermore we remove till storage from the model by setting $\Wtilmax=0$.

Recall that the flux has two expressions $\bq = - k W \grad \psi = \bV W - D \grad W$.  Here is the steady form of model \eqref{eq:bluebox} written in terms of $\bV,\bq,W,P$:
\begin{align}
\bV &= - k \grad \left(P + \rho_w g b\right), \label{eq:Vsteady} \\
\bq &= \bV W - \rho_w g k W \grad W, \label{eq:qsteady} \\
0 &= - \Div \bq + \frac{m}{\rho_w}, \label{eq:masscontsteady} \\
0 &= c_2 A (P_o - P)^3 W - c_1 |\bv_b| (W_r - W)_+. \label{eq:openclosesteady}
\end{align}
Note that we also have bounds $W\ge 0$ and $0 \le P \le P_o$.

Relative to the time-dependent form \eqref{eq:bluebox}, processes have become decoupled in the steady state equations \eqref{eq:Vsteady}--\eqref{eq:openclosesteady}.  There are separate balances between the divergence of the flux and the water input on the one hand (i.e.~equation \eqref{eq:masscontsteady}), and the opening and closing processes on the other hand (i.e.~equation \eqref{eq:openclosesteady}).  Steady state equations \eqref{eq:Vsteady}--\eqref{eq:openclosesteady} are also stated by \cite{Schoofetal2012} model, where the decoupling is also noted.  Specifically, in the one-dimensional case the above equations reduce to (5.8) and (5.10) from \cite{Schoofetal2012}.


\subsection*{Functional relationship for pressure in steady state}  Equation \eqref{eq:openclosesteady} allows us to write the pressure $P=P(W)$ in steady state as a continuous function of the water amount $W$.  (This fundamental fact was already pointed out in considering the steady states of equation \eqref{eq:hewittcapacity}.)  Steady state is only possible if condition \eqref{eq:steadyOCbound} holds, but here with $W$ for $Y$:
\begin{equation}
c_1 |\bv_b| (W_r - W)_+ \le c_2 A P_o^3 W \qquad \text{ in steady state}. \label{eq:steadyboundfirst}
\end{equation}
In this and later formulas define the following scaled basal sliding speed which has units of pressure:
\begin{equation}
s_b =  \left(\frac{c_1 |\bv_b|}{c_2 A}\right)^{1/3}.  \label{eq:definesb}
\end{equation}
One may think of $s_b$ as a scale for the pressure drop associated to cavitation in steady state.  Then \eqref{eq:steadyboundfirst} is equivalent to
\begin{equation}
W \ge W_c := \frac{s_b^3}{s_b^3 + P_o^3} W_r  \qquad \text{ in steady state}. \label{eq:steadyboundsecond}
\end{equation}
This condition says that the water amount is above a critical level that depends on the sliding and the overburden pressure.  If the pressure effect sliding is large ($s_b \gg P_o$) then $W_c\approx W_r$.

If \eqref{eq:steadyboundfirst} or \eqref{eq:steadyboundsecond} holds then
\begin{equation}
P(W) = P_o - s_b \left(\frac{(W_r - W)_+}{W}\right)^{1/3} \qquad \text{ in steady state}.  \label{eq:PofWsteady}
\end{equation}
Note that in \eqref{eq:PofWsteady} we have $P(W_c)=0$.  Underpressure ($P=0$) with subcritical water amount ($W<W_c$) does not occur \emph{in steady state} though it can occur in nonsteady conditions.  Formula \eqref{eq:PofWsteady} may apply even if $W\ge W_r$, in which case the water pressure takes the overburden value $P = P_o$.  However, if $P_o=0$ then \eqref{eq:steadyboundfirst} implies that either $W\ge W_r$ or $|\bv_b|=0$.  This describes the values of $W$ and $|\bv_b|$ at ice margins where $H\to 0$ and therefore $P_o\to 0$.  These are all restrictions that apply when $W=Y$, that is, when the cavities are full, whereas partially-filled cavities can occur in the \cite{Schoofetal2012} theory.

\newcommand{\upto}{ \!\!\nearrow\! }
\newcommand{\downto}{ \!\searrow\! }
Figure \ref{fig:psteady-vb} shows the function $P(W)$ from \eqref{eq:PofWsteady} for several cases of sliding speed $|\bv_b|$.  Figure \ref{fig:psteady-Po} shows $P(W)$ for several cases of overburden pressure $P_o$.  We see that as the water amount reaches the roughness scale ($W\upto W_r$) the pressure rises rapidly to overburden ($P(W) \upto P_o$).  At the other extreme, we see that $P(W) \downto 0$ if $W \downto W_c$.  The curves $P(W)$ in Figures \ref{fig:psteady-vb} and \ref{fig:psteady-Po}, which describe steady state, do not include the interval $0\le W < W_c$ because such underpressure conditions are not achievable in steady state.

\begin{figure}[ht]
\includegraphics[width=3.5in,keepaspectratio=true]{psteady-vb}
\medskip
\caption{The steady state function $P(W)$ defined by equation \eqref{eq:PofWsteady} depends on the sliding speed.  Four cases are shown (in color) using a fixed uniform ice thickness of $H=1000$ m: $|\bv_b|=0$ m/a (blue), $10$ m/a (green), $100$ m/a (red), and $1000$ m/a (cyan).  The values of $W_c$ for these cases are indicated by black dots at $P=0$.  Relations \eqref{eq:PofWFC} (dashed black) and \eqref{eq:PofWBB} (dash-dot black) are shown with $W_{\text{crit}}=W_r=1$ m for comparison.}
\label{fig:psteady-vb}
\end{figure}

\begin{figure}[ht]
\includegraphics[width=3.5in,keepaspectratio=true]{psteady-Po}
\medskip
\caption{Function $P(W)$ defined by \eqref{eq:PofWsteady} also depends on overburden pressure $P_o=\rho_i g H$.  We fix $|\bv_b|=100$ m/a and $W_r=1$ m and consider four cases of uniform thickness $H=$ $2000$ m (blue), $1000$ m (green), $500$ m (red), and $200$ m (cyan).}
\label{fig:psteady-Po}
\end{figure}

Recall that \cite{FlowersClarke2002_theory} propose function $P_{FC}(W)$ (equation \eqref{eq:PofWFC}) for both steady and nonsteady circumstances.  Both functions $P(W)$ in \eqref{eq:PofWsteady} and $P_{FC}(W)$ are increasing and both relate the water pressure to the overburden pressure $P_o$.  However, while in \eqref{eq:PofWsteady} the relation to $P_o$ is additive, in \eqref{eq:PofWFC} it is a multiplicative scaling.  The power law form \eqref{eq:PofWFC} is not justified by the physical reasoning which led to equation \eqref{eq:PofWsteady}, even in steady state.   It would appear that any functional relationship $P(W)$ should also depend on the sliding velocity, as it does here, if cavitation is to influence the water pressure.  Of course the $W>W_{\text{crit}}$ case gives $P_{FC}(W) > P_o$ in \eqref{eq:PofWFC}, a problematic condition already noted by \cite{Schoofetal2012}, but this condition does not arise in \eqref{eq:PofWsteady}.  In conclusion, an important contrast between the \cite{FlowersClarke2002_theory} theory and the current paper is that we will not assume a relationship $P=P(W)$ in nonsteady conditions, even though such a relation does emerge from our theory in steady state.

\subsection*{Water velocity in steady state}  We now consider how the steady state water velocity $\bV$ depends on other quantities.  Equation \eqref{eq:PofWsteady} defines $P=P(W,P_o,s_b)$ while $\bV$ depends on $\grad P$.  Thus we need derivatives of $P$.  In steady state we have
\begin{equation}
\frac{\partial P}{\partial W} =
    \begin{cases}
      \text{undefined}, & W \le W_c, \\
      \frac{1}{3} s_b W_r W^{-4/3} (W_r - W)^{-2/3}, & W_c < W < W_r, \\
      \text{undefined}, & W = W_r, \\
      0, & W > W_r.
    \end{cases}  \label{eq:dPdWsteady}
\end{equation}
Note that the condition $W_c < W < W_r$ is identical to the normal pressure condition $0 < P < P_o$ in steady state.  Formula \eqref{eq:dPdWsteady} and Figures \ref{fig:psteady-vb} and \ref{fig:psteady-Po} agree that $\partial P / \partial W \to \infty$ as $W \upto W_r$.  

Equations \eqref{eq:Vsteady}, \eqref{eq:PofWsteady}, and \eqref{eq:dPdWsteady} now yield a formula for the velocity in steady state which applies in the normal pressure cases:
\begin{align}
\bV &= - k \grad P - \rho_w g k \grad b = - k \left[\frac{\partial P}{\partial P_o} \grad P_o + \frac{\partial P}{\partial s_b} \grad s_b + \frac{\partial P}{\partial W} \grad W\right] - \rho_w g k \grad b  \notag \\
    &= - k \left[\grad \left(P_o + \rho_w g b\right) - \left(\frac{W_r - W}{W}\right)^{1/3} \grad s_b + \frac{s_b W_r}{3 W^{4/3} (W_r - W)^{2/3}} \grad W\right]. \label{eq:Vsteadyexpand}
\end{align}
Formula \eqref{eq:Vsteadyexpand} helps us understand the steady state meaning of the advective flux ``$\bV W$'' in $\bq=\bV W - D \grad W$.  The direction of water velocity $\bV$ is determined by a combination of a geometric direction ($-\grad \left(P_o + \rho_w g b\right)$), a direction derived from spatial variations in the sliding speed ($\grad s_b$), and a diffusive direction ($-\grad W$).  Therefore a fraction of $\bV W$ is diffusive in steady state,\footnote{To our knowledge, the identification of the diffusiveness of this steady velocity is new to this paper.} in addition to the \emph{a priori} diffusive flux $- D \grad W$.

Let $\psi_o = P_o + \rho_w g b$ for simplicity.  We see that in steady state we can write the flux as a linear combination of gradients,
\begin{equation}
\bq = \bV W - D \grad W = - A_1 \grad \psi_o + A_2 \grad s_b - A_3 \grad W,  \label{eq:qabstract}
\end{equation}
with coefficients
\begin{equation}
A_1 = k W, \quad
A_2 = k \left(W_r - W\right)^{1/3} W^{2/3}, \quad
A_3 = \frac{k s_b W_r}{3 (W_r - W)^{2/3}}\, W^{-1/3} + \rho_w g k W.
\end{equation}
The first two coefficients $A_{1,2}$ go to zero as $W\to 0$.  However, $A_3$ remains large when $W\to 0$, even if $k$ is small, as long as sliding is occurring ($s_b > 0$).  Thus for low water amount and sustained sliding we should think of the water as diffusing in the layer; the apparently advective term ``$\bV W$'' is acting diffusively.  When the water thickness is much greater, specifically if it approximates the roughness scale ($W\approx W_r$), then $A_1$ is more significant, $A_2$ is again small, and $A_3$ may be significant in sliding cases ($s_b>0$).

Thinking more generally, it would be no surprise that when the ice thickness, bed elevation, sliding velocity, or water thickness fields is highly-variable in space then we can expect larger speeds $|\bV|$ in steady state.  Considering the various gradients in it, formula \eqref{eq:qabstract} reflects this general intuition.  Because the magnitude of the velocity determines the CFL time step restriction \citep{MortonMayers} associated to numerically solving the mass conservation equation, large variations in these spatial fields will also generally reduce the time steps taken by a numerical model.

\subsection*{The radial steady-state equations}  The above steady equations are the basis on which we now build a nearly-exact solution for $W$ and $P$ in the map-plane.  This solution, which is useful for verifying time-stepping or time-independent numerical schemes, will depend on the numerical solution of a scalar first-order ODE initial value problem, something we can do with high accuracy.  Exact solutions in one horizontal dimension (waves) also appear in \cite{Schoofetal2012}.

Consider steady state equations \eqref{eq:Vsteady}--\eqref{eq:masscontsteady}, and assume all quantities only depend on the radial coordinate $r = \sqrt{x^2+y^2}$.  One may eliminate $\bV$.  In the flat bed case the resulting pair of equations is
\begin{align}
q &= - k W\, \left(\frac{dP}{dr} + \rho_w g \frac{dW}{dr}\right), \label{eq:rsflux} \\
\frac{1}{r}\frac{d}{dr}\left(r\,q\right) &= \frac{m}{\rho_w}. \label{eq:rsconserve}
\end{align}

In the case of constant water input where $m = m_0 > 0$, which we assume for the exact solution, we can integrate \eqref{eq:rsconserve} from $0$ to $r$ and use symmetry ($q(0)=0$) to get
\begin{equation}
q(r) = \frac{m_0}{2\rho_w} \, r. \label{eq:qradial}
\end{equation}
On the other hand, equation \eqref{eq:PofWsteady} gives $P$ as a function of $W$ in steady state; here equation \eqref{eq:openclosesteady} has played the key role.  Suppose $h(r)$ is given so that $P_o(r)$ is also determined.  Assume that the scaled sliding speed $s_b(r)$ has a bounded derivative and that the solution $W(r)$ satisfies the normal pressure conditions $W_c < W < W_r$; both of these properties must be verified later for the constructed solution.  Now, by combining \eqref{eq:PofWsteady}, \eqref{eq:dPdWsteady}, \eqref{eq:rsflux}, and \eqref{eq:qradial} we can eliminate $q$ and $P$ to find
\begin{equation}
\omega_0\, r = - W\, \left(\frac{dP_o}{dr} - \frac{ds_b}{dr} \left(\frac{W_r - W}{W}\right)^{1/3} + \left(\frac{s_b W_r}{3 W^{4/3} (W_r - W)^{2/3}} + \rho_w g\right) \frac{dW}{dr}\right)  \label{eq:ODEfirst}
\end{equation}
where $\omega_0 = m_0 / (2 \rho_w k)$.

Equation \eqref{eq:ODEfirst} is a first-order ordinary differential equation (ODE) for $W(r)$.  To put it in the standard form expected by a numerical ODE solver we solve it for $dW/dr$:
\begin{equation}
\frac{dW}{dr} = \frac{\frac{ds_b}{dr} W (W_r - W) - \Big[\omega_0\, r W^{-1} + \frac{dP_o}{dr}\Big] W^{4/3} \left(W_r - W\right)^{2/3}}{\frac{1}{3} s_b W_r + \rho_w g W^{4/3} (W_r - W)^{2/3}}.
\label{eq:WradialODE}
\end{equation}
Equation \eqref{eq:WradialODE} has a constant solution $W(r)=W_r$.

\subsection*{Generating a nontrivial exact solution}  To generate a nontrivial exact solution, however, we will have a positive thickness of ice at the margin so that $P_o(L^-)>0$; Figure \ref{fig:Pexact} shows this small cliff at the margin.  We also assume that at the margin there is some sliding so that $s_b(L^-)>0$, and we require that $s_b(L^-) W_r > P_o(L^-)^3 Y_0$.  At the ice margin $r=L$ we have water pressure $P=0$ so $W(L)=W_c(L^-)$ is the boundary (initial) condition for the ODE.  The initial condition at $r=L$ also satisfies $W(L) < W_r$.  Then we integrate \eqref{eq:WradialODE} from $r=L$ to $r=0$.  The central water thickness value $W(0)$ is determined as part of the solution.

It is useful to have an ice cap geometry in which the surface gradient formula is simple so that $dP_o/dr$ in \eqref{eq:WradialODE} is also simple.  The plug flow, flat bed surface elevation solution of \cite{Bodvardsson} has this property.  Extending to the radial case, equations (23) and (24) of \citep{Bodvardsson} give ice thickness
\begin{equation}
H(r) = H_0 \left(1 - \frac{r^2}{R_0^2} \right) \label{eq:choosebodvardssonh}
\end{equation}
where $H(0)=H_0$ is the height of the center of the ice cap.  It follows that $dP_o/dr = - C r$ where $C=2\rho_i g h_0 R_0^{-2}$.  We choose $L=0.9 R_0$ and we note that $H(L)=0.19 h_0$ in \eqref{eq:choosebodvardssonh}.

The sliding speed could be determined by a model for stresses at the ice base and within the ice \citep{GreveBlatter2009}, but a coupled ice and water dynamics solution is too advanced for initial model verification.  Instead we choose a well-behaved sliding speed function which has no sliding near the ice cap center, and which increases in the radial direction:
\begin{equation}
|\bv_b|(r) = \begin{cases} 0, & 0 \le r \le R_1, \\
                           v_0  \left(\frac{r-R_1}{L-R_1}\right)^5, & R_1 < r \le L.
             \end{cases}  \label{eq:choosevb}
\end{equation}
It follows from \eqref{eq:definesb} and \eqref{eq:choosevb} that $ds_b/dr$ in \eqref{eq:WradialODE} is bounded and continuous on $0\le r \le L$.

Now we solve ODE \eqref{eq:WradialODE} with initial condition $W(L)=W_c(L)$ and the specific values in Table \ref{tab:verifconstants}.  We use adaptive numerical ODE solvers, both a Runge-Kutta 4(5) Dormand-Prince method and a variable-order stiff solver, with relative tolerance $10^{-12}$ and absolute tolerance $10^{-9}$, and with essentially identical results.  Modest stiffness \citep{AscherPetzold} of ODE \eqref{eq:WradialODE} is observed at $r\approx R_1$.  The result $W(r)$ is shown in Figure \ref{fig:Wexact}.

Because equations \eqref{eq:choosebodvardssonh} and \eqref{eq:choosevb} imply a pressure functional relation $P=P(W,r)$ from \eqref{eq:PofWsteady}, we can also show in Figure \ref{fig:Wexact} the regions of the $r,W$ plane which correspond to overpressure, normal pressure, and underpressure.  We see that $W(r)$ is in the normal pressure region as $r$ decreases from $r=L$ to $r=R_1$.  At $r=R_1$ the function $W(r)$ switches into the overpressure case because there is no sliding.  Figure \ref{fig:Pexact} shows the corresponding pressure solution $P(r)=P(W(r))$ from \eqref{eq:PofWsteady}.

\begin{table}[ht]
  \centering
  \caption{Constants used in constructing the exact solution.}
  \begin{tabular}{lllp{3.0in}}
    \textbf{Name} & \textbf{Value} & \textbf{Units} & \textbf{Description}\\
\hline
    $\alpha$ & $1$ & & power in flux \\
    $\beta$  & $2$ & & power in flux \\
    $H_0$ & $500$ & m & ice cap center thickness \\
    $k$   & $0.01/(\rho_w g)$ & $\text{m}^3\,\text{s}\,\text{kg}^{-1}$ & hydraulic conductivity; FIXME: 4 orders of magnitude smaller than in \cite{Hewittetal2012} \\
    $L$   & $22.5$& km & $=0.9 R_0$; actual ice cap margin \\
    $m_0$ & $0.2\rho_w$ & $\text{kg}\,\text{m}^{-2}\,\text{a}^{-1}$ & constant water input rate; $= 20 \,\text{cm}\,\text{a}^{-1}$ \\
    $R_0$ & $25$  & km & ideal ice cap radius \\
    $R_1$ & $5$   & km & radial location $r=R_1$ of onset of sliding \\
    $v_0$ & $100$ & $\text{m}\,\text{a}^{-1}$ & sliding speed scale \\
    \hline
  \end{tabular}
 \label{tab:verifconstants}
\end{table}

\begin{figure}[ht]
\includegraphics[width=3.5in,keepaspectratio=true]{exact-W-plot-onu}
\caption{An exact radial, steady solution for water thickness $W(r)$ (dashed).  In $r$-versus-$W$ space the overpressure (O), normal pressure (N), and underpressure (U) regions are determined by ice geometry and sliding velocity (solid curves; see text).}
\label{fig:Wexact}
\end{figure}

The reason for stiffness near $R_1$ is that as the sliding goes to zero the cavitation also goes to zero.  Because creep closure balances cavitation in steady state, effective pressure also goes to zero ($P\to P_o$).  The remaining active mechanisms in the model are the variable overburden pressure and the rate of water input.  They must exactly balance.  In fact, in this case \eqref{eq:WradialODE} reduces to the much simpler form
\begin{equation}
\frac{dW}{dr} = - \frac{\varphi_o r W^{-1} + \frac{dP_o}{dr}}{\rho_w g}. \label{eq:WradialODEnoslide}
\end{equation}
Though we have not derived it this way, Equation \eqref{eq:WradialODEnoslide} is the steady radial form of the mass conservation equation under the ``$P=P_o$'' closure, namely equation \eqref{eq:PisoverConservation}.

In equation \eqref{eq:WradialODEnoslide} we see that $dW/dr=0$ if $W$ satisfies $W = - \varphi_0 r / (dP_o/dr)$.  In our case with geometry \eqref{eq:choosebodvardssonh} this reduces to a constant value $W=\tilde W= 0.21764$ m because $\Phi_0$ is constant and $dP_o/dr$ is linear in $r$.  Both numerical ODE solvers used here confirm that $W(r)$ is asymptotic to this constant value $\tilde W$ as $r\to 0$, and that $W(r)\approx \tilde W$ within about 1\% on all of $0\le r \le R_1$.  This is seen in Figure \ref{fig:Wexact}.

\begin{figure}[ht]
\includegraphics[width=3.5in,keepaspectratio=true]{exact-P-plot}
\caption{An exact radial, steady solution pressure $P(r)$ (dashed) and overburden pressure $P_o$ (solid).}
\label{fig:Pexact}
\end{figure}


\section{Numerical schemes}  \label{sec:num}

\subsection*{Discretization of the mass conservation equation}  Mass conservation equation \eqref{eq:adeqn}, which is part of the combined mathematical model \eqref{eq:bluebox}, will be discretized by an explicit, conservative finite difference method.   A centered, second-order scheme will be applied to the diffusion part.  A pair of schemes for the advection part will be compared, namely first-order upwinding and a higher-order flux-limited upwind-biased method.

We first consider stable time steps.  Stability for either advection scheme occurs with a time step $\Delta t \le \Delta t_{\text{CFL}}$ where
\begin{equation}
\Delta t_{\text{CFL}} \left(\frac{\max |u|}{\Delta x} + \frac{\max |v|}{\Delta y}\right) = \frac{1}{2}. \label{eq:dtCFL}
\end{equation}
Here $\bV=(u,v)$ is the water velocity, as in previous sections.  Because of the additional diffusion process, for stability the time step should also satisfy $\Delta t \le \Delta t_{W}$  where \citep{MortonMayers}
\begin{equation}
\Delta t_W\, 2 \max D \left(\frac{1}{\Delta x^2} + \frac{1}{\Delta y^2}\right) = \frac{1}{2}. \label{eq:dtDIFFW}
\end{equation}
The condition $\Delta t \le \min\{\Delta t_{\text{CFL}}, \Delta t_W\}$ is sufficient for stability and convergence of the overall scheme for \eqref{eq:adeqn}.  Indeed, Appendix \ref{app:positivestable} shows that this is sufficient for the first-order upwind scheme, while standard theory suggests the same conclusion for the higher-order flux-limited advection scheme \citep{HundsdorferVerwer2010}.

We can understand the scale of these restrictions better by considering an example using the parameter values in Table \ref{tab:constants}, especially $\alpha=5/4$, $\beta=3/2$, $k=0.01$, and $W_r=0.1$.  We ran the model on a $\Delta x = \Delta y = 250$ m grid with initial condition $W=0$ to approximate steady state for \Nbreen ice and bedrock geometry, a hypothesized water input distribution with average value about 1 m $\text{a}^{-1}$, and a glacier-wide constant sliding rate of $50$ m $\text{a}^{-1}$.  The result is that the maximum (over space) water speed $|\bV| = \max |u| = \max |v|$ is about $0.2$ m $\text{s}^{-1}$, with temporal variation in this value, and that the diffusivity $D$ has maximum value (over space) that varies in time: $0.1 \le \max D \le 5 \,\text{m}^2\,\text{s}^{-1}$.  Thus the advective restriction \eqref{eq:dtCFL} is $\Delta t_{\text{CFL}} \approx 300\,\text{s} \approx 10^{-5}\,\text{a}$.  The diffusive restriction \eqref{eq:dtDIFFW} with $\max D=1\,\text{m}^2\,\text{s}^{-1}$ as a typical value is $\Delta t_W \approx 8000\,\text{s} \approx 2.5 \times 10^{-4}\,\text{a}$.   That is, $\Delta t_W \approx 25 \Delta t_{CFL}$.

This example suggest that, unless the global peak velocity is unusually slow, or unless deep subglacial lakes develop so that $D = \rho_w g K W$ is large and $\Delta t_W$ is correspondingly small, the diffusive time scale is significantly longer than the CFL time scale for a $250$ m grid.  However, the scaling $\Delta t_W = O(\Delta x^2)$ versus $\Delta t_{CFL} = O(\Delta x^1)$ makes it clear that refining or coarsening the grid may change the controlling time step.  We will see below, however, that the time step restriction associated to an explicit time-stepping method for the pressure equation is typically even shorter.  By contrast, if implicit time-stepping is used for the pressure equation \cite{Hewittetal2012,Schoofetal2012} then the time scales $\Delta t_W, \Delta t_{CFL}$ addressed here are the only restrictions.

\begin{figure}[ht]
\centering
\includegraphics[width=2.5in,keepaspectratio=true]{diffstencil}
\bigskip
\caption{Numerical schemes \eqref{eq:Wfd} and \eqref{eq:Pfd} use a grid-point-centered cell.  Velocities, diffusivities, and fluxes are evaluated at staggered grid locations (triangles at centers of cell edges denoted $e,w,n,s$).  State functions $W,P$ are evaluated at regular grid points (diamonds).}
\label{fig:stencil}
\end{figure}

To set notation, suppose our rectangular computational domain has $M_x \times M_y$ gridpoints $(x_i,y_j)$ with uniform spacing $\Delta x,\Delta y$.  Let $\Wlij \approx W(t_l,x_i,y_j)$ and $\Plij \approx P(t_l,x_i,y_j)$ be the numerical approximations.  Recall that $\bV$ is determined from pressure and bed elevation; see \eqref{eq:bluebox}.  We will compute velocity components and flux components at the staggered (cell-face-centered) points shown in Figure \ref{fig:stencil}.  We compute these values based on centered finite difference approximations of equations \eqref{eq:vexpression} and \eqref{eq:qexpression}.

We use ``compass'' indices such as $u_e = u_{i+1/2,j}$ for the ``east'' staggered component.  Similarly we use compass indices for staggered grid values of the water layer thickness, and these are computed by averaging regular grid values:
\begin{equation}
W_e = (W_{i,j}^l + W_{i+1,j}^l)/2, \qquad W_n = (W_{i,j}^l + W_{i,j+1}^l)/2. \label{eq:stagW}
\end{equation}
In this case we can compute the ``west'' and ``south'' values by shifting indices: $W_w = W_e\big|_{(i-1,j)}$ and $W_s = W_n\big|_{(i,j-1)}$.  Thus there are only two distinct staggered grid values (e.g.~east and north) to compute per regular grid location $(x_i,y_j)$.
The nonlinear effective conductivity $K=K(W,\grad P,\grad b)$ from \eqref{eq:Kdefine} is also needed at staggered locations.  As a notational convenience define $R=P+\rho_w g b$ and define these staggered approximations of $|\grad(P+\rho_w g b)|^2$ \citep[compare][]{Mahaffy}:
\begin{align*}
\Pi_e &= \left|\frac{R_{i+1,j}-R_{i,j}}{\Delta x}\right|^2 + \left|\frac{R_{i+1,j+1}+R_{i,j+1} - R_{i+1,j-1}-R_{i,j-1}}{4\Delta y}\right|^2, \\
\Pi_n &= \left|\frac{R_{i+1,j+1}+R_{i+1,j} - R_{i-1,j+1}-R_{i-1,j}}{4\Delta x}\right|^2 + \left|\frac{R_{i,j+1}-R_{i,j}}{\Delta y}\right|^2.
\end{align*}
Thereby define
\begin{equation}
K_e = k W_e^{\alpha-1} \Pi_e^{(\beta-2)/2}, \qquad K_n = k W_n^{\alpha-1} \Pi_n^{(\beta-2)/2}.  \label{eq:stagK}
\end{equation}
When the power $\beta-2$ is negative we avoid division by zero and cap the conductivities: $K_e = 1000 k$ if $\Pi_e = 0$ and similarly for $K_n$.

We can find the velocity components needed at staggered locations by differencing:
\begin{align}
u_e &= - K_e \left(\frac{P_{i+1,j}-P_{i,j}}{\Delta x} + \rho_w g \frac{b_{i+1,j}-b_{i,j}}{\Delta x}\right),  \label{eq:velocitycomp} \\
v_n &= - K_n \left(\frac{P_{i,j+1}-P_{i,j}}{\Delta y} + \rho_w g \frac{b_{i,j+1}-b_{i,j}}{\Delta y}\right). \notag
\end{align}
Similarly for diffusivity we have
\begin{equation}
D_e = \rho_w g K_e W_e, \qquad D_n = \rho_w g K_n W_n.  \label{eq:diffusivitycomp}
\end{equation}
We get the remaining staggered-grid quantities by shift:
\begin{align*}
u_w &= u_e\big|_{(i-1,j)}, \qquad v_s = v_n\big|_{(i,j-1)}, \\
K_w &= K_e\big|_{(i-1,j)}, \qquad K_s = K_n\big|_{(i,j-1)}, \\
D_w &= D_e\big|_{(i-1,j)}, \qquad D_s = D_n\big|_{(i,j-1)}.
\end{align*}

Now we define $Q_e(u_e)$, $Q_w(u_w)$, $Q_n(v_n)$, and $Q_s(v_s)$ as the face-centered (staggered-grid) normal components of the advective flux $\bV W$.  These quantities are described in more detail in the next subsection.  They use only the staggered velocity component but there is upwinding to determine which $W$ value(s) are used.  Here is the scheme for equation \eqref{eq:adeqn} using \eqref{eq:stagK} and \eqref{eq:velocitycomp} above:
\begin{align}
 &\frac{W_{i,j}^{l+1} - W_{i,j}^l}{\Delta t} = - \frac{Q_e(u_e) - Q_w(u_w)}{\Delta x} - \frac{Q_n(v_n) - Q_s(v_s)}{\Delta y} \label{eq:Wfd} \\
      &\hspace{-0.5in} + \frac{D_e (W_{i+1,j}^l - \Wlij) - D_w (\Wlij - W_{i-1,j}^l)}{\Delta x^2} + \frac{D_n (W_{i,j+1}^l - \Wlij) - D_s (\Wlij - W_{i,j-1}^l)}{\Delta y^2}  + \frac{m_{ij}}{\rho_w} - T_{ij}^l \notag
\end{align}
where $T_{ij}^l=T(W_{i,j}^l,{\Wtil}_{i,j}^{l+1})$ is the right side of \eqref{eq:tilldynamics}.

BBB FIXME START

Assuming no error in the flux components $Q$, the local truncation error \citep{MortonMayers} of scheme \eqref{eq:Wfd} would be $O(\Delta t^1 + \Delta x^2 + \Delta y^2)$ as an approximation of \eqref{eq:adeqn}.  The actual truncation error depends on the nature of the discrete fluxes, which we address next.

It is useful to rewrite scheme \eqref{eq:Wfd} in ``update'' form, and we should address the englacial storage update.  Let $\nu_x = \Delta t/\Delta x$, $\nu_y = \Delta t/\Delta y$, $\mu_x = \Delta t/\Delta x^2$, and $\mu_y = \Delta t/\Delta y^2$.  The following is equivalent to \eqref{eq:Wfd}:
\begin{align}
 (W+\Wen)_{i,j}^{l+1} &= (\Wen)_{i,j}^l + W_{i,j}^l - \nu_x \left(Q_e(u_e) - Q_w(u_w)\right) - \nu_y \left(Q_n(v_n) - Q_s(v_s)\right) \label{eq:Wupdate} \\
  &\qquad + \mu_x \left(D_e (W_{i+1,j}^l - \Wlij) - D_w (\Wlij - W_{i-1,j}^l)\right) \notag \\
  &\qquad + \mu_y \left(D_n (W_{i,j+1}^l - \Wlij) - D_s (\Wlij - W_{i,j-1}^l)\right) + \Delta t \frac{m_{ij}}{\rho_w}. \notag
\end{align}

BBB FIXME END


\subsection*{Fluxes in the mass conservation equation}  Appendix \ref{app:fluxlimiters} reviews some of the now-standard theory of flux-limiters for transport schemes.  Here we state the two flux-discretization alternatives which we actually test as flux discretizations for scheme \eqref{eq:Wupdate}.  These choices are choices of functions $\Psi$ from Table \ref{tab:fluxlimiters} in Appendix \ref{app:fluxlimiters}.  The methods we try are first-order upwind and the Koren flux-limiter \citep{HundsdorferVerwer2010}.

Recall that the velocity has components $\bV=(u,v)$ which are evaluated at these staggered locations following \eqref{eq:velocitycomp}.  Also recall that the flux at the staggered grid location $(x_{i+1/2},y_j)$ is denoted ``$Q_e(u_e)$'' and that the flux at $(x_i,y_{j+1/2})$ is denoted ``$Q_n(v_n)$.''  To ensure conservation we must have a single formula for $Q_{i+1/2,j}$ whether this flux is ``$Q_e$'' for $(x_i,y_j)$ or ``$Q_w$'' for $(x_{i+1},y_j)$; similar comments apply to ``$Q_n$'' versus ``$Q_s$'':
\begin{align}
Q_e(u_e) &= \begin{cases} u_e \left[W_{i,j} + \Psi(\theta_{i}) (W_{i+1,j} - W_{i,j})\right], & u_e \ge 0, \\ u_e \left[W_{i+1,j} + \Psi\left((\theta_{i+1})^{-1}\right) (W_{i,j} - W_{i+1,j})\right], & u_e < 0, \end{cases} \label{eq:adfluxes} \\
Q_n(v_n) &= \begin{cases} v_n \left[W_{i,j} + \Psi(\theta_{j}) (W_{i,j+1} - W_{i,j})\right], & v_n \ge 0, \\ v_n \left[W_{i,j+1} + \Psi\left((\theta_{j+1})^{-1}\right) (W_{i,j} - W_{i,j+1})\right], & v_n < 0. \end{cases} \notag
\end{align}
The subscripted $\theta$ quotients are as follows:
\begin{align*}
\theta_i &= \frac{W_{i,j}-W_{i-1,j}}{W_{i+1,j} - W_{i,j}}, & (\theta_{i+1})^{-1} &= \frac{W_{i+2,j}-W_{i+1,j}}{W_{i+1,j} - W_{i,j}}, \\
\theta_j &= \frac{W_{i,j}-W_{i,j-1}}{W_{i,j+1} - W_{i,j}}, & (\theta_{j+1})^{-1} &= \frac{W_{i,j+2}-W_{i,j+1}}{W_{i,j+1} - W_{i,j}}.
\end{align*}
One does not even compute these ``$\theta$s'' when using first-order upwind.  On the other hand, when using the Koren flux-limiter the stencil in Figure \ref{fig:stencil} is extended because regular grid neighbors $W_{i+2,j}$, $W_{i-2,j}$, $W_{i,j+2}$, $W_{i,j-2}$ are potentially involved in updating $W_{i,j}$.

For either the first-order or Koren schemes, if the water input $m$ is negative then we must actively enforce the positivity of the water thickness $W$.  That is, positivity of the advection-diffusion scheme is a desirable property but it does not ensure positivity of the solution if there is actual water removal ($(m/\rho_w) + T < 0$).  Therefore we project (reset) $W$ to be nonnegative at the end of each time step.

\subsection*{Discretization of the pressure evolution equation}  The pressure evolution equation \eqref{eq:regpressureequation} is a nonlinear diffusion with additional ``reaction'' terms associated to opening and closing.  Unlike solving \eqref{eq:adeqn} for $W$, when solving \eqref{eq:regpressureequation} for $P$ there is no dominating advection term.  Therefore we discretize it using a centered second-order scheme.  Because this is also an explicit scheme, we consider stable time steps immediately.

The time step restriction we identify is comparable to \eqref{eq:dtDIFFW}, though the proof technique in Appendix \ref{app:positivestable} does not suffice to \emph{prove} stability under this condition because of the additional reaction terms.  If the time step satisfies $\Delta t \le \Delta t_P$, where
\begin{equation}
\Delta t_P\, \left(\frac{2 \max D}{\phi_0}\right) \left(\frac{1}{\Delta x^2} + \frac{1}{\Delta y^2}\right) = 1 \label{eq:dtDIFFP}
\end{equation}
then the scheme is stable.  The resulting time step $\Delta t_P$ is a fraction of $\Delta t_W$ from \eqref{eq:dtDIFFW}:
\begin{equation}
\Delta t_P = 2 \phi_0\, \Delta t_W.  \label{eq:dtDIFFPfromW}
\end{equation}

We can again by quantitative in a particular computed example.  Considering the same 250 m simulation of the hydrology of \Nbreen as earlier, and with $\phi_0 = 0.005$, we have $\Delta t_P$ which is 100 times smaller than $\Delta t_W$.  With these values, and the others used earlier, we get
\begin{align*}
  \Delta t_W            &\approx 8000 \text{ s} &&\text{ from \eqref{eq:dtDIFFW}}, \\
  \Delta t_{\text{CFL}} &\approx 300  \text{ s} &&\text{ from \eqref{eq:dtCFL}}, \\
  \Delta t_P            &\approx 80   \text{ s} &&\text{ from \eqref{eq:dtDIFFPfromW}.}
\end{align*}
This analysis suggests that the numerical scheme for pressure diffusion, given next, will usually have the shortest time step.  Note that $\Delta t_{\text{CFL}}=O(\Delta x)$ while $\Delta t_W$ and $\Delta t_P$ are $O(\Delta x^2)$.  In actual computation it seems to be common that $\Delta t_P$ is $1$ to $50$ times shorter than $\Delta t_{\text{CFL}}$.  The CFL time step restriction $\Delta t_{\text{CFL}}$ essentially cannot be avoided by switching to implicit methods, while the time step restriction associated to $\Delta t_P$ certainly can be \citep{Schoofetal2012,Hewittetal2012}.  However, as the size of the stable time step $\Delta t_P$ scales with the adjustable regularizing porosity $\phi_0$, by choosing $\phi_0$ smaller or larger we can make the time step restriction on $\Delta t_P$ more or less severe, respectively \citep[compare][]{Clarke2003}.

The scheme we use for the pressure equation \eqref{eq:regpressureequation} is similar to the scheme we have just presented for the mass continuity equation \eqref{eq:adeqn}, but it requires no advection approximation.  Denote $\psi_{i,j}^l = P_{i,j}^l + \rho_w g (b_{i,j}^l + W_{i,j}^l)$.  Let
	$$\mathcal{O}_{ij} = c_1 |\bv_b|_{i,j} \left(W_r - \Wlij\right)_+, \qquad \mathcal{C}_{ij} = c_2 A \left(\rho_i g H_{i,j} - \Plij\right)^3 \Wlij$$
be the gridded values of the cavitation-opening and creep-closure rates.  The scheme is
\begin{align}
\frac{\phi_0}{\rho_w g} \frac{P_{i,j}^{l+1} - \Plij}{\Delta t} &= \frac{K_e W_e \left(\psi_{i+1,j}^l - \psi_{i,j}^l\right) - K_w W_w \left(\psi_{i,j}^l - \psi_{i-1,j}^l\right)}{\Delta x^2}  \label{eq:Pfd} \\
      &\quad + \frac{K_n W_n \left(\psi_{i,j+1}^l - \psi_{i,j}^l\right) - K_s W_s \left(\psi_{i,j}^l - \psi_{i,j-1}^l\right)}{\Delta y^2} + \mathcal{C}_{ij} - \mathcal{O}_{ij} + \frac{m_{ij}}{\rho_w} - T_{ij}^l. \notag
\end{align}
Again it is useful to restate \eqref{eq:Pfd} in explicit update form.  First define $\omega_x = 1/\Delta x^2$, $\omega_y = 1/\Delta y^2$.   Then scheme \eqref{eq:Pfd} is equivalent to this form written in terms of $D=\rho_w g K W$:
\begin{align}
P_{i,j}^{l+1} = \Plij &+  \frac{\,\Delta t}{\phi_0} \bigg[\omega_x D_e \left(\psi_{i+1,j}^l - \psi_{i,j}^l\right) - \omega_x D_w \left(\psi_{i,j}^l - \psi_{i-1,j}^l\right) \label{eq:Pfdupdate} \\
      &\qquad\qquad\qquad + \omega_y D_n \left(\psi_{i,j+1}^l - \psi_{i,j}^l\right) - \omega_y D_s \left(\psi_{i,j}^l - \psi_{i,j-1}^l\right)\bigg] \notag \\
      &\quad + \frac{\rho_w g\,\Delta t}{\phi_0} \left[\mathcal{C}_{ij} - \mathcal{O}_{ij} + \frac{m_{ij}}{\rho_w} - T_{ij}^l\right]. \notag
\end{align}

There are special cases at the boundaries of the active subglacial layer: (\emph{i}) where there is no ice $H_{i,j}=0$ and land ($b_{i,j}>0$) we set $P_{i,j}^{l+1}=0$, (\emph{ii}) where the ice is floating we set $P_{i,j}^{l+1}=(P_o)_{i,j}$, and (\emph{iii}) where there is grounded ice ($H_{i,j}>0$) and no water ($W_{i,j}^l=0$) we again set $P_{i,j}^{l+1}=(P_o)_{i,j}$. 

\subsection*{One time step of the model}  Mathematical model \eqref{eq:bluebox} evolves $W$, $\Wtil$, and $P$.  One time step of the fully-discretized evolution is described next as a ``recipe''.  We treat the ice and bedrock geometry, and the ice sliding speed, as fixed so that $h_{i,j}$, $b_{i,j}$, $(P_o)_{i,j}$, and $|\bv_b|_{i,j}$ are denoted as time-independent.

The ice geometry may be quite general, with ice-free land and floating ice allowed.  In fact, the ice geometry determines boolean masks for grid cell state based on a sea level of elevation zero:
\begin{align*}
\text{\texttt{icefree}}_{i,j} &= (h_{i,j} > 0)\, \&\, (h_{i,j} = b_{i,j}), \\
\text{\texttt{float}}_{i,j}   &= (\rho_i (H_{\text{float}})_{i,j} < - \rho_{sw}\, b_{i,j}).
\end{align*}
Here we take a sea-water density $\rho_{sw}=1028.0$ and $H_{\text{float}}=h_{i,j} / (1 - r)$  is the thickness of the ice if it is floating, where $r=\rho_i / \rho_{sw}$.  Note that $\text{\texttt{float}}_{i,j}$ is true in ice-free ocean.  The subglacial layer we are attempting to model exists only for grounded ice, that is, only if both \texttt{icefree} and \texttt{float} masks are false.  The other mask cases provide boundary conditions when they are neighbors to grounded ice cells.

One time step follows this algorithm:

\bigskip\medskip
\renewcommand{\labelenumi}{\emph{(\roman{enumi})}}
\begin{enumerate}
\item Start with values $\Wlij$, $(\Wtil)_{i,j}^l$, $\Plij$ which satisfy the bounds $W\ge 0$, $0\le \Wtil \le \Wtilmax$, and $0 \le P \le P_o$.
\item Compute the current values of the hydraulic potential $\psi_{i,j}^l$, but with $\psi_{i,j}^l=(P_o)_{i,j}$ where $\text{\texttt{float}}_{i,j}$.
\item Compute velocity components $u_e$, $v_n$ at staggered grid locations from \eqref{eq:velocitycomp}.
\item FIXME: get $(\Wtil)_{i,j}^{l+1}$ by implicit step.  this determines $T_{i,j}^l$
\item Get $W$ values averaged onto the staggered grid from \eqref{eq:stagW}.
%FIXME: however, Wea and Wno should not average from outside the ice domain?
\item Get time step $\Delta t = \min\{\Delta t_{\text{CFL}}, \Delta t_W, \Delta t_P\}$ using criteria \eqref{eq:dtCFL}, \eqref{eq:dtDIFFW}, and \eqref{eq:dtDIFFPfromW}, but based on the actual gridded values of $\bV$ and $W$.
\item If $\text{\texttt{icefree}}_{i,j}$ set $P_{i,j}^{l+1}=0$.  If $\text{\texttt{float}}_{i,j}$ then set $P_{i,j}^{l+1} = (P_o)_{i,j}$; this is the pressure of sea water at the base of the ice.  If $\Wlij=0$ and $\text{\texttt{icefree}}_{i,j}$ and $\text{\texttt{float}}_{i,j}$ are both false, then set $P_{i,j}^{l+1} = (P_o)_{i,j}$.  Otherwise use \eqref{eq:Pfdupdate} to compute preliminary values for $P_{i,j}^{l+1}$ at the remaining locations, but do not compute the $x$-($y$-)direction divided-difference contribution to the flux divergence in \eqref{eq:Pfdupdate} when either $x$-($y$-)neighbor is \texttt{icefree} or \texttt{float}.
\item If $P_{i,j}^{l+1}$ does not satisfy bounds $0 \le P \le P_o$ then reset (project) into this range.
\item Using \eqref{eq:adfluxes} and a particular flux-limiter, compute the advective fluxes $Q_e(\alpha_e)$ at all staggered-grid points $(i+1/2,j)$ and $Q_n(\beta_n)$ at all staggered-grid points $(i,j+1/2)$.  
\item If $\text{\texttt{icefree}}_{i,j}$ or $\text{\texttt{float}}_{i,j}$ then set $W_{i,j}^{l+1}=0$ and $(\Wtil)_{i,j}^{l+1}=0$.  Otherwise use \eqref{eq:Wupdate} to compute values for $W_{i,j}^{l+1}$.
\item If $W_{i,j}^{l+1}<0$ then reset (project) $W_{i,j}^{l+1}=0$.
\item Update time $t_{l+1}=t_l+\Delta t$ and repeat at \emph{(i)}.
\end{enumerate}

\medskip
This recipe goes with a reporting scheme for mass conservation.  Note that water is lost or gained at the margin where either the thickness goes to zero on land (margins), or at locations where the ice becomes floating (grounding lines), according to the hydraulic potential differences which cause transport.  Because such loss/gain may be the modeling goal---users want hydrological discharge---these amounts are reported.  This reporting scheme also tracks the projections in step \emph{(xi)}, which represent a mass conservation error which goes to zero under the continuum limit $\Delta t\to 0$.

\subsection*{Verification of the coupled model}  By using the coupled steady-state nearly-exact solution constructed in section \ref{sec:steadyverif} we can verify most of the numerical schemes described above.  Verification is the process of actually measuring the errors made by the numerical scheme, especially as the numerical grid is refined.

Our exact solution is for the steady-state.  Therefore we initialize our time-stepping numerical scheme with the exact steady solution and we measure the error relative to the steady exact values after some period of time integration.  The continuum time-dependent model \eqref{eq:bluebox} would cause no drift but we can measure the drift away from the exact steady solution generated by the numerical approximation to \eqref{eq:bluebox}, that is, by the scheme which is the ``recipe'' above.  The magnitude of the drift depends on the spatial grid size.  The rate of convergence under grid refinement is determined by our spatial discretizations.

For the verification runs we use the values in Table \ref{tab:verifconstants}.  The exact solution is shown in Figures \ref{fig:Wexact} and \ref{fig:Pexact}.  We do a one model-month run on grids with spacing decreasing by factors of two from $5$ km to $156$ m.  Figure \ref{fig:refineWPpism} shows the results based on first-order upwinding for the fluxes.

\begin{figure}[ht]
\includegraphics[width=3.0in,keepaspectratio=true]{refineWpism} \quad \includegraphics[width=3.0in,keepaspectratio=true]{refinePpism}
\caption{Left: Average water thickness error $|W-W_{exact}|$ decays as $O(\Delta x^{0.87})$.  Right: Average pressure error $|P-P_{exact}|$ decays as $O(\Delta x^{1.20})$}
\label{fig:refineWPpism}
\end{figure}

Because they give evidence for numerical convergence, these results suggest that our numerical solution method for these coupled advection-diffusion (for $W$) and diffusion-reaction (for $P$) equations are being solved correctly.  The rate of convergence is not very good, however.  The location of the large errors is entirely in the neighborhood of the ice margin $r=L$ (not shown).  The method for handling boundary conditions is critical to determining the magnitude of the error and its rate of decay.  Further research is needed to decide how to handle this boundary in a more-nearly optimal manner.

The rates of convergence for average errors are nearly identical for the higher resolution flux-limited (Koren) scheme and for the first-order upwinding scheme (not shown).  Our problem is a combined advection-diffusion problem in which both the advection velocity and the diffusivity are solution-dependent, and thus it is difficult to separate the errors arising from the numerical treatments of advection and diffusion.  The first-order upwinding scheme for the advection has much larger numerical diffusivity but this diffusivity may be compatible with the continuum model (i.e.~desirable) diffusivity.  Based on this verification evidence it is reasonable to choose first-order upwinding for applications.  This scheme is simpler to implement and it requires fewer floating point operations.  It also uses a smaller stencil for less communication in a parallel implementation.



\section{Results}  \label{sec:results}

\subsection*{Steady results for a tidewater glacier}

FIXME: decide on where this paper is going

FIXME: redo results from nbreen to use PISM and make better figures


\begin{comment}
\begin{figure}[ht]
\includegraphics[width=7.0in,keepaspectratio=true]{icethk-icefree-float-250m}
\caption{FIXME}
%\label{fig:X}
\end{figure}

\begin{figure}[ht]
\includegraphics[width=7.0in,keepaspectratio=true]{outline-input-250m}
\caption{FIXME}
%\label{fig:X}
\end{figure}

\begin{figure}[ht]
\includegraphics[width=7.0in,keepaspectratio=true]{W-Pmask-250m}
\caption{FIXME}
%\label{fig:X}
\end{figure}

\begin{figure}[ht]
\includegraphics[width=7.0in,keepaspectratio=true]{Po-P-250m}
\caption{FIXME}
%\label{fig:X}
\end{figure}
\end{comment}

FIXME:  text about F\&C equation \eqref{eq:PofWFC}

\begin{figure}[ht]
\includegraphics[width=3.0in,keepaspectratio=true]{isPofW-250m} \,
\includegraphics[width=3.0in,keepaspectratio=true]{isPofW-250m-month}
\caption{Left: Scatter plot of $(W,P)$ pairs for all cells at end of 5 year steady-input simulation on a 250 m grid.  Red dashed is equation \eqref{eq:PofWFC} with $W_{\text{crit}} = W_r = 1$ m.  Right: Same except at the end of a one month simulation, and with equation \eqref{eq:PofWFC} using $W_{\text{crit}} = W_r / 3$.}
\label{fig:isPofWnbreen}
\end{figure}


\section{Conclusion}  \label{sec:conclusion}

MOST IMPORTANT:  IMPLEMENTED COMMON (I.E.~SIMULTANEOUS) EXTENSION OF 4 MAJOR MODELS


%\clearpage\newpage
\small
\bibliography{ice_bib}  % generally requires link to pism/doc/ice_bib.bib
\bibliographystyle{agu}
\normalsize


%\clearpage\newpage
\appendix
\small

\section{Relation to the Bartholomaus et al.~(2011) model}  \label{app:barth}

The model in \cite{Bartholomausetal2011} describes the evolution of the hydrology of the Kennicott glacier in Alaska.  It is a significantly different model from the distributed one of \cite{Schoofetal2012}, which we compare to in the main text, but similarities exist.  Both models describe the evolution of linked-cavity systems, and both include physical cavity opening and closing processes.  On the other hand the \cite{Schoofetal2012} theory is distributed and entirely subglacial while the \cite{Bartholomausetal2011} is ``lumped'' (i.e.~the entire glacier is represented by one cell) and is both subglacial and englacial.

In this Appendix we restate the published equations of the \cite{Bartholomausetal2011} model and then we derive a pressure evolution equation which applies in that model.  The form of this pressure equation is suggested by equation (12) in \cite{Bartholomausetal2011}, but its complete form is not stated.  By extending it to the distributed case, this pressure equation can be recognized as a diffusive (parabolic) version of the elliptic pressure equation in \cite{Schoofetal2012}.  This is one method for deriving our pressure equation \eqref{eq:regpressureequation} in the main text.

We start by describing the variables and equations of the \cite{Bartholomausetal2011} model.  The total volume of liquid water stored in the glacier is $S(t)$.  This is split into englacial $S_{en}(t)$ and subglacial $S_{sub}(t)$ portions.  The cavities have geometry partially-determined by bedrock bumps which have cartesian spacing $\lambda_x,\lambda_y$, height $h$, and width $w_c$.  These combine to give a dimensionless capacity parameter $f=(h w_c)/(\lambda_x \lambda_y)$; the value $f=0.05$ is used for the Kennicott glacier application.  Each cavity has cross-sectional area $A_c(t)$ and volume $w_c A_c$.  The glacier occupies a rectangle of dimensions $L\times W$ in the map-plane so that the number of cavities is $\nu = (LW)/(\lambda_x\lambda_y)$.  It follows that the subglacial storage volume is $S_{sub} = (w_c A_c) \nu = (f L W/h) A_c$.

Englacial water is assumed to fill crevasses and moulins up to a level $z_w(t)$ above the bedrock, in a system which has macroporosity $\phi$ (dimensionless).  Thus the englacial storage is $S_{en}=L W \phi z_w$.  In summary, the \cite{Bartholomausetal2011} model includes these equations describing the water amount variables $S_{en}$ and $S_{sub}$:
\begin{equation}
S = S_{en} + S_{sub}, \qquad S_{en} = L W \phi \zen, \qquad S_{sub} = \frac{f L W}{h} A_c.  \label{eq:barth:kinematics}
\end{equation}

Mass conservation in the model is the simple statement  \citep{Bartholomausetal2008}
\begin{equation}
\frac{dS}{dt} = Q_{in}(t) - Q_{out}(t). \label{eq:barth:massconserve}
\end{equation}
In the Kennicott glacier application, fluxes $Q_{in}$ and $Q_{out}$ are given by observations.

Let $P_o=\rho_i g H$ denote the overburden pressure, $P(t)$ the water pressure, and $N(t)=P_o-P(t)$ the effective pressure applied by the glacier to its bed.  Knowledge of the water pressure $P$ is equivalent to knowledge of the amount of englacial storage because there is an assumed efficient connection of the macroporous glacier to the subglacial system, namely the equation $S_{en}=L W \phi \zen$.  Englacial water therefore applies the hydrostatic pressure to the subglacier, and thus also
\begin{equation}
P = \rho_w g z_w.  \label{eq:barth:englacialpressure}
\end{equation}

In the \cite{Bartholomausetal2011} model the cavity cross-sectional area $A_c$ evolves by physical opening and closure processes.  A wall melt parameterization is also given but, in keeping with the cavity evolution in the rest of the current paper, we simply denote it as a melt term $\dot m$.  Denote the sliding speed by $u_b$ and let $C_c = (2 A)/n^n$, where $A$ and $n$ are parameters in the (Glen) ice flow law \citep{Paterson}.  Then the cavity area evolution equation (4) in \cite{Bartholomausetal2011} is
\begin{equation}
\frac{dA_c}{dt} = u_b h + \dot m - C_c A_c (P_o-P)^n.  \label{eq:barth:cavityevolution}
\end{equation}
The three terms on the right are opening by cavitation, melt, and closure by creep, respectively.

Equations \eqref{eq:barth:kinematics}, \eqref{eq:barth:massconserve}, \eqref{eq:barth:englacialpressure}, and \eqref{eq:barth:cavityevolution} combine to give an evolution equation for the pressure, though this may not have been recognized by \cite{Bartholomausetal2011}.  From \eqref{eq:barth:kinematics} and \eqref{eq:barth:englacialpressure} we can write the pressure rate of change in terms of the englacial storage rate of change:
	$$\frac{dP}{dt} = \rho_w g \frac{dz_w}{dt} = \frac{\rho_w g}{L W \phi} \frac{d S_{en}}{dt}.$$
But \eqref{eq:barth:kinematics} and \eqref{eq:barth:massconserve}  allow us to rewrite in terms of fluxes and cavity area:
    $$\frac{dP}{dt} = \frac{\rho_w g}{L W \phi} \left(\frac{d S}{dt} - \frac{d S_{sub}}{dt}\right) = \frac{\rho_w g}{L W \phi} \left(Q_{in} - Q_{out} - \frac{d S_{sub}}{dt}\right).$$
Now use \eqref{eq:barth:kinematics} to write the evolution in terms of the rate of change of $A_c$:
    $$\frac{dP}{dt} = \frac{\rho_w g}{L W \phi} \left(Q_{in} - Q_{out} - \frac{f L }{h} \frac{d A_c}{dt}\right).$$
Finally incorporate equation \eqref{eq:barth:cavityevolution} to eliminate $dA_c/dt$:
\begin{equation}
\frac{dP}{dt} = \frac{\rho_w g}{L W \phi} \left(Q_{in} - Q_{out} - \frac{f L }{h} \left[u_b h + \dot m - C_c A_c (P_o-P)^n\right]\right)  \label{eq:barth:fullpressure}
\end{equation}
We emphasize that, though it is not stated there, equation \eqref{eq:barth:fullpressure} follows from the model equations in \cite{Bartholomausetal2011}.

Equation \eqref{eq:barth:fullpressure} suggests how to extend the \cite{Bartholomausetal2011} theory from ``lumped'' into ``distributed.''  Consider a one-dimensional glacier flowing in the positive $x$ direction.  Let the transverse width be $W$ and replace $L$ by $\Delta x$.  Note that ``$Q_{in}$'' would, in a distributed theory, be the upstream flux while ``$Q_{out}$'' would be downstream.  Thus we rewrite \eqref{eq:barth:fullpressure} as
\begin{equation*}
\frac{\phi}{\rho_w g}\frac{dP}{dt} = \frac{1}{W} \left(- \frac{Q_{out} - Q_{in}}{\Delta x} - \frac{f}{h} \left[u_b h + \dot m - C_c A_c (P_o-P)^n\right]\right).
\end{equation*}
The continuum limit is then clear, with $(Q_{out} - Q_{in})/\Delta x \to \partial Q/\partial x$.  Thus a distributed flowline form of the \cite{Bartholomausetal2011} theory is the partial differential equation
\begin{equation}
\frac{\phi}{\rho_w g} \frac{\partial P}{\partial t} = \frac{1}{W} \left(- \frac{\partial Q}{\partial x} - \frac{f}{h} \left[u_b h + \dot m - C_c A_c (P_o-P)^n\right]\right).  \label{eq:barth:distpressure}
\end{equation}
Equation \eqref{eq:barth:distpressure} is, for all practical purposes, the same as equation \eqref{eq:regpressureequation} in the main text.

A distributed extension of the \cite{Bartholomausetal2011} theory is not, however, viable without a Darcy or other flux expression for $Q$.  One must also use a distributed mass conservation equation like \eqref{eq:conserve}.  By contrast, a flux expression, Darcy or otherwise, was not needed in the Kennicott glacier case because the lumped input and output from the hydrological system were available as data.

An implication of the \cite{Bartholomausetal2011} model, again not stated there, regards the pressure in steady state.  Steady state in equation \eqref{eq:barth:cavityevolution} gives
\begin{equation*}
0 = u_b h + \dot m - C_c A_c (P_o-P)^n.
\end{equation*}
This is a relationship between pressure $P$ and cavity area $A_c$ in steady state:
\begin{equation}
P = P_o - \left(\frac{u_b h + \dot m}{C_c A_c}\right)^{1/n}. \label{eq:barth:steadypressure}
\end{equation}
This equation, which is just like equation \eqref{eq:PofWsteady} in the main text, says that in steady state the pressure is a function of the amount of water.  It is interesting to observe, however, that \eqref{eq:barth:steadypressure} shows that the steady water pressure does not depend on the englacial macroporosity $\phi$.  Though the englacial pressure is parameterized by $P=\rho_w g z_w$ in \cite{Bartholomausetal2011}, its \emph{steady} value is entirely determined by the balance between sliding, wall melt, and creep closure in the subglacial system.  The englacial system is passive in determining steady state.

Returning to equation \eqref{eq:barth:distpressure} we can now connect the theory outlined in this Appendix with the main theory in the paper.  Namely, if we extend \eqref{eq:barth:distpressure} to two horizontal dimensions ($\partial Q/\partial t \to \Div \bq$ and so on) and we add Darcy relation \eqref{eq:flux} then we get \eqref{eq:regpressureequation}.  Under any reasonable Darcy-type formulation for the flux $Q$ in \eqref{eq:barth:distpressure}, the $\phi\to 0$ limit of \eqref{eq:barth:distpressure} is an elliptic equation for the water pressure.  This $\phi\to 0$ limit of the distributed version of the \cite{Bartholomausetal2011} model is the model in \cite{Schoofetal2012}.  The main text of the current paper simultaneously describes a distributed extension of the \cite{Bartholomausetal2011} model and an englacial-storage-regularized extension of the \cite{Schoofetal2012} model.


\section{Flux-limiter methods} \label{app:fluxlimiters}

Because \eqref{eq:adeqn} in the main text is an advection-dominated PDE, the well-known goals for discretizing the fluxes in \eqref{eq:adeqn} include non-oscillation and positivity \citep{HundsdorferVerwer2010} in addition to reduced truncation error.  Schemes addressing these goals are often described as finite volume discretizations \citep{LeVeque}.  We now introduce a few such schemes using an advection-only model equation in which an abstract quantity $u(t,x)$ is transported by an abstract velocity $v(x)$:
\begin{equation} \label{eq:modeladvect}
\ddt{u} + \ddx{}\left(v(x) u\right) = 0
\end{equation}

Suppose that the grid points $x_i$ are the centers of equally-spaced cells with $x_{i+1}-x_i=\Delta x$.  Cell interfaces are at $x_{i-1/2}=x_w$ (``\emph{w}'' for ``west'') and $x_{i+1/2}=x_e$ (``east'').  Suppose we do not discretize time and instead we denote $u(t,x_i)$ by $u_i(t)$.  We say a spatial discretization of \eqref{eq:modeladvect} is \emph{positive} if $u_i(0) \ge 0$ for all $i$ implies $u_i(t)\ge 0$ for all $t\ge 0$ and all $i$.  In particular, such a property is desirable for any conservation scheme for the water thickness because thicknesses are intrinsically nonnegative.

The discretizations of \eqref{eq:modeladvect} we consider here all use the fluxes $Q=v(x) u$ at the cell interfaces $x_w$ and $x_e$, namely
\begin{equation}
\frac{du_i}{dt} + \frac{Q_e - Q_w}{\Delta x} = 0. \label{eq:basicmodelFD}
\end{equation}
But we must choose a flux parameterization.  The simplest positive scheme for the fluxes is first-order upwinding in conservative form \citep[section I.4.3]{HundsdorferVerwer2010}.  We use the upwind value $u_i$ in $Q_e$ if $v_e = v(x_e) \ge 0$:
\begin{equation}
Q_e = v_e u_i \label{eq:upwindfluxfirst}
\end{equation}
If $v_e < 0$ then \eqref{eq:upwindfluxfirst} is replaced by $Q_e = v_e u_{i+1}$. 
Also $Q_w = v_w x_{i-1}$ if $v_w \ge 0$, and $Q_w = v_w x_{i}$ if $v_w < 0$.  Scheme \eqref{eq:upwindfluxfirst} is also called the ``donor cell'' upwind method \citep{LeVeque} because there is a decision, based on the sign of the cell-interface velocity, on which cell contributes the value of $u$.

A second-order truncation error scheme which is \emph{not} positive uses a centered average in computing the flux.  We can regard this as a correction to the first-order upwind form:
\begin{equation}
Q_e = v_e \frac{u_i+u_{i+1}}{2} = v_e \left[u_i + \frac{1}{2} (u_{i+1} - u_i)\right]. \label{eq:centerfluxfirst}
\end{equation}
A yet higher-resolution scheme is third-order upwind-biased fluxes which can again be written as a correction to first-order upwinding:
\begin{equation}
Q_e = v_e \frac{-u_{j-1} + 5 u_i + 2 u_{i+1}}{6} = v_e \left[u_i + \left(\frac{1}{3}+\frac{1}{6} \theta_i \right) (u_{i+1} - u_i)\right] \label{eq:thirdfluxfirst}
\end{equation}
where
\begin{equation}
\theta_i = \frac{u_{i} - u_{i-1}}{u_{i+1} - u_i}.  \label{eq:thetadefine}
\end{equation}
Unfortunately, despite the upwind-biasing in the third-order scheme, both \eqref{eq:centerfluxfirst} and \eqref{eq:thirdfluxfirst} cause oscillations and are not positive.

The ``flux-limiting'' (i.e.~flux-correction-limited) approach regards the corrections in \eqref{eq:centerfluxfirst} and \eqref{eq:thirdfluxfirst} as too large to allow positivity, specifically near local extrema of $u$ \citep[section III.1.1]{HundsdorferVerwer2010}.  Godunov's barrier theorem \citep[section I.7.1]{HundsdorferVerwer2010} says that we cannot get truncation error better than first-order with a discretization that is both linear and positive.  However, there exist nonlinear flux-limited formulas which restore these properties, and we list and try them here.  Because the schemes under consideration are explicit, the nonlinearity does not represent a significant computational cost.

These flux-limited schemes can be written in the same general form as above, with a correction to the first-order upwinded flux:
\begin{equation}
Q_e = v_e \left[u_i + \Psi(\theta_i) (u_{i+1} - u_i)\right], \qquad v_e \ge 0. \label{eq:fluxlimiterform}
\end{equation}
If we have $v(x)<0$ in \eqref{eq:modeladvect} then the same function $\Psi$ should be used but with the direction reversed \citep[section III.1.1]{HundsdorferVerwer2010}:
\begin{equation}
Q_e = v_e \left[u_{i+1} + \Psi\left((\theta_{i+1})^{-1}\right) (u_i - u_{i+1})\right], \qquad v_e < 0. \label{eq:fluxlimiterformreversed}
\end{equation}
The forms for $Q_w$ in \eqref{eq:basicmodelFD} simply replace $v_e \to v_w$, $i\to i-1$, and $i+1\to i$.  Table \ref{tab:fluxlimiters} shows several cases for $\Psi$ including those already considered.

The difference ratio $\theta_i$ in \eqref{eq:thetadefine} can take any real value.  For the time-discretizations we will consider, sufficient conditions on $\Psi(\theta)$ to give a positive advection scheme for model equation \eqref{eq:modeladvect} are
\begin{equation}
0 \le \Psi(\theta) \le 1, \qquad 0 \le \frac{\Psi(\theta)}{\theta} \le 1 \qquad \text{ for all } \theta \in \RR.
\end{equation}
The schemes in Table \ref{tab:fluxlimiters} which satisfy these conditions are marked with ``$\ast$''.

For smooth functions and fine grids we have $\theta_i\approx 1$ except near extrema where $u_{i+1} - u_i$ and $u_i - u_{i-1}$ are both near zero.  Note that in every case in Table \ref{tab:fluxlimiters} where there is a correction ($\Psi(\theta)\ne 0$) we have $\Psi(1)=1/2$.  The Koren flux-limiter in Table \ref{tab:fluxlimiters} \citep{HundsdorferVerwer2010} has $\Psi(\theta) = \frac{1}{3}+\frac{1}{6} \theta$ for $\frac{2}{5} \le \theta \le 4$.  We can think of the Koren scheme as a positive form of the third-order upwind-biased scheme.

\begin{table}[ht]
  \centering
  \caption{Flux schemes written in limiter form \eqref{eq:fluxlimiterform} using $\theta=\theta_i$ from \eqref{eq:thetadefine}.}
  \begin{tabular}{ll}
    \textbf{scheme ($\ast=$ positive)} & \textbf{formula} \\
\hline
    $\ast$ first-order upwinding               & $\phantom{\Big|}\Psi(\theta) = 0$ \\
    \phantom{$\ast$} second-order centered     & $\phantom{\Big|}\Psi(\theta) = \frac{1}{2}$  \\
    \phantom{$\ast$} third-order upwind-biased & $\phantom{\Big|}\Psi(\theta) = \frac{1}{3}+\frac{1}{6} \theta$  \\
    $\ast$ van Leer 1974                       & $\phantom{\Big|}\Psi(\theta) = \frac{1}{2} \frac{\theta + |\theta|}{1+\theta}$  \\
    $\ast$ Koren 1993                          & $\phantom{\Big|}\Psi(\theta) = \max\left\{0,\min\{1,\theta,\frac{1}{3}+\frac{1}{6} \theta\}\right\}$  \\
    \hline
  \end{tabular}
 \label{tab:fluxlimiters}
\end{table}

To summarize the flux discretization choices for model equation \eqref{eq:modeladvect}, first-order upwinding \eqref{eq:upwindfluxfirst} gives positivity but only $O(\Delta x^1)$ truncation error while second-order centered differencing \eqref{eq:centerfluxfirst} and third-order upwind-biased differencing \eqref{eq:thirdfluxfirst} give better truncation error (i.e.~$O(\Delta x^2)$ and $O(\Delta x^3)$, respectively), but not positivity.  The flux-limited Koren and van Leer schemes in Table \ref{tab:fluxlimiters} give positivity and also better truncation error away from ``difficult areas''.  Such areas, which are where $\theta_i$ in \eqref{eq:thetadefine} is $\gg 1$ or $\ll 1$, tend to be near extrema and non-smooth areas.  Thus these schemes are high-order but they revert to first-order when they get in trouble.

The above discussion was limited to the spatial (semi-)discretization.  A time discretization is also required, and for this we simply choose forward Euler.  The resulting fully discrete system is positive if
\begin{equation}
\max_x \frac{|v(x)|\Delta t}{\Delta x} \le \frac{1}{2} \label{eq:CFL}
\end{equation}
\citep[section III.1.1]{HundsdorferVerwer2010}.  One can show that these positive schemes are also total variation diminishing (TVD) if the velocity is constant ($v(x)=v_0$).  We identify condition \eqref{eq:CFL} as simply ``CFL'' even though it is more strict than the CFL condition that suffices for stability \citep{MortonMayers}.


\section{Positivity and stability of the mass conservation scheme} \label{app:positivestable}

Explicit numerical scheme \eqref{eq:Wupdate} for the mass conservation PDE \eqref{eq:adeqn}, combined with the first-order upwind case of formulas \eqref{eq:adfluxes}, is sufficiently simple so that we can analyze its stability properties.  For this scheme we sketch a maximum principle argument which shows stability \citep{MortonMayers}.  The argument also shows positivity \citep{HundsdorferVerwer2010} as long as the total water input is nonnegative; here only the case $m = 0$ is shown.  We consider only the upwinding case where the discrete velocities at cell interfaces are nonnegative: $u_e\ge 0$, $u_w\ge 0$, $v_n\ge 0$, $v_s\ge 0$.  The other upwinding cases can be handled by similar special-case arguments like this one.

We start by restating equation \eqref{eq:Wupdate} with the above simplifications:
\begin{align*}
 W_{i,j}^{l+1} &= \Wlij - \nu_x \left(u_e \Wlij - u_w W_{i-1,j}^l\right) - \nu_y \left(v_n \Wlij - v_s W_{i,j-1}^l\right)  \\
      &\qquad + \mu_x \left[D_e \left(W_{i+1,j}^l - \Wlij\right) - D_w \left(\Wlij - W_{i-1,j}^l\right)\right]  \\
      &\qquad + \mu_y \left[D_n \left(W_{i,j+1}^l - \Wlij\right) - D_s \left(\Wlij - W_{i,j-1}^l\right)\right].
\end{align*}
Collecting terms to write the new value as a linear combination of the old values, we get
\begin{align}
 W_{i,j}^{l+1} &= (\nu_x u_w + \mu_x D_w) W_{i-1,j}^l + (\mu_x D_e) W_{i+1,j}^l + (\nu_y v_s + \mu_y D_s) W_{i,j-1}^l + (\mu_y D_n) W_{i,j+1}^l  \notag \\
      &\qquad + \Big[1 - \nu_x u_e - \nu_y v_n - \mu_x (D_e + D_w) - \mu_y (D_n + D_s)\Big] \Wlij \notag \\
  &= \tilde A\, W_{i-1,j}^l + \tilde B\, W_{i+1,j}^l + \tilde C\, W_{i,j-1}^l + \tilde D\, W_{i,j+1}^l + \tilde E\, \Wlij. \label{eq:lincomb}
\end{align}
Because of our assumption about nonnegative velocities, and noting that the diffusivities are nonnegative, we see that coefficients $\tilde A,\tilde B,\tilde C,\tilde D$ are all nonnegative.  Only $\tilde E$ could be negative, depending on values of $\nu_x, \nu_y, \mu_x$, and $\mu_y$.  Requiring it to be nonnegative will generate a sufficient stability condition \citep{MortonMayers}.

We state such a condition based on an equal split between advective and diffusive parts.  First there is a CFL restriction for the advection terms; compare \eqref{eq:dtCFL}:
\begin{equation}
\nu_x \alpha_e + \nu_y \beta_n = \Delta t \left(\frac{u_e}{\Delta x} + \frac{u_n}{\Delta y}\right) \le \frac{1}{2}. \label{eq:adstabcond}
\end{equation}
The second is a time-step restriction on the diffusion; compare \eqref{eq:dtDIFFW}:
\begin{equation}
\mu_x (D_e + D_w) + \mu_y (D_n + D_s) = \Delta t \left(\frac{D_e + D_w}{\Delta x^2} + \frac{D_n + D_s}{\Delta y^2}\right) \le \frac{1}{2}. \label{eq:diffstabcond}
\end{equation}
If \eqref{eq:adstabcond} and \eqref{eq:diffstabcond} hold then the coefficient $\tilde E$ in \eqref{eq:lincomb} is nonnegative:
	$$\tilde E = 1 - \nu_x u_e - \nu_y v_n - \mu_x (D_e + D_w) - \mu_y (D_n + D_s) \ge 0.$$

Because the coefficients in linear combination \eqref{eq:lincomb} add to one, as the reader may check, it follows  from \eqref{eq:adstabcond} and \eqref{eq:diffstabcond} that the scheme is stable \citep{MortonMayers}.  It also follows from \eqref{eq:adstabcond} and \eqref{eq:diffstabcond} that if $\Wlij\ge 0$ for all $i,j$ then \eqref{eq:lincomb} gives $W_{ij}^{l+1}\ge 0$, which is our positivity claim.  More generally, under conditions \eqref{eq:dtCFL} and \eqref{eq:dtDIFFW}, the conditions which describe all of the upwinding cases, scheme \eqref{eq:Wupdate} is stable and positivity-preserving.


\section{An abstract view of the transport-plus-storage problem} \label{app:transportstorage}

The problem considered in sections \ref{sec:elements} and \ref{sec:capacity} includes mass conservation equation \eqref{eq:conserve} for the total water amount and also equation \eqref{eq:tilldynamics} for the evolution of the water stored locally in till pore spaces.  An abstract view of this problem is the following pair of equations for abstract quantities $u(t,x,y),v(t,x,y)$ satisfying bounds $u\ge 0$ and $0 \le v \le v_{\text{max}}$:
\begin{align}
\frac{\partial}{\partial t} \left(u+v\right) &= \Div \left(\mathbf{q}(x,u,\grad u)\right) + g(x), \label{eq:abs:transport} \\
   \frac{\partial v}{\partial t} &= b \left(f(u) - v\right).  \label{eq:abs:storage}
\end{align}
Here the conserved quantity is the sum $u+v \sim \Wtot$.  The flux $\mathbf{q}$ only depends on the transported portion $u \sim W$ and its gradient, while the second equation describes local (i.e.~with no spatial derivatives) evolution of the locally-stored quantity $v \sim \Wtil$.  The flux $\mathbf{q}$ and the source term $g$ are quite general, as is the transfer term $b f(u)$, and the details generally don't concern us.  Note that the constant $b\ge 0$ is the exponential rate at which the locally-stored quantity ``drains'' in the absence of transfer of $u$ (i.e.~if $f(u)=0$).

The first idea explained in this appendix is that the storage quantity $v$ \emph{could} be eliminated from the system of equations \eqref{eq:abs:transport} and \eqref{eq:abs:storage}.  The result is a single transport equation for $u$ with an additional term which is an integral over previous states of $u$.  Therefore implementing this form would require storage of prior states.  In the main text we instead maintain the two-equation structure  \eqref{eq:abs:transport} and \eqref{eq:abs:storage}.  The second idea in this appendix shows that an implicit time-discretization of equation \eqref{eq:abs:storage} will guarantee that the bounds $0 \le v \le v_{\text{max}}$ are maintained (compare section \ref{sec:num}).   

To show how to eliminate equation \eqref{eq:abs:storage} we multiply it by an integrating factor $e^{-bt}$ and simplify:
    $$\frac{\partial v}{\partial t} + b v = b f(u) \qquad \iff \qquad \frac{\partial}{\partial t} \left(e^{bt} v\right) = b e^{bt} f(u).$$
If $v$ is known at the initial time, say $v=v_0(x,y)$ at $t=0$, then we can integrate from $0$ to $t$ and multiply both sides by $e^{-bt}$ to get
    $$v(t,x,y) = e^{-bt} \left(v_0(x,y) + b \int_0^t e^{bs} f(u(s,x,y))\,ds\right).$$
This allows us to eliminate $v$ from \eqref{eq:abs:transport} by moving $\partial v/\partial t$ to the right side:
\begin{equation}
\frac{\partial u}{\partial t} = \Div \left(\mathbf{q}(x,u,\grad u)\right) + g(x) + b e^{-bt} v_0(x,y) - b f(u) + b^2 \int_0^t e^{-b(t-s)} f(u(s,x,y))\,ds. \label{eq:abs:transportagain} 
\end{equation}
Thus we can eliminate the locally-stored quantity $v$ at the cost of maintaining a complete history of the past values of the transported quantity $u$.  While equation \eqref{eq:abs:transportagain} may or may not be conceptually advantageous, it is not effectively-implementable.

Next we will consider discretizations of \eqref{eq:abs:storage}.  In the till-porewater application it is important that the storage capacity of the till not be exceeded, and so in this abstract description we add bounds $0 \le v \le v_{\text{max}}$.  The same requirement also implies $\partial v/\partial t \to 0$ as $v \upto v_{\text{max}}$, so we assume $0 \le f(u) \le v_{\text{max}}$ also.

Equation \eqref{eq:abs:storage} is linear in $v$ which allows an implicit scheme to be used without solving nontrivial equations.  In particular, if $t^l$ is the time-step and $t^{l+1}-t^l = \Delta t$ and if $U^l,V^l \approx u,v$ then the stable (specifically ``A-stable'' \citep{AscherPetzold}) backwards Euler approximation of \eqref{eq:abs:storage} using explicit coupling is
    $$\frac{V^{l+1} - V^l}{\Delta t} = b \left(f(U^l) - V^{l+1}\right).$$
Solving for $V^{l+1}$ gives
\begin{equation}
V^{l+1} = \frac{V^l + b \Delta t f(U^l)}{1 + b \Delta t}
\end{equation}
Noting $b \Delta t \ge 0$ we see that $V^{l+1}$ is an average of $V^l$ and $f(U^l)$.  Therefore
    $$0 \le V^l, f(U^l) \le v_{\text{max}} \quad \implies \quad 0 \le V^{l+1} \le v_{\text{max}},$$
so that the discrete evolution of the locally-stored quantity $v$ maintains the desired constraints.




\end{document}
