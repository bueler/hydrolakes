%\documentclass[gmd,hvmath,online]{copernicus_discussions}  % to submit use this style
\documentclass[gmd]{copernicus}   % two-column layout

\usepackage{amssymb,xspace}
\usepackage[T1]{fontenc}

% hyperref should be the last package we load
\usepackage[pdftex,
                colorlinks=true,
                plainpages=false, % only if colorlinks=true
                linkcolor=blue,   % only if colorlinks=true
                citecolor=black,   % only if colorlinks=true
                urlcolor=magenta     % only if colorlinks=true
]{hyperref}

\ifx\text\undefined
\newcommand{\text}{\textrm}
\else
\fi

% math macros
\newcommand\bv{\mathbf{v}}
\newcommand\bV{\mathbf{V}}
\newcommand\bq{\mathbf{q}}
\newcommand\bQ{\mathbf{Q}}

\newcommand{\ddt}[1]{\ensuremath{\frac{\partial #1}{\partial t}}}
\newcommand{\Div}{\nabla\cdot}
\newcommand\eps{\epsilon}
\newcommand{\grad}{\nabla}

\newcommand{\Ntil}{N_{\text{til}}}
\newcommand{\hatNtil}{\hat N_{\text{til}}}
\newcommand{\Wtil}{W_{\text{til}}}
\newcommand{\Wtilmax}{W_{\text{til}}^{\text{max}}}

\newcommand{\Wlij}{W^l_{i,j}}
\newcommand{\Wij}{W_{i,j}}
\newcommand{\Plij}{P^l_{i,j}}
\newcommand{\Pij}{P_{i,j}}

\newcommand{\Nbreen}{Nordenski\"oldbreen\xspace}

\usepackage[]{lineno}



\begin{document}
\graphicspath{{figszip/}}

%\linenumbers

\title{Mass-conserving subglacial hydrology in the Parallel Ice Sheet Model version 0.6}


\author[1]{E.~Bueler}
\author[2]{W.~{van} Pelt}

\affil[1]{Department of Mathematics and Statistics, University of Alaska Fairbanks, Alaska, USA}
\affil[2]{Institute for Marine and Atmospheric Research Utrecht, The Netherlands\footnote{Current address: Norwegian Polar Institute, Tromsø, Norway}}

%% The [] brackets identify the author to the corresponding affiliation, 1, 2, 3, etc. should be inserted.

\runningtitle{Subglacial hydrology in PISM}
\runningauthor{Bueler and {van} Pelt}

\correspondence{Ed Bueler (\texttt{elbueler@alaska.edu})}

%% These dates will be inserted by the Publication Production Office during the typesetting process.
\received{}
\pubdiscuss{} %% only important for two-stage journals
\revised{}
\accepted{}
\published{}

\firstpage{1}

\maketitle

\begin{abstract}
We describe and test a two-horizontal-dimension subglacial hydrology model which combines till with a distributed system of water-filled, linked cavities which open through sliding and close through ice creep.  The addition of this sub-model to the Parallel Ice Sheet Model accomplishes three specific goals: (1) conservation of the mass of water, (2) simulation of spatially- and temporally-variable basal shear stress from physical mechanisms based on a minimal number of free parameters, and (3) convergence under grid refinement.  The model is a common generalization of four others: (i) the undrained plastic bed model of \cite{Tulaczyketal2000b}, (ii) a standard ``routing'' model used for identifying locations of subglacial lakes, (iii) the lumped englacial/subglacial model of \cite{Bartholomausetal2011}, and (iv) the elliptic-pressure-equation model of \cite{Schoofetal2012}.  We preserve physical bounds on the pressure.  In steady state a functional relationship between water amount and pressure emerges.  We construct an exact solution of the coupled, steady equations and use it for verification of our explicit time-stepping, parallel numerical implementation.  We demonstrate the model at scale by five year simulations of the entire Greenland ice sheet at 2 km horizontal resolution, with one million nodes in the hydrology grid.
\end{abstract}


\introduction

Any continuum-physics-based dynamical model of the liquid water underneath and within a glacier or ice sheet has at least these two elements: the mass of the water is conserved and the water flows from high to low values of the modeled hydraulic potential.  Beyond that there are many variations considered in the literature.  Modeled aquifer geometry might be a system of linked cavities \citep{Kamb1987}, conduits \citep{Nye1976}, or a sheet \citep{CreytsSchoof2009}.  Geometry evolution processes might include the opening of cavities by sliding of the overlying ice past bedrock bumps \citep{Schoof2005cavitation}, the creation of cavities by interaction of the ice with deformable sediment \citep{Schoof2007deformable}, closure of cavities and conduits by creep \citep{Hewitt2011}, or melt on the walls of cavities and conduits which causes them to open \citep{Clarke05}.  Water could move in a macro-porous englacial system \citep{Bartholomausetal2011} or it could be stored in a porous till \citep{Tulaczyketal2000}.

Models have combined subsets of these different morphologies and processes \citep{FlowersClarke2002_theory,Hewitt2013,HoffmanPrice2014,vanderWeletal2013,Werderetal2013,deFleurianetal2014}.  However, the completeness of the modeled processes must be balanced against the number of uncertain model parameters and the ultimate availability of observations with which to constrain them.

This paper describes a carefully-selected model for a distributed system of linked subglacial cavities, with additional storage of water in the pore spaces of subglacial till.  Water in excess of the capacity of the till passes into the distributed transport system.  In this sense the model could be called a ``drained-and-conserved'' extension of the ``undrained'' plastic bed model of \cite{Tulaczyketal2000b}.

The cavities in our modeled distributed system open by sliding of the ice over bedrock roughness and close by ice creep.  These two physical processes combine to determine the relationship between water amount and pressure.  Pressure is thereby determined non-locally over each connected component of the hydrological system.  No functional relation between subglacial water amount and pressure is assumed \citep[compare][]{FlowersClarke2002_theory}.  The pressure solves an equation which is a parabolic regularization of the distributed pressure equation given in elliptic variational inequality form by \cite{Schoofetal2012}.

In cases where boreholes have actually been drilled to the ice base, till is often observed \citep{Hookeetal1997,Tulaczyketal2000,TrufferHarrisonEchelmeyer2000,TrufferHarrison2006}.  Laboratory experiments on the rheology of till \citep{Kamb1991,Hookeetal1997,Tulaczyketal2000,TrufferEchelmeyerHarrison2001} generally conclude that its deformation is well-approximated by a Mohr-Coulomb relation \citep{SchoofTill}.  For this reason we adopt a compressible-Coulomb-plastic till model when determining the effective pressure on the till as a function of the amount of water stored in it \citep{Tulaczyketal2000}.  Existing models which combine till and a mass conservation equation for the subglacial water are rather different from ours, as they either have only one-horizontal dimension \citep{vanderWeletal2013} or have a pressure equation which directly ties water pressure to water amount, which generates a porous medium equation form \citep{FlowersClarke2002_theory,deFleurianetal2014}.

The major goals here are to implement, verify, and demonstrate this two-dimensional subglacial hydrology model.  The model is applicable at a wide variety of spatial and temporal scales but it has relatively-few parameters.  It is parallelized and it exhibits convergence of solutions under grid refinement.  It is a sub-model of a comprehensive three-dimensional ice sheet model, the open-source Parallel Ice Sheet Model (PISM; \texttt{pism-docs.org}); the sub-model can be added to any PISM run by a simple run-time option.

Conduits are not included.  Existing theories of conduits apparently require their locations to be fixed a priori \citep{Schoofmeltsupply,Hewitt2013,Werderetal2013}.  Such lattice models have no known continuum limit in the map plane.  By contrast with conduits, linked-cavity models do not put the cavities at the nodes of a pre-determined lattice, exactly because the continuum limit of such a lattice model is known \citep{Hewitt2011}, namely partial differential equation (PDE) \eqref{eq:hewittcapacity} in the current paper.  Regarding lattice models, because all PISM usage involves a run-time determination of grid resolution, all parameters must have grid-spacing-independent meaning.  Lattice or other input-grid-based models are therefore not acceptable as components of PISM.

Wall melt in the linked-cavity system, which is believed to be small \citep{Kamb1987}, is not added into the mass conservation equation in our model.  (It can be calculated diagnostically from the modeled flux and hydraulic gradient, however.)  If included in mass conservation, the addition of wall melt can generate an unstable distributed system \citep{Walder1982}, though such a system can be stabilized to some degree by bedrock bumps \citep{CreytsSchoof2009}.

The structure of the paper is as follows: Section \ref{sec:elements} considers basic physical principles, culminating with a fundamental advection-diffusion form of the mass conservation equation.  Section \ref{sec:tillmechanics} reviews what is known about till mechanical properties, water in till pore spaces, and shear stress at the base of a glacier.  In section \ref{sec:closures} we compare ``closures'' which directly or indirectly determine the subglacial water pressure.  Based on all these elements, in section \ref{sec:newmodel} we summarize the new model and the role of its major fields.  In this section we also show how the model extends several published models, we note properties of its steady states (see also Appendix A), and we compute a nearly-exact steady solution in the map-plane, a useful tool for verification.  In section \ref{sec:num} we present the numerical schemes, with particular attention to time step restrictions and the treatment of advection, and we document the PISM options and parameters seen by users.  Section \ref{sec:results} shows numerical results from the model, starting with convergence under grid refinement in the verification case.  We then demonstrate the model in five year runs on a 2 km grid covering the entire Greenland ice sheet.


\section{Elements of subglacial hydrology} \label{sec:elements}

\subsection{Mass conservation}  We assume that liquid water is of constant density $\rho_w$; see Table \ref{tab:constants} for constants.  Thus the thickness of the layer of laterally-moving water, denoted by $W(t,x,y)$, determines its mass; see Table \ref{tab:symbols} for variable names and meanings.  In addition there is liquid water stored locally in the pore spaces of till \citep{Tulaczyketal2000b} which is also described by an effective thickness $\Wtil(t,x,y)$.  Such thicknesses are only meaningful compared to observations if they are regarded as averages over a horizontal scale of meters to hundreds of meters \citep{FlowersClarke2002_theory}.  Thus the total effective thickness of the water at map-plane location $(x,y)$ and time $t$ is $W + \Wtil$.  This sum is the conserved quantity in our model.  In two map-plane dimensions the mass conservation equation is \citep[compare][]{Clarke05}
\begin{equation} \label{eq:conserve}
\frac{\partial W}{\partial t} + \frac{\partial \Wtil}{\partial t} + \Div \bq = \frac{m}{\rho_w}
\end{equation}
where $\Div = \partial/\partial x + \partial/\partial y$ denotes divergence, $\bq$ is the vector water flux ($\text{m}^2\,\text{s}^{-1}$), and $m$ is the total input to the subglacial hydrology ($\text{kg}\,\text{m}^{-2}\,\text{s}^{-1}$).  Note that the water flux $\bq$ is concentrated within the two-dimensional subglacial layer.

The water source $m$ in equation \eqref{eq:conserve} includes both melt on the base of the glacier and drainage to the bed from the glacier surface.  In portions of ice sheets with cold surface conditions, such as Antarctica and the interior of Greenland, the basal melt rate part of $m$ is dominated by the energy balance at the base of the ice \citep{AschwandenBuelerKhroulevBlatter}, and the Greenland results in section \ref{sec:results} use only that basal melt for $m$.  Drainage from the surface has also been added to $m$ in applications of our model \citep{vanPeltthesis}, but modelling such drainage is outside the scope of this paper.

\subsection{Hydraulic potential and water pressure}  The  hydraulic potential $\psi(t,x,y)$ combines the pressure $P(t,x,y)$ of the transportable subglacial water and the gravitational potential of the top of the water layer \citep{Goelleretal2013,Hewittetal2012},
\begin{equation} \label{eq:potential}
\psi = P + \rho_w g\, (b+W).
\end{equation}
Here $z=b(x,y)$ is the bedrock elevation.

We have added the term ``$\rho_w g W$'' to the standard hydraulic potential formula $\psi = P + \rho_w g b$ \citep{Clarke05,Shreve1972} because differences in the potential at the \emph{top} of the subglacial water layer determine the driving potential gradient for a fluid layer.  When the water depth becomes substantial ($W\gg 1$ m), as it would be in a subglacial lake, this term keeps the modeled lakes from being singularities of the water thickness field \citep[compare][]{LeBrocqetal2009}.

Ice is a viscous fluid which has a stress field of its own.  The basal value of the downward normal stress, called the overburden pressure, is denoted by $P_o$.  We accept the shallow approximation that this stress is hydrostatic \citep{GreveBlatter2009},
\begin{equation} \label{eq:hydrostatic}
  P_o = \rho_i g H,
\end{equation}
where $H$ is the ice thickness.

Overpressure $P>P_o$ has been observed in ice sheets, but only for short durations \citep{Dasetal08}.  In our model and others \citep{Schoofetal2012}, however, because the condition $P>P_o$ is presumed to cause the ice to lift and thus reduce the pressure back to overburden $P=P_o$, the pressure solution is subject to inequalities
\begin{equation}
0 \le P \le P_o. \label{eq:bounds}
\end{equation}

\subsection{Darcy flow}  \label{subsec:darcy}  Subglacial water flows from high to low hydraulic potential.  The simplest expression of this is a Darcy flux model for a water sheet,
\begin{equation}  \label{eq:fluxearly}
\bq = - K \,W\, \grad \psi
\end{equation}
where the hydraulic conductivity $K$ is a constant \citep{Clarke05}.  More generally \cite{Schoofetal2012} suggests a power law form
\begin{equation}  \label{eq:flux}
\bq = - k\, W^\alpha\, |\grad \psi|^{\beta-2} \grad \psi
\end{equation}
for $\alpha \ge 1$, $\beta>1$, and a coefficient $k>0$ with units that depend on $\alpha$ and $\beta$ (see Table \ref{tab:constants}).  \cite{Clarke05} suggests $\alpha=1$ and $\beta=2$, to give \eqref{eq:fluxearly} above, \cite{CreytsSchoof2009} use $\alpha=3/2$ and $\beta=3/2$, \cite{Hewitt2011,Hewitt2013} uses $\alpha=3$ and $\beta = 2$, and \cite{Hewittetal2012} suggest $\alpha=5/4$ and $\beta=3/2$.  The current paper implements law \eqref{eq:flux} generally but uses the \cite{Clarke05} and \cite{Hewittetal2012} exponents in an exact solution and in numerical experiments, respectively.  We call $K = k\, W^{\alpha-1}\, |\grad \psi|^{\beta-2}$ the \emph{effective} hydraulic conductivity, so that equation \eqref{eq:fluxearly} applies formally throughout.

\subsection{Advection-diffusion decomposition}  \label{subsec:advectdiffus}  Combining \eqref{eq:potential} and \eqref{eq:flux}, and separating the term proportional to $\grad W$, we get the flux expression
\begin{align}
\bq &= - k  W^\alpha \left|\grad \psi\right|^{\beta-2} \grad \left(P + \rho_w g b\right)  \label{eq:firstfluxdecomp} \\
    &\qquad\qquad - \rho_w g k W^\alpha \left|\grad \psi\right|^{\beta-2} \grad W  \notag
\end{align}
which suggests a mix of mechanisms.  If $P$ scales with the overburden pressure $P_o$, and if $|\grad (H+b)| \gg |\grad W|$, then the first flux term in \eqref{eq:firstfluxdecomp} will dominate.  The second term with $\grad W$ acts diffusively in the mass conservation equation \eqref{eq:conserve}.  We will see that in near-steady-state circumstances where there is significant sliding, the first term with $\grad P$ is also significantly diffusive in the mass conservation equation (subsection \ref{subsec:steady}).  In conditions far from steady state, however, the direction of $\grad P$ is presumably different from the direction $\grad W$.

We will construct our numerical scheme based on decomposition \eqref{eq:firstfluxdecomp}.  To simplify the expression slightly, the small thickness approximation $W\approx 0$ is made inside the absolute value signs in \eqref{eq:firstfluxdecomp}, namely
\begin{equation}
\left|\grad \psi\right| \approx \left|\grad \left(P + \rho_w g b \right)\right|.  \label{eq:Wsmall}
\end{equation}
This simplification, which makes no change in the $\beta=2$ case (see subsection \ref{subsec:darcy}), lets us redefine the effective hydraulic conductivity as
\begin{equation}
K = k W^{\alpha-1} \left|\grad(P+\rho_w g b)\right|^{\beta - 2}. \label{eq:Kdefine}
\end{equation}
In terms of $K$ we define a velocity field and a diffusivity coefficient:
\begin{equation} \label{eq:vexpression}
  \bV = - K\, \grad \left(P + \rho_w g b\right), \qquad D = \rho_w g K W,
\end{equation}
so that \eqref{eq:firstfluxdecomp} is a clean advection-diffusion decomposition,
\begin{equation} \label{eq:qexpression}
  \bq = \bV\, W - D \grad W.
\end{equation}

From equations \eqref{eq:conserve} and \eqref{eq:qexpression} we now have an advection-diffusion-production equation for the evolution of the conserved water amount $W+W_{til}$:
\begin{equation} \label{eq:adeqn}
  \frac{\partial W}{\partial t} + \frac{\partial \Wtil}{\partial t} = - \Div\left(\bV\, W\right) + \Div \left(D \grad W\right) + \frac{m}{\rho_w}.
\end{equation}
There are distinct numerical approximations for the advection term $\Div\left(\bV\, W\right)$ and the diffusion term $\Div \left(D \grad W\right)$, with time-step restrictions of different magnitudes (section \ref{sec:num}).  Equation \eqref{eq:adeqn} is often advection-dominated in the sense that $|\bV W| \gg |D \grad W|$, and numerical schemes for advection and diffusion must be tested in combination (section \ref{sec:results}).

\subsection{Capacity of a linked-cavity distributed system}  \label{subsec:cavities}  The rate of change of the area-averaged thickness of the cavities in a distributed linked-cavity system is the difference of opening and closing rates \citep{Hewitt2011}.  This thickness $Y$, also called ``bed separation'' \citep{Bartholomausetal2011}, has generic evolution equation
\begin{equation}
\frac{\partial Y}{\partial t} = \mathcal{O}(|\bv_b|,Y) - \mathcal{C}(N,Y) \label{eq:hewittcapacity}
\end{equation}
where $\bv_b$ is the ice base (sliding) velocity and $N=P_o-P$ is the effective pressure on the cavity system.  Denoting $X_+= \max\{0,X\}$, we choose a nonnegative opening term based on cavitation only:
\begin{equation}
\mathcal{O}(|\bv_b|,Y) = c_1 |\bv_b| (W_r - Y)_+. \label{eq:openingform}
\end{equation}
Here $c_1$ is a scaling coefficient and $W_r$ is a maximum roughness scale of the basal topography \citep{Schoofetal2012}; see Table \ref{tab:constants}.  The closing term models ice creep only \citep{Hewitt2011,Schoofetal2012}:
\begin{equation}
\mathcal{C}(N,Y) = c_2 A N^3 Y, \label{eq:closingform}
\end{equation}
where $c_2$ is a scaling coefficient and $A$ is the softness of the ice.  We have used Glen exponent $n=3$ for concreteness and simplicity.  The closing term $\mathcal{C}$ in \eqref{eq:closingform} is nonnegative because our modeled pressure $P$ satisfies bounds $0\le P \le P_o$.

\section{Till hydrology and mechanics} \label{sec:tillmechanics}

Till with pressurized liquid water in its pore spaces is expected to support much of the ice overburden.  When present, such saturated till is central to the complicated relationship between the amount of subglacial water and the speed of sliding.  Our model includes storage of subglacial water in till both because of its role in conserving the mass of liquid water and its role in determining basal shear stress.

We will assume throughout that liquid water or ice fills pore spaces in the till, and that there are no air- or vapor-filled pore spaces.  When $\Wtil=0$ in the model, the pore spaces in the till are regarded as filled with ice and the basal shear stress is correspondingly high.  When $\Wtil$ attains sufficiently large values, however, the till is regarded as saturated with liquid, and a drop in effective pressure becomes possible (subsection \ref{subsec:effectivepressuretill} below).

\subsection{Evolution of till-stored water layer thickness}  The water in till pore spaces is much less mobile than that in the linked-cavity system because of the very low hydraulic conductivity of till \citep{LingleBrown1987,TrufferEchelmeyerHarrison2001}.  Therefore we choose an evolution equation for $\Wtil$ without horizontal transport for simplicity \citep{BBssasliding,Tulaczyketal2000}, namely
\begin{equation}
\frac{\partial \Wtil}{\partial t} = \frac{m}{\rho_w} - C_d. \label{eq:tilldynamics}
\end{equation}
Here $C_d\ge 0$ is a fixed rate that makes the till gradually drain in the absence of water input; we choose $C_d=1$ mm/a, which is small compared to typical values of $m/\rho_w$.  Refreeze is also allowed, as a negative value for $m$.

As in \citep{BBssasliding}, we constrain the layer thickness by
\begin{equation}
0 \le \Wtil \le \Wtilmax.  \label{eq:Wtilbounds}
\end{equation}
Any water in excess of the capacity of the till, i.e.~$\Wtilmax$, ``overflows'' the till and enters the transport system; it is conserved.  Because the source term $m$ in equation \eqref{eq:tilldynamics}, or the whole right side, can be negative, the lower bound in \eqref{eq:Wtilbounds} must be actively-enforced.  The upper bound in \eqref{eq:Wtilbounds} also relates to the effective pressure on the till, as we explain next.

\subsection{Effective pressure on the till} \label{subsec:effectivepressuretill}  Deformation of saturated till is well-modeled by a plastic (Coulomb friction) or nearly-plastic rheology \citep{Hookeetal1997,TrufferHarrisonEchelmeyer2000,Tulaczyketal2000,SchoofTill}.  Its yield stress $\tau_c$ satisfies the Mohr-Coulomb relation
\begin{equation}
\tau_c = c_0 + (\tan \varphi) \Ntil  \label{eq:mohrcoulomb}
\end{equation}
where $c_0$ is the till cohesion, $\varphi$ is the till friction angle, and $\Ntil$ is the effective pressure of the overlying ice on the saturated till \citep{CuffeyPaterson}.  (Note that the effective pressure $N=P_o-P$ used in section \ref{subsec:cavities} for modeling cavity closure is distinct from $\Ntil$ in \eqref{eq:mohrcoulomb}.  This distinction is again explained by the very low hydraulic conductivity of till.)

Let $e = V_w / V_s$ be the till void ratio, where $V_w$ is the volume of water in the pore spaces and $V_s$ is the volume of mineral solids \citep{Tulaczyketal2000}.  From the standard theory of soil mechanics and from laboratory experiments on till \citep{Hookeetal1997,Tulaczyketal2000}, a linear relation exists between the logarithm of $\Ntil$ and $e$,
\begin{equation}
e = e_0 - C_c \log_{10}\left(\Ntil / N_0\right).  \label{eq:voidpressure}
\end{equation}
Figure \ref{fig:Ntilfunctions}(a) shows a graph of \eqref{eq:voidpressure}.  Here $e_0$ is the void ratio at a reference effective pressure $N_0$ and $C_c$ is the coefficient of compressibility of the till.  Equivalently to \eqref{eq:voidpressure}, $\Ntil$ is an exponential function of $e$, namely $\Ntil = N_0 10^{(e_0 - e)/C_c}$ \citep[equation (15)]{vanderWeletal2013}, so $\Ntil$ is nonzero for all finite values of $e$.

While \eqref{eq:voidpressure} suggests that the effective pressure could be any positive number, in fact the area-averaged value of $\Ntil$ under ice sheets and glaciers has limits.  It cannot exceed the overburden pressure for any sustained period.  Furthermore, once the till is close to its maximum capacity then the excess water will be ``drained'' into a transport system.  We suppose this occurs at a small, fixed fraction $\delta$ of the overburden pressure.  Thus we assume bounds
\begin{equation}
\delta P_o \le \Ntil \le P_o  \label{eq:Ntilbounds}
\end{equation}
where $\delta = 0.02$ in the experiments in this paper.

The void ratio $e$ and the effective water layer thickness $\Wtil$ are describing the same thing, namely the amount of liquid water.  In fact, if $\Delta x$, $\Delta y$ are the horizontal dimensions of a rectangular patch of till with (mineral-portion) thickness $\eta$ then $V_w = \Wtil \,\Delta x \,\Delta y$ and $V_s = \eta \,\Delta x \,\Delta y$.  Thus
\begin{equation}
e = \frac{\Wtil}{\eta}.  \label{eq:voidequivalent}
\end{equation}
On the other hand we specify a maximum $\Wtilmax$ on the water layer thickness, in bounds \eqref{eq:Wtilbounds}.  The minimum $\Ntil=\delta P_o$ of the effective pressure occurs at maximum values of void ratio $e$ and effective thickness $\Wtil$, so equations \eqref{eq:voidpressure} and \eqref{eq:voidequivalent} allow us to express the solid-till thickness $\eta$ in terms of our preferred parameters $\Wtilmax$, $\delta$, $e_0$, $N_0$, and $C_c$:
\begin{equation}
\eta = \frac{\Wtilmax}{e_0 - C_c \log_{10}\left(\delta P_o / N_0\right)}.  \label{eq:etaexpression}
\end{equation}

From \eqref{eq:voidpressure}, \eqref{eq:voidequivalent}, and \eqref{eq:etaexpression}, the effective pressure $\Ntil$ can now be written as the following function of $\Wtil$:
\begin{equation}
\hatNtil = N_0 \left(\frac{\delta P_o}{N_0}\right)^s 10^{(e_0/C_c)\,(1-s)}
\label{eq:NtilofWtil}
\end{equation}
where $s = \Wtil/\Wtilmax$.  However, as noted above, $\Ntil$ is bounded:
\begin{equation}
\Ntil = \min\left\{P_o,\, \hatNtil\right\}.
\label{eq:Wtilpressure}
\end{equation}
This function is shown in Figure \ref{fig:Ntilfunctions}(b).

It follows from equations \eqref{eq:mohrcoulomb}, \eqref{eq:NtilofWtil}, and \eqref{eq:Wtilpressure} that the yield stress $\tau_c$ is determined by the layer thickness $\Wtil$.  Regarding the parameters in this relation:
\renewcommand{\labelenumi}{(\emph{\roman{enumi}})}
\begin{enumerate}
\item Experiments on till suggest small values for cohesion $c_0$ in \eqref{eq:mohrcoulomb}, $0 \le c_0 \lesssim 1$ kPa \citep{Tulaczyketal2000}, and we choose $c_0=0$ for concreteness.
\item Measured till friction angles $\varphi$ are in a $18^\circ$--\,$40^\circ$ range \citep{CuffeyPaterson}.  Simulations of the whole Antarctic \citep{Martinetal2011} and Greenlandic \citep{AschwandenAdalgeirsdottirKhroulev} ice sheets have been based on a hypothesis that the till friction angle $\varphi$ depends on bed elevation so as to model the submarine history of low-elevation sediments.
\item The ratio $e_0/C_c$ can be determined by laboratory experiments on till samples \citep[e.g.][]{Hookeetal1997,Tulaczyketal2000}.  Values for the dimensionless constants $e_0$ and $C_c$ used here (Table \ref{tab:constants}) are from till samples from ice stream B in Antarctica \citep{Tulaczyketal2000}, and they give $e_0/C_c=5.75$ in \eqref{eq:NtilofWtil}.
\item The till capacity parameter $\Wtilmax$ could be set in a location-dependent manner from in situ \citep{Tulaczyketal2000} or seismic reflection \citep{Rooneyetal1987} evidence, but for simplicity we set it to a constant $2$ meters.
\end{enumerate}

\subsection{Sliding law}  Observe that the ice sliding velocity $\mathbf{v}_b$ is an input into the subglacial hydrology model we are building, because of equation \eqref{eq:openingform}.  On the other hand, the yield stress $\tau_c$ is an output of the till-related part of the hydrology model (subsection \ref{subsec:effectivepressuretill}).  In an ice dynamics model like PISM, $\mathbf{v}_b$ is determined by solving a stress balance in which the vector basal shear stress $\boldsymbol\tau_b$ appears either as a boundary condition \citep{SchoofCoulombBlatter} or as a term in a vertically-integrated balance \citep{SchoofStream,BBssasliding}.  In PISM, $\tau_c$ and $\mathbf{v}_b$ combine to determine $\boldsymbol\tau_b$ through a sliding law
\begin{equation}
\boldsymbol\tau_b = - \tau_c \frac{\mathbf{v}_b}{|\mathbf{v}_b|^{1-q} v_0^q}. \label{eq:pseudobasalstress}
\end{equation}
where $0\le q \le 1$ and $v_0$ is a threshold sliding speed \citep{AschwandenAdalgeirsdottirKhroulev}.

Power law \eqref{eq:pseudobasalstress} generalizes, and includes as the case $q=0$, the purely-plastic (Coulomb) relation $\boldsymbol\tau_b = - \tau_c \mathbf{v}_b/|\mathbf{v}_b|$.  At least in the $q\ll 1$ cases, under \eqref{eq:pseudobasalstress} the till ``yields'' and the magnitude of the basal shear stress becomes nearly independent of $|\mathbf{v}_b|$, when $|\mathbf{v}_b| \gg v_0$.  Equation \eqref{eq:pseudobasalstress} could also be written in generic power-law form $\boldsymbol\tau_b = - \beta |\mathbf{v}_b|^{q-1} \mathbf{v}_b$ with coefficient $\beta = \tau_c / v_0^q$; in the linear case $q=1$ we have $\beta = \tau_c/v_0$.


\section{Closures to determine pressure} \label{sec:closures}

The evolution equations listed so far, namely \eqref{eq:adeqn}, \eqref{eq:hewittcapacity}, and \eqref{eq:tilldynamics}, can be simplified to three equations in the four major variables $W$, $\Wtil$, $Y$, and $P$.  We do not yet know how to compute the water pressure $P$ or its rate of change $\partial P/\partial t$ given the other variables and data of the problem.  A closure is needed.

\subsection{Simplified closures without cavity evolution}  \label{subsec:simplifiedclosures}  We first consider two simple closures which appear in the literature but which do not use cavity evolution equation \eqref{eq:hewittcapacity} or similar physics.  We list them because the resulting simplified conservation equations emerge as reductions of our more complete theory.  For simplicity we present them without till storage ($\Wtilmax=0$) and only in the constant conductivity case ($\alpha=1$ and $\beta=2$).

Setting the pressure equal to the overburden pressure is the simplest closure \citep{LeBrocqetal2009,Shreve1972}:
\begin{equation}
P = P_o.\label{eq:Pisoverburden}
\end{equation}
This model is sometimes used for ``routing'' subglacial water under ice sheets so as to identify subglacial lake locations \citep{Goeller2014,Livingstoneetal2013,Siegertetal2009}.  Straightforward calculations using equations \eqref{eq:conserve}, \eqref{eq:flux}, and \eqref{eq:Pisoverburden} show that the advection-diffusion form \eqref{eq:adeqn} has an ice-geometry-determined velocity $\tilde\bV$,
\begin{align}
  \frac{\partial W}{\partial t} &= - \Div\left(\tilde\bV\, W\right) + \Div\left(\rho_w g k \,W\, \grad W\right) + \frac{m}{\rho_w},   \label{eq:PisoverConservation} \\
\tilde\bV &= - \rho_w g k \left[\frac{\rho_i}{\rho_w} \grad H + \left(1-\frac{\rho_i}{\rho_w}\right) \grad b\right]. \notag
\end{align}

Because the approximation $W\ll H$ is usually accepted, so that the hydraulic potential is insensitive to the water layer thickness, i.e.~$\psi = P_o + \rho_w g b$ \citep{LeBrocqetal2009}, the diffusion term in \eqref{eq:PisoverConservation} is usually not included.  With this common simplification, equation \eqref{eq:PisoverConservation} becomes an advection equation with a source term.  It therefore possesses characteristic curves, trajectories of the water flow or ``pathways'' \citep{Livingstoneetal2013}, which are determined by ice sheet geometry.  However, without the diffusion term, equation \eqref{eq:PisoverConservation} exhibits continuum solutions with infinite water concentration at every location where the simplified potential $\psi = P_o + \rho_w g b$ has a minimum.  Applications therefore only compute the characteristic curves themselves.  We prefer equation \eqref{eq:PisoverConservation} as stated, \emph{with} the diffusion term, because it is well-posed for positive initial and boundary values on $W$ \citep[compare][]{Hewittetal2012}, so that numerical solutions can converge under sufficient grid refinement.

At an almost opposite extreme, our second simplified closure makes the water pressure a function of the amount of water.  Specifically, \cite{FlowersClarke2002_theory} propose
\begin{equation}
P_{FC}(W) = P_o \left(\frac{W}{W_{\text{crit}}}\right)^{7/2}, \label{eq:PofWFC}
\end{equation}
where, for Trapridge glacier, \cite{FlowersClarke2002_trapridge} use $W_{\text{crit}}=0.1$ m.  Thus no separate pressure evolution equation needs to be solved.

One concern with form \eqref{eq:PofWFC} is that $P_{FC}(W)$ can be arbitrarily larger than overburden pressure \citep{Schoofetal2012}.  In any case, \eqref{eq:PofWFC} is used in equations \eqref{eq:conserve} and \eqref{eq:flux} to yield a nonlinear diffusion which generalizes the porous-medium equation $\partial W/\partial t = \grad^2 (W^\gamma)$ \citep{VazquezPME}.  The main idea in such a nonlinear diffusion is that the direction of the flux is $-\grad W$.  However, a Darcy-type model $\bq \sim -\grad \psi$ like \eqref{eq:flux} normally gives flux directions different from $-\grad W$ in many cases, especially in rapidly-evolving hydrologic systems, if the pressure is determined by a more physical closure.  We consider such a closure next.

\subsection{Full-cavity closure}  Simply requiring the subglacial layer to be full of water is also a closure \citep{Bartholomausetal2011}, which we adopt:
\begin{equation}
W = Y.\label{eq:strongclosure}
\end{equation}
The consequences of this closure are explored at some length by \cite{Schoofetal2012}, \cite{Hewittetal2012}, and \cite{Werderetal2013}, who describe the full-cavity case as the ``normal pressure'' condition.

Equation \eqref{eq:strongclosure} obviously allows us to eliminate either $W$ or $Y$ as a state variable.  We choose to eliminate $Y$ because $W$ is part of the conserved mass $W + \Wtil$.  In the zero till storage case, equations \eqref{eq:conserve}, \eqref{eq:hewittcapacity}, and \eqref{eq:strongclosure} imply
\begin{equation}
\mathcal{O}(|\bv_b|,W) - \mathcal{C}(N,W) + \Div\bq = \frac{m}{\rho_w}. \label{eq:elliptic}
\end{equation}
which is exactly elliptic pressure equation (2.12) of \cite{Schoofetal2012}.  They argue that a model based on \eqref{eq:elliptic} should accommodate the possibility of partially-empty cavities with $W<Y$ when $P=0$.  However, like \cite{Werderetal2013} who implement the model in two dimensions, we accept a potential loss of model completeness by using a full-cavity model.

\subsection{Englacial porosity as a pressure regularization}  Englacial systems of cracks, crevasses, and moulins have been observed in glaciers \citep[for example]{Fountainetal2005,Bartholomausetal2008,Harperetal2010}, and these have been included in combined englacial/subglacial hydrology models \citep{FlowersClarke2002_theory,Bartholomausetal2011,Hewitt2013,Werderetal2013}.  The englacial system is generally parameterized as having macroporosity $0\le \phi < 1$.  If the englacial system is efficiently-connected to the subglacial water then the amount of englacial water is equivalent to the subglacial pressure, which is reflected by an englacial ``water table'' in such models.

\cite{Bueler2014correspondence} shows that a distributed extension of the lumped englacial/subglacial model in \cite{Bartholomausetal2011} gives an equation similar to \eqref{eq:elliptic}.  The crucial difference from \eqref{eq:elliptic} is that the equation is parabolic for the pressure and not elliptic (compare \cite{Hewittetal2012}).  Based on this analysis, our model uses a parabolic regularization of \eqref{eq:elliptic} which has constant (notional) englacial porosity $\phi_0$:
\begin{align}
\frac{\phi_0}{\rho_w g} \ddt{P} &= - \Div \bq + \frac{m}{\rho_w} + \mathcal{C}(N,W)  \label{eq:regpressureequation} \\
  &\qquad\qquad - \mathcal{O}(|\bv_b|,W) - \ddt{\Wtil}. \notag
\end{align}
Compare equations (7) in \citep{Hewitt2013} and (24) in \citep{Werderetal2013}.  Unlike \cite{Werderetal2013}, however, we do not add an englacial water amount variable to the conservation equation, and in this sense the porosity only serves to regularize the pressure equation.

Using englacial porosity as a regularization, as in \eqref{eq:regpressureequation}, allows a user-adjustable trade-off between temporal detail in the pressure evolution versus computational effort \citep{vanPeltthesis}.  If the englacial porosity $\phi_0$ is small then there is a nearly impermeable ``cap'' on the subglacial system and equation \eqref{eq:regpressureequation} is stiff \citep{AscherPetzold}.  Equation \eqref{eq:regpressureequation} is then similar, in terms of numerical solution, to elliptic equation \eqref{eq:elliptic}.  Indeed, if elliptic equation \eqref{eq:elliptic} is used instead of \eqref{eq:regpressureequation} then the coupled hydrological equations system is differential-algebraic \citep{AscherPetzold}, and hardest to solve numerically.  By contrast, if $\phi_0$ is larger then equation \eqref{eq:regpressureequation} causes local changes in subglacial pressure $P$ to be damped in the speed and range of influence, on other parts of the connected subglacial hydrologic system, and the numerical solution is easier.

\cite{Schoofetal2012} show that the mathematical problem encompassing \eqref{eq:elliptic}, constraints \eqref{eq:bounds}, and appropriate pressure boundary conditions can be written as an elliptic variational inequality \citep{KinderlehrerStampacchia}.  Solving this variational inequality problem in two dimensions, at each time step, is asserted to be ``prohibitively expensive'' by \cite{Werderetal2013}.  Our adaptive explicit time-stepping scheme (section \ref{sec:num}), by contrast, solves \eqref{eq:regpressureequation}, while satisfying constraints \eqref{eq:bounds}, at demonstrably-reasonable computational cost (section \ref{sec:results}).

Stiffness in these pressure equations ultimately follows from the incompressibility of water and the relative non-distensibility (i.e.~hardness) of the ice and bedrock.  \cite{Clarke2003} addresses this in a physically-different manner from englacial porosity.  He includes a relaxation (damping) parameter  ``$\beta$'' which is based on the small compressibility of water, but which is more than two orders of magnitude larger than the physical value.  Clarke's parameter $\beta$ appears in his equation exactly as $\phi_0$ appears in equation \eqref{eq:regpressureequation}, multiplying the pressure time derivative.


\section{The new subglacial hydrology model in PISM} \label{sec:newmodel}

\subsection{Summary of the model}  The major evolution equations for the model are mass conservation \eqref{eq:adeqn}, till-stored water layer thickness evolution \eqref{eq:tilldynamics}, and pressure evolution \eqref{eq:regpressureequation}.  Collected here for clarity they are:
\begin{align}
&\ddt{W} + \ddt{\Wtil} = - \Div\left(\bV\, W\right) + \Div \left(D \grad W\right) + \frac{m}{\rho_w}, \label{eq:bluebox} \\
&\frac{\partial \Wtil}{\partial t} = \frac{m}{\rho_w} - C_d, \notag \\
&\frac{\phi_0}{\rho_w g} \ddt{P} + \ddt{\Wtil} = - \Div\left(\bV\, W\right) + \Div \left(D \grad W\right) + \frac{m}{\rho_w} \notag \\
& \qquad \qquad \qquad + c_2 A (P_o - P)^3 W - c_1 |\bv_b| (W_r - W)_+. \notag
\end{align}
Also recall these definitions:
\begin{align*}
D    &= \rho_w g K W && \text{diffusivity,} \\
K    &= k W^{\alpha-1} \left|\grad(P+\rho_w g b)\right|^{\beta-2} && \text{effective conductivity,} \\
P_o  &= \rho_i g H && \text{overburden pressure, and} \\
\bV  &= - K \grad\left(P + \rho_w g b\right) && \text{velocity}.
\end{align*}
Equations \eqref{eq:bluebox} are coupled to ice dynamics by Mohr-Coulomb equation \eqref{eq:mohrcoulomb} and till effective pressure equations \eqref{eq:NtilofWtil}, \eqref{eq:Wtilpressure}.

The model includes these bounds on major variables:
\begin{equation}
0\le W, \quad 0\le \Wtil \le \Wtilmax, \quad 0 \le P \le P_o.  \label{eq:allbounds}
\end{equation}
As a result of inequalities \eqref{eq:allbounds}, free boundaries arise in the domain at locations where, in particular, $m$ is sufficiently negative to drive $W$ to zero or where the pressure $P$ goes to zero or overburden.

A coupled weak formulation of equations \eqref{eq:bluebox} and constraints \eqref{eq:allbounds} would be a mathematically-rigorous unified description of the free boundary conditions, but this paper takes a more pragmatic approach, as follows.  First, PISM uses a periodic domain for whole ice sheet computations (section \ref{sec:results}), so the computational domain has no classical boundary.  Second, inequalities \eqref{eq:allbounds} are enforced in our coupled explicit scheme by truncation/projection (section \ref{sec:num}).  Third, at ice-free land and ocean (i.e.~ice shelf or ice-free ocean) grid locations, pressure $P$ is determined by atmospheric or ocean pressure, respectively.  Fourth and finally, at ice-free land and ocean grid locations the mass conservation equation effectively have $m$ sufficiently negative so that water which flows or diffuses into that grid location during a time step is fully removed and thus $W=0$ and $\Wtil=0$; see the ``mask'' variables in section \ref{sec:num}.

As in Table \ref{tab:symbols}, the functions in the model can be categorized into \emph{state} functions, which must be provided with initial values, \emph{input} functions, which are either supplied by observations or by other components of an ice sheet model, and \emph{output} functions which are supplied to other components of the ice sheet model.  In two-way coupling the ice dynamics model passes $H$, $m$, and $|\bv_b|$ to the subglacial hydrology model, and $\tau_c$ is returned.

\subsection{Reduction to existing models}  Four reductions (limiting cases) of model \eqref{eq:bluebox} can now be stated precisely:

\renewcommand{\labelenumi}{\textbf{(\roman{enumi})}}
\begin{enumerate}
\item The zero till storage ($\Wtilmax=0$) and zero englacial porosity ($\phi_0=0$) case of \eqref{eq:bluebox} is essentially the model described by \cite{Schoofetal2012}.  Recalling that $\bq = - K W \grad \psi$, the equations are
\begin{align}
&\frac{\partial W}{\partial t} = - \Div\left( K W \grad \psi \right) + \frac{m}{\rho_w}, \label{eq:schoofsmodel} \\
&0 = \Div \left( K W \grad \psi \right) + \frac{m}{\rho_w} \notag \\
&\qquad \qquad + c_2 A (P_o - P)^3 W - c_1 |\bv_b| (W_r - W)_+.  \notag
\end{align}
The bounds $W \ge 0$ and $0 \le P \le P_o$ are unchanged.  Model \eqref{eq:bluebox} is a parabolic version of \eqref{eq:schoofsmodel}, regularized using a notional connection to porous englacial storage, and with coupling to till storage.

\item The $P=P_o$ limit of \eqref{eq:bluebox}, in which the evolution equation for pressure is ignored, is essentially the model for ``routing'' water to subglacial lakes under cold ice sheets used by \cite{Siegertetal2009} and \cite{Livingstoneetal2013}.  As noted in section \ref{sec:closures}, the $\Wtilmax=0$ and $\alpha=1$ case of this model routes water with a velocity which is determined entirely by ice and bedrock geometry.

\item The non-distributed ``lumped'' form of \eqref{eq:bluebox}, in which, in particular, $\Div \bq = (q_{out} - q_{in})/L$ where $L$ is the length of the glacier and $q_{out},q_{in}$ are given by observations, is the model of \cite{Bartholomausetal2011}; see \cite{Bueler2014correspondence}.

\item The undrained plastic bed (UPB) model of \cite{Tulaczyketal2000b} arises as the $W=0,\bq=0,\phi_0=0$ reduction of \eqref{eq:bluebox}.  This model depends on friction-heating feedback to keep $\Wtil$ bounded, which is not effective if local friction heating is a non-local function of changes in till strength.  \cite{BBssasliding} therefore enforce $\Wtil \le \Wtilmax$ by removing water above the capacity $\Wtilmax$, giving a minimal non-conservative, but ``drained,'' version of the UPB model.
\end{enumerate}

The above list does not imply that all possible subglacial hydrology models are reductions of ours.  For example, the subglacial hydrology model of \cite{JohnsonFastook} is a variation on idea \textbf{(ii)} above but it is not a reduction.  The \cite{FlowersClarke2002_theory} model mentioned in subsection \ref{subsec:simplifiedclosures} is also not a reduction, though a significant connection is explained in the Appendix.

Two-dimensional models which include conduits \citep{Schoofmeltsupply} are not reductions of our model.  Conduit evolution is numerically-straightforward to implement in one-dimensional hydrology models \citep{Hewittetal2012,vanderWeletal2013}, but when extended to two-horizontal dimensions all existing models \citep{Schoofmeltsupply,Hewitt2013,Werderetal2013} become ``lattice'' models without a known continuum limit.  Our model has no conduit-like evolution equations at all, though the gradient-descent locations of characteristic curves from models using idea \textbf{(ii)} may correspond to the locations of conduits in some cases.

\subsection{Steady states}  \label{subsec:steady}  The steady form of model \eqref{eq:bluebox}, stated using $\alpha=1$, $\beta=2$, and $\Wtilmax=0$ for simplicity, can be written as follows in terms of $\bV,\bq,W,P$:
\begin{align}
\bV &= - k \grad \left(P + \rho_w g b\right), \label{eq:Vsteady} \\
\bq &= \bV W - \rho_w g k W \grad W, \label{eq:qsteady} \\
0 &= - \Div \bq + \frac{m}{\rho_w}, \label{eq:masscontsteady} \\
0 &= c_2 A (P_o - P)^3 W - c_1 |\bv_b| (W_r - W)_+. \label{eq:openclosesteady}
\end{align}
These steady state equations are also stated in the one-dimensional case by \cite{Schoofetal2012}, where traveling-wave exact solutions are also found.  Observe that the equations describing mass conservation \eqref{eq:masscontsteady} and cavity opening/closing processes \eqref{eq:openclosesteady} have become decoupled.

We make three observations about solutions to \eqref{eq:Vsteady}--\eqref{eq:openclosesteady}:
\renewcommand{\labelenumi}{(\emph{\roman{enumi}})}
\begin{enumerate}
\item from \eqref{eq:openclosesteady} there is a functional relationship $P=P(W)$,
\item by \eqref{eq:Vsteady} and \eqref{eq:openclosesteady}, the apparently advective flux ``$\bV W$'' in \eqref{eq:qsteady} actually acts diffusively, and
\item radial nearly-exact solutions can be constructed.
\end{enumerate}
In Appendix A we detail points (\emph{i}) and (\emph{ii}).  Observation (\emph{iii}) is addressed next.

\subsection{A nearly-exact steady state solution}  \label{subsec:exactsolution}  For the purpose of verifying numerical schemes we have built a two-dimensional, nearly-exact solution for $W$ and $P$, in a case with nontrivial overburden pressure and ice sliding.  It depends on the numerical solution of a scalar first-order ordinary differential equation (ODE) initial value problem, something we can do with high accuracy.

We solve the flat bed ($b=0$) angularly-symmetric case of coupled equations \eqref{eq:Vsteady}--\eqref{eq:openclosesteady}.  By assuming spatially-constant water input ($m=m_0$), a parabolic ice thickness profile in the radial coordinate $r$, and a particular profile of sliding---namely a function $|\bv_b(r)|$ with onset of sliding at location $r=5$ km, about one-fourth of the ice cap radius $r=22.5$ km---the equations reduce to a single first-order ODE in $r$ for the water thickness $W(r)$.  The pressure $P(r)$ is then determined from $W(r)$ by the functional relationship \eqref{eq:PofWsteady} which arises in steady state (Appendix A).

To compute the nearly-exact solution we use adaptive numerical ODE solvers, both a Runge-Kutta method and a variable-order stiff solver, with relative tolerance $10^{-12}$ and absolute tolerance $10^{-9}$.  The two solvers gave identical results to more than six digits.  The result $W(r)$ is shown in Figure \ref{fig:Wexact}, which also shows the regions of the $r,W$ plane which correspond to overpressure ($P=P_o$ in our model), normal pressure ($0<P<P_o$), and underpressure ($P=0$).  Figure \ref{fig:Pexact} shows the corresponding pressure solution $P(r)$.  Starting at the margin, we see that the solution is in the normal pressure region as $r$ decreases, until the onset of sliding ($r=5$ km).  At that location it switches into the overpressure case because there is no sliding.

Verification results using the nearly-exact solution appear in section \ref{sec:results}.  The numerical methods (next section) use a cartesian $(x,y)$ grid unrelated to the radial nearly-exact solution.  Thus numerical error comes from generic relationships between exact solution features and the grid.


\section{Numerical schemes}  \label{sec:num}

The equations in model \eqref{eq:bluebox} are discretized by explicit finite difference methods \citep{MortonMayers}.   A centered, second-order scheme is applied to the diffusion part of the mass conservation equation in \eqref{eq:bluebox}, but two upwind-type schemes for the advection part are compared, namely first-order ``donor cell'' upwinding \citep{LeVeque} and a higher-order flux-limited upwind-biased method \citep{HundsdorferVerwer2010}.  All the numerical schemes are implemented in parallel using the PETSc library \citep{petsc-user-ref}.

\subsection{Discretization of the mass conservation equation}  \label{subsec:Wfd}  To set notation, suppose the rectangular computational domain has $M_x \times M_y$ gridpoints $(x_i,y_j)$ with uniform spacing $\Delta x,\Delta y$.  Let $\Wlij \approx W(t_l,x_i,y_j)$, $(\Wtil)_{ij}^l \approx \Wtil(t_l,x_i,y_j)$, and $\Plij \approx P(t_l,x_i,y_j)$ denote the numerical approximations.

We compute velocity components and flux components at the staggered (cell-face-centered) points, shown in Figure \ref{fig:stencil}, from centered finite difference approximations of equations \eqref{eq:vexpression} and \eqref{eq:qexpression}.  We use ``compass'' indices for such staggered values, so that, for example, the ``east'' and ``north'' staggered water layer thicknesses are computed by averaging regular grid values:
\begin{equation}
W_e = (W_{i,j}^l + W_{i+1,j}^l)/2, \quad W_n = (W_{i,j}^l + W_{i,j+1}^l)/2. \label{eq:stagW}
\end{equation}

The nonlinear effective conductivity $K$ from \eqref{eq:Kdefine} is also needed at staggered locations.  As a notational convenience define $R=P+\rho_w g b$ and define these staggered-grid values \citep[compare][]{Mahaffy}:
\begin{align*}
&\Pi_e = \left|\frac{R_{i+1,j}-R_{i,j}}{\Delta x}\right|^2 \\
&\qquad \qquad + \left|\frac{R_{i+1,j+1}+R_{i,j+1} - R_{i+1,j-1}-R_{i,j-1}}{4\Delta y}\right|^2, \\
&\Pi_n = \left|\frac{R_{i+1,j+1}+R_{i+1,j} - R_{i-1,j+1}-R_{i-1,j}}{4\Delta x}\right|^2 \\
&\qquad \qquad + \left|\frac{R_{i,j+1}-R_{i,j}}{\Delta y}\right|^2.
\end{align*}
Thereby define
\begin{equation}
K_e = k W_e^{\alpha-1} \Pi_e^{(\beta-2)/2}, \quad K_n = k W_n^{\alpha-1} \Pi_n^{(\beta-2)/2}.  \label{eq:stagK}
\end{equation}
The velocity components $(u,v)$ of the water velocity $\mathbf{V}$ are then found by differencing:
\begin{equation}
u_e = - K_e \frac{R_{i+1,j}-R_{i,j}}{\Delta x}, \quad v_n = - K_n \frac{R_{i,j+1}-R_{i,j}}{\Delta y}.  \label{eq:velocitycomp}
\end{equation}
For diffusivity we simply have
\begin{equation}
D_e = \rho_w g K_e W_e, \quad D_n = \rho_w g K_n W_n.  \label{eq:diffusivitycomp}
\end{equation}
We get the remaining staggered-grid quantities by shifting indices.

Define $Q_e(u_e)$, $Q_w(u_w)$, $Q_n(v_n)$, and $Q_s(v_s)$ as the face-centered (staggered-grid) normal components of the advective flux $\bV W$; more detail is given in the next subsection.  The grid values of $\mathcal{D} = \Div \bq = \Div (\bV W) - \Div (D \grad W)$ using \eqref{eq:velocitycomp} and \eqref{eq:diffusivitycomp} now become:
\begin{align}
&\mathcal{D}_{i,j} =  \frac{Q_e(u_e) - Q_w(u_w)}{\Delta x} + \frac{Q_n(v_n) - Q_s(v_s)}{\Delta y}  \label{eq:fluxdivgrid} \\
&\qquad \qquad - \frac{D_e (W_{i+1,j}^l - \Wlij) - D_w (\Wlij - W_{i-1,j}^l)}{\Delta x^2} \notag \\
&\qquad \qquad - \frac{D_n (W_{i,j+1}^l - \Wlij) - D_s (\Wlij - W_{i,j-1}^l)}{\Delta y^2}.  \notag
\end{align}
Local conservation is ensured by using $Q_e(u_e)$ in computing $\mathcal{D}_{i,j}$ equal to $Q_w(u_w)$ used in $\mathcal{D}_{i+1,j}$, and so on.

Our scheme for approximating mass conservation equation \eqref{eq:adeqn} is
\begin{equation}
\frac{W_{i,j}^{l+1} - W_{i,j}^l}{\Delta t} + \frac{(\Wtil)_{i,j}^{l+1} - (\Wtil)_{i,j}^l}{\Delta t} = - \mathcal{D}_{i,j} + \frac{m_{ij}}{\rho_w}.    \label{eq:Wfd}
\end{equation}
The updated value of $\Wtil$, which appears on the left side of \eqref{eq:Wfd}, is computed by trivial integration of equation \eqref{eq:tilldynamics},
\begin{equation}
(\Wtil)_{i,j}^{l+1} = (\Wtil)_{i,j}^l + \Delta t \left(\frac{m_{ij}}{\rho_w} - C_d\right).  \label{eq:tillupdatefd}
\end{equation}
The given value $\Wtil^{l+1}$ is used if it is in the closed interval $[0,\Wtilmax]$, but otherwise the bounds $0\le \Wtil \le \Wtilmax$ are enforced.  Once $\Wtil^{l+1}$ is computed, the value of $W^{l+1}$ can be updated by \eqref{eq:Wfd} in a mass-conserving way.

Assuming no error in the flux components $Q$, the local truncation error \citep{MortonMayers} of scheme \eqref{eq:Wfd} would be $O(\Delta t^1 + \Delta x^2 + \Delta y^2)$ as an approximation of \eqref{eq:adeqn}.  The actual truncation error depends on the approximation of the discrete fluxes, addressed next.

\subsection{Discrete advective fluxes}  \label{subsect:discretefluxes}  We test two flux-discretization schemes, namely a first-order upwind scheme and the Koren flux-limited third-order scheme \citep{HundsdorferVerwer2010}.  Both schemes achieve non-oscillation and positivity, but with different local truncation error and complexity of implementation.  The third-order scheme is best explained as a modification of our conservative (``donor cell''; \cite{LeVeque}) first-order upwind scheme.

For a flux-limited scheme, the following formulas apply in the cases $u_e \ge 0$, $u_e < 0$, $v_n \ge 0$, and $v_n < 0$, respectively:
\begin{align}
&Q_e(u_e) = u_e \left[W_{i,j} + \Psi(\theta_{i}) (W_{i+1,j} - W_{i,j})\right], \label{eq:adfluxes} \\ 
&Q_e(u_e) = u_e \left[W_{i+1,j} + \Psi\left((\theta_{i+1})^{-1}\right) (W_{i,j} - W_{i+1,j})\right],  \notag \\
&Q_n(v_n) = v_n \left[W_{i,j} + \Psi(\theta_{j}) (W_{i,j+1} - W_{i,j})\right], \notag \\
&Q_n(v_n) = v_n \left[W_{i,j+1} + \Psi\left((\theta_{j+1})^{-1}\right) (W_{i,j} - W_{i,j+1})\right],\notag
\end{align}
where the subscripted $\theta$ quotients are
\begin{equation*}
\theta_i = \frac{W_{i,j}-W_{i-1,j}}{W_{i+1,j} - W_{i,j}}, \quad \theta_j = \frac{W_{i,j}-W_{i,j-1}}{W_{i,j+1} - W_{i,j}}.
\end{equation*}

The first-order upwind scheme simply sets $\Psi(\theta)=0$ in formulas \eqref{eq:adfluxes}.  The Koren scheme limits its third-order and positive-coefficient correction to the upwind scheme by using this formula \citep{HundsdorferVerwer2010}:
\begin{equation}
\Psi(\theta) = \max\left\{0,\min\Big\{1,\theta,\frac{1}{3}+\frac{1}{6} \theta\Big\}\right\}.  \label{eq:korenPsi}
\end{equation}
When using the Koren flux-limiter the stencil in Figure \ref{fig:stencil} is extended because regular grid neighbors $W_{i+2,j}$, $W_{i-2,j}$, $W_{i,j+2}$, $W_{i,j-2}$ are also involved in updating $W_{i,j}$.  The flux-correction-limited Koren third-order scheme bypasses the first-order limitation of positive linear finite difference/volume schemes imposed by Godunov's barrier theorem \citep[section I.7.1]{HundsdorferVerwer2010} by using a nonlinear correction formula.  Though the Koren scheme is third-order where smoothness allows, it reverts to first-order at extrema and jumps where $\theta \gg 1$ or $\theta \ll 1$.

For either scheme, if the water input $m$ is negative then we must actively enforce, by truncation, the positivity of the water thickness $W$.  In fact, positivity of the source-free advection-diffusion scheme, a desirable property which we can show by standard methods \citep{HundsdorferVerwer2010}, does not ensure positivity of the solution if there is  water removal, i.e.~if $m/\rho_w - \partial \Wtil/\partial t < 0$.

\subsection{Discretization of the pressure equation}  \label{subsec:Pfd}  Pressure evolution equation \eqref{eq:regpressureequation} is a nonlinear diffusion with ``reaction'' terms from the opening and closing of cavities.  However, our numerical scheme for this equation is similar to the scheme for the mass conservation equation (section \ref{subsec:Wfd}) because the spatial derivatives are actually the same in each equation, namely $\Div\bq$.  Thus we reuse the computation of those derivatives, namely scheme \eqref{eq:fluxdivgrid}, which gives $\mathcal{D}_{i,j}$.

Let $\mathcal{O}_{ij}$, $\mathcal{C}_{ij}$ be the gridded values of the zeroth-order (i.e.~without spatial derivatives) opening and closing rates; see equations \eqref{eq:openingform}, \eqref{eq:closingform}.  Define the sum of all zeroth-order terms:
\begin{equation}
Z_{ij} = \mathcal{C}_{ij} - \mathcal{O}_{ij} + \frac{m_{ij}}{\rho_w} - \frac{(\Wtil)_{ij}^{l+1} - (\Wtil)_{ij}^l}{\Delta t}.  \label{eq:defnZ}
\end{equation}
Using \eqref{eq:fluxdivgrid} for the flux divergence, the scheme for pressure equation \eqref{eq:regpressureequation} is
\begin{equation}
\frac{\phi_0}{\rho_w g} \frac{P_{i,j}^{l+1} - \Plij}{\Delta t} = - \mathcal{D}_{i,j} + Z_{ij}. \label{eq:Pfd}
\end{equation}
Because equation \eqref{eq:defnZ} uses the updated value $(\Wtil)_{ij}^{l+1}$, equation \eqref{eq:tillupdatefd} must be applied before \eqref{eq:Pfd} can be used to update $P$.  There are also special cases at the boundaries of the region where $W>0$; see subsection \ref{subsec:onestep}.

\subsection{Stability of time-stepping}  A sufficient condition for stability of mass-conservation scheme \eqref{eq:Wfd} comes from combining sufficient conditions for stability of the advection and diffusion parts.  For the advection part we first define $\Delta t_{\text{CFL}}$, after the well-known Courant-Friedrichs-Lewy restriction for advection schemes \citep{MortonMayers}, by
\begin{equation}
\Delta t_{\text{CFL}} \left(\frac{\max |u|}{\Delta x} + \frac{\max |v|}{\Delta y}\right) = \frac{1}{2}, \label{eq:dtCFL}
\end{equation}
where $\mathbf{V}=(u,v)$ is the velocity of the water in the distributed system.  For the diffusion part we define $\Delta t_{W}$ by
\begin{equation}
\Delta t_W\, \max D \left(\frac{1}{\Delta x^2} + \frac{1}{\Delta y^2}\right) = \frac{1}{4}. \label{eq:dtDIFFW}
\end{equation}
The condition $\Delta t \le \min\{\Delta t_{\text{CFL}}, \Delta t_W\}$ is sufficient for stability and convergence of scheme \eqref{eq:Wfd} if $\mathbf{V}$, $D$, and $m$ were all externally-provided functions, i.e.~in the case where the equations of \eqref{eq:bluebox} are decoupled.  We can show this by maximum principle arguments for the first-order upwind advection choice \citep{MortonMayers}, but standard theory at least suggests the same conclusion for the higher-order flux-limited advection scheme \citep{HundsdorferVerwer2010}.

These time-step restrictions can be understood by considering an example.  We ran the model on a $\Delta x = \Delta y = 250$ m grid to approximate steady state for the subglacial hydrology of \Nbreen \citep{vanPeltthesis}.  We used realistic inputs for $H$, $b$, and $m$, but a spatially-constant ice sliding rate of $|\mathbf{v}_b|=50$ m $\text{a}^{-1}$; other parameter values were from Table \ref{tab:constants}.  The result is that the maximum computed water speed $|\bV|$ is about $0.2$ m $\text{s}^{-1}$ so \eqref{eq:dtCFL} gives $\Delta t_{\text{CFL}} \approx 300\,\text{s}$.  Computed diffusivity $D = \rho_w g K W$ has a maximum value that varies significantly in time, $0.1 \le \max D \le 5 \,\text{m}^2\,\text{s}^{-1}$.  Using a typical value $\max D=1\,\text{m}^2\,\text{s}^{-1}$ in \eqref{eq:dtDIFFW} gives $\Delta t_W \approx 8000\,\text{s}$.  Thus in this simulation $\Delta t_W \approx 25 \Delta t_{CFL}$.  This example suggests that, unless both the maximum speed $|\mathbf{V}|$ is unusually low, and deep subglacial lakes develop so that $\max D$ is large, the diffusive time scale is significantly longer than the CFL time scale.  The scaling $\Delta t_W = O(\Delta x^2)$ versus $\Delta t_{CFL} = O(\Delta x^1)$ makes it clear that under sufficient spatial grid refinement $\Delta t_W$ is controlling, but we suspect that $\Delta t_{CFL}$ is controlling for $\Delta x > 100$ m.

However, the time step restriction from the pressure equation scheme is typically shorter than either $\Delta t_W$ or $\Delta t_{CFL}$.  The time step restriction for scheme \eqref{eq:Pfd} is comparable to $\Delta t_W$, though the proof above for the stability of the mass conservation scheme does not suffice to prove stability.  Nonetheless we define $\Delta t_P$ by
\begin{equation}
\Delta t_P\, \left(\frac{2 \max D}{\phi_0}\right) \left(\frac{1}{\Delta x^2} + \frac{1}{\Delta y^2}\right) = 1. \label{eq:dtDIFFP}
\end{equation}
If the time step is set by
\begin{equation}
\Delta t = \min\{\Delta t_{\text{CFL}}, \Delta t_W, \Delta t_P\}.  \label{eq:stablecombined}
\end{equation}
then we observe in practice that the coupled scheme consisting of \eqref{eq:Wfd}, \eqref{eq:tillupdatefd}, and \eqref{eq:Pfd} is stable.

Recalling \eqref{eq:dtDIFFW}, however, $\Delta t_P$ is actually a fraction of $\Delta t_W$, namely $\Delta t_P = 2 \phi_0\, \Delta t_W$.  If we return to the above example for \Nbreen, with $\phi_0 = 0.01$ we have $\Delta t_W \approx 8000$ s, $\Delta t_{\text{CFL}} \approx 300$ s, and $\Delta t_P \approx 160$ s.  In this case the pressure scheme has the shortest time step, but it is comparable to CFL.  Because $\Delta t_P$ is $O(\Delta x^2)$, the pressure scheme restriction is certainly controlling for sufficiently-fine grids.  However, the time step $\Delta t_P$ also scales with porosity $\phi_0$, so we can make it more or less severe by adjusting that parameter.

If implicit time-stepping were instead used for the pressure equation, which requires overt variational inequality treatment to preserve physical pressure bounds \citep{Schoofetal2012}, then the time scales $\Delta t_W, \Delta t_{CFL}$ addressed here are the only restrictions.  The time step restriction $\Delta t_W$ could also be removed by implicit steps for the mass-conservation equation, though again this requires a variational inequality formulation because of the lower bound $W\ge 0$.  Our observation above that $\Delta t_{CFL} \ll \Delta t_W$ for practical ice sheet grids suggests that implicit time-stepping for the mass-conservation equation is not beneficial.

\subsection{One time step of the model}  \label{subsec:onestep}  Mathematical model \eqref{eq:bluebox} evolves the fields $W$, $\Wtil$, and $P$.  Here we describe one time step of the fully-discretized coupled evolution.

For convenience only we denote the ice geometry, bed geometry, and sliding speed (i.e.~$H_{i,j}$, $b_{i,j}$, $(P_o)_{i,j}$, and $|\bv_b|_{i,j}$) as though they were all time-independent.  The geometry may be quite general, with ice-free land, floating ice shelf, or ice-free ocean allowed at any location $(x_i,y_j)$.  In fact, the geometry data determines boolean ``masks'' on the grid, based on zero as the sea level elevation:
\begin{align*}
\text{\texttt{icefree}}_{i,j} &= (H_{i,j} = 0)\, \&\, (b_{i,j} > 0), \\
\text{\texttt{float}}_{i,j}   &= (\rho_i H_{i,j} < - \rho_{sw}\, b_{i,j}),
\end{align*}
where $\rho_{sw}=1028.0$ is sea-water density.  Note $\text{\texttt{float}}_{i,j}$ is true both where there is floating ice shelf and where the ocean is ice-free.  The subglacial hydrology model exists only for grounded ice, that is, only if both flags \texttt{icefree} and \texttt{float} are false.

One time step follows this algorithm:

\renewcommand{\labelenumi}{\emph{(\roman{enumi})}}
\begin{enumerate}
\item Start with values $\Wlij$, $(\Wtil)_{i,j}^l$, $\Plij$ which satisfy bounds $W\ge 0$, $0\le \Wtil \le \Wtilmax$, and $0 \le P \le P_o$.
\item Get $(\Wtil)_{i,j}^{l+1}$ by \eqref{eq:tillupdatefd}.  Enforce $0\le \Wtil \le \Wtilmax$.  If $\text{\texttt{icefree}}_{i,j}$ or $\text{\texttt{float}}_{i,j}$ then set $(\Wtil)_{i,j}^{l+1}=0$.
\item Get $W$ values averaged onto the staggered grid from \eqref{eq:stagW}, staggered grid values of the effective conductivity $K$ from \eqref{eq:stagK}, velocity components $u$, $v$ at staggered grid locations from \eqref{eq:velocitycomp}, and staggered grid values of the diffusivity $D$ from \eqref{eq:diffusivitycomp}.
\item Get time step $\Delta t$ from \eqref{eq:stablecombined}.
\item Using \eqref{eq:adfluxes} and a flux-limiter $\psi(\theta)$, compute the advective fluxes $Q_e(\alpha_e)$ and $Q_n(\beta_n)$ at all staggered-grid points.
\item Get flux divergence approximations $\mathcal{D}_{i,j}$ from \eqref{eq:fluxdivgrid}.
\item If $\text{\texttt{icefree}}_{i,j}$ then set $P_{i,j}^{l+1}=0$.  If $\text{\texttt{float}}_{i,j}$ then set $P_{i,j}^{l+1} = (P_o)_{i,j}$.  If $\Wlij=0$, and if $\text{\texttt{icefree}}_{i,j}$ and $\text{\texttt{float}}_{i,j}$ are both false, then either set $P_{i,j}^{l+1} = (P_o)_{i,j}$ (no sliding) or $P_{i,j}^{l+1} = 0$ (any sliding).  Otherwise use \eqref{eq:Pfd} to compute $P_{i,j}^{l+1}$.
\item If $P_{i,j}^{l+1}$ does not satisfy bounds $0 \le P \le P_o$ then truncate/project into this range.
\item If $\text{\texttt{icefree}}_{i,j}$ or $\text{\texttt{float}}_{i,j}$ then set $W_{i,j}^{l+1}=0$.  Otherwise use \eqref{eq:Wfd} to compute values for $W_{i,j}^{l+1}$.
\item If $W_{i,j}^{l+1}<0$ then truncate/project to get $W_{i,j}^{l+1}=0$.
\item Update time $t_{l+1}=t_l+\Delta t$ and repeat at \emph{(i)}.
\end{enumerate}

\medskip
This recipe goes with a reporting scheme for mass conservation.  Note that in steps \emph{(ii)} and \emph{(ix)} water is lost or gained at the margin where either the ice thickness goes to zero on land (margins), or at locations where the ice becomes floating (grounding lines).  Because such loss/gain may be the modeling goal---users want hydrological discharge---these amounts are reported.  This reporting scheme also tracks the projections in step \emph{(x)}, which represent a mass conservation error which goes to zero in the continuum limit $\Delta t\to 0$.


\subsection{Run-time options for hydrology models} \label{subsec:pismdoc}  Option \texttt{-hydrology NAME}, where \texttt{NAME} is one of the three headings below, chooses the model equations.

\medskip
\noindent \texttt{distributed}:\quad  This model is governed by the full set of equations \eqref{eq:bluebox} in section \ref{sec:newmodel}.  The full set of parameters (Table \ref{tab:constants}) and variables (Table \ref{tab:symbols}) are active in this model.

\medskip
\noindent \texttt{routing}:\quad  In this reduced model the equation for pressure evolution is replaced by $P = P_o$.  The evolution equations for the state variables $W$ and $W_{til}$, and the bounds $0\le W$ and $0\le \Wtil \le \Wtilmax$, are unchanged.

\medskip
\noindent \texttt{null}:\quad  This further-reduced model is non-conserving.  It has only the state variable $\Wtil$ which is subject to bounds $0\le \Wtil \le \Wtilmax$ and evolves by equation \eqref{eq:tilldynamics}.

\medskip
The correspondence between the notation in this paper and PISM's configurable parameters is shown in Table \ref{tab:correspondence}.  These parameters can be set at runtime by using the parameter name as an option, or by setting a \verb|pism_overrides| variable in a NetCDF file which is read with the \verb|-config_override | option \citep{pism-user-manual}.  File \verb|src/pism_config.cdl| determines the default values and units.


\section{Results}  \label{sec:results}

\subsection{Verification of the coupled model}  By using the coupled, steady-state, nearly-exact solution (subsection \ref{subsec:exactsolution}) we verified most of the numerical schemes described above.  (Verification is the process of measuring and analysing the errors made by the numerical scheme, especially as the numerical grid is refined \citep{Wesseling}.)  To do this we initialized our time-stepping numerical scheme with the nearly-exact steady solution and we measured the error relative to the exact values after one model-month.  The continuum time-dependent model \eqref{eq:bluebox} would cause no drift away from steady state, so any drift is numerical error.  We did runs on grids decreasing by factors of two from $2$ km to $125$ m.  Figure \ref{fig:refineWPpism} shows the results based on first-order upwinding for the fluxes.

This convergence evidence suggests that we have implemented the numerical schemes in section \ref{sec:num}, for the coupled advection-diffusion-reaction equations for $W$ and $P$, correctly.  The rate of convergence in this verification case is roughly linear (i.e.~about $O(\Delta x^1)$) because the largest errors arise at locations of low regularity of the exact solution, including the radius $r=5$ km where $P$ quickly drops from $P_o$, and at the ice sheet margin where there is a jump in $W$ to zero.

The rates of convergence for average errors are nearly identical for the higher-resolution flux-limited scheme and for the first-order upwinding scheme (not shown).  Because our problem is an advection-diffusion problem in which both the advection velocity and the diffusivity are solution-dependent, it is difficult to separate the errors arising from numerical treatments of advection and diffusion.  The first-order upwinding scheme for the advection has much larger numerical diffusivity but this diffusivity is masked by the physical diffusivity.  Based on our verification evidence it is reasonable to choose the simpler first-order upwinding for applications, as it requires less interprocess communication.


\subsection{Application to the Greenland ice sheet}

We now apply our hydrology models to the entire Greenland ice sheet at 2 km grid resolution.  This nontrivial example demonstrates the model at large computational scale using real ice sheet geometry, with one-way coupling from ice dynamics giving a realistic distributions of overburden pressure, ice sliding speed, and basal melt rate.

\subsubsection{Spun-up initial state}  The PISM dynamics and thermodynamics model \citep{BBssasliding,Winkelmannetal2011,AschwandenBuelerKhroulevBlatter}, using the non-mass-conserving \texttt{null} hydrology model (subsection \ref{subsec:pismdoc}), was used to compute a consistent and nearly-steady model of the ice sheet, a ``spun-up'' initial state, following the procedures in \cite{AschwandenAdalgeirsdottirKhroulev}.  Our model uses no spatially-variable parameter values, such as basal shear stresses, found by inversion of surface velocities.  The bed elevations and present-day climate of the ice sheet, especially surface temperature and surface mass balance \citep{Ettemaetal2009}, were from the SeaRISE data set for Greenland \citep{Bindschadler2013SeaRISE}.

The spin-up grid sequence was to run 50 ka on a 20 km grid, 20 ka on a 10 km grid, 2 ka on a 5 km grid, and finally 200 a on a 2 km grid, with bilinear interpolation at each refinement stage.  The final 2 km stage, on a horizontal grid of 1.05 million grid points, used uniform 10 m vertical spacing so that the ice sheet flow was modelled on a structured 3D grid of 460 million velocity/temperature points.  This whole spin-up used 2800 total processor-hours on 72 2.2 GHz AMD Opteron processors, a small computation for modern supercomputers.

The results of this spin-up were validated by comparing results to present-day observations.  In the last 100 a of this run the ice sheet volume varied by less than $0.04$ percent, so the model is in nearly steady state, though the actual Greenland ice sheet may not be as close to steady.  The spun-up ice sheet volume of $3.094\times 10^{6}\,\textrm{km}^3$ is close to the present-day volume of $3.088\times 10^{6}\,\textrm{km}^3$ computed from the SeaRISE data on the same grid.  Compared to volume alone, a better evaluation of dynamical quality is to compare the modeled and observed \citep{Joughinetal2010} surface speed, with a very similar result to the comparison described in \cite{AschwandenAdalgeirsdottirKhroulev}.

The spun-up initial state includes, in particular, modelled ice thickness $H$, basal melt rate $m$, and sliding velocity $|\bv_b|$; the latter two fields are shown in Figure \ref{fig:Greenhydroinputs}.  Areas of sliding roughly coincide with areas of basal melt because heat-producing (modeled) basal drag comes from the yield stress parameterized in section \ref{sec:tillmechanics}.

\subsubsection{Experimental setup and model runs}  We used fields $H$, $m$, $|\bv_b|$ from the spun-up state as steady data in five model-year runs of the \texttt{routing} and \texttt{distributed}  hydrology models; see subsection \ref{subsec:pismdoc} for model descriptions.  Thus only one-way coupling was tested: a steady ice dynamics model fed its fields to an evolving subglacial hydrology model.  The hydrology model was initialized with the $\Wtil$ values from the spun-up state, but with $W=0$ initial values for both models, and also $P=0$ initial values for \texttt{distributed}.

In the runs, variables $W$, $\Wtil$, and $P$ were recomputed at each time-step, at each of 1.05 million subglacial hydrology grid points, using parameter values from Table \ref{tab:constants}.  In both \texttt{routing} and \texttt{distributed} models the hydrological system became steady after the first three model years.

Adaptively-determined time-steps reached a steady level of about 4 model-hours for the \texttt{routing} model based on maximum subglacial water speeds $|\bV|$ of 0.05 $\text{m}\,\text{s}^{-1}$ and maximum diffusivity $D$ of 10.6 $\text{m}^2\,\text{s}^{-1}$.  For the \texttt{distributed} model the time steps were actually slightly longer, primarily because \texttt{routing} concentrates large water amounts and fluxes along steepest-descent paths; the time steps were about 6 model-hours based on maximum speeds $|\bV|$ of 0.03 $\text{m}\,\text{s}^{-1}$ and much smaller maximum diffusivities $D$ of about 0.25 $\text{m}^2\,\text{s}^{-1}$.    These hydrology-only runs used much less computation than the spin-up: 14.7 processor-hours for the \texttt{routing} run and 14.2 for \texttt{distributed}. 

\subsubsection{\texttt{routing} results}  The final $\Wtil$ and $W$ fields from the \texttt{routing} run are shown in Figure \ref{fig:Greenroutingresults}.  The till is fully saturated ($\Wtil=2$ m) in essentially all areas where basal melt occurs.  In the outlet glacier areas the transportable water $W$ concentrates along curves of steepest descent of the hydraulic potential; see detail in Figure \ref{fig:Greenroutingdetail}.  The location of the pathways is determined primarily by the bedrock elevation detail provided by the SeaRISE data set, which is limited.  Furthermore, the grid resolution of 2 km, while very high for whole ice sheet models, still causes spatial ``smearing'' of the flow pathways.

The continuum limit of the model would have concentrated pathways of a few meters to tens of meters width.  These concentrated pathways could be regarded as minimal ``conduit-like'' features of the subglacial hydrology.  As noted in the introduction, however, our model has no ``R-channel'' conduit mechanism, in which dissipation heating of the flowing water generates wall melt-back.

\subsubsection{\texttt{distributed} results}  The final values of $W$ and the relative water pressure $P/P_o$ for the five model-year \texttt{distributed} run are shown in Figure \ref{fig:Greendistributedresults}.  The till is full ($\Wtil=2$ m) in essentially all areas where basal melt occurs, so, as $\Wtil$ is nearly-identical to the \texttt{routing} result, it is not shown.

Recall that $|\bv_b|$ determines the pressure drop caused by sliding-generated cavities.  The effect is to spread out the water $W$ relative to the \texttt{routing} model, as clearly seen in Figure \ref{fig:Greendistributedresults}.  There is now no strong concentration of $W$ along curves of steepest descent of the hydraulic potential, but the spreading depends on opening and closing parameters in the \texttt{distributed} model, especially parameters $c_1,c_2,\phi_0,W_r$.  Darcy flux model parameters $\alpha,\beta,k$ are also important.  Parameter identification using observed surface, in situ, basal-reflectivity, discharge, and other data, though needed, is beyond our current scope.

We can examine the local relationship between water layer thickness $W$ and pressure $P$ in the \texttt{distributed} results.  Though the model is near steady state, the basal melt rate, sliding speed, and overburden pressure all show realistically-large spatial variations.  In Figure \ref{fig:GreenisPofW} we ``bin'' pairs $(W,P)$ by relatively-narrow sliding speed ranges (each sub-plot) and color the points by the ice thickness.  There is an increasing relationship between $W$ and the relative pressure $P/P_o$ in each bin.  While in the fast-sliding case $W$ is often comparable to the bed roughness scale $W_r$, for slow sliding we see generally lower water amounts ($W \lesssim W_r/10$) but a full range of pressures.  In thick ice the pressure $P$ is close to overburden even if there is fast sliding.  Locations with high sliding, high water amount, and low pressure always have low ice thickness.

\conclusions  \label{sec:conclusion}  This paper documents additions made to the Parallel Ice Sheet Model in its 0.6 version released February 2014.  It describes and demonstrates a subglacial hydrology model which is novel in having these features:\begin{itemize}
\item a 2D parallel implementation of a coupled till-and-linked-cavities model,
\item a pressure-equation regularization, using notional englacial porosity, which eases implementation and improves numerical performance,
\item a scheme for maintaining physical pressure bounds ($0\le P \le P_o$) at all times,
\item verification using a nearly-exact solution of the coupled mass-conservation and pressure equations, in the steady radial case, and
\item demonstration at high resolution and whole ice-sheet scale on a million-point hydrology grid.
\end{itemize}

Furthermore, the comprehensive exposition here clarifies the relationship among several pressure-determining ``closures'' (section \ref{sec:closures}), and it allows us to understand our model as a common extension of several seemingly-disparate published models (section \ref{sec:newmodel}).  Additional analysis (Appendix A) shows that in that in steady state a functional relationship ``$P=P(W)$'' arises between pressure and water layer thickness.  This analysis reveals the diffusive nature of the apparently-advective part of the steady-state flux.

The current paper only demonstrates one-way coupling, in which the PISM ice flow and thermodynamics model feeds basal melt rate and sliding velocities to the hydrology model.  Two-way coupling  will appear in future work.


\section{Code availability}  \label{sec:availability}

The source code for all versions of PISM is available through host website \url{https://github.com/pism/pism}.  Extensive PDF and searchable browser documentation for PISM is contained both in the source code and online through PISM homepage \url{http://www.pism-docs.org/}.  PISM is licensed under the GNU General Public License (version 3).


\begin{acknowledgements}
Comments by editor Dan Goldberg, reviewer Tim Bartholomaus, and two anonymous reviewers improved the focus and quality of the paper.  Detailed comments by Andy Aschwanden and Martin Truffer were much appreciated.  Constantine Khroulev helped with the PISM implementation.  The first author was supported by NASA grant \#NNX13AM16G.  This work was supported by a grant of high-performance computing resources from the Arctic Region Supercomputing Center.
\end{acknowledgements}


\begin{thebibliography}{}
\providecommand{\natexlab}[1]{#1}
\providecommand{\url}[1]{{\tt #1}}
\providecommand{\urlprefix}{URL }
\expandafter\ifx\csname urlstyle\endcsname\relax
  \providecommand{\doi}[1]{doi:\discretionary{}{}{}#1}\else
  \providecommand{\doi}{doi:\discretionary{}{}{}\begingroup
  \urlstyle{rm}\Url}\fi

\bibitem[{Ascher and Petzold(1998)}]{AscherPetzold}
Ascher, U. and Petzold, L.: Computer {M}ethods for {O}rdinary {D}ifferential
  {E}quations and {D}ifferential-algebraic {E}quations, SIAM Press,
  Philadelphia, PA, 1998.

\bibitem[{Aschwanden et~al.(2012)Aschwanden, Bueler, Khroulev, and
  Blatter}]{AschwandenBuelerKhroulevBlatter}
Aschwanden, A., Bueler, E., Khroulev, C., and Blatter, H.: An enthalpy
  formulation for glaciers and ice sheets, J. Glaciol., 58, 441--457,
  \doi{10.3189/2012JoG11J088}, 2012.

\bibitem[{Aschwanden et~al.(2013)Aschwanden, Adalgeirsd{\'o}ttir, and
  Khroulev}]{AschwandenAdalgeirsdottirKhroulev}
Aschwanden, A., Adalgeirsd{\'o}ttir, G., and Khroulev, C.: Hindcasting to
  measure ice sheet model sensitivity to initial states, The Cryosphere, 7,
  1083--1093, \doi{10.5194/tc-7-1083-2013}, 2013.

\bibitem[{Balay et~al.(2011)}]{petsc-user-ref}
Balay, S. et~al.: {PETS}c {U}sers {M}anual, Tech. Rep. ANL-95/11 - Revision
  3.2, Argonne National Laboratory, 2011.

\bibitem[{Bartholomaus et~al.(2008)}]{Bartholomausetal2008}
Bartholomaus, T.~C., Anderson, R.~S., and Anderson, S.~P.: Response of glacier
  basal motion to transient water storage, Nature Geosci., 1, 33--37,
  \doi{10.1038/ngeo.2007.52}, 2008.

\bibitem[{Bartholomaus et~al.(2011)}]{Bartholomausetal2011}
Bartholomaus, T.~C., Anderson, R.~S., and Anderson, S.~P.: Growth and collapse
  of the distributed subglacial hydrologic system of {K}ennicott {G}lacier,
  {A}laska, {USA}, and its effects on basal motion, J. Glaciol., 57, 985--1002,
  2011.

\bibitem[{Bindschadler et~al.(2013)}]{Bindschadler2013SeaRISE}
Bindschadler, R. et~al.: Ice-sheet model sensitivities to environmental forcing
  and their use in projecting future sea-level ({T}he {S}ea{RISE} {P}roject),
  J. Glaciol, 59, 195--224, 2013.

\bibitem[{Bueler(2014)}]{Bueler2014correspondence}
Bueler, E.: Correspondence: Extensions of the lumped subglacial-englacial
  hydrology model of {B}artholomaus, et al.~(2011), submitted to J.~Glaciol.,
  2014.

\bibitem[{Bueler and Brown(2009)}]{BBssasliding}
Bueler, E. and Brown, J.: Shallow shelf approximation as a ``sliding law'' in a
  thermodynamically coupled ice sheet model, J. Geophys. Res., 114, f03008,
  doi:10.1029/2008JF001179, 2009.

\bibitem[{Bueler et~al.(2005)}]{BLKCB}
Bueler, E. et~al.: Exact solutions and numerical verification for isothermal ice sheets,
  J. Glaciol., 51, 291--306, 2005.

\bibitem[{Clarke(2003)}]{Clarke2003}
Clarke, G.~K.: Hydraulics of subglacial outburst floods: new insights from the
  {Spring-Hutter} formulation, J. Glaciol., 49, 299--313,
  \doi{10.3189/172756503781830728}, 2003.

\bibitem[{Clarke(2005)}]{Clarke05}
Clarke, G.: Subglacial processes, Annu. Rev. Earth Planet. Sci., 33,
  247--276, \doi{10.1146/annurev.earth.33.092203.122621}, 2005.

\bibitem[{Creyts and Schoof(2009)}]{CreytsSchoof2009}
Creyts, T. and Schoof, C.: Drainage through subglacial water sheets, J.
  Geophys. Res., 114, \doi{10.1029/2008JF001215}, 2009.

\bibitem[{Cuffey and Paterson(2010)}]{CuffeyPaterson}
Cuffey, K.~M. and Paterson, W. S.~B.: The {P}hysics of {G}laciers, Elsevier,
  4th edn., 2010.

\bibitem[{Das et~al.(2008)Das, Joughin, Behn, Howat, King, Lizarralde, and
  Bhatia}]{Dasetal08}
Das, S.~B., Joughin, I., Behn, M.~D., Howat, I.~M., King, M.~A., Lizarralde,
  D., and Bhatia, M.~P.: {Fracture Propagation to the Base of the Greenland Ice
  Sheet During Supraglacial Lake Drainage}, Science, 320, 778--781,
  \doi{10.1126/science.1153360}, 2008.

\bibitem[{de~Fleurian et~al.(2014)de~Fleurian, Gagliardini, Zwinger, Durand,
  Le~Meur, Mair, and R{\aa}back}]{deFleurianetal2014}
de~Fleurian, B., Gagliardini, O., Zwinger, T., Durand, G., Le~Meur, E., Mair,
  D., and R{\aa}back, P.: A double continuum hydrological model for glacier
  applications, The Cryosphere, 8, 137--153, \doi{10.5194/tc-8-137-2014}, 2014.

\bibitem[{Ettema et~al.(2009)}]{Ettemaetal2009}
Ettema J. et~al.: Higher surface mass balance of the Greenland ice sheet
  revealed by high-resolution climate modeling, Geophys. Res. Lett., 36, L12501,
  \doi{10.1029/2009GL038110}, 2009.

\bibitem[{Flowers and Clarke(2002{\natexlab{a}})}]{FlowersClarke2002_theory}
Flowers, G.~E. and Clarke, G. K.~C.: A multicomponent coupled model of glacier
  hydrology 1. {T}heory and synthetic examples, J. Geophys. Res., 107, 2287,
  \doi{10.1029/2001JB001122}, 2002{\natexlab{a}}.

\bibitem[{Flowers and Clarke(2002{\natexlab{b}})}]{FlowersClarke2002_trapridge}
Flowers, G.~E. and Clarke, G. K.~C.: A multicomponent coupled model of glacier
  hydrology 2. {A}pplication to {T}rapridge {G}lacier, {Y}ukon, {C}anada, J.
  Geophys. Res., 107, 2288, \doi{10.1029/2001JB001124}, 2002{\natexlab{b}}.

\bibitem[{Fountain et~al.(2005)}]{Fountainetal2005}
Fountain, A., et al.: Fractures as the main pathways of water flow in temperate glaciers, Nature 433, 618--621, \doi{doi:10.1038/nature03296}, 2005.

\bibitem[{Goeller et~al.(2013)Goeller, Thoma, Grosfeld, and
  Miller}]{Goelleretal2013}
Goeller, S., Thoma, M., Grosfeld, K., and Miller, H.: A balanced water layer
  concept for subglacial hydrology in large-scale ice sheet models, The
  Cryosphere, 7, 1095--1106, 2013.

\bibitem[{Goeller(2014)}]{Goeller2014}
Goeller, S.: Antarctic {S}ubglacial {H}ydrology: {I}nteractions of subglacial lakes, basal water flow, and ice dynamics, PhD Dissertation, Universit\"at Bremen, 2014.

\bibitem[{Greve and Blatter(2009)}]{GreveBlatter2009}
Greve, R. and Blatter, H.: Dynamics of {I}ce {S}heets and {G}laciers, Advances
  in Geophysical and Environmental Mechanics and Mathematics, Springer, 2009.

\bibitem[{Harper et~al.(2010)Harper, Bradford, Humphrey, and
  Meierbachtol}]{Harperetal2010}
Harper, J., Bradford, J., Humphrey, N., and Meierbachtol, T.: Vertical
  extension of the subglacial drainage system into basal crevasses, Nature,
  467, 579--582, \doi{10.1038/nature09398}, 2010.

\bibitem[{Hewitt(2011)}]{Hewitt2011}
Hewitt, I.~J.: Modelling distributed and channelized subglacial drainage: the
  spacing of channels, J. Glaciol., 57, 302--314, 2011.

\bibitem[{Hewitt(2013)}]{Hewitt2013}
Hewitt, I.~J.: Seasonal changes in ice sheet motion due to melt water
  lubrication, Earth Planet. Sci. Lett., 371--372, 16--25,
  \doi{10.1016/j.epsl.2013.04.022}, 2013.

\bibitem[{Hewitt et~al.(2012)Hewitt, Schoof, and Werder}]{Hewittetal2012}
Hewitt, I.~J., Schoof, C., and Werder, M.~A.: Flotation and free surface flow
  in a model for subglacial drainage. {P}art {II}: {C}hannel flow, J. Fluid
  Mech., 702, 157--188, 2012.

\bibitem[{Hoffman and Price(2014)}]{HoffmanPrice2014}
Hoffman, M. J. and Price, S.: Feedbacks between coupled subglacial hydrology and glacier dynamics, J. Geophys. Res. Earth Surf., 119, \doi{10.1002/2013JF002943}, 2014.

\bibitem[{Hooke et~al.(1997)}]{Hookeetal1997}
Hooke, R. et~al.: Rheology of
  till beneath {S}t\"orglaciaren, {S}weden, J. Glaciol., 43, 172--179, 1997.

\bibitem[{Hundsdorfer and Verwer(2010)}]{HundsdorferVerwer2010}
Hundsdorfer, W. and Verwer, J.~G.: Numerical {S}olution of {T}ime-{D}ependent
  {A}dvection-{D}iffusion-{R}eaction {E}quations, Springer Series in
  Computational Mathematics, Springer, 2010.

\bibitem[{Huybrechts et~al.(1996)}]{EISMINT96}
Huybrechts, P. et~al.: The {EISMINT} benchmarks for testing ice-sheet models,
  Ann. Glaciol., 23, 1--12, 1996.

\bibitem[{Johnson and Fastook(2002)}]{JohnsonFastook}
Johnson, J. and Fastook, J.~L.: Northern {H}emisphere glaciation and its
  sensitivity to basal melt water, Quat. Int., 95, 65--74, 2002.

\bibitem[{Joughin et~al.(2010)}]{Joughinetal2010}
Joughin, I. et~al.: Greenland flow variability from ice-sheet-wide velocity
  mapping, J. Glaciol., 56, 415--430, 2010.

\bibitem[{Kamb(1987)}]{Kamb1987}
Kamb, B.: Glacier surge mechanism based on linked cavity configuration of the
  basal water conduit system, J. Geophys. Res., 92, 9083--9100, 1987.

\bibitem[{Kamb(1991)}]{Kamb1991}
Kamb, B.: Rheological nonlinearity and flow instability in the deforming bed
  mechanism of ice stream motion, J. Geophys. Res.: Solid Earth, 96, 16585--16595, 1991.

\bibitem[{Kinderlehrer and Stampacchia(1980)}]{KinderlehrerStampacchia}
Kinderlehrer, D. and Stampacchia, G.: An {I}ntroduction to {V}ariational
  {I}nequalities and their {A}pplications, Pure and Applied Mathematics,
  Academic Press, 1980.

\bibitem[{Le~Brocq et~al.(2009)Le~Brocq, Payne, Siegert, and
  Alley}]{LeBrocqetal2009}
Le~Brocq, A., Payne, A., Siegert, M., and Alley, R.: A subglacial water-flow
  model for {W}est {A}ntarctica, J. Glaciol., 55, 879--888,
  \doi{10.3189/002214309790152564}, 2009.

\bibitem[{LeVeque(2002)}]{LeVeque}
LeVeque, R.~J.: Finite Volume Methods for Hyperbolic Problems, Cambridge Texts
  in Applied Mathematics, Cambridge University Press, 2002.

\bibitem[{Lingle and Brown(1987)}]{LingleBrown1987}
Lingle, C.~S. and Brown, T.~J.: A subglacial aquifer bed model and water
  pressure-dependent basal sliding relationship for a {W}est {A}ntarctic ice
  stream, in: Dynamics of the {W}est {A}ntarctic {I}ce {S}heet, edited by der
  Veen, C. J.~V. and Oerlemans, J., D. Reidel, 1987.

\bibitem[{Livingstone et~al.(2013)Livingstone, Clark, Woodward, and
  Kingslake}]{Livingstoneetal2013}
Livingstone, S.~J., Clark, C.~D., Woodward, J., and Kingslake, J.: Potential
  subglacial lake locations and meltwater drainage pathways beneath the
  {A}ntarctic and {G}reenland ice sheets, The Cryosphere, 7, 1721--1740,
  \doi{10.5194/tc-7-1721-2013}, 2013.

\bibitem[{Mahaffy(1976)}]{Mahaffy}
Mahaffy, M.~W.: A three--dimensional numerical model of ice sheets: tests on
  the {B}arnes {I}ce {C}ap, {N}orthwest {T}erritories, J. Geophys. Res., 81,
  1059--1066, 1976.

\bibitem[{Martin et~al.(2011)}]{Martinetal2011}
Martin, M.~A. et~al.: The {P}otsdam {P}arallel {I}ce {S}heet
  {M}odel ({PISM-PIK}) --{P}art 2: {D}ynamic equilibrium simulation of the
  {A}ntarctic ice sheet, The Cryosphere, 5, 727--740, 2011.

\bibitem[{Morton and Mayers(2005)}]{MortonMayers}
Morton, K.~W. and Mayers, D.~F.: Numerical {S}olutions of {P}artial
  {D}ifferential {E}quations: {A}n {I}ntroduction, Cambridge University Press,
  2nd edn., 2005.

\bibitem[{Nye(1976)}]{Nye1976}
Nye, J.~F.: Water flow in glaciers: {J}\"okulhlaups, tunnels and veins, J.
  Glaciol., 17, 181--207, 1976.

\bibitem[{{PISM authors}(2013)}]{pism-user-manual}
{PISM authors}: {PISM}, a {P}arallel {I}ce {S}heet {M}odel: {U}ser's {M}anual,
  \url{http://www.pism-docs.org}, 2013.

\bibitem[{Rooney et~al.(1987)}]{Rooneyetal1987}
Rooney, S. T., et~al.: Till beneath ice stream {B}: 2. structure and continuity,
  J. Geophys. Res.: Solid Earth, 92, 8913--8920, 1987.

\bibitem[{Schoof(2005)}]{Schoof2005cavitation}
Schoof, C.: The effect of cavitation on glacier sliding, Proc. R. Soc. A, 461,
  609--627, \doi{10.1098/rspa.2004.1350}, 2005.

\bibitem[{Schoof(2006{\natexlab{a}})}]{SchoofStream}
Schoof, C.: A variational approach to ice stream flow, J. Fluid Mech., 556,
  227--251, 2006{\natexlab{a}}.

\bibitem[{Schoof(2006{\natexlab{b}})}]{SchoofTill}
Schoof, C.: Variational methods for glacier flow over plastic till, J. Fluid
  Mech., 555, 299--320, 2006{\natexlab{b}}.

\bibitem[{Schoof(2007)}]{Schoof2007deformable}
Schoof, C.: Cavitation on deformable glacier beds, SIAM J. Appl. Math., 67,
  1633--1653, 2007.

\bibitem[{Schoof(2010{\natexlab{a}})}]{SchoofCoulombBlatter}
Schoof, C.: Coulomb friction and other sliding laws in a higher order glacier
  flow model, Math. Models Methods Appl. Sci. (M3AS), 20, 157--189,
  \doi{10.1142/S0218202510004180}, 2010{\natexlab{a}}.

\bibitem[{Schoof(2010{\natexlab{b}})}]{Schoofmeltsupply}
Schoof, C.: Ice sheet acceleration driven by melt supply variability, Nature,
  468, 803--806, 2010{\natexlab{b}}.

\bibitem[{Schoof et~al.(2012)Schoof, Hewitt, and Werder}]{Schoofetal2012}
Schoof, C., Hewitt, I.~J., and Werder, M.~A.: Flotation and free surface flow
  in a model for subglacial drainage. {P}art {I}: {D}istributed drainage, J.
  Fluid Mech., 702, 126--156, 2012.

\bibitem[{Shreve(1972)}]{Shreve1972}
Shreve, R.: Movement of water in glaciers, J. Glaciol, 11, 205--214, 1972.

\bibitem[{Siegert et~al.(2009)Siegert, Le~Brocq, and Payne}]{Siegertetal2009}
Siegert, M., Le~Brocq, A., and Payne, A.: Hydrological connections between
  Antarctic subglacial lakes, the flow of water beneath the East Antarctic Ice
  Sheet and implications for sedimentary processes, pp. 3--10, Wiley-Blackwell,
  2009.

\bibitem[{Truffer and Harrison(2006)}]{TrufferHarrison2006}
Truffer, M. and Harrison, W.: In situ measurements of till deformation and
  water pressure, J. Glaciol., 52, 175--182, 2006.

\bibitem[{Truffer et~al.(2000)Truffer, Echelmeyer, and
  Harrison}]{TrufferHarrisonEchelmeyer2000}
Truffer, M., Echelmeyer, K., and Harrison, W.: Glacier motion dominated by
  processes deep in underlying till, J. Glaciol., 46, 213--221, 2000.

\bibitem[{Truffer et~al.(2001)Truffer, Echelmeyer, and
  Harrison}]{TrufferEchelmeyerHarrison2001}
Truffer, M., Echelmeyer, K., and Harrison, W.: Implications of till deformation
  on glacier dynamics, J. Glaciol., 47, 123--134,
  \doi{10.3189/172756501781832449}, 2001.

\bibitem[{Tulaczyk et~al.(2000{\natexlab{a}})Tulaczyk, Kamb, and
  Engelhardt}]{Tulaczyketal2000}
Tulaczyk, S., Kamb, W.~B., and Engelhardt, H.~F.: Basal mechanics of {I}ce
  {S}tream {B}, {W}est {A}ntarctica 1.~{T}ill mechanics, J. Geophys. Res., 105,
  463--481, 2000{\natexlab{a}}.

\bibitem[{Tulaczyk et~al.(2000{\natexlab{b}})Tulaczyk, Kamb, and
  Engelhardt}]{Tulaczyketal2000b}
Tulaczyk, S., Kamb, W.~B., and Engelhardt, H.~F.: Basal mechanics of {I}ce
  {S}tream {B}, {W}est {A}ntarctica 2.~{U}ndrained plastic bed model, J.
  Geophys. Res., 105, 483--494, 2000{\natexlab{b}}.

\bibitem[{van~der Wel et~al.(2013)van~der Wel, Christoffersen, and
  Bougamont}]{vanderWeletal2013}
van~der Wel, N., Christoffersen, P., and Bougamont, M.: The influence of
  subglacial hydrology on the flow of {K}amb {I}ce {S}tream, {W}est
  {A}ntarctica, J. Geophys. Res.: Earth Surface, 118, 1--14,
  \doi{10.1029/2012JF002570}, 2013.

\bibitem[{van Pelt(2013)}]{vanPeltthesis}
van Pelt, W.: Modelling the dynamics and boundary processes of {S}valbard
  glaciers, Ph.D. thesis, Institute for Marine and Atmospheric Research Utrecht
  (IMAU), The Netherlands, 2013.

\bibitem[{V{\'a}zquez(2007)}]{VazquezPME}
V{\'a}zquez, J.~L.: The {P}orous {M}edium {E}quation, Oxford Mathematical
  Monographs, The Clarendon Press Oxford University Press, Oxford, 2007.

\bibitem[{Walder(1982)}]{Walder1982}
Walder, J.~S.: Stability of sheet flow of water beneath temperate glaciers and
  implications for glacier surging, J. Glaciol., 28, 273--293, 1982.

\bibitem[{Werder et~al.(2013)Werder, Hewitt, Schoof, and
  Flowers}]{Werderetal2013}
Werder, M., Hewitt, I., Schoof, C., and Flowers, G.: Modeling channelized and
  distributed subglacial drainage in two dimensions, J Geophys. Res.: Earth
  Surface, 118, 2140--2158, 2013.

\bibitem[{Wesseling(2001)}]{Wesseling}
Wesseling, P.: Principles of {C}omputational {F}luid {D}ynamics,
  Springer-Verlag, 2001.

\bibitem[{Winkelmann et~al.(2011)}]{Winkelmannetal2011}
Winkelmann, R. et~al.: The {P}otsdam {P}arallel {I}ce {S}heet
  {M}odel ({PISM-PIK}) {P}art 1: {M}odel description, The Cryosphere, 5,
  715--726, 2011.
\end{thebibliography}


\appendix

\section{Analysis of steady states}  Relative to the time-dependent model equations \eqref{eq:bluebox}, steady-state equations \eqref{eq:Vsteady}--\eqref{eq:openclosesteady} have separate balances between the divergence of the flux and the water input, and the opening and closing processes.  In particular, equation \eqref{eq:openclosesteady} allows us to write the pressure $P=P(W)$ in steady state as a continuous function of the layer thickness $W$.  However, steady state is only possible if a condition holds:
\begin{equation}
c_1 |\bv_b| (W_r - W)_+ \le c_2 A P_o^3 W. \label{eq:steadyboundfirst}
\end{equation}
This condition says that the maximum closing rate $\mathcal{C}(N,W)$, which occurs at zero water pressure, must equal or exceed the sliding-generated opening rate $\mathcal{O}(|\bv_b|,W)$.

We define a scaled basal sliding speed which has units of pressure; it is a scale for the pressure drop from cavitation:
\begin{equation}
s_b =  \left(\frac{c_1 |\bv_b|}{c_2 A}\right)^{1/3}.  \label{eq:definesb}
\end{equation}
Then \eqref{eq:steadyboundfirst} is equivalent to the condition $W\ge W_c$, where $W_c = W_r s_b^3/(s_b^3 + P_o^3)$ is a critical water thickness.  If $W\ge W_c$ then
\begin{equation}
P(W) = P_o - s_b \left(\frac{(W_r - W)_+}{W}\right)^{1/3}.  \label{eq:PofWsteady}
\end{equation}
Formula \eqref{eq:PofWsteady} applies even if $W\ge W_r$, in which case $P = P_o$.  Underpressure ($P=0$) with subcritical water amount ($W<W_c$) does not occur in steady state, though it can occur in nonsteady conditions; note $P(W_c)=0$.  Figure \ref{fig:psteady-vb} shows the function $P(W)$ for different values of sliding speed $|\bv_b|$, and Figure \ref{fig:psteady-Po} shows it for different values of overburden pressure $P_o$.

\cite{FlowersClarke2002_theory} propose function $P_{FC}(W)$---see equation \eqref{eq:PofWFC}---for both steady and nonsteady circumstances.  Both functions $P(W)$ and $P_{FC}(W)$ are increasing, and both relate $P$ to the overburden pressure $P_o$.  However, in \eqref{eq:PofWsteady} the relation of $P$ to $P_o$ is additive, while in \eqref{eq:PofWFC} they are proportional.  Power law form \eqref{eq:PofWFC} is not justified by the physical reasoning which led to equation \eqref{eq:PofWsteady}, even in steady state.   It would appear that any functional relationship $P(W)$ should also depend on the sliding velocity, as it does here, if cavitation influences the water pressure.  In any case, in the current paper we do not impose a relationship $P=P(W)$ at all, though such a relation emerges in steady state.

We now consider how the steady state water velocity $\bV$, and the associated flux $\bq$, depends on other quantities.  First, from equations \eqref{eq:Vsteady} and \eqref{eq:PofWsteady}, in steady state we have
\begin{equation}
\frac{\partial P}{\partial W} = \frac{s_b W_r}{3 W^{4/3} (W_r - W)^{2/3}} \label{eq:dPdWsteady}
\end{equation}
if $W_c < W < W_r$.  If $W\le W_c$ then $\partial P/\partial W$ is undefined, and if $W>W_r$ then $\partial P/\partial W=0$.  Formula \eqref{eq:dPdWsteady} and Figures \ref{fig:psteady-vb} and \ref{fig:psteady-Po} agree that $\partial P / \partial W \to \infty$ as $W \nearrow W_r$.  

Now note that equations \eqref{eq:Vsteady}, \eqref{eq:PofWsteady}, and \eqref{eq:dPdWsteady} imply a formula for the velocity in steady state:
\begin{align}
\bV &= - k \bigg[\grad \psi_o - \left(\frac{W_r - W}{W}\right)^{1/3} \grad s_b \label{eq:Vsteadyexpand} \\
    & \qquad \qquad + \frac{s_b W_r}{3 W^{4/3} (W_r - W)^{2/3}} \grad W\bigg], \notag
\end{align}
where $\psi_o = P_o + \rho_w g b$.  Thus the direction of water velocity $\bV$ is determined by a combination of a geometric direction ($\grad \psi_o$), a direction derived from spatial variations in the sliding speed ($\grad s_b$), and a diffusive direction ($\grad W$).  Indeed, a portion of the advective flux $\bV W$ is diffusive in steady state, in addition to the \emph{a priori} diffusive flux $- D \grad W$; recall equation \eqref{eq:qexpression} in subsection \ref{subsec:advectdiffus}.  In fact, because the coefficient of $\grad W$ in \eqref{eq:Vsteadyexpand} remains large when $W\to 0$, as long as sliding is occurring ($s_b > 0$), then for low water amount and sustained sliding we should think of the water as diffusing in the layer.  On the other hand, when the water thickness is almost at the roughness scale ($W\lesssim W_r$), then the same coefficient is also large in sliding cases ($s_b>0$); again the effect is diffusive.


%%%%%%%%%%%%%%%%%    TABLES    %%%%%%%%%%%%%%%%%%%%

\begin{table*}[ht]
  \centering
  \caption{Physical constants and model parameters.  All values are configurable in PISM; see Table \ref{tab:correspondence}.}
  \begin{tabular}{lllp{3.0in}} 
    \textbf{Name} & \textbf{Default} & \textbf{Units} & \textbf{Description}\\
\hline
    $A$ & $3.1689\times 10^{-24}$ & $\text{Pa}^{-3}\,\text{s}^{-1}$ & ice softness \citep{EISMINT96} \phantom{$\Big|$} \\
    $\alpha$ & $5/4$ & & power in flux formula  \citep{Schoofetal2012} \\
    $\beta$ & $3/2$ & & power in flux formula  \citep{Schoofetal2012} \\
    $c_0$ & 0 & Pa & till cohesion \citep{Tulaczyketal2000} \\
    $c_1$ & $0.5$ & $\text{m}^{-1}$ & cavitation coefficient \citep{Schoofetal2012} \\
    $c_2$ & $0.04$ & & creep closure coefficient \\
    $C_c$ & 0.12 &  & till compressibility \citep{Tulaczyketal2000} \\
    $C_d$ & $0.001$ &  $\text{m}\,\text{a}^{-1}$ & background till drainage rate \\
    $\delta$ & 0.02 &  & $\Ntil$ lower bound, as fraction of overburden pressure \\
    $e_0$ & 0.69 &  & reference void ratio at $N_0$ \citep{Tulaczyketal2000} \\
    $\phi_0$ & $0.01$ & & notional (regularizing) englacial porosity \\
    $g$ & $9.81$ & m $\text{s}^{-2}$ & acceleration of gravity \\
    $k$ & $0.001$ & $\text{m}^{2\beta-\alpha} \text{s}^{2\beta-3} \text{kg}^{1-\beta}$ & conductivity coefficient \citep{Schoofetal2012} \\
    $N_0$ & $1000$ & Pa & reference effective pressure \citep{Tulaczyketal2000} \\
    $\rho_i$ & $910$ & $\text{kg}\,\text{m}^{-3}$ & ice density \citep{GreveBlatter2009} \\
    $\rho_w$ & $1000$ & $\text{kg}\,\text{m}^{-3}$ & fresh water density \citep{GreveBlatter2009} \\
    $W_r$ & $0.1$ & $\text{m}$ & roughness scale \citep{Hewittetal2012} \\
    $\Wtilmax$ & $2\phantom{\Big|}$ & $\text{m}$ & maximum water in till \citep{BBssasliding} \\
    \hline
  \end{tabular}
 \label{tab:constants}
\end{table*}

\begin{table}[ht]
\caption{Functions used in subglacial hydrology model \eqref{eq:bluebox}.}
\begin{tabular}{l|l}
\textbf{Type} & \textbf{Description (symbol, units, meaning)} \\
\hline
\emph{state} & \begin{tabular}{lll}
        $W$ & m \phantom{llllllllllll\,} & transportable water thickness \\
        $\Wtil$ & m & till-stored water thickness \\
        $P$ & Pa & transportable water pressure \\
        \end{tabular} \\ \hline
\emph{input} &  \begin{tabular}{lll}
        $b$ & m & bedrock elevation \\
        $\varphi$ &  & till friction angle \\
        $H$ & m & ice thickness \\
        $m$ & $\text{kg}\,\text{m}^{-2}\,\text{s}^{-1}$ & total melt water input \\
        $|\bv_b|$ & $\text{m}\,\text{s}^{-1}$ & ice sliding speed \\
        \end{tabular} \\ \hline
\emph{output} &  \begin{tabular}{lll}
        $\Ntil$ & Pa & till effective pressure \\
        $\tau_c$ \phantom{l\;} & Pa \phantom{llllllllllll} & till yield stress \\
        \end{tabular} \\ \hline
\end{tabular}
\label{tab:symbols}
\end{table}

\begin{table*}[t]
  \caption{Correspondence between PISM parameter names and symbols in this paper (Table \ref{tab:constants}).  All are used in the \texttt{distributed} model, with indicated subsets used in the \texttt{routing} and \texttt{null} models.} \vskip4mm \centering
  \begin{tabular}{llcc} \tophline
    \textbf{PISM parameter name} & \textbf{Symbol} & \textbf{\texttt{routing}} & \textbf{\texttt{null}} \\
    \middlehline
    \verb|fresh_water_density| & $\rho_w$ & $\times$ & $\times$ \\
    \verb|hydrology_cavitation_opening_coefficient| & $c_1$ \\
    \verb|hydrology_creep_closure_coefficient| & $c_2$ \\
    \verb|hydrology_gradient_power_in_flux| &  $\beta$ & $\times$ & \\
    \verb|hydrology_hydraulic_conductivity| & $k$ & $\times$ & \\
    \verb|hydrology_regularizing_porosity| & $\phi_0$ \\
    \verb|hydrology_roughness_scale| & $W_r$ \\
    \verb|hydrology_thickness_power_in_flux| &  $\alpha$ & $\times$ & \\
    \verb|hydrology_tillwat_decay_rate| & $C_d$ & $\times$ & $\times$ \\
    \verb|hydrology_tillwat_max| & $\Wtilmax$ & $\times$ & $\times$ \\
    \verb|ice_density| & $\rho_i$ & $\times$ & $\times$ \\
    \verb|ice_softness| & $A$ \\
    \verb|standard_gravity| & $g$ & $\times$ & $\times$ \\
    \verb|till_c_0| & $c_0$ & $\times$ & $\times$ \\
    \verb|till_compressibility_coefficient| & $C_c$ & $\times$ & $\times$ \\
    \verb|till_effective_fraction_overburden| & $\delta$ & $\times$ & $\times$ \\
    \verb|till_reference_effective_pressure| & $N_0$ & $\times$ & $\times$ \\
    \verb|till_reference_void_ratio| & $e_0$ & $\times$ & $\times$ \\
    \bottomhline
  \end{tabular}
  \label{tab:correspondence}
\end{table*}


%%%%%%%%%%%%%%%%%    FIGURES    %%%%%%%%%%%%%%%%%%%%

\begin{figure}[ht]
\includegraphics[width=3.0in,keepaspectratio=true]{Ntilfunctions}
\caption{\textbf{(a)} Equation \eqref{eq:voidpressure} determines the effective pressure $\Ntil$ as a function of the void ratio $e$; reference values $e_0$,$N_0$ are indicated.  \textbf{(b)}  The same curve, but with $\Ntil$ as a function of $\Wtil$, with bounds above by overburden pressure $P_o$ and below by a fixed fraction $\delta$ of $P_o$; the solid curve is used in our model.  The case shown has 1000 meters ice thickness.}
\label{fig:Ntilfunctions}
\end{figure}

\begin{figure}[ht]
\includegraphics[width=3.0in,keepaspectratio=true]{exact-W-plot-onu}
\caption{A nearly-exact radial, steady solution for water thickness $W(r)$ (dashed).  In $r$-versus-$W$ space the overpressure (O), normal pressure (N), and underpressure (U) regions (solid curves) are determined by ice geometry and sliding velocity, because this is steady state.}
\label{fig:Wexact}
\end{figure}

\begin{figure}[ht]
\includegraphics[width=3.0in,keepaspectratio=true]{exact-P-plot}
\caption{A nearly-exact radial, steady solution for pressure $P(r)$ (dashed) and overburden pressure $P_o$ (solid).}
\label{fig:Pexact}
\end{figure}

\begin{figure}[ht]
\centering
\includegraphics[width=2.2in,keepaspectratio=true]{diffstencil}
\bigskip
\caption{Numerical schemes \eqref{eq:Wfd} and \eqref{eq:Pfd} use a grid-point-centered cell.  Velocities, diffusivities, and fluxes are evaluated at staggered grid locations (triangles at centers of cell edges) denoted by compass notation $e,w,n,s$.  State functions $W,W_{til},P$ are located at regular grid points (diamonds).}
\label{fig:stencil}
\end{figure}

\begin{figure}[ht]
\includegraphics[width=3.0in,keepaspectratio=true]{refineWPpism}
\caption{Average water thickness error $|W-W_{exact}|$ decays as $O(\Delta x^{0.91})$, and average pressure error $|P-P_{exact}|$ decays as $O(\Delta x^{0.92})$, for grids with spacing $250 \le \Delta x = \Delta y \le 2000$ m.}
\label{fig:refineWPpism}
\end{figure}

\newcommand{\grnht}{3.4in}

\begin{figure*}[ht]
\mbox{\includegraphics[height=\grnht,keepaspectratio=true]{g2km-init-bmelt} \,
\includegraphics[height=\grnht,keepaspectratio=true]{g2km-init-velbase-mag}}
\caption{The inputs to the hydrology model are the modeled basal melt rate $m/\rho_w$ (left; $\mathrm{m}\,\mathrm{a}^{-1}$) and sliding speed $|\bv_b|$ (right; $\mathrm{m}\,\mathrm{a}^{-1}$) from the spun-up state.}
\label{fig:Greenhydroinputs}
\end{figure*}

\begin{figure*}[ht]
\mbox{\includegraphics[height=\grnht,keepaspectratio=true]{routing-decoupled-tillwat} \,
\includegraphics[height=\grnht,keepaspectratio=true]{routing-decoupled-bwat}}
\caption{Outputs from the \texttt{routing} hydrology model are the modelled till-stored water layer thickness $\Wtil$ (left; $\mathrm{m}$) and modelled transportable water layer thickness $W$ (right; $\mathrm{m}$).}
\label{fig:Greenroutingresults}
\end{figure*}

\begin{figure*}[ht]
\includegraphics[height=2.7in,keepaspectratio=true]{detail-routing-decoupled-bwat}
\caption{Detail of transportable water $W$ plotted in Figure \ref{fig:Greenroutingresults}, covering Jakobshavn (J), Helheim (H), and Kangerdlugssuaq (K) outlet glaciers}
\label{fig:Greenroutingdetail}
\end{figure*}

\begin{figure*}[ht]
\mbox{\includegraphics[height=\grnht,keepaspectratio=true]{distributed-decoupled-bwat} \,
\includegraphics[height=\grnht,keepaspectratio=true]{distributed-decoupled-bwprel}}
\caption{Outputs from the \texttt{distributed} hydrology model include the modelled transportable water layer thickness $W$ (left; $\mathrm{m}$), and the modelled transportable water layer pressure $P$, shown relative to overburden pressure (i.e.~$P/P_o$; right).}
\label{fig:Greendistributedresults}
\end{figure*}

\newcommand{\myheight}{1.8in}
\begin{figure*}[ht]
\mbox{\includegraphics[height=\myheight,keepaspectratio=true]{bin1-g2km} \, \includegraphics[height=\myheight,keepaspectratio=true]{bin10-g2km} \, \includegraphics[height=\myheight,keepaspectratio=true]{bin30-g2km}}
\mbox{\includegraphics[height=\myheight,keepaspectratio=true]{bin100-g2km} \,
\includegraphics[height=1.85in,keepaspectratio=true]{bin300-g2km}}
\caption{Scatter plots of $(W,P/P_o)$ pairs for all cells from the \texttt{distributed} model run, which used roughness scale $W_r = 0.1$ m.  Each sub-plot only shows pairs from the indicated range of ice sliding speeds.  Points are colored by ice thickness using a common scale shown beside last figure.}
\label{fig:GreenisPofW}
\end{figure*}

\begin{figure}[ht]
\includegraphics[width=2.8in,keepaspectratio=true]{psteady-vb}
\caption{The steady state function $P(W)$ defined by equation \eqref{eq:PofWsteady}, using $W_r=1$ m and $H=1000$ m (solid curves).  Values of $W_c$ are indicated by black dots at $P=0$.  For comparison, \cite{FlowersClarke2002_theory} relation \eqref{eq:PofWFC} is shown with $W_{\text{crit}}=1$ m (dashed black).}
\label{fig:psteady-vb}
\end{figure}

\begin{figure}[ht]
\includegraphics[width=2.8in,keepaspectratio=true]{psteady-Po}
\caption{The graph of $P(W)$ defined by \eqref{eq:PofWsteady} also depends on overburden pressure $P_o=\rho_i g H$, shown using $|\bv_b|=100$ m/a and $W_r=1$ m.}
\label{fig:psteady-Po}
\end{figure}

\end{document}
