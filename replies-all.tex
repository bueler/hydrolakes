\documentclass[11pt,reqno]{amsart}
%prepared in AMSLaTeX, under LaTeX2e
\addtolength{\oddsidemargin}{-.65in}
\addtolength{\evensidemargin}{-.65in}
\addtolength{\topmargin}{-.3in}
\addtolength{\textwidth}{1.5in}
\addtolength{\textheight}{.6in}

\renewcommand{\baselinestretch}{1.1}

\usepackage{verbatim} % for "comment" environment

\usepackage[pdftex, colorlinks=true, plainpages=false, linkcolor=blue, citecolor=red, urlcolor=blue]{hyperref}

\newtheorem*{thm}{Theorem}
\newtheorem*{defn}{Definition}
\newtheorem*{example}{Example}
\newtheorem*{problem}{Problem}
\newtheorem*{remark}{Remark}

\newcommand{\mtt}{\texttt}
\usepackage{alltt,xspace}
\usepackage[normalem]{ulem}
\newcommand{\mfile}[1]
{\medskip\begin{quote}\scriptsize \begin{alltt}\input{#1.m}\end{alltt} \normalsize\end{quote}\medskip}

\usepackage[final]{graphicx}
\newcommand{\mfigure}[1]{\includegraphics[height=2.5in,
width=3.5in]{#1.eps}}
\newcommand{\regfigure}[2]{\includegraphics[height=#2in,
keepaspectratio=true]{#1.eps}}
\newcommand{\widefigure}[3]{\includegraphics[height=#2in,
width=#3in]{#1.eps}}

% macros
\usepackage{amssymb}

\usepackage[T1, OT1]{fontenc}
\renewcommand{\dh}{\fontencoding{T1}\selectfont{\symbol{240}}}

\newcommand{\bod}{B\"o\dh varsson\xspace}
\newcommand{\bods}{B\"o\dh varsson's}
\newcommand{\citebod}{B\"o\dh varsson (1955)\xspace}
\newcommand{\citepbod}{(B\"o\dh varsson, 1955)\xspace}

\newcommand{\bA}{\mathbf{A}}
\newcommand{\bB}{\mathbf{B}}
\newcommand{\bE}{\mathbf{E}}
\newcommand{\bF}{\mathbf{F}}
\newcommand{\bJ}{\mathbf{J}}
\newcommand{\br}{\mathbf{r}}
\newcommand{\bx}{\mathbf{x}}
\newcommand{\hbi}{\mathbf{\hat i}}
\newcommand{\hbj}{\mathbf{\hat j}}
\newcommand{\hbk}{\mathbf{\hat k}}
\newcommand{\hbn}{\mathbf{\hat n}}
\newcommand{\hbr}{\mathbf{\hat r}}
\newcommand{\hbt}{\mathbf{\hat t}}
\newcommand{\hbx}{\mathbf{\hat x}}
\newcommand{\hby}{\mathbf{\hat y}}
\newcommand{\hbz}{\mathbf{\hat z}}
\newcommand{\hbphi}{\mathbf{\hat \phi}}
\newcommand{\hbtheta}{\mathbf{\hat \theta}}
\newcommand{\complex}{\mathbb{C}}
\newcommand{\ppr}[1]{\frac{\partial #1}{\partial r}}
\newcommand{\ppt}[1]{\frac{\partial #1}{\partial t}}
\newcommand{\ppx}[1]{\frac{\partial #1}{\partial x}}
\newcommand{\ppy}[1]{\frac{\partial #1}{\partial y}}
\newcommand{\ppz}[1]{\frac{\partial #1}{\partial z}}
\newcommand{\pptheta}[1]{\frac{\partial #1}{\partial \theta}}
\newcommand{\ppphi}[1]{\frac{\partial #1}{\partial \phi}}
\newcommand{\pp}[2]{\frac{\partial #1}{\partial #2}}
\newcommand{\ppp}[2]{\frac{\partial^2 #1}{\partial^2 #2}}
\newcommand{\pppp}[3]{\frac{\partial^2 #1}{\partial #2 \partial #3}}
\newcommand{\Div}{\ensuremath{\nabla\cdot}}
\newcommand{\Curl}{\ensuremath{\nabla\times}}
\newcommand{\curl}[3]{\ensuremath{\begin{vmatrix} \hbi & \hbj & \hbk \\ \partial_x & \partial_y & \partial_z \\ #1 & #2 & #3 \end{vmatrix}}}
\newcommand{\cross}[6]{\ensuremath{\begin{vmatrix} \hbi & \hbj & \hbk \\ #1 & #2 & #3 \\ #4 & #5 & #6 \end{vmatrix}}}
\newcommand{\eps}{\epsilon}
\newcommand{\grad}{\nabla}
\newcommand{\image}{\operatorname{im}}
\newcommand{\integers}{\mathbb{Z}}
\newcommand{\ip}[2]{\ensuremath{\left<#1,#2\right>}}
\newcommand{\lam}{\lambda}
\newcommand{\lap}{\triangle}
\newcommand{\Matlab}{\textsc{Matlab}\xspace}
\newcommand{\exers}[1]{\bigskip\noindent\textbf{Exercises} #1}
\newcommand{\fexer}[2]{\bigskip\noindent\textbf{Lesson #1, \##2}\quad }
\newcommand{\prob}[1]{\bigskip\noindent\textbf{#1} }
\newcommand{\pts}[1]{(\emph{#1 pts}) }
\newcommand{\epart}[1]{\medskip\noindent\textbf{(#1)}\quad }
\newcommand{\ppart}[1]{\,\textbf{(#1)}\quad }
\newcommand{\note}[1]{[\scriptsize #1 \normalsize]}
\newcommand{\MatIN}[1]{\mtt{>> #1}}
\newcommand{\onull}{\operatorname{null}}
\newcommand{\rank}{\operatorname{rank}}
\newcommand{\range}{\operatorname{range}}
\renewcommand{\P}{\mathcal{P}}
\newcommand{\real}{\mathbb{R}}
\newcommand{\trace}{\operatorname{tr}}
\renewcommand{\Re}{\operatorname{Re}}
\renewcommand{\Im}{\operatorname{Im}}
\newcommand{\Arg}{\operatorname{Arg}}

\newcommand{\comm}[2]{\item \emph{#1}:\, #2}

\renewcommand{\ln}[2]{\comm{line #1}{#2}}
\newcommand{\lnpage}[3]{\comm{line #1 \underline{on page #2}}{#3}}
\newcommand{\lns}[2]{\comm{lines #1}{#2}}
\newcommand{\lnspage}[3]{\comm{lines #1 \underline{on page #2}}{#3}}
\newcommand{\fg}[2]{\comm{Figure #1}{#2}}
\newcommand{\eqn}[2]{\comm{equation #1}{#2}}

\newcommand{\reply}[2]{
\medskip\medskip
\item  \begin{quote}
\emph{#1}
\end{quote}

\medskip
\noindent #2}


\title[Replies to Referee Comments]{Replies to all referee comments on \\ \emph{Mass-conserving subglacial hydrology in PISM}}

\author{Ed Bueler}

\date{\today}

\begin{document}
\maketitle

\thispagestyle{empty}

We thank the reviewers for many thoughtful comments.  Indeed, there are 19 pages worth of reviewer comments below, and our replies take another FIXME pages, yielding this long (FIXME page) document.

As emphasized by the editor and some reviewers, the paper should be shortened, and we have done so.  Specifically we replace section 6 (``An exact steady state solution'') and subsection 7.4 by brief text summaries, and Table A3 has been removed.  Just as significant, certain exposition has been made much briefer.  As a result the two-column-format length of the paper has shrunk from 26 pages to FIXME pages.  Also, the accompanying \texttt{latexdiff} output is enormous because essentially every paragraph has been revised, and most shortened.

While the reviewers disagree, at times, with the way we describe our results, and the processes we choose to include in the model, by our understanding these reviewers are not asserting that anything in the paper is \emph{wrong}.  There is no assertion that our model is wrong in the sense of not being compatible with physical principles, equations being deduced wrongly, or that the numerical schemes are inconsistent or nonconvergent (in a numerical sense).

Instead, the majority of the reviewers' comments amount to asking us to add process models and add commentary.  This has been resisted.  Furthermore, it is frustrating that none of the reviewers seem to be seriously interested in, or have apparent experience in, applying models of any type at whole ice sheet scale, which is clearly our emphasis.  Our model is already a super-model of four (identified) important subglacial hydrology models in the literature, and we have demonstrated it at unprecedented scale.  We clearly explain why we make the choices we do, and we document the nontrivial numerics and configurability that make it useful.  We have a sense that this is all overlooked in the quest for conduits (in particular), which we have very good reasons for not including.  We make clear our intent to stick to continuum physics (below, and in the text), and as such are the only accepted models in climate and fluid modeling, we are shocked by the general insistence that we must sin too.

We should also emphasize that \emph{scalability}, so that the model can apply at high resolution to a whole ice sheet, and \emph{configurability}, so that climate modelers inexperienced in fiddling with subglacial hydrology models can still use it, are goals which dominate the design of the model.  These goal motivates many of the replies below.


\subsection*{Comments by Dr.~Bartholomaus}\begin{itemize}

\reply{In this manuscript, the authors present a novel extension of the existing Parallel Ice Sheet Model that includes the most complete treatment to date of subglacial hydrology in a large-scale ice sheet model.  Subglacial hydrology is immensely important in glacier dynamics, but is often neglected in the major ice sheet models used to predict future sea level rise.  The computational expense of tracking changes in the rapidly evolving subglacial environment has generally prevented all but the crudest of parameterizations (see table 2 of Bindschadler 2013's summary of the SeaRISE experiment).\\
Thus, the present work is novel and worthy of publication in GMD.  The writing is generally clear and fluent.  Both the theoretical development of the continuum equations and the numerical implementation are clearly outlined.}
{We appreciate this summary and the comments below, which have improved the text.}

\reply{Beyond these over-arching strengths, I have four critiques that I believe would significantly enhance the impact and accessibility of the manuscript. These four opportunities for improvement are below. My line edits follow these more significant points.\\
---Four significant opportunities---\\
+ The authors offer some comparison between
their model and those of Werder, Hewitt, Flowers, Schoof, etc., but these are generally smaller scale models that have yet to be implemented or applied at the ice sheet
scale, and rarely to the complex geometries of existing glaciers or ice sheets. Some
discussion regarding how the new PISM hydrology model compares with the hydrology models of other major ice sheet models, such as those discussed in the SeaRISE
project would be very valuable.  At present, comparison to existing ice sheet models is
entirely lacking.  Without much knowledge of these models myself, I suspect that the
present model may represent a significant advance over the implementations in other
ice sheet models.  If appropriate, the authors may consider adding a sentence regarding this comparison to the abstract.  Also, by way of review, please consider adding a
table comparing features of presently-used ice sheet models.}
{We \emph{do} compare to existing large-scale models, by describing and citing the work of \cite{Goeller2014,LeBrocqetal2009,Livingstoneetal2013,Siegertetal2009}.  Such find-the-subglacial-lakes-and-drainage-paths modeling, which either uses an ill-posed version of our well-posed overburden-pressure-based \texttt{routing} model, or a balance velocity model, is the only whole-ice-sheet-scale work we know about.\footnote{We care about this whole ice sheet scale application, among others.  The fact that we are building an improvement of the \cite{Goeller2014,LeBrocqetal2009,Livingstoneetal2013,Siegertetal2009} models is important to us, clearly stated in the paper, and apparently of no interest to reviewers.}  We have also added a citation to \cite{HoffmanPrice2014}, which describes the construction of a related hydrology submodel within the Community Ice Sheet Model, but which is applied only at the scale of a single idealized mountain glacier.\footnote{It has not yet been applied at whole ice sheet scale (S.~Price personal communication).}\\
\indent We believe it would be inappropriate, and a surprising use of space, to add a table comparing features of presently-used ice sheet models in a PISM model-description paper.  It is already an inherent deficiency of model description papers that they can only describe a snapshot of an evolving piece of software.  Snapshotting other software projects, many of which do not have open development heads, unlike PISM, would only make this worse.}

\reply{+ Considering that efficient, low-pressure conduits are such important features of the subglacial hydrologic system, some discussion/justification of why a model without conduits is useful is necessary.  While consistent model behavior under grid refinement is certainly tremendously valuable, if one of the fundamental processes (i.e., transport of water in conduits) is entirely neglected, then all the model results may be called into question.  The present model is still an improvement on the general lack of subglacial hydrology in existing ice sheet models, but ideally, conduits will be included in future generations of ice sheets.}
{We discuss why a conduit model is not included in our model, and we have amplified our points in the revised version.\\
\indent  At present, \emph{all} 2D conduit models are \emph{not physics} by the normal standards of the field (e.g.~the field ``climate modeling'' or ``fluid modeling'' or ``continuum physics'', according to taste).  Note that ``consistent model behavior under grid refinement,'' in the sense used by this reviewer, is normally called ``continuum physics'', and has been the standard for physics since Fourier and Maxwell.\footnote{Numerical model behaviour under ``grid refinement'', as used in the field of numerical approximations of continuum physics, and as normal in geoscientific models, is the very different concept of numerical convergence to the solution of a differential equation.  It is directly addressed by our verification case.} \\
\indent We clearly state that the existing lattice models of 2D conduits can't work in a user-controlled large-scale ice dynamics model, and that for that reason we do not add them.  That these models are useful for process exploration is not denied or disputed.\\
\indent Linked-cavities could also be forced onto the nodes of a 2D lattice, but we, and \emph{all} existing 2D models, of which we cite the ones we know, do not put them on a lattice.  That is because we (collectively) \emph{do} have the PDE which describes the effect of a linked cavity system as continuum physics.  We have added this point to the paper. \\
\indent We encourage this reviewer, and the other reviewers who we also believe (by their questions) are interested in including conduits into subglacial hydrology models, to proceed in the normal manner of physics and attempt to develop a PDE description (i.e.~a lattice-free formulation) of conduit effects.  Or they can apply an actual conduit formation model to whole ice sheet scale, that is, one the causes a conduit to appear at the location where the data suggests it should, not where an input grid forces it.  To complain that we have not invented the former ourselves, or made a model trillions of times more efficient than existing models so that we can claim that latter, in addition to the other actions of this paper, is bizarre.\\
\indent That ``all the model results may be called into question'' is the normal state of affairs in climate modeling.  But this phrase profoundly explains why we \emph{don't} use lattice models.  It would be sad for a user to use PISM coupled to a GCM and then have a reviewer of the result correctly point out that there was a single subsystem in the entire coupled mess which was not using the usual translation-invariant structures of physics \dots namely a 2D lattice model of subglacial conduits.\\
\indent We completely agree that ``ideally conduits will be included in future
generations of ice sheet [models]''.}

\reply{+ This manuscript and model includes an ambiguous mixing of hard-bedded and soft-bedded ideas. For example, the model includes opening and closing of cavities at
the glacier bed, driven by basal sliding (section 2.5).  This is generally considered
a hard-bedded view of basal hydrology and motion.  However, the description of the
Mohr-Coulomb yield stress for till (section 3.2) is appropriate for soft beds and basal
motion accomplished by deformation \emph{within} the till, not at the interface between the
till and the glacier ice. Similarly, the sliding law that depends on the till's yield stress
(section 3.3) is also a soft-bedded concept. The combination of soft- and hard-bedded
ideas in this model appears to be inappropriate or at least confusing. \dots}
{We are not quite sure why our mixing of hard- and soft-bedded morphology is ``ambiguous''.  The equations are clear.\\
\indent Though the reviewer may not have read it, we cite \cite{Schoof2007deformable} which models the formation of cavities, by sliding, in a deformable subglacial till.  This is precisely a ``combination of soft- and hard-bedded ideas'' for ``opening and closing of cavities at the glacier bed, driven by basal sliding'' in the sense used by the reviewer.  We also cite   \cite{vanderWeletal2013} which uses till essentially as we do, combined with a conduit in a 1D model. \\
\indent The literature of subglacial hydrology usually avoids including till in model-based exploration of hard-bed processes, but we can't find a single published (or unpublished) example of a till-free bed-reaching borehole in ice of any substantial thickness, and none are offered by reviewers.  Because we expect based on the few existing basal observations that the majority of the ice overburden pressure of an ice sheet, in wet-based areas, is supported by saturated till, we include a model for its strength, namely Mohr-Coulomb. \\
\indent Yes, the Mohr-Coulomb model for the yield stress for till is appropriate for soft beds and basal motion accomplished by deformation within the till.  However, basal \emph{ice} deformation may occur in a thin (meters) layer of temperate ice with high water and sediment content. This deformation, and also notional hard-bed sliding if it occurs, are all modeled in the current literature by power-law sliding relations.  An ice sheet model, and the actual data available to constrain it,\footnote{Esp.~DEM, surface velocity, and bed elevation.} cannot distinguish these mechanisms occurring close to the bed. \\
\indent Finally, as stated in section 3.3, our computed yield stress value $\tau_c$ is used as a physically-meaningful coefficient in a power law for sliding \cite{AschwandenAdalgeirsdottirKhroulev}, and such power laws are effectively regularized Coulomb stress models in the range of powers we use \cite{SchoofCoulombBlatter}.  Having the coefficient of the sliding law be physically meaningful, and being tied to modeled basal water pressure so that it can physically evolve, is both conceptually and practically better then providing a sliding law with no physical meaning of, or physically-based way to model the temporal- or spatial-variation of, the coefficient.}

\reply{\dots  Furthermore, the description of 1-way and 2-way coupling could be more clear. If the rate of basal motion ($u_b$ or $v_b$) is an input to the model, then why is there a section on the sliding law (section 3.3)?}
{Section 3.3 is included exactly to give meaning to the yield stress $\tau_c$ as a submodel output.  Because this is a model description paper for a submodel of PISM, we are obliged to state what the inputs and outputs of the submodel are.\\
\indent One connection between $\mathbf{v}_b$ and $\tau_c$ is our hydrology model.  The other is the whole ice dynamics model of PISM, which ice-sheet-modeler readers know takes boundary stress as an input and produces velocity as an output.  This ice dynamics model is well-described in literature we cite. \\
\indent As hinted above, the normal use of sliding laws in ice sheet modeling is to have \emph{no} physical meaning to the coefficient of the sliding law, and often to set it by inversion of surface velocities, thus totally by-passing a process-based description of how it might evolve.  We think that having mass-conservation for liquid water in the subglacial system, and using a physically-based computation of the coefficient in a sliding law, is a preferred situation.}

\reply{\label{inversepage} + It is interesting and surprising to note that you find an inverse relationship between
water pressure and basal motion for systems at steady state. This is contrary to almost
all prevailing sliding laws. Is a result of the 1-way coupling ($v_b$ that does not directly
depend on water pressure)? Whatever the cause, it is sufficiently surprising to warrant
additional discussion.}
{We include an analysis of steady states because this analysis is (mostly) not done elsewhere for the (now) standard linked-cavity system equations (e.g.~from \cite{Hewitt2011,Schoofetal2012}).  Our first point is that these equations imply a functional form $P(W)$ at steady state, and thus that Flowers and Clarke \cite{FlowersClarke2002} are not crazy to propose such.\\
\indent Indeed there emerges a ``inverse relationship between water pressure and basal motion for systems at steady state'' from this analysis.  Why is this surprising?  Basal motion generates cavities (i.e.~space for the water to fill), so the pressure drops.  Said another way (as we do in the paper), sliding increases the opening rate, so if creep closure must balance it then the pressure will drop, to speed the closure, unless there is a simultaneous increase of water into the system (which does not happen at steady state).  We presumed this observation, which is not offered as a ``sliding law'' at all, and indeed should not be used that way, was standard.  In any case it follows from the equations, the reviewers don't believe our analysis is in error, and we have included it with the prominence we believe it deserves. \\
\indent Presumably the idea is surprising because the reviewer believes in sliding laws.  Regarding the idea that the sliding velocity ``does not directly
depend on water pressure'', we remark that sliding laws usually relate the basal shear stress applied to the ice \emph{and} the water pressure \emph{and} the ice base velocity.  Indeed, the presence of longitudinal stresses in the ice implies that there is globalized connection from water pressure back to sliding velocity, via (e.g.) stress boundary conditions at the boundary of the ice fluid mass (i.e.~the glacier).  This connection via a stress balance is outside of the scope of this submodel description paper, but is described at length in \cite{BBssasliding}, which we cite.\\
\indent Our analysis of steady states is not offered as a sliding law, and in that sense the reviewer is correct that our analysis ``is a result of the 1-way coupling,'' in the sense that conservation of momentum in the ice plays no role in the relationship.  The relationship simply follows from equations (13), (14), (15) in their steady state cases, as clear in the paper.}

\reply{---Line Edits---\\
p.~4706, l.~26 Also consider citing Walder 1982 if your purpose is to highlight some of
the early work here.}
{We cite Walder 1982 line 22 of page 4708.  We use Creyts and Schoof at this point on page 4706 because only a stable (i.e.~viable) models aquifer geometry are worth listing as alternatives which might go into a subglacial hydrology numerical model.}

\reply{p.~4707 l.~5 I think the best reference for englacial porosity is Fountain et al.~2005, from Storglaciaren.}
{These citations here are about models.  We (already) cite Fountain later when describing observations.}

\reply{p.~4708, l.~24 It may be worth mentioning that wall melt in linked cavities is generally
expected to be small (Kamb, 1987 and Bartholomaus et al., 2011)}
{We have added this comment, thanks.}

\reply{p.~4709, l.~1 What are the ramifications of neglecting to model conduits? Many observational studies (including work by Nienow, Mair, Anderson, Cowton, Harper) have
shown that efficient conduits are a fundamental component of subglacial hydrology.
How will your model provide insightful and realistic results without a conduit component?}
{We had assumed that models like \cite{Bartholomausetal2011}, which seem to explain the behavior of (rare) well-observed hydrological-plus-glacier-dynamics systems without using conduits, were of some value, but we may be wrong.\\
\indent Remember that the model is intended for whole ice sheet use.  We want, therefore, a \emph{well-posed extension} of the models used for identifying subglacial lakes \cite{LeBrocqetal2009,Siegertetal2009}, and we want a \emph{mass-conserving extension} of a successful (in terms of explaining surface velocity observations) ice stream basal stress model \cite{AschwandenAdalgeirsdottirKhroulev,BBssasliding,Tulaczyketal2000}, and we want ice streams to have significant width as they are observed to, but which they would not without a pressure-evolution equation.\footnote{If mechanisms like linked cavities do not draw water out of the hydraulic-gradient descent paths then the water is concentrated along characteristics.  Compare \texttt{routing} and \texttt{distributed} figures in section 9.}\\
\indent While the above purposes are all quite prominent in the paper we actually wrote, they are essentially ignored by these and other reviewer comments.}

\reply{p.~4712, l.~17 It is not immediately clear to me why $\nabla H \gg \nabla W$. Where does this suggestion/observation come from?}
{We have revised the relevant sentence to say: ``If $P$ scales with the overburden pressure $P_o$ then the first term will dominate in the situation $|\nabla (H+b)| \gg |\nabla W|$''.  We no longer assert the situation is ``common''.\\
\indent This comment is in the context of explaining why, for nearly the first time, we have included ``$\rho_w g W$'' into the formula for hydraulic potential.  We presumed that the reason it is left out of nearly all prior literature is because that literature assumes that flow along the gradient of the (prior) hydraulic potential $\psi = P + \rho_w g b$ dominates over the gradient of the other part (i.e.~the gradient of the added part ``$\rho_w g W$'').  This is the only case in which it would be acceptable to leave out the added part, as almost all prior literature does. \\
\indent So either: (i) all the prior literature is worthless (perfectly possible), or (ii) the case $|\nabla (H+b)| \gg |\nabla W|$ is worth considering as a way to relate our formulas to prior literature, and that is the spirit in which we consider it.  Our model only \emph{assumes} $|\nabla (H+b)| \gg |\nabla W|$ in a minor part of the given formula for the flux; see the next reply.}

\reply{p.~4712, l.~23 If here you assume that W $\ll$ b or P, and thus can be neglected in eq. 8, why have you made the distinction in eq. 2 and the discussion that follows to include the W term?}
{This question only makes sense if ``you assume that W $\ll$ b or P'' is interpreted as ``you assume that $|\nabla W| \ll |\nabla (H+b)|$''.  We do not compare values of $W$ (a thickness) to $b$ (an elevation) to $P$ (a pressure); we are comparing only gradients of distances to each other.\\
\indent As noted, the simplification is for simplicity, in particular for simplicity in the final implementation.  Despite the simplification we \emph{keep} the part of the flux proportional to $\grad W$, so the model is more complete than any other applied at ice sheet scale.  Furthermore, the simplified model is always diffusive for any pressure closure, and so, in particular, the \texttt{routing} model is well-posed unlike the related models in the prior literature \cite{LeBrocqetal2009,Siegertetal2009}. \\
\indent The simplification occurs inside our formula for the effective hydraulic conductivity $K$, which is wildy-uncertain in practice anyway.  That is, in the formula
  $$K = k W^{\alpha-1} |\grad(P+\rho_w g b)|^{\beta-2}$$
the correct values for coefficient $k$ and the powers $\alpha,\beta$ are all subject to the most rank and data-free speculation, as we imply in citations in subsection 2.3 which give a wide range of values.  The simplification makes no difference at all if $\beta=2$, which is used in more than half the cited work which picks a value for $\beta$.}

\reply{p.~4713, l.~1 Near here, or somewhere else within the paper, please compare your
values for hydraulic conductivity [$k$] with those that may be calculated from field observations. Are your values in line with those found in the field? Googling ``subglacial
hydraulic conductivity'' yields several points of comparison.}
{We have looked and not found.  There is not a single observational paper we can find which argues that a directly-measurable value of the hydraulic conductivity is describing the average effect of a linked cavity system over the area of a grid cell relevant to this work (100 m to 5 km squares, say).  Values are, of course, always given when these papers include a \emph{model}---our Table A1 cites the default value of $k$ as from \cite{Schoofetal2012}---but one should be very skeptical that a value from applying one model to the data (supposing this is done) is still the right value when applying a different model to the data. \\
\indent Of course hydraulic conductivity for \emph{till} is given in literature, based on specific \emph{in situ} observational work.  Such values appear in the literature we cite, and they dominate the results from Googling ``subglacial hydraulic conductivity''.  But the till hydraulic conductivity value should not be used as $k$.  The conductivity of till is so low that water does not move laterally through till in a time, and over distances, which could explain any of the apparent behavior of water under glaciers and ice sheets.  Rather it is the macroscopic conductivity of the connected cavity network which is relevant.  (Such a network can be present even as there is sediment (till) lying around; this is the situation we are modelling. \\
\indent  To quote \cite{Bartholomausetal2011}, which we already cite, \begin{quote}
\emph{Each of these three parameters, $\gamma$, $[k=]C\tau_b^n$, and $\phi$, is only weakly constrained by observations reported in the literature.  \dots $[k=]C\tau_b^n$ has units that depend on the exponent, and varies from $1.5\times 10^{-5}\, \text{m}\,\text{s}^{-1}\,\text{Pa}^{0.18}$ to $1.1\times 10^{-3}\, \text{m}\,\text{s}^{-1}\,\text{Pa}^{0.4}$ (Jansson, 1995; Sugiyama and Gudmundsson, 2004).}
\end{quote}
Is this ``weakly constrained'' result the kind of ``field observations'' meant by the reviewer?  Why should space in this model description paper be used to recapitulate such a weak and uncertain state of affairs?  We want to avoid, in a model description paper, asserting that any particular value of any constant is correct.  We are building the model so users can relate its relatively few parameters ($k$ among them) to rich, but often indirect, available data.  As pointed out in our paper, ``Darcy flux parameters $\alpha,\beta,k$ are also important [to the distribution of water thickness in the model results].  Parameter identification using observed surface data, though needed, is beyond the scope of this paper.''  Indeed, the journal is \emph{Geoscientific Model Development} not \emph{Journal of Glaciology}, and the conductivity $k$ is an adjustable parameter in the model.  The source of the default value (Table A1) is cited.  This situation should suffice.}

\reply{p.~4714, l.~15 Here and nearby: define $c_1$, $c_2$, and A.}
{We have done so.}

\reply{p.~4715, l.~6 Phrasing is ambiguous, as it makes it sound as though your model
potentially does not include till water storage beneath some parts of the ice sheet.}
{The issue is that the majority (by area) under the world's ice sheets does not have \emph{liquid} water under it, though it may have till.  The equations for till storage, transfer into the transport system, and weakening of the saturated till, must all reflect the amount of \emph{liquid} water there, not frozen water.  We model frozen locations as not having liquid water in the till, so $W_{til}=0$.  We have attempted to make this point clearer, without increased length.}

\reply{p.~4715, l.~20 Why not include lateral transport of water through till if vertical transport is included?  Till is often regarded as having an anisotropic hydraulic conductivity (e.g., Jones, 1993, ``A comparison of pumping and slug tests. . .'' in Ground Water vol.~31(6)).  Horizontal conductivity can be at least several times greater than vertical conductivity.}
{The reason for not including horizontal transport is the standard fact of ice sheet modeling generally, but more so here: flowing layers are \emph{thin}.  In particular, any till thickness ever given in the literature is 1/1000 (or less) of the lateral distances traveled by subglacial water.  Anisotropy is irrelevant unless the horizontal conductivity can make up for this thinness.  As the reviewer says, the conductivity in the horizontal is \emph{not} one thousand or more times the conductivity in the vertical.\\
\indent Of course, there is presumably a transport network of cavities, conduits, or thin sheets in addition, which has a low macroscopic conductivity.  We attempt to model the first of these morphologies because continuum physics evolution equations are available for it, and it is stable.  The overall structure of the model is exactly what we believe is appropriate for water moving underneath ice sheets which have much of their overburden supported by saturated till: we model transport in combination with till storage, and the till is modeled as Mohr-Coulomb.}

\reply{p.~4715, l.~20 Is m in eq. 16 the same as m in eq. 1?  If so, these terms cancel out of
eq. 1.}
{Yes, $m$ in equation (16) is the same as in equation (1).  Yes, they cancel out when $W_{til}< W_{til}^{max}$, so that no water enters the transport network (i.e.~so that $\partial W/\partial t=0$ in (1)) in that case.  But we conserve water.  Thus if the right side of (16) is positive and also $W_{til} = W_{til}^{max}$, at a given location, then $\partial W_{til}/\partial t=0$, i.e.~we put no more water in till, and the water goes into the transport network ($W$) according to equation (1). \\
\indent We have attempted to clarify this logic here in section 3, and also in section 7 where numerical schemes are nailed down.}

\reply{p.~4715, l.~20 If $m/\rho$ is almost always bigger than $C_d$, then $dW_{til}/dt$ is always increasing up to the cap $W_{til}^{max}$. It would be useful to lay this out more explicitly, and include eq. 21 in this subsection. Essentially, you have a Boolean relationship, where in some places there is wet till and other places the till is frozen. Is model sensitive to selection of $W_{til}^{max}$?}
{Yes, when there is positive\footnote{Note that the majority by area of ice sheets are assumed to have frozen base, so $m\le 0$ there.} basal melt, then $m/\rho$ is almost always bigger than $C_d$, so that $dW_{til}/dt$ is always increasing up to the cap $W_{til}^{max}$.  The figures in section 9 reflect this. \\
\indent Though we would not say we have a ``Boolean relationship,'' we agree with the spirit of the reviewer's assertion.  We repeatedly emphasize that we enforce inequalities including (21), the bounds on $W_{til}$.  It follows that in some places there is wet till and other places the till is frozen; well-known reference \cite{BBssasliding} covers these ideas.\\
\indent No, the model is not very sensitive to the selection of $W_{til}^{max}$, at least in areas of substantial basal melt rate.  This is because of our Figure 1(b), which shows that the effective pressure $N_{til}$ is a weak function of $W_{til}$ once the till is weak enough (i.e.~$N_{til}$ is low enough) to allow sliding.}

\reply{p.~4715, l.~24 Inclusion of $C_d$ with fixed value is poorly justified and seems very ad hoc.  Even if used by Tulaczyk, why is it necessary here and what is the model sensitivity to the selection of 1 mm a$^{-1}$?  A constant rate of till water drainage into the subglacial hydrologic system, that does not depend on pressure gradients, seems very odd.}
{We agree that Tulaczyk's use of $C_d$ is ad hoc.  We need such a background loss of till-stored water, like him, but unlike him we have implemented a \emph{conservation} model.  We keep track of all the water globally, but we need previously wet areas which no longer have water input to not eternally remain weak (i.e.~with till full of liquid water).\\
\indent In areas resembling anything in the northern hemisphere, with relatively high basal melt rates, the model is insensitive to $C_d$.  In areas of very low melt rate (e.g.~EAIS) there is a time-scale sensitivity.  We have no time-dependent information about changes in EAIS subglacial melt rates with which to constrain values.  The implication that we should use something complicated (which is the only alternative to ``ad hoc'' here) implies many unconstrained parameters instead of few.   If the reviewer has 2D, data-supported, physics-based, applicable-at-large-scale models of how till and a linked-cavity (or other) system interact, then we hope he publishes that.  We can't find it.  We seek relative simplicity and few parameters, instead of implementing idle process speculation.}

\reply{p.~4717, l.~1 What is the effect of this choice? How was it selected?}
{The value $\delta=0.02$ is based on the observations that subglacial water pressure at the bottoms of boreholes, i.e.~in till, have pressure within a few percent of overburden pressure.  The particular value used means that fully saturated till has water pressure which is $(1-0.02)P_o=0.98 P_o$.  This parameter \emph{is} very influential on sliding, and is explored the right way (i.e.~by using lots of observations of surface velocity) in \cite{AschwandenAdalgeirsdottirKhroulev}, though using a non-conserving subglacial hydrology model there.}

\reply{p.~4718, l.~16 I recommend changing the title of this section to ``Basal motion relation'' or some other phrase. ``Sliding law'' implies slip at the interface between the ice and its bed, whether bedrock or sediment, whereas your equation for yield stress (eq. 17) is appropriate for till deformation.}
{Ice sheet modelers use ``sliding law'' the way we do, and ice sheet models can't make the distinction implied by this reviewer.  That is, there is no distinction in results in any existing ice sheet model between modeling slip at the ice-bed interface and a meter down within the till.  Vertical resolution like this is only in the heads of process modelers. \\
\indent Equation (25) is called a sliding law by all readers familiar with ice sheet modeling, the target audience of this paper.  The reviewer also calls it a sliding law in other comments.}

\reply{p.~4718, l.~21 q is already used for flux (even if printed in bold-face to identify its vector character).  I suggest using another variable name.}
{Exponent $q$ is used in prior literature, including \cite{AschwandenAdalgeirsdottirKhroulev}, and in the PISM users manual.\footnote{Despite the content of all reviews \dots oddly enough we are actually trying to publish a model description.}  There is consistent use of bold for vectors in the paper, thus the flux is $\mathbf{q}$ while the power is $q$, so no confusion will arise.}

\reply{p.~4718, l.~23 Previously (eq. 14), $v_b$ was the rate of basal motion. $u$ and $v_b$ are used inconsistently throughout the paper.}
{This has been corrected.  Only ``$\mathbf{v}_b$'' is used for the ice base sliding speed.}

\reply{p.~4719, l.~4 What value of $q$ have you selected for your simulations? Justification?}
{Value $q=0.25$ was used in the spinup that preceded the hydrology run \cite{AschwandenAdalgeirsdottirKhroulev}.  The sliding law equation (25) is, as stated above, included so that the reader knows that $\tau_c$ is a model output and how it is used, so the particular $q$ value is unimportant.  More important content, explaining the meaning of the $q=0$ and $q=1$ extremes, is given instead.}

\reply{p.~4720, l.~1  While ``velocity'' is technically correct, it is an odd choice for a thickness change. I suggest using ``rate.''}
{Sorry.  We mean that $\tilde{\mathbf{V}}$ is a velocity, not $\partial W/\partial t$.  This has been clarified.}

\reply{p.~4720, l.~5 Define $h$- the ice surface elevation.}
{This is simply a typo.  It should be $H$, the ice thickness.  Corrected.}

\reply{p.~4722, l.~7 ``\dots does not exist for tidewater glaciers or ice sheets.'' This may not be strictly true see Gulley et al, 2009, in QSR, where they report exploring many englacial conduits.  In subsequent work, Gulley has mapped subglacial conduits.  A safer statement would be that ``vapor/air-filled cavities are not known to exist far from glacier margins.''  The distinction regarding tidewater glaciers or ice sheets is unnecessary.}
{The relevant sentence has been removed in the revisions which shortened the paper.  This idea, that vapor/air filled cavities are not a concern supported well by observations, in a model intended for ice sheets, is not important enough to comment on.}

\reply{p.~4722, l.~10 ``observed in ice sheets and glaciers`` instead of ``observed in ice sheets''}
{We have clarified that we only mean ice sheets here by only citing Das 2008.}

\reply{p.~4722, l.~21 Add that the englacial water table is intended to represent the mean
over some large area of glacier, perhaps $>1$ km$^2$. Here, it is best to avoid the extreme
complications of, e.g., Fudge, 2008 in J Glac, where subglacial water pressures vary
significantly over very short distances.}
{Our point is not that there is variation over any particular scale, but that efficient connection to the subglacier implies a close connection between subglacial pressure and the height of water englacially.  This is not, fundamentally, contradicted by Fudge 2008.  We agree that our (notional) englacial water table represents a spatial-average of the nearly unobservable englacial macroporous network.}

\reply{p.~4723, l.~8 You might add that we can expect phi to be large everywhere that $dP/dt$ would be large (a highly fractured temperate glacier in coastal Alaska), and that phi would be small only where $dP/dt$ is small (ice sheet interiors). Thus, even hydraulically/numerically ``stiff'' ice sheets shouldn’t experience physical or numerical shocks.}
{Actually, we think $dP/dt$ may be very large in ice sheet ``interiors'', namely during abrupt subglacial lake filling or drainage (observed in Antarctica) or moulin drainage of supraglacial lakes (observed in Greenland).  We don't really expect the model to be good for either highly-fractured temperate glaciers in Alaska, or in modeling the temporal detail associated to the above ice sheet dramas.  Instead of discussing at length all possible consequences, regarding this material we must let the equations themselves do some of the job.}

\reply{p.~4724, eq. 34 As before, are these m's supposed to be the same?}
{Yes.  See comment above.}

\reply{p.~4724, eq. 34 This is an odd combination of equations, because the top equation is a component of the bottom equation, but the middle equation has not been incorporated in the bottom equation.}
{Yes.  This is ``odd,'' but that is different from ``incorrect,'' and the situation is complicated. \\
\indent The context: As clearly stated in subsections 2.3 and 2.4, one can write either of two expressions for the flux, namely $\mathbf{q} = -K W \grad \psi$ or $\mathbf{q} = \mathbf{V} W - D \grad W$.  And a term $\nabla\cdot\mathbf{q}$ appears in both the water amount evolution equation and the pressure evolution equation.  \\
\indent The complications are that (i) we are indeed enforcing inequalities on $W,W_{til},P$, and (ii) we want to handle the $\nabla\cdot\mathbf{q}$ terms by the same numerics, with a split between diffusive and advective parts, in both equations in which it appears.\\
\indent  These complications are both made most clear, given that we don't want to write a paper using variational inequalities which would only be understandable to mathematicians, when we describe the numerical scheme in section 7.  After writing (and tossing) our paper many times with different expository choices, we find the current exposition most clear.  If the reviewer finds it incorrect he should say so.}

\reply{p.~4726, l.~3 Note that this is essentially the same as eq. 27.}
{Yes.  This redundancy in the exposition has been removed.}

\reply{p.~4726, l.~23 Another connection is presented on p.~4721.}
{Yes, we have noted this connection now.}

\reply{p.~4727, l.~20 Around here, discuss that, in steady state, eq. 41 suggests that at water pressure decreases, the rate of basal motion increases. This flies in the face of most sliding laws. Can you offer any insight as to how we are to incorporate these two views in our understanding of hydrology and glacier dynamics?  Is the one-way coupling of your hydrology model with a glacier dynamics model sufficient to gain insight?}
{The reviewer has already made this comment above (page \pageref{inversepage}), and we address it there.  In summary, we are not stating a sliding law, we are stating a consequence of the equations (esp.~equation (13) which comes from Hewitt 2011 among other places), and the reviewer does not assert that this deduction is in error.}

\reply{p.~4727, l.~20 Also note that $P$ depends also on $v_b$, not on $W$ alone.}
{Here $P$ depends on a lot of things, sorry.  Our notation ``$P(W)$'' emphasizes that one can write the equation as a function which yields $P$ given $W$, if all other symbols in the equation are defined.  This is the usual convention in more than a century of exposition in theories using mathematics.\footnote{A mathematician would presumably be equally happy calling it more precisely ``$P(A,c_1,c_2,W_r,P_o,|\mathbf{v}_b|,W)$,'' but probably not the reader.}}

\reply{p.~4727, l.~23 I don’t see the relationship between eq. 41 and the $VW$ advective flux.  Please elaborate.}
{As noted immediately after this claim about equations (41) \emph{and} (38), we explain it in the Appendix.  It is nontrivial, which is why we explain it.}

\reply{p.~4729, l.~5 Readers should not have to turn to the appendix to learn what $s_b$ is. Move essential material out of the appendix and into the main text.}
{Good point.  As noted at the top of this document we have replaced section 6 by a brief text summary, in section 5.4, and so this issue with $s_b$ does not arise.}

\reply{p.~4729, l.~6 Defining this new $\omega_0$ variable seems unnecessary.}
{The relevant text has been removed, and the issue does not arise.}

\reply{p.~4730, l.~5 What is the justification of the 5th power in the sliding speed?}
{In constructing an exact solution for the purpose of verification, specificity is essential, and qualitative reasonableness is of some importance, but uniqueness is not asserted or important.  Here the power is used for simplicity (thus it is an integer) and smoothness of the solution (noting the power on $|\mathbf{v}_b|$ is 1/3 in the formula $P(W)$, smoothness requires that this power exceed 3).  However, the reader is no longer bothered with the particular power because section 6 is replaced by a short text description of the construction of the nearly-exact solution.  Figures A2 and A3 remain; they show the solution and the reader can see the smoothness at sliding onset.}

\reply{p.~4730, l.~17 Define what you mean by ``under'', ``normal'' and ``over'' pressure.}
{Yes.  We have added this into the text.}

\reply{p.~4731, l.~13 Give a few sentence introduction to the numerics here. The point is to
discretize eq. 34. What is the order of calculations? What will feed into what over the
next sub-sections of section 7? A thumbnail sketch similar to what is presented in 7.6
would be useful to guide the reader.}
{We believe that our organization of the exposition of the numerics is already appropriate.  We want a serious reader who has made it this far to see an adequately-precise description.  Precision requires defining most symbols before 7.6, where the summary happens.}

\reply{p.~4731, l .19 Near here, is it necessary for a model development paper to include a
reference for ``CFL'' and ``upwind''}
{We beg to differ.  We believe that readers of this material will normally prefer not to see these defined, as they are so common.  Note that in the same text we do not define ``explicit'', ``finite difference'', ``centered,'' ``second-order,'' ``stability,'' or ``convergence'' either.  Should we really define all of these?}

\reply{p.~4731, l.~21 Be sure to clarify that $u$ and $v$ are not components of $v_b$, but are for the water speed.}
{Thanks for the reminder.  We have now defined $u,v$ components at their first use.}

\reply{p.~4732, l.~8 Parenthesis around the citation}
{Yes, got it.}

\reply{p.~4736, l.~7 Is it important that the reader understand what it means for a scheme to be ``flux-limited?''  Without modeling expertise myself, I’m not sure what this means.}
{Readers experienced with numerical advection schemes will know, but in any case the references to Morton\&Mayers, LeVeque, and Hundsdorfer\&Verwer are adequate.}

\reply{p.~4742, l.~17 Because you report that your scheme is mass conserving so prominently
in the abstract, you should report how much error is involved with step (x), where
negative water thicknesses are discarded. This could be for the Greenland run of
section 9.2.}
{The quantity in question, the error in step (x), is much bigger than we would like, but that is a result of the implementation of the \emph{energy} conservation scheme at the ice base \cite{AschwandenBuelerKhroulevBlatter}, not the hydrology scheme.  We only discard negative water thicknesses when, essentially, the energy-conservation-computed basal melt rate $m$ is \emph{negative}, i.e.~in the refreeze case.  The size of this quantity has to do with the magnitude and extent of large negative basal melt rates, and not the quality of the hydrology model at all.  The hydrology scheme itself is positivity-preserving, as shown in the manuscript. \\
\indent Thus the size of this quantity is thus not at issue in this model, but the fact that ours is the only paper which has ever even mentioned this quantity \emph{should be} at issue.  Noticing this error for the first time in a hydrology scheme is a feature not a bug.  Figuring out the best way to reduce it, and demonstrating its reduction to zero under grid refinement, are for future research.}

\reply{p.~4745, l.~22 Are the 2800 processor-hours on each of the 72 processors or divided amongst the processors?}
{It is the total number of processor-hours in the computation, as is standard in describing parallel computations and when using the units ``processor-hours''.}

\reply{p.~4746, l.~16 Do you specify a geothermal heat flux? The handling (or lack thereof) of geothermal heat should also be specified earlier, where the model setup is described.}
{We specify geothermal flux from input data to the model.  The relevant Shapiro and Fitzwoller data is addressed in the SeaRISE description paper (\cite{Bindschadler2013SeaRISE}, cited), and the ``handling of geothermal heat'' is handled and documented carefully at full paper length (\cite{AschwandenBuelerKhroulevBlatter}, cited).\footnote{If anyone tells you that ``handling the geothermal flux'' requires less than a full polythermal thermodynamics scheme for the entire ice sheet, they are lying.}}

\reply{p.~4746, l.~18 Please report if you identified any basal freeze-on ($m < 0$) consistent with Bell et al., 2014, Nat. Geosci., vol.~7?}
{Why this particular order?  As noted, we allow basal freeze-on.  Comparison to all existing observations is not the role of a model description paper.}

\reply{p.~4747, l.~25 Again, this is a good place to discuss the ramifications of a model without R-channels. What are the limitations of your model? Is there a way that aspects of R-channels emerge in your model without explicit channel modeling?}
{Yes, ``aspects of R-channels emerge \dots''.  In fact we say (page 4747, line 25):
\begin{quote}
\emph{This model could be regarded as a minimal ``conduit-like'' description of the subglacial flow, because of these concentrated pathways. As noted in the introduction, however, our model has no ``R-channel'' conduit mechanism, in which dissipation heating of the flowing water generates wall melt-back.}
\end{quote}
We are wondering, after the many lines we have already spent explaining something we \emph{don't do}, why more and more is needed.}

\reply{p.~4748, l.~2 What about the eastern outlet glacier results makes them particularly
suspect? 7, C1774--C1781, 2014}
{They are suspect because of the quality of the ``bed elevation detail provided by the SeaRISE data set,'' as stated in the paper, given that the bed elevation field there is from few flight lines.}

\reply{p.~4748, l.~4 You report on the run time for your spin-up with the null hydrology model, but what are the processor demands for the distributed model described here?}
{Good point.  We have added the numbers, which are very small because we are not modeling ice dynamics, namely 14.2 processor hours for \texttt{distributed} and 14.7 for \texttt{routing}.  As noted, the higher modeled water velocities and modeled diffusivities in the \texttt{routing} model decrease the time step, which implies more computation, but on the other hand the per-time-step work in \texttt{routing} is less.}
% 14.2286 proc.-hours
% 14.6556 proc.-hours

\reply{p.~4748, l.~5 Another statement regarding the sensitivity of results to $W_{til}^{max}$ would be useful here.}
{Though this model has a substantially-reduced number of parameters than the model the reviewer would want, there are still far too many to examine sensitivity of all of these parameters.}

\reply{p.~4748, l.~20 Around here, worth mentioning that pressure as an increasing function of $W$ is vaguely in line with the results of the Flowers (2002) model, although your model reveals additional complexity.}
{We already make this connection at four different spots in the paper.  Cluttering up the description of model results is, we believe, unnecessary.}

\reply{p.~4749, l.~17 ``seemingly-disparate''}
{Yes, thanks.}

\reply{p.~4750, l.~20 Again, reference the observation that steady pressure here increases as sliding decreases, which is inconsistent with almost all sliding laws.}
{As noted, we are not giving a sliding law, the inverse relation in question applies to all models which have both cavity formation through sliding and cavity collapse through creep, and this is not an important result of the paper.}

\reply{Table 3 Odd to present the melt rate as a function of water density. Change this to a
straight scalar (i.e., 200).}
{Yes.  But Table A3, and other details of the construction of the exact solution, have been removed, so the issue does not arise.}
\end{itemize}



\subsection*{Comments by Anonymous Referee \#2}\begin{itemize}
\reply{This paper describes a new sub-component of the open source ice-sheet model PISM,
which accounts for subglacial drainage of meltwater.  The model and a number of
subcases are described in considerable detail and then the numerical implementation
is described.  A simple steady state solution is used to test the numerical method, and
the model is then applied to the whole of the Greenland ice sheet.\\
\indent I enjoyed reading this paper.  It represents to my knowledge the first serious attempt
to include an evolving subglacial drainage model within an ice-sheet scale ice-sheet
model, and the results are encouraging.  As such, I would like to recommend publication.  However, I have a few issues that I think need to be clarified or thought about first.}
{We appreciate this summary of the paper.}

\reply{The major comments are here, followed by some specific but more minor points.\\
\indent 1. The first term of (33), involving the pressure derivative and which represents
changes in englacial water content, ought to appear in (34a) also, since this term
derives from the mass flux into/out of the englacial system, and it is the addition
of this term to the mass conservation equation (34a) that gives rise to its appearance in (33).  As it stands in (34), subtraction of the first and third equations
puts the $\partial P/\partial t$ term into the opening/closure equation $\partial W/\partial t$, which I don't see justification for.}
{First, as we state, we only have \emph{notional} englacial porosity, because it is used as a regularization.  The englacial storage itself would require another mass variable (e.g.~$W_{eng}$ used below) and then more coupling parameters between subsystems\footnote{I.e.~transfers between $W_{til}$, $W$, and $W_{eng}$ all need to be parameterized.  We wrote such a model and paper and threw it away.} would be required. \\
\indent The reviewer's statement that the term ``represents changes in englacial water content'' is wrong.  Rather, changes to englacial water content (i.e.~the time-derivative term in (33)) is balanced by the net result of all the terms on the right side.\\
\indent Apparently this is the first time that the reviewer has noticed that the divergence of flux term $\Div\mathbf{q}$ appears in both the water thickness and pressure evolution equations, in a model which has three components: (i) mass conservation, (ii) cavity thickness evolution, and (iii) full cavities.  Such models appear in \cite{Hewitt2011,Hewittetal2012,Schoofetal2012,Werderetal2013}, and we would think that they are familiar to the reviewer.  For instance, the ``first term of (33) involving the pressure derivative'' is also a term in (32), which is verbatim from \cite{Schoofetal2012}.\\
\indent In any case we cannot know what ``justification'' should be in the reviewer's head.  If the reviewer asserts our deduction is wrong here then he/she should say so.}

\reply{2. p4738, l11, and this section generally---is it clear that these arguments prove
stability for the \emph{system} of equations in this model (in which the coefficients in (60),
say are varying at each timestep due to the pressure evolution)? The analysis
here seems to be for a standard advection-diffusion equation on its own, but it
is not immediately clear to me that standard results can be used here. I have
no doubt that the method is stable, but I think if the stability properties are to be
discussed in this much detail, it needs to be done for the whole system together,
and not for the individual components of the operator splitting separately. Or if
there is an argument as to why this is sufficient, that should be included.}
{Yes, these arguments prove the stability of the numerical scheme for the particular equation in the presence of irregular coefficients which might come from coupling.  We are not using a linearized stability analysis here,\footnote{E.g.~von Neumann or Fourier analysis \cite{MortonMayers}.} which would ``see'' the coupling if present, but a \emph{maximum principle} analysis, which often gives overly-pessimistic stability conditions.  One of the benefits of max principle analysis, however, is that the details of the coefficients, including the possibility of them coming from coupling to other equations, don't enter into the analysis.\footnote{Thus the max principle analysis of stability of schemes for abstract heat-like PDEs $u_t=(1+x^2)u_{xx}$ and $u_t=(1+w^2)u_{xx}$, where $w$ is the solution of another equation, is the same.  But note that the \emph{positivity} of the coefficient is important here, not its origin.}  If the reviewer's ``standard results'' for advection-diffusion equations come from a linear stability analysis, which is the most standard we suppose, then no we are not using these ``standard results.''\\
\indent On the other hand, the case of negative source term, which in this case comes from outside the model (i.e.~through the melt rate $m$ not through the involvement of coupled equations for $P$ and $W_{til}$), requires enforcing inequalities, and this is not in our stability analysis, as we state.  Furthermore, so as to shorten the paper we have removed the subsection in which the scheme is shown to have positivity-preserving and stable properties.  We have replaced this subsection with a brief text description.}

\reply{3. The boundary conditions should really be described in more detail.  It'd be helpful to state mathematically what boundary conditions are imposed (in section 5 say), rather than having it algorithmically described in section 7.  In particular, the diffusive nature of the $W$ equation suggests that one should apply some sort of conditions on $W$ at all boundaries, but these are rather hidden, \dots}
{On the one hand, the situation is much worse than portrayed by the reviewer, and it is not primarily about what is inside this paper.  On the other hand, there is no boundary in PISM in the sense meant by the reviewer.  We now explain these statements.\\
\indent Our model is, as clearly stated in the paper, subject to inequalities on water amount (i.e.~$W\ge 0$) and on pressure (i.e.~$0\le P \le P_o$).  These constraints imply that a mathematically rigorous description of the equations has \emph{free} boundary conditions essentially determined by a variational inequality or similar weak formulation.\\
\indent This is clearly understood \emph{for the pressure equation} by Schoof et al 2012 \cite{Schoofetal2012}.  However, \cite{Schoofetal2012} does not consider negative basal melt rate (i.e.~the case $m<0$ in equation (4.5) for the evolution of water thickness $h_w$ in \cite{Schoofetal2012}).  As a result they miss the fact that \emph{both} evolution equations, i.e.~for water amount $W$ and for pressure $P$ (our notation), are subject to variational inequalities.  Indeed, diffusive or otherwise, the continuum equation for water thickness $W$, as stated in \cite{Schoofetal2012} or \cite{Hewitt2011} or our manuscript or elsewhere, does not maintain positive values of $W$ if $m$ can be negative (i.e.~refreeze), so a free boundary appears which we must deal with.\footnote{It is a free boundary not seen in \cite{Schoofetal2012} primarily because there is no allowance for coupling to an energy-conserving basal melt rate, which would sometimes be negative.  In this sense the pre-determined water input of most subglacial hydrology modeling is addressing an easier problem than ours.}  A numerical scheme for $W$ evolution must actively enforce the inequality in some way, such as by restricting admissible functions or by truncation/projection in an explicit scheme such as ours or that of \cite{Schoofetal2012}. \\
\indent So the only mathematically honest treatment of our continuum model would require coupled variational inequalities.  As just one variational inequality is hard to handle---see \cite{Werderetal2013}, who say it is ``prohibitive'' in 2D and then skip it---the complications of a coupled pair are great.  We actually believe we \emph{are} correctly (i.e.~convergently) numerically solving this coupled pair of free boundary problems by an explicit scheme which truncates/projects to enforce the inequalities, but we are not close to proving that.  If we took on this topic with mathematical precision then (i) the paper would be enormous and (ii) no one would read it. \\
\indent On the other hand, note that the periodic domain (i.e.~flat torus) version of the model in \cite{Schoofetal2012} would have no classically-defined boundary conditions because the domain on which the continuum model is solved has no boundary.  For whole ice sheet simulations in PISM, the ocean or ice free land surrounding the ice sheet is has exactly such a periodic extension, that is, no boundary.  This has little disadvantage in practice, and the advantage that every grid point in PISM, on every processor, has the same physics.  We do indeed state how all free boundaries are handled numerically---this is what we are doing ``algorithmically in section 7'', by stating where inequalities are enforced by truncation/projection---but we don't have classical boundaries in the sense meant by the reviewer. \\
\indent In summary, we describe what the numerical scheme actually does in section 7.6.  Then we show verification results in a case where the exact continuum solution, subject to the two coupled (but unstated) variational inequalities, with free boundary, is known.  We think this is actually addressing the boundary conditions in a manner which is more helpful to the GMD reader than some expository alternatives.}

\reply{\dots and in section 9.1 it is claimed that there are convergence issues associated with a jump in $W$, which seems at odds with the diffusive term.}
{No, we do not say such a thing in section 9.1. \\
\indent We have attempted to clarify the sentence to which we think the reviewer is alluding:
\begin{quote}
\emph{The rate of convergence is roughly linear (i.e.~about $O(\Delta x^1)$) because largest errors arise at locations of low regularity of the solution, \dots}
\end{quote}
Thus we \emph{do} say that the exact solution has low regularity.  As well-known in numerical analysis, a perfectly-implemented numerical scheme of arbitrarily high order will still have slow convergence if the continuum solution being approximated has low regularity (e.g.~has discontinuous derivatives of some order).  Here we use ``$O(\Delta x^1)$'' to quantify the slow observed convergence of a scheme which should have $O(\Delta x^2)$ convergence on a high-regularity solution.  As seen in figures from \cite[see Figure 9(a)]{Schoofetal2012}, the variational inequalities defining the continuum model for pressure imply that the continuum solution has low regularity at places where $P$ hits zero or overburden.  This happens in our nontrivial 2D exact solution, which is also an exact solution for the \cite{Schoofetal2012} model.  We think that such low regularity at pressure limits is realistic for this model, so we want it in our exact solution.}

\reply{ \dots  I suspect the boundary conditions are mostly imposed by step (vi) on p4742, but I was not entirely clear on what is meant by 'not computing' the divided difference contribution to the flux divergence.}
{The sentence in question on step (vi) is simply wrong, and should not be there at all as it describes an old state of the code.  It has been removed.  \\
\indent Equation (55) is used as stated at all grid points, regardless of neighbor mask state.\footnote{See method \texttt{raw\_update\_W()} in file \texttt{PISMRoutingHydrology.cc} in branch \texttt{stable0.6} of the PISM source code.}  Thus the boundary conditions, which can all be interpreted as free boundary conditions and which are motivated by concerns listed in replies above, are applied in steps (ii), (vii), (viii), (ix), and (x) in the list given in section 7.6. \\
\indent In fact, we have added the following ideas about boundary conditions to the revised text: \emph{(i)} PISM always has a periodic grid for whole ice sheet computations, so there is no classical boundary to the hydrology system.  \emph{(ii)} Free boundaries occur all over the place as a result of enforcement of inequalities.  \emph{(iii)} In ice-free land and ocean (i.e.~ice shelf or ice-free ocean) grid points, the hydrology model effectively sees such a large (in magnitude) negative value for $m$ that any water which flows to, or diffuses to, that location during a time step is immediately removed.  \emph{(iv)} The ice-free land and ocean grid points have pressure determined by external factors (e.g.~atmospheric or ocean-base pressures).}

\reply{Finally, I felt the paper might be shortened without losing detail; there are a number of places where the discussion of relatively simple points is laboured. Sections that might be reduced include section 2.4, section 4.3, section 6.2, section 7.1, section 9.2.1, (could just reference Aschwanden et al for much of this?), section 9.2.4, and the appendix.}
{With this comment we heartily agree.  Our efforts to shorten the paper have reduced its length by FIXME percent.\\
\indent Regarding the specific recommendations, we have modestly shortened section 2.4, improved 4.3 without much reduction, removed almost all of section 6.2, mostly kept 7.1 but shortened section 7.2 substantially, and kept much of 9.2.1 and 9.2.4 so as to make sense of the inputs into and outputs from the hydrology model in a real application.  We have halved the length of the Appendix.  In addition we have made substantial reductions in section 5.2, we have removed sections 6.1 and 7.4 and replaced them by a sentence or two, and we have substantially shortend sections 7.5 and 8.}

\reply{Specific comments\\
1. p4708, l3, also throughout - I do not see why the parabolic equation is always described as a 'regularization', which suggests some element of artifice.  For the physical system described, the equation is parabolic, and there is no need to treat it as a regularization.}
{}

\reply{2. p4708, l7 - I'd temper this by saying that till is 'sometimes' observed, as I don't think it is true that it is always observed.}
{}

\reply{3. p4708, l20 - it is not the inclusion of wall melt in the mass conservation equation that leads to the instability but rather then inclusion of wall melt in the kinematic opening-closure equation.}
{}

\reply{4. p4711, l9 - given the coupling with PISM, it seems a bit odd to say that you 'accept' the hydrostatic approximation, since you should be calculating $P_o$ consistently with the ice flow. As I understand it $P_o$ is always hydrostatic for the level of approximation in PISM, so this would seem a better justification.}
{}

\reply{5. p4713, l11, also throughout - I find the repeated reference to the 'advection-diffusion equation' a bit misleading as although it has advection and diffusion terms, it is rather different from what is normally associated with that term, as the velocity depends on the pressure which is evolving simultaneously.  Perhaps this is my own connotation of advection-diffusion, but I think it should be emphasized that (12) is not stand-alone and is inherently coupled to more equations.}
{}

\reply{6. p4717 - the prescription of a minimum value for $N$ seems a bit arbitrary---could it be explained briefly what this physically represents? (e.g.~this is the level at which the till becomes sufficiently deformable that a cavity system is developed and that effectively caps the water pressure?) I would have thought a critical pressure, rather than a critical fraction of overburden, might be more reasonable? That aside, I found the prescription of $W_{til}^{max}$, and subsequent derivation of till thickness $\eta$ (22) rather odd, since it seems more natural to prescribe the thickness of till $\eta$ and have $W_{til}^{max}$ derived from that (and $\delta$ and $P_o$).  As it is, $\eta$ varies as the overburden varies (when coupled with ice flow), so that there is implicit redistribution of sediment.}
{}

\reply{7. p4721, (30) and following sentences - it is a bit confusing to write $P = P_{FC}(W)$ here (and in (29), and similarly in the appendix), as the formula depends upon $P_o$ and therefore space, as well as on $W$.  It'd be clearer to include $x$ as an additional argument here ((30) is not then a clean porous-medium equation).}
{}

\reply{8. p4723, l5 - this sentence reads rather strangely.  Aren't most of the parameters 'user-adjustable'?  What is meant by temporal 'detail' in the pressure evolution - is it suggesting that $\phi_0=0$ is 'correct'?  Later that paragraph, what is meant by diffusive 'range', and would it not scale as $\phi_0^{1/2}$?}
{}

\reply{9. p4723, l16-22 - this algorithm is certainly a lot more computationally efficient than the method used to solve the elliptic variational problem of Schoof et al (2012), but it should be noted that the schemes are not solving exactly the same problem (at least, for non-steady states, which is where the computational cost lies).  Difficulties of Schoof et al's method stemmed notably from discontinuities in $W$ associated with unfilled cavities, which are absent in the current problem.}
{}

\reply{10. p4727, l6 - I'm not sure how much we know that the system is close to steady state 'much of the time', so I'd recommend removing this; justification for looking at steady states is probably not required.}
{}

\reply{11. p4728, l1 - clarify that this statement is for a given discharge?}
{}

\reply{12. p4729, l11 - I am confused by the 'solution' $W = W_r$ to (45). This would only be a solution if the ice surface were a very particular shape?}
{}

\reply{13. Section 6.2 - the discussion of the boundary conditions here seems unnecessarily confusing and it could be much clearer just to state the shape, sliding velocity, and boundary conditions that are used, rather than explaining in generality how the solution works. Note that $W_c$ has only been defined in the appendix so comes out of the blue here. Since $r = L$ is the edge of the domain, the distinction between $L_-$ and $L$ seems pedantic (the definition of variables outside of the domain has not yet been given, and is more of an algorithmic issue).}
{}

\reply{14. p4731, (48) - $\varphi_0$ is $\omega_0$ ?}
{}

\reply{15. p4731, l7 - presumably the numerical value for $W^*$ given here corresponds to a particular parameter set? It must depend upon $k$, $H_0$ etc?}
{}

\reply{16. p4736, l20 - the right hand column here seems unnecessary?}
{}

\reply{17. p4739, l25 - The numerical values of timesteps here and on p4732 could be brought together to save space and avoid repetition. The value of $\phi_0$ used seems rather large; if a smaller value were used (going towards the elliptic limit) might the timestep restriction become restrictive?}
{}

\reply{18. p4748, l15, and figure 11 - I was a bit confused by the comparison of $W$ and $P/P_o$; what significance is $P/P_o$ believed to have? Doesn't a lot of this information come just from the steady state relationship between $W$ and $P$ in (A4)?, The caption is a bit confusing when it refers to 'pairs' $(W,P)$, but what is plotted is really $P/P_o$.}
{}

\reply{19. p4749, l9 - what is the 'actual diffusivity of the advective flux'?  'diffusive nature of the advective flux' might be clearer.}
{}

\reply{20. p4749, l15 - this statement is rather vague, and I'm not sure what it's trying to say.}
{}

\reply{21. p4751, l17 - something missing from this sentence?}
{}
\end{itemize}



\subsection*{Comments by Anonymous Referee \#3}\begin{itemize}
\reply{\textbf{Summary of the manuscript}\\
The manuscript (MS) describes a novel subglacial hydrology model implemented as
part of the PISM ice sheet model. To my knowledge, this model together with the
model of de Fleurian et al. (2014) (Elmer/Ice) are the currently most complex hydrology
models included in large scale ice sheet models. The hydrology model consists of a
cavity-like layer which can conduct the water horizontally, and two storage components:
a till layer and an englacial aquifer. The coupling to ice flow would be through the yield
strength of the till, which in terms depends on the amount of water stored (although no
two-way coupled runs are demonstrated). The model performs well on test cases with
analytic solutions and on an application to the Greenland ice sheet.\\
The MS is very detailed and describes the mathematical model, some analytic solutions, the numerical implementation and some test applications. The MS is suitable for
publication in GMD after the comments below are addressed.}
{}

\reply{\textbf{Mathematical model}\\
My main comments are that water in englacial storage is not accounted for, that the
statements $0 \le P \le P_o$ and $W = Y$ are inconsistent, and that boundary conditions are
omitted. Further, in the mathematical sections it is never explained how in detail the
bounds on P and also W till are enforced, although it can be deciphered from the later
sections on numerical implementation. Also the authors mention that their pressure
regularisation is necessary to allow enforcing $0 \le P \le P_o$ (by projection).  Why is this
so? Why could this not be done using the elliptic pressure equation?}
{}

\reply{Mass conservation (Eq. 1, 34a) should also take into account $W_{eng}$, the equivalent
layer of water stored englacially:
   $$\frac{\partial W}{\partial t} + \frac{\partial W_{till}}{\partial t} + \frac{\partial W_{eng}}{\partial t} + \nabla \cdot \mathbf{q} = m /\rho_w. \qquad (1)$$
In particular, for the void ratios ($\phi_0 = 0.01$) considered in this MS the $W_{eff}$ term is
important.  For instance, a relatively small pressure difference of 10 m water head leads
to a change in $W_{eff}$ of $0.1 m$ which is on the order of $W$. In fact, having $\phi_0 = 0.01$ is
probably beyond what may be considered a regularisation (i.e. having negligible effect
on the solution), and the MS should be updated accordingly.}
{}

\reply{If possible, it would be nice to state the bounds on the various state equations more
explicitly, e.g.:
  $$\frac{\partial W_{till}}{\partial t} = \begin{cases} m /\rho_w - C_d & \text{if \dots} \\
  0 & \text{otherwise}  \end{cases} \qquad (2)$$
Or if that is not possible, state the bounds next to the equations.}
{}

\reply{For the pressure, according to the numerics outlined in section 7.6, the authors solve
Eq.33 on the whole domain for $P$ and then project/update P such that $0 \le P \le P_o$
(except where $W = 0$ also $P = P_o$ ). Therefore $Y = P$ is only true in the so-called
''normal-pressure'' regions, which should be stated. In the overpressure or underpres-
sure regions the authors instead use the mathematical closures $P = P_o$ and $P = 0$,
which should also be stated. Also, it seems that the pressure equation is solved for
the whole domain using boundary conditions at the edge of the domain, which is in
contrast to Schoof et al. (2012). This difference needs to be discussed in a section
about boundary conditions.}
{}

\reply{Even apart from the storage term (which the authors acknowledge), the presented
scheme is not quite equivalent to the one in Schoof et al. (2012): To determine the
regions where pressure equation needs to be solved (Eq.34c in this MS) Schoof et al.
(2012) uses constraints on $W$ and not on $P$ (see their equations 4.1, 4.7 and 4.11). In
the region where the pressure equation is solved, Schoof et al. (2012) uses appropri-
ate boundary condition to link to the adjacent regions.  Also in underpressure regions
Schoof et al. (2012) solve both for $Y$ and $W$ (their $h$ and $h_w$).}
{}

\reply{To illustrate the impact of the different models, here a pathological case which (I think)
the mathematical model of Schoof et al. (2012) handles fine but the one in this MS less
so:}
{}

\reply{Starting with an initial, steady state with a region where $W > W_r$ and $P = P_o$.  Decrease input into that region until $P < P_o$, i.e.~something like a draining subglacial lake.
Now (as far as I understand the equations in the MS) $W$ in that region would evolve
according to Eq.13, i.e.~shrink by viscous creep (unless $P < 0$ at which point it would
again evolve according to Eq.34a). This contrasts to Schoof et al. (2012) which keeps
$P = P_o$ until $W \le W_r$.}
{}

\reply{Not getting this and other corner cases right is not bad and still results in a great
subglacial hydrology model, in particular for the application intended here. However,
Schoof et al. (2012) gets them right(er) (as far as I understand) and thus the authors'
claims that they successfully solve that problem should be a bit more qualified (see
line-comments below).}
{}

\reply{Other comments\\
The manuscript is quite lengthy and could do with some streamlining. Among others,
Section 4.1 and 5.2 should be merged, Section 9.2.1 should be shortened and Fig.~6
removed.}
{}

\reply{It would help if the authors would state the unknown variables at the beginning of the
mathematical description.}
{}

\reply{The authors mention frozen conditions but never go into details about them.  What
happens to the cavity sheet and till layer when input is negative?  What does the water
pressure do?  What do the cavities and thus $W$ do?  In fact, the evolution equation for
$Y$ does not contain a melt/freeze term so $Y > 0$ even when frozen.  How does this link
to setting $P = P_o$ when $W = 0$ (p.4742 l.4).  This should warrant at least a paragraph.}
{}

\reply{\textbf{Comments by line}\\
Comments by page and line number (add 4700 to the page number):\\
p.6 l.8 State how many parameters are used}
{}

\reply{p.6 l.8 Instead of ``We use englacial porosity as a regularization, and we preserve
physical bounds on the pressure.'' write ``We use englacial porosity as a regularization to impose physical bounds on the pressure.''  But in fact, I am not sure this statement is right, as bounds on the pressure are enforced by projecting it onto $0 \le P \le P_o$.}
{}

\reply{p.6 l.21 reword ``reasonable''}
{}

\reply{p.8 l.4-6 This is not quite right, see my Section above.}
{}

\reply{p.8 l.19-24 Maybe this paragraph should be moved to start at line 7.}
{}

\reply{p.8 l.29 Whilst no mathematical proven of convergence of grid-based models is available,  they do seem to converge under grid refinement in a statistical sense (see appendix of Werder et al. (2013)).  Also, their parameters are independent of the grid.  Thus automatic grid-resolution determination should be possible.}
{}

\reply{p.9 l.8 ``closures'' here and elsewhere can be confused with ``creep closure,'' reword.}
{}

\reply{p.9 l.19 It would help to briefly introduce which processes will be described and in
particular which are the unknown variables (or major variables as the authors call
them later).}
{}

\reply{p.10 Eq.1 add a term $\partial W_{eng}/\partial t$}
{}

\reply{p.10 l.9 it is not quite clear what ``the two-dimensional subglacial layer'' is. Presumably
it is the layer which has thickness $W$.}
{}

\reply{p.10 l.18 Specify that the pressure $P$ is at the top of the water layer too.}
{}

\reply{p.16 l.8 write ``and $N_{til} = P_o-P_{til}$ is the effective pressure of the overlying ice on the saturated till \dots''}
{}

\reply{p.16 l.10 Should be ``previous section'' but specify section number instead.}
{}

\reply{p.16 l.19 I find $N_0$ confusing. The very similar looking subscript ``o'' in $P_o$ refers overburden but the ``$0$'' is something else. Maybe $N_r$ or $N_{ref}$?}
{}

\reply{p.17 l.8-16 What follows in this part is unclear. Reformulate of this introductory sentence ``On the other hand we will describe the maximum capacity of the till by specifying \dots'' to prepare the reader that instead of working with $\delta$ you change to $W_{til}^{max}$.}
{}

\reply{p.17 l.10 Should this not just be $W_{til} < W_{til}^{max}$.  The lower bound is never used, or is it?}
{}

\reply{p.19 l.10 For this section the $Y$ equation is not needed/decoupled. That should be
mentioned.}
{}

\reply{p.20 l.22 comma after ``consider.''}
{}

\reply{p.22 l.10 Expand here (or maybe elsewhere) on how $P \le P_o$ is enforced.}
{}

\reply{p.23 l.15-22 This paragraph is a bit misplaced in this section.  Maybe the enforcement
of the various constraints, including $0 \le P \le P_o$, warrants its own section.  Which
is where this paragraph would belong.}
{}

\reply{p.23 l.19-22 These two sentences suggest that the authors have solved the ``prohibitively expensive'' problem of Werder et al. (2013).  But as discussed above, they only solve a simplified version of Schoof et al. (2012) without channels.  Reword.}
{}

\reply{p.24 Sec.5.1 I like this summary. One suggestion: write the equations in Eq.34 all as
``time derivative of unknown = something.'' Add the boundary conditions.}
{}

\reply{p.25 l.1-8 Either be specific about which functions are what type or leave the paragraph away.}
{}

\reply{p.25 l.1-8 Either be specific about which functions are what type or leave the paragraph away.}
{}

\reply{25 Sec.5.2 This section should be merged with section 4.1, probably at this location in
the MS.}
{}

\reply{p.25 l.11-19 as stated above, I don’t think this is quite the Schoof et al. (2012) model.}
{}

\reply{p.27 l.23 write ``layer thickness'' instead of ``amount''}
{}

\reply{p.28 l.1 write ``layer thickness'' instead of ``amount'' (and other places in the MS)}
{}

\reply{p.37 l.15 What happens when $W < 0$ should probably be discussed in the mathematical section too.}
{}

\reply{p.39 l.14 For a mountain glacier porosity seems to be around $0.01$ (Bartholomaus
et al., 2011).  Porosity for an ice sheet may be more on the order of $10^{-4}$.}
{}

\reply{p.40 l.13 What is the ``active subglacial layer''?}
{}

\reply{p.42 l.17 is this connected to the statement on p. 37, l.15? How?}
{}

\reply{p.45 l.20 write ``The spin-up grid sequence...''}
{}

\reply{45 Sec.9.2.1 This section is too long and detailed considering this is not about ice flow modelling.  Is this spin-up different from others used before?  Also in a similar
vein, Fig.~6~could be removed.}
{}

\reply{p.48 l.5-7 The till is either completely full or empty.  If I understand the dependence of sliding on the till hydrology correctly, this means either fully slippery on not at all.
So, is there no dependence of sliding on hydrology?  Maybe this point could be
briefly discussed.}
{}

\reply{\textbf{Comments for tables and figures}\\
Tab.~3 Why is $W_r$ so much higher here?}
{}

\reply{Fig.~2 Label $R_1$ and $L$.}
{}

\reply{Fig 2 \& 3 they could be combined.}
{}

\reply{Fig.~6 could be left away}
{}

\reply{Fig.~8 \& 11 mention what model run this is for}
{}

\reply{Fig.~11 Add a label to the colour-scale. Also, I think there is a inconsistency between
the caption and the text (p.48, l.18), one says ice thickness one says sliding
speed.}
{}
\end{itemize}



\subsection*{Comments by Editor Goldberg}\begin{itemize}
\reply{There are now 3 very helpful and insightful reviews from 3 very qualified and industrious
referees. I think that all their major concerns have merit, and I ask that you make efforts
to address these concerns. There maybe a couple of typos in reviewer 3's review, and
I disagree that $W_{eng}$ is unaccounted for (it just may need to be added to some early
equations)---but there are some very good points made about the difference between
your model and the Schoof 2012 model with respect to the regions where pressure is
either overburden or zero.}
{}

\reply{I want to highlight something that Dr Bartholomaus mentioned, offhand that the coupling is essentially one-way, because melt rate affects $N_til$, and thus yield stress, locally and $P/W$ do not in any way feed back on it. This is why I asked initially if there was
some way of allowing conduit pressure to influence till storage. I don't remember this
being emphasized anywhere in the text, and that it should be. (this also bears on Dr
Bartholomaus's comment on the mixing it is indeed odd for the ice flow to be opening
up cavities, and yet the normal stress of the asperities not affecting basal velocity.)
I hope that you can address all of these concerns, as I expect this to be a very valuable
addition to GMD.}
{}

\end{itemize}


\begin{thebibliography}{}
\providecommand{\natexlab}[1]{#1}
\providecommand{\url}[1]{{\tt #1}}
\providecommand{\urlprefix}{URL }
\expandafter\ifx\csname urlstyle\endcsname\relax
  \providecommand{\doi}[1]{doi:\discretionary{}{}{}#1}\else
  \providecommand{\doi}{doi:\discretionary{}{}{}\begingroup
  \urlstyle{rm}\Url}\fi

\bibitem{AschwandenAdalgeirsdottirKhroulev}
Aschwanden, A., Adalgeirsd{\'o}ttir, G., and Khroulev, C.: Hindcasting to
  measure ice sheet model sensitivity to initial states, The Cryosphere, 7,
  1083--1093, \doi{10.5194/tc-7-1083-2013}, 2013.

\bibitem{AschwandenBuelerKhroulevBlatter}
Aschwanden, A., Bueler, E., Khroulev, C., and Blatter, H.: An enthalpy
  formulation for glaciers and ice sheets, J. Glaciol., 58, 441--457,
  \doi{10.3189/2012JoG11J088}, 2012.

\bibitem{Bartholomausetal2011}
Bartholomaus, T.~C., Anderson, R.~S., and Anderson, S.~P.: Growth and collapse
  of the distributed subglacial hydrologic system of {K}ennicott {G}lacier,
  {A}laska, {USA}, and its effects on basal motion, J. Glaciol., 57, 985--1002,
  2011.

\bibitem{Bindschadler2013SeaRISE}
Bindschadler, R. et~al.: Ice-sheet model sensitivities to environmental forcing
  and their use in projecting future sea-level ({T}he {S}ea{RISE} {P}roject),
  J. Glaciol, 59, 195--224, 2013.

\bibitem{BBssasliding}
Bueler, E. and Brown, J.: Shallow shelf approximation as a ``sliding law'' in a
  thermodynamically coupled ice sheet model, J. Geophys. Res., 114, f03008,
  doi:10.1029/2008JF001179, 2009.

\bibitem{FlowersClarke2002}
Flowers, G.~E. and Clarke, G. K.~C.: A multicomponent coupled model of glacier
  hydrology 1. {T}heory and synthetic examples, J. Geophys. Res., 107, 2287,
  \doi{10.1029/2001JB001122}, 2002{\natexlab{a}}.

\bibitem{Goeller2014}
Goeller, S.: Antarctic {S}ubglacial {H}ydrology: {I}nteractions of subglacial lakes, basal water flow, and ice dynamics, PhD Dissertation, Universit\"at Bremen, 2014.

\bibitem{Hewitt2011}
Hewitt, I.~J.: Modelling distributed and channelized subglacial drainage: the
  spacing of channels, J. Glaciol., 57, 302--314, 2011.

\bibitem{Hewittetal2012}
Hewitt, I.~J., Schoof, C., and Werder, M.~A.: Flotation and free surface flow
  in a model for subglacial drainage. {P}art {II}: {C}hannel flow, J. Fluid
  Mech., 702, 157--188, 2012.

\bibitem{HoffmanPrice2014}
Hoffman, M. J. and Price, S.: Feedbacks between coupled subglacial hydrology and glacier dynamics, J. Geophys. Res. Earth Surf., 119, \doi{10.1002/2013JF002943}, 2014.
  
\bibitem{LeBrocqetal2009}
Le~Brocq, A., Payne, A., Siegert, M., and Alley, R.: A subglacial water-flow
  model for {W}est {A}ntarctica, J. Glaciol., 55, 879--888,
  \doi{10.3189/002214309790152564}, 2009.

\bibitem{Livingstoneetal2013}
Livingstone, S.~J., Clark, C.~D., Woodward, J., and Kingslake, J.: Potential
  subglacial lake locations and meltwater drainage pathways beneath the
  {A}ntarctic and {G}reenland ice sheets, The Cryosphere, 7, 1721--1740,
  \doi{10.5194/tc-7-1721-2013}, 2013.

\bibitem{MortonMayers}
Morton, K.~W. and Mayers, D.~F.: Numerical {S}olutions of {P}artial
  {D}ifferential {E}quations: {A}n {I}ntroduction, Cambridge University Press,
  2nd edn., 2005.

\bibitem{SchoofCoulombBlatter}
Schoof, C.: Coulomb friction and other sliding laws in a higher order glacier
  flow model, Math. Models Methods Appl. Sci. (M3AS), 20, 157--189,
  \doi{10.1142/S0218202510004180}, 2010{\natexlab{a}}.

\bibitem{Schoof2007deformable}
Schoof, C.: Cavitation on deformable glacier beds, SIAM J. Appl. Math., 67,
  1633--1653, 2007.

\bibitem{Schoofetal2012}
Schoof, C., Hewitt, I.~J., and Werder, M.~A.: Flotation and free surface flow
  in a model for subglacial drainage. {P}art {I}: {D}istributed drainage, J.
  Fluid Mech., 702, 126--156, 2012.

\bibitem{Siegertetal2009}
Siegert, M., Le~Brocq, A., and Payne, A.: Hydrological connections between
  Antarctic subglacial lakes, the flow of water beneath the East Antarctic Ice
  Sheet and implications for sedimentary processes, pp. 3--10, Wiley-Blackwell,
  2009.

\bibitem{Tulaczyketal2000}
Tulaczyk, S., Kamb, W.~B., and Engelhardt, H.~F.: Basal mechanics of {I}ce
  {S}tream {B}, {W}est {A}ntarctica 2.~{U}ndrained plastic bed model, J.
  Geophys. Res., 105, 483--494, 2000{\natexlab{b}}.

\bibitem{vanderWeletal2013}
van~der Wel, N., Christoffersen, P., and Bougamont, M.: The influence of
  subglacial hydrology on the flow of {K}amb {I}ce {S}tream, {W}est
  {A}ntarctica, J. Geophys. Res.: Earth Surface, 118, 1--14,
  \doi{10.1029/2012JF002570}, 2013.

\bibitem{Werderetal2013}
Werder, M., Hewitt, I., Schoof, C., and Flowers, G.: Modeling channelized and
  distributed subglacial drainage in two dimensions, J Geophys. Res.: Earth
  Surface, 118, 2140--2158, 2013.
\end{thebibliography}


\end{document}

