\documentclass[11pt,reqno]{amsart}
%prepared in AMSLaTeX, under LaTeX2e
\addtolength{\oddsidemargin}{-.65in}
\addtolength{\evensidemargin}{-.65in}
\addtolength{\topmargin}{-.3in}
\addtolength{\textwidth}{1.5in}
\addtolength{\textheight}{.6in}

\renewcommand{\baselinestretch}{1.1}

\usepackage{verbatim} % for "comment" environment

\usepackage[pdftex, colorlinks=true, plainpages=false, linkcolor=blue, citecolor=red, urlcolor=blue]{hyperref}

\newtheorem*{thm}{Theorem}
\newtheorem*{defn}{Definition}
\newtheorem*{example}{Example}
\newtheorem*{problem}{Problem}
\newtheorem*{remark}{Remark}

\newcommand{\mtt}{\texttt}
\usepackage{alltt,xspace}
\usepackage[normalem]{ulem}
\newcommand{\mfile}[1]
{\medskip\begin{quote}\scriptsize \begin{alltt}\input{#1.m}\end{alltt} \normalsize\end{quote}\medskip}

\usepackage[final]{graphicx}
\newcommand{\mfigure}[1]{\includegraphics[height=2.5in,
width=3.5in]{#1.eps}}
\newcommand{\regfigure}[2]{\includegraphics[height=#2in,
keepaspectratio=true]{#1.eps}}
\newcommand{\widefigure}[3]{\includegraphics[height=#2in,
width=#3in]{#1.eps}}

% macros
\usepackage{amssymb}

\usepackage[T1, OT1]{fontenc}
\renewcommand{\dh}{\fontencoding{T1}\selectfont{\symbol{240}}}

\newcommand{\bod}{B\"o\dh varsson\xspace}
\newcommand{\bods}{B\"o\dh varsson's}
\newcommand{\citebod}{B\"o\dh varsson (1955)\xspace}
\newcommand{\citepbod}{(B\"o\dh varsson, 1955)\xspace}

\newcommand{\bA}{\mathbf{A}}
\newcommand{\bB}{\mathbf{B}}
\newcommand{\bE}{\mathbf{E}}
\newcommand{\bF}{\mathbf{F}}
\newcommand{\bJ}{\mathbf{J}}
\newcommand{\br}{\mathbf{r}}
\newcommand{\bx}{\mathbf{x}}
\newcommand{\hbi}{\mathbf{\hat i}}
\newcommand{\hbj}{\mathbf{\hat j}}
\newcommand{\hbk}{\mathbf{\hat k}}
\newcommand{\hbn}{\mathbf{\hat n}}
\newcommand{\hbr}{\mathbf{\hat r}}
\newcommand{\hbt}{\mathbf{\hat t}}
\newcommand{\hbx}{\mathbf{\hat x}}
\newcommand{\hby}{\mathbf{\hat y}}
\newcommand{\hbz}{\mathbf{\hat z}}
\newcommand{\hbphi}{\mathbf{\hat \phi}}
\newcommand{\hbtheta}{\mathbf{\hat \theta}}
\newcommand{\complex}{\mathbb{C}}
\newcommand{\ppr}[1]{\frac{\partial #1}{\partial r}}
\newcommand{\ppt}[1]{\frac{\partial #1}{\partial t}}
\newcommand{\ppx}[1]{\frac{\partial #1}{\partial x}}
\newcommand{\ppy}[1]{\frac{\partial #1}{\partial y}}
\newcommand{\ppz}[1]{\frac{\partial #1}{\partial z}}
\newcommand{\pptheta}[1]{\frac{\partial #1}{\partial \theta}}
\newcommand{\ppphi}[1]{\frac{\partial #1}{\partial \phi}}
\newcommand{\pp}[2]{\frac{\partial #1}{\partial #2}}
\newcommand{\ppp}[2]{\frac{\partial^2 #1}{\partial^2 #2}}
\newcommand{\pppp}[3]{\frac{\partial^2 #1}{\partial #2 \partial #3}}
\newcommand{\Div}{\ensuremath{\nabla\cdot}}
\newcommand{\Curl}{\ensuremath{\nabla\times}}
\newcommand{\curl}[3]{\ensuremath{\begin{vmatrix} \hbi & \hbj & \hbk \\ \partial_x & \partial_y & \partial_z \\ #1 & #2 & #3 \end{vmatrix}}}
\newcommand{\cross}[6]{\ensuremath{\begin{vmatrix} \hbi & \hbj & \hbk \\ #1 & #2 & #3 \\ #4 & #5 & #6 \end{vmatrix}}}
\newcommand{\eps}{\epsilon}
\newcommand{\grad}{\nabla}
\newcommand{\image}{\operatorname{im}}
\newcommand{\integers}{\mathbb{Z}}
\newcommand{\ip}[2]{\ensuremath{\left<#1,#2\right>}}
\newcommand{\lam}{\lambda}
\newcommand{\lap}{\triangle}
\newcommand{\Matlab}{\textsc{Matlab}\xspace}
\newcommand{\exers}[1]{\bigskip\noindent\textbf{Exercises} #1}
\newcommand{\fexer}[2]{\bigskip\noindent\textbf{Lesson #1, \##2}\quad }
\newcommand{\prob}[1]{\bigskip\noindent\textbf{#1} }
\newcommand{\pts}[1]{(\emph{#1 pts}) }
\newcommand{\epart}[1]{\medskip\noindent\textbf{(#1)}\quad }
\newcommand{\ppart}[1]{\,\textbf{(#1)}\quad }
\newcommand{\note}[1]{[\scriptsize #1 \normalsize]}
\newcommand{\MatIN}[1]{\mtt{>> #1}}
\newcommand{\onull}{\operatorname{null}}
\newcommand{\rank}{\operatorname{rank}}
\newcommand{\range}{\operatorname{range}}
\renewcommand{\P}{\mathcal{P}}
\newcommand{\real}{\mathbb{R}}
\newcommand{\trace}{\operatorname{tr}}
\renewcommand{\Re}{\operatorname{Re}}
\renewcommand{\Im}{\operatorname{Im}}
\newcommand{\Arg}{\operatorname{Arg}}

\newcommand{\comm}[2]{\item \emph{#1}:\, #2}

\renewcommand{\ln}[2]{\comm{line #1}{#2}}
\newcommand{\lnpage}[3]{\comm{line #1 \underline{on page #2}}{#3}}
\newcommand{\lns}[2]{\comm{lines #1}{#2}}
\newcommand{\lnspage}[3]{\comm{lines #1 \underline{on page #2}}{#3}}
\newcommand{\fg}[2]{\comm{Figure #1}{#2}}
\newcommand{\eqn}[2]{\comm{equation #1}{#2}}

\newcommand{\reply}[2]{
\medskip\medskip
\item  \begin{quote}
\emph{#1}
\end{quote}

\medskip
\noindent #2}


\title[Replies to Referee Comments]{Replies to all referee comments on \\ \emph{Mass-conserving subglacial hydrology in PISM}}

\author{Ed Bueler}

\date{\today}

\begin{document}
\maketitle

\thispagestyle{empty}


\subsection*{Comments by Dr.~Bartholomaus}\begin{itemize}

\reply{In this manuscript, the authors present a novel extension of the existing Parallel Ice Sheet Model that includes the most complete treatment to date of subglacial hydrology in a large-scale ice sheet model.  Subglacial hydrology is immensely important in glacier dynamics, but is often neglected in the major ice sheet models used to predict future sea level rise.  The computational expense of tracking changes in the rapidly evolving subglacial environment has generally prevented all but the crudest of parameterizations (see table 2 of Bindschadler 2013's summary of the SeaRISE experiment).\\
Thus, the present work is novel and worthy of publication in GMD.  The writing is generally clear and fluent.  Both the theoretical development of the continuum equations and the numerical implementation are clearly outlined.}
{We appreciate this summary and the comments below, which have improved the text.  We note that scalability, so that the model can apply at high resolution to a whole ice sheet, dominates the design of the model, and will motivate many of the replies below.}

\reply{Beyond these over-arching strengths, I have four critiques that I believe would significantly enhance the impact and accessibility of the manuscript. These four opportunities for improvement are below. My line edits follow these more significant points.\\
---Four significant opportunities---\\
+ The authors offer some comparison between
their model and those of Werder, Hewitt, Flowers, Schoof, etc., but these are generally smaller scale models that have yet to be implemented or applied at the ice sheet
scale, and rarely to the complex geometries of existing glaciers or ice sheets. Some
discussion regarding how the new PISM hydrology model compares with the hydrology models of other major ice sheet models, such as those discussed in the SeaRISE
project would be very valuable.  At present, comparison to existing ice sheet models is
entirely lacking.  Without much knowledge of these models myself, I suspect that the
present model may represent a significant advance over the implementations in other
ice sheet models.  If appropriate, the authors may consider adding a sentence regarding this comparison to the abstract.  Also, by way of review, please consider adding a
table comparing features of presently-used ice sheet models.}
{Roughly-speaking, we \emph{do} compare to existing large-scale models, by describing and citing the work of \cite{JohnsonFastook,LeBrocqetal2009,Siegertetal2009}.  Such find-the-subglacial-lakes modeling, which essentially uses the overburden-pressure-based \texttt{routing} model described in our manuscript, is the only whole-ice-sheet-scale work we know about.\footnote{None of the reviewers mention this whole ice sheet scale application, but we care about it.  The fact that we are building an improvement of the \cite{LeBrocqetal2009,Siegertetal2009} model is important to us, clearly stated in the paper, and apparently of no interest to reviewers, who just want to complain about our lack of their preferred process modeling.}  We have also added a citation to \cite{HoffmanPrice2014}, which describes the construction of a related hydrology submodel within the Community Ice Sheet Model, but which is applied only at the scale of a single idealized mountain glacier.  It has not yet been applied at whole ice sheet scale (S.~Price personal communication).\\
\indent We believe it would be inappropriate, and a surprising use of space, to add a table comparing features of presently-used ice sheet models in a PISM model-description paper.}

\reply{+ Considering that efficient, low-pressure conduits are such important features of the
subglacial hydrologic system, some discussion/justification of why a model without conduits is useful is necessary.  While consistent model behavior under grid refinement is
certainly tremendously valuable, if one of the fundamental processes (i.e., transport of
water in conduits) is entirely neglected, then all the model results may be called into
question.  The present model is still an improvement on the general lack of subglacial
hydrology in existing ice sheet models, but ideally, conduits will be included in future
generations of ice sheets.}
{We discuss why a conduit model is not included in our model, and we have amplified the point in the revised version.\\
\indent  At present, \emph{all} 2D conduit models are \emph{not physics} by the normal standards of the field (e.g.~the field ``climate modeling'' or ``fluid modeling'' or ``continuum physics'', according to taste).  Note that ``consistent model behavior under grid refinement,'' in the sense used by this reviewer, is normally called ``continuum physics'' not ``grid refinement''.\footnote{Numerical model behaviour under grid refinement, as normally meant by the field of numerical approximations of continuum physics, is the very different concept of numerical convergence to the solution of a differential equation.  It is directly addressed by our verification case.} \\
\indent We clearly state that the existing lattice models of 2D conduits can't work in a user-controlled large-scale ice dynamics model, and that for that reason we do not add them.  That these models are useful for process exploration is not denied or disputed.\\
\indent We encourage this reviewer, and the other reviewers who we also believe (by their questions) are interested in including conduits into subglacial hydrology models, to proceed in the normal manner of physics and attempt to develop a PDE description (i.e.~a lattice-free formulation) of conduit effects.  Or they can apply an actual conduit formation model to whole ice sheet scale, that is, one the causes a conduit to appear at the location where the data suggests it should, not where an input grid forces it.  To complain that we have not invented the former ourselves, or made a model trillions of times more efficient than existing models so that we can claim that latter, in addition to the other actions of this paper, is bizarre.\\
\indent Note that 2D linked-cavities could be forced onto the nodes of a 2D lattice also.  We, and \emph{all} existing 2D models, of which we cite the ones we know, do not put linked-cavities on a lattice because we (collectively) \emph{do} have the PDE which describes the effect of a linked cavity system as continuum physics.  We have added this point to the paper. \\
\indent That ``all the model results may be called into question'' is the normal state of affairs in climate modeling.  But this phrase profoundly explains why we \emph{don't} use lattice models.  It would be sad for a user to use PISM coupled to a GCM and then have a reviewer of the result correctly point out that there was a single subsystem in the entire coupled mess which was not using the usual translation-invariant structures of physics \dots namely a 2D lattice model of subglacial conduits.\\
\indent Finally, we completely agree that ``ideally conduits will be included in future
generations of ice sheet [models]''.}

\reply{+ This manuscript and model includes an ambiguous mixing of hard-bedded and soft-bedded ideas. For example, the model includes opening and closing of cavities at
the glacier bed, driven by basal sliding (section 2.5).  This is generally considered
a hard-bedded view of basal hydrology and motion.  However, the description of the
Mohr-Coulomb yield stress for till (section 3.2) is appropriate for soft beds and basal
motion accomplished by deformation \emph{within} the till, not at the interface between the
till and the glacier ice. Similarly, the sliding law that depends on the till's yield stress
(section 3.3) is also a soft-bedded concept. The combination of soft- and hard-bedded
ideas in this model appears to be inappropriate or at least confusing. \dots}
{We are not quite sure why our mixing of hard- and soft-bedded morphology is ``ambiguous''.  The equations are clear.\\
\indent Though the reviewer may not have read it, we cite \cite{Schoof2007deformable} which models the formation of cavities in a deformable subglacial till.  This is precisely a ``combination of soft- and hard-bedded ideas'' for ``opening and closing of cavities at
the glacier bed, driven by basal sliding'' in the sense used by the reviewer.  We also cite   \cite{vanderWeletal2013} which uses till essentially as we do, combined with a conduit in a 1D model. \\
\indent We do not believe our use is inappropriate or confusing.  We continue to be mystified that the field often avoids including till in model-based exploration of hard-bed processes, given that we can't find a single published (or unpublished) example of a till-free borehole to the bed in ice of any substantial thickness.  (And none are offered by reviewers.) \\
\indent Indeed the Mohr-Coulomb yield stress for till is appropriate for soft beds and basal motion accomplished by deformation within the till.  However, basal \emph{ice} deformation may occur in a thin (meters) layer of temperate ice with high water and sediment content. This deformation, and notional hard-bed sliding if it occurs, are all modeled in the current literature by power-law sliding laws.  An ice sheet model, and the data (such as surface velocities and uncertain bed elevation) available to it, cannot distinguish these mechanisms occurring close to the bed.  As stated in section 3.3, our yield stress is used as a physically-meaningful coefficient in a power law for sliding---see also recent PISM literature like \cite{AschwandenAdalgeirsdottirKhroulev}---and such power laws are effectively regularized yield stress models \cite{SchoofCoulombBlatter} in the range of powers we use.  Having the coefficient of the sliding law be physically meaningful, with units of stress, and being tied to modeled basal water pressure, is both conceptually and practically better then providing a sliding law to PISM users with no physical meaning of, or physically-based way to model the variation of, the coefficient.}

\reply{\dots  Furthermore, the description of 1-way and 2-way coupling could be more clear. If the rate of basal motion ($u_b$ or $v_b$) is an input to the model, then why is there a section on the sliding law (section 3.3)?}
{Section 3.3 is included exactly to give meaning to the yield stress $\tau_c$ as a submodel output.  Because this is a model description paper for a submodel of PISM, we are obliged to state what the inputs and outputs of the model are.\\
\indent One connection between $\mathbf{v}_b$ and $\tau_c$ is our hydrology model.  The other is the whole ice dynamics model of PISM, which readers know takes boundary stress as an input and produces velocity as an output.  This ice dynamics model is well-described elsewhere, in literature we cite. \\
\indent Roughly-speaking, the normal use of sliding laws in ice sheet modeling is to have \emph{no} physical meaning to the coefficient of the sliding law, and often to set it by inversion of surface velocities (i.e.~totally by-passing a process which might set it).  We think that having mass-conservation for liquid water in the subglacial system, and using a physically-based computation of the coefficient in a sliding law, is a preferred situation.}

\reply{+ It is interesting and surprising to note that you find an inverse relationship between
water pressure and basal motion for systems at steady state. This is contrary to almost
all prevailing sliding laws. Is a result of the 1-way coupling ($v_b$ that does not directly
depend on water pressure)? Whatever the cause, it is sufficiently surprising to warrant
additional discussion.}
{Despite complaints elsewhere about inclusion of it in the Appendix, we include an analysis of steady states because this analysis is (mostly) not done elsewhere for the (now) standard linked-cavity system equations (e.g.~from \cite{Hewitt2011,Schoofetal2012}).  Our first point is that these equations imply a functional form $P(W)$ at steady state, and thus that Flowers and Clarke \cite{FlowersClarke2002} are not crazy to propose such.\\
\indent There emerges a ``inverse relationship between water pressure and basal motion for systems at steady state'' from this analysis.  Why is this surprising?  Basal motion generates cavities (i.e.~space for the water to fill), so the pressure drops.  Said another way (as we do in the paper), sliding increases the opening rate, so if creep closure must balance it then the pressure must drop to speed the closure.  We presumed this observation, which is not offered as a ``sliding law'' at all, and indeed should not be used that way, was standard.  In any case it follows from the equations, the reviewers don't believe our analysis is in error, and we have included it with the prominence we believe it deserves. \\
\indent Presumably the idea is surprising because the reviewer believes in sliding laws.  Regarding the idea that the sliding velocity ``does not directly
depend on water pressure'', we remark that sliding laws usually relate basal shear stress \emph{and} water pressure \emph{and} velocity.  The presence of longitudinal stresses in the ice implies that there should ultimately be a \emph{globalized} connection from water pressure back to sliding velocity, via (e.g.) stress boundary conditions at the edges of the glacier; such connection is outside of the scope of this submodel description paper, but is described at length in \cite{BBssasliding}, which we cite.  Our analysis of steady states is not offered as a sliding law, and in that sense the reviewer is correct that our analysis ``is a result of the 1-way coupling'' in the sense that conservation of momentum in the ice plays no role in the relationship.  The relationship simply follows from equations (13), (14), (15) in their steady state cases; the reviewer does not hint here that any of these equations, or our analysis, is incorrect.}

\reply{---Line Edits---\\
p.~4706, l.~26 Also consider citing Walder 1982 if your purpose is to highlight some of
the early work here.}
{We cite Walder 1982 line 22 of page 4708.  We use Creyts and Schoof at this point on page 4706 because only a stable (i.e.~viable) models aquifer geometry are worth listing as alternatives which might go into a subglacial hydrology numerical model.}

\reply{p.~4707 l.~5 I think the best reference for englacial porosity is Fountain et al.~2005, from Storglaciaren.}
{These citations here are about models.  We (already) cite Fountain later when describing observations.}

\reply{p.~4708, l.~24 It may be worth mentioning that wall melt in linked cavities is generally
expected to be small (Kamb, 1987 and Bartholomaus et al., 2011)}
{We have added this comment, thanks.}

\reply{p.~4709, l.~1 What are the ramifications of neglecting to model conduits? Many observational studies (including work by Nienow, Mair, Anderson, Cowton, Harper) have
shown that efficient conduits are a fundamental component of subglacial hydrology.
How will your model provide insightful and realistic results without a conduit component?}
{Remember that the model is intended for whole ice sheet use.  We want, therefore, a \emph{well-posed extension} of the models used for identifying subglacial lakes \cite{LeBrocqetal2009,Siegertetal2009}, and we want a \emph{mass-conserving extension} of a successful (in terms of explaining surface velocity observations) ice stream basal stress model \cite{AschwandenAdalgeirsdottirKhroulev,BBssasliding,Tulaczyketal2000}, and we want ice streams to have significant width as they are observed to, but which they would not without a pressure-evolution equation.\footnote{If mechanisms like linked cavities do not draw water out of the hydraulic-gradient descent paths then the water is concentrated along characteristics.  Compare \texttt{routing} and \texttt{distributed} figures in section 9.}\\
\indent While the above purposes are all quite prominent in the paper we actually wrote, they are essentially ignored by these and other reviewer comments, which tend to focus on current controversies in modeling mountain glaciers, and on the short-time behavior of drainage events in the warmer areas of the smaller ice sheets, as these are the topics which are current among researchers trying to model conduits. \\
\indent We had also assumed that models like \cite{Bartholomausetal2011}, which seem to explain the behavior of (rare) well-observed hydrological-plus-glacier-dynamics systems without using conduits, were of some value, but we may be wrong.}

\reply{p.~4712, l.~17 It is not immediately clear to me why $\nabla H \gg \nabla W$. Where does this suggestion/observation come from?}
{We have revised the relevant sentence to say: ``If $P$ scales with the overburden pressure $P_o$ then the first term will dominate in the situation $|\nabla (H+b)| \gg |\nabla W|$''.  We no longer assert the situation is ``common''.\\
\indent This comment is in the context of explaining why, for nearly the first time, we have included ``$\rho_w g W$'' into the formula for hydraulic potential.  We presumed that the reason it is left out of nearly all prior literature is because that literature assumes that flow along the gradient of the (prior) hydraulic potential $\psi = P + \rho_w g b$ dominates over the gradient of the other part (i.e.~the gradient of the added part ``$\rho_w g W$'').  This is the only case in which it would be acceptable to leave out the added part, as almost all prior literature does. \\
\indent So either: (i) all the prior literature is worthless (perfectly possible), or (ii) the case $|\nabla (H+b)| \gg |\nabla W|$ is worth considering as a way to relate our formulas to prior literature, and that is the spirit in which we consider it.  (In our model we only assume $|\nabla (H+b)| \gg |\nabla W|$ in a minor part of the given formula for the flux.)  We are here primarily trying to describe the complicated relationship between flux and the two gradients.}

\reply{p.~4712, l.~23 If here you assume that W $\ll$ b or P, and thus can be neglected in eq. 8, why have you made the distinction in eq. 2 and the discussion that follows to include the W term?}
{This question only makes sense if ``you assume that W $\ll$ b or P'' is interpreted as ``you assume that $|\nabla W| \ll |\nabla (H+b)|$''.  We do not compare values of $W$ (a thickness) to $b$ (an elevation) to $P$ (a pressure); we are comparing \emph{gradients} of distances.\\
\indent As noted, the simplification is for simplicity, in particular for simplicity in the final implementation code.  Despite the simplification we \emph{keep} the part of the flux proportional to $\grad W$, thus the simplified model is always diffusive for any pressure closure, and so, in particular, the \texttt{routing} model is well-posed unlike the related models in the prior literature \cite{LeBrocqetal2009,Siegertetal2009}.  As we also say, the simplification makes no difference at all if $\beta=2$, which is used in more than half the cited work which picks a value for $\beta$.\\
\indent Finally, the simplification occurs inside our formula for the effective hydraulic conductivity $K$, which is wildy-uncertain in practice anyway.  That is, in the formula
  $$K = k W^{\alpha-1} |\grad(P+\rho_w g b)|^{\beta-2}$$
the correct values for coefficient $k$ and the powers $\alpha,\beta$ are all subject to the most rank and data-free speculation, as we imply in citations in subsection 2.3 which give a wide range of values.}

\reply{p.~4713, l.~1 Near here, or somewhere else within the paper, please compare your
values for hydraulic conductivity [$k$] with those that may be calculated from field observations. Are your values in line with those found in the field? Googling ``subglacial
hydraulic conductivity'' yields several points of comparison.}
{We have looked and not found.  There is not a single observational paper we can find which argues that a directly-measurable value of the hydraulic conductivity is describing the average effect of a linked cavity system over the area of a grid cell relevant to this work (100 m to 5 km squares, say).  Values are, of course, always given when these papers include a \emph{model}---our Table A1 cites the default value of $k$ as from \cite{Schoofetal2012}---but one should be very skeptical that a value from applying one model to the data is still the right value when applying a different model to the data. \\
\indent Of course hydraulic conductivity for \emph{till} is given, based on specific \emph{in situ} observational work, and such values appear in the literature we cite (and dominate the results from Googling ``subglacial hydraulic conductivity'').  But the till hydraulic conductivity value should not be used as $k$.  The conductivity of till is so low that water does not move laterally through till in a time, and over distances, which could explain any of the apparent behavior of water under glaciers and ice sheets.  Rather it is the macroscopic conductivity of the connected cavity network which is relevant.  Such a network can be present even as there is sediment (till) lying around, and this is the situation we are attempting to model. \\
\indent  To quote \cite{Bartholomausetal2011}, which we cite, \begin{quote}
\emph{Each of these three parameters, $\gamma$, $[k=]C\tau_b^n$, and $\phi$, is only weakly constrained by observations reported in the literature.  \dots $[k=]C\tau_b^n$ has units that depend on the exponent, and varies from $1.5\times 10^{-5}\, \text{m}\,\text{s}^{-1}\,\text{Pa}^{0.18}$ to $1.1\times 10^{-3}\, \text{m}\,\text{s}^{-1}\,\text{Pa}^{0.4}$ (Jansson, 1995; Sugiyama and Gudmundsson, 2004).}
\end{quote}
Is this ``weakly constrained'' result the kind of ``field observations'' meant by the reviewer?  Why should space in this model description paper be used to recapitulate such a weak and uncertain state of affairs?  We want to avoid, in a model description paper, asserting that any particular value of any constant is correct.  We are building the model so users can relate its relatively few parameters ($k$ among them) to rich, but often indirect, available data.  As pointed out in our paper, ``Darcy flux parameters $\alpha,\beta,k$ are also important [to the distribution of water thickness in the model results].  Parameter identification using observed surface data, though needed, is beyond the scope of this paper.''  Indeed, the journal is \emph{Geoscientific Model Development} not \emph{Journal of Glaciology}, and the conductivity $k$ is an adjustable parameter in the model.  The source of the default value (Table A1) is cited.  This situation should suffice.}

\reply{p.~4714, l.~15 Here and nearby: define $c_1$, $c_2$, and A.}
{}

\reply{p.~4715, l.~6 Phrasing is ambiguous, as it makes it sound as though your model
potentially does not include till water storage beneath some parts of the ice sheet.}
{}

\reply{p.~4715, l.~20 Why not include lateral transport of water through till if vertical transport
is included? Till is often regarded as having an anisotropic hydraulic conductivity (e.g.,
Jones, 1993, ``A comparison of pumping and slug tests. . .'' in Ground Water vol.~31(6)).
Horizontal conductivity can be at least several times greater than vertical conductivity.}
{}

\reply{p.~4715, l.~20 Is m in eq. 16 the same as m in eq. 1? If so, these terms cancel out of
eq. 1.}
{}

\reply{p.~4715, l.~20 If $m/\rho$ is almost always bigger than $C_d$, then $dW_{til}/dt$ is always
increasing up to the cap $W_{til}^{max}$. It would be useful to lay this out more explicitly,
and include eq. 21 in this subsection. Essentially, you have a Boolean relationship,
where in some places there is wet till and other places the till is frozen. Is model
sensitive to selection of $W_{til}^{max}$?}
{}

\reply{p.~4715, l.~24 Inclusion of $C_d$ with fixed value is poorly justified and seems very
ad hoc. Even if used by Tulaczyk, why is it necessary here and what is the model
sensitivity to the selection of 1 mm a$^{-1}$? A constant rate of till water drainage into the
subglacial hydrologic system, that does not depend on pressure gradients, seems very
odd.}
{}

\reply{p.~4717, l.~1 What is the effect of this choice? How was it selected?}
{}

\reply{p.~4718, l.~16 I recommend changing the title of this section to ``Basal motion relation''
or some other phrase. ``Sliding law'' implies slip at the interface between the ice and its
bed, whether bedrock or sediment, whereas your equation for yield stress (eq. 17) is
appropriate for till deformation.}
{}

\reply{p.~4718, l.~21 q is already used for flux (even if printed in bold-face to identify its vector
character). I suggest using another variable name.}
{}

\reply{p.~4718, l.~23 Previously (eq. 14), $v_b$ was the rate of basal motion. $u$ and $v_b$ are
used inconsistently throughout the paper.}
{}

\reply{p.~4719, l.~4 What value of $q$ have you selected for your simulations? Justification?}
{}

\reply{p.~4720, l.~1  While ``velocity'' is technically correct, it is an odd choice for a thickness
change. I suggest using ``rate.''}
{}

\reply{p.~4720, l.~5 Define $h$- the ice surface elevation.}
{}

\reply{p.~4722, l.~7 ``\dots does not exist for tidewater glaciers or ice sheets.'' This may not
be strictly true see Gulley et al, 2009, in QSR, where they report exploring many
englacial conduits. In subsequent work, Gulley has mapped subglacial conduits. A
safer statement would be that ``vapor/air-filled cavities are not known to exist far from
glacier margins.''  The distinction regarding tidewater glaciers or ice sheets is unnecessary.}
{}

\reply{p.~4722, l.~10 ``observed in ice sheets and glaciers`` instead of ``observed in ice sheets''}
{}

\reply{p.~4722, l.~21 Add that the englacial water table is intended to represent the mean
over some large area of glacier, perhaps $>1$ km$^2$. Here, it is best to avoid the extreme
complications of, e.g., Fudge, 2008 in J Glac, where subglacial water pressures vary
significantly over very short distances.}
{}

\reply{p.~4723, l.~8 You might add that we can expect phi to be large everywhere that $dP/dt$
would be large (a highly fractured temperate glacier in coastal Alaska), and that phi
would be small only where $dP/dt$ is small (ice sheet interiors). Thus, even hydrauli-
cally/numerically ``stiff'' ice sheets shouldn’t experience physical or numerical shocks.}
{}

\reply{p.~4724, eq. 34 As before, are these m's supposed to be the same?}
{}

\reply{p.~4724, eq. 34 This is an odd combination of equations, because the top equation is a
component of the bottom equation, but the middle equation has not been incorporated
in the bottom equation.}
{}

\reply{p.~4726, l.~3 Note that this is essentially the same as eq. 27.}
{}

\reply{p.~4726, l.~23 Another connection is presented on p.~4721.}
{}

\reply{p.~4727, l.~20 Around here, discuss that, in steady state, eq. 41 suggests that at water
pressure decreases, the rate of basal motion increases. This flies in the face of most
sliding laws. Can you offer any insight as to how we are to incorporate these two views
in our understanding of hydrology and glacier dynamics? Is the one-way coupling of
your hydrology model with a glacier dynamics model sufficient to gain insight?}
{}

\reply{p.~4727, l.~20 Also note that $P$ depends also on $v_b$, not on $W$ alone.}
{}

\reply{p.~4727, l.~23 I don’t see the relationship between eq. 41 and the $VW$ advective flux.
Please elaborate.}
{}

\reply{p.~4729, l.~5 Readers should not have to turn to the appendix to learn what $s_b$ is.
Move essential material out of the appendix and into the main text.}
{}

\reply{p.~4729, l.~6 Defining this new $\omega_0$ variable seems unnecessary.}
{}

\reply{p.~4730, l.~5 What is the justification of the 5th power in the sliding speed?}
{}

\reply{p.~4730, l.~17 Define what you mean by ``under'', ``normal'' and ``over'' pressure.}
{}

\reply{p.~4731, l.~13 Give a few sentence introduction to the numerics here. The point is to
discretize eq. 34. What is the order of calculations? What will feed into what over the
next sub-sections of section 7? A thumbnail sketch similar to what is presented in 7.6
would be useful to guide the reader.}
{}

\reply{p.~4731, l .19 Near here, is it necessary for a model development paper to include a
reference for ``CFL'' and ``upwind''}
{}

\reply{p.~4731, l.~21 Be sure to clarify that $u$ and $v$ are not components of $v_b$, but are for the water speed.}
{}

\reply{p.~4732, l.~8 Parenthesis around the citation}
{}

\reply{p.~4736, l.~7 Is it important that the reader understand what it means for a scheme to
be ``flux-limited?'' Without modeling expertise myself, I’m not sure what this means.}
{}

\reply{p.~4742, l.~17 Because you report that your scheme is mass conserving so prominently
in the abstract, you should report how much error is involved with step (x), where
negative water thicknesses are discarded. This could be for the Greenland run of
section 9.2.}
{}

\reply{p.~4745, l.~22 Are the 2800 processor-hours on each of the 72 processors or divided
amongst the processors?}
{}

\reply{p.~4746, l.~16 Do you specify a geothermal heat flux? The handling (or lack thereof) of
geothermal heat should also be specified earlier, where the model setup is described.}
{}

\reply{p.~4746, l.~18 Please report if you identified any basal freeze-on ($m < 0$) consistent with
Bell et al., 2014, Nat. Geosci., vol.~7?}
{}

\reply{p.~4747, l.~25 Again, this is a good place to discuss the ramifications of a model without
R-channels. What are the limitations of your model? Is there a way that aspects of R-channels emerge in your model without explicit channel modeling?}
{}

\reply{p.~4748, l.~2 What about the eastern outlet glacier results makes them particularly
suspect? 7, C1774--C1781, 2014}
{}

\reply{p.~4748, l.~4 You report on the run time for your spin-up with the null hydrology model,
but what are the processor demands for the distributed model described here?}
{}

\reply{p.~4748, l.~5 Another statement regarding the sensitivity of results to $W_{til}ˆ{max}$ would
be useful here.}
{}

\reply{p.~4748, l.~20 Around here, worth mentioning that pressure as an increasing function of
$W$ is vaguely in line with the results of the Flowers (2002) model, although your model
reveals additional complexity.}
{}

\reply{p.~4749, l.~17 ``seemingly-disparate''}
{}

\reply{p.~4750, l.~20 Again, reference the observation that steady pressure here increases as
sliding decreases, which is inconsistent with almost all sliding laws.}
{}

\reply{Table 3 Odd to present the melt rate as a function of water density. Change this to a
straight scalar (i.e., 200).}
{}
\end{itemize}



\subsection*{Comments by Anonymous Referee \#2}\begin{itemize}
\reply{This paper describes a new sub-component of the open source ice-sheet model PISM,
which accounts for subglacial drainage of meltwater.  The model and a number of
subcases are described in considerable detail and then the numerical implementation
is described.  A simple steady state solution is used to test the numerical method, and
the model is then applied to the whole of the Greenland ice sheet.\\
I enjoyed reading this paper.  It represents to my knowledge the first serious attempt
to include an evolving subglacial drainage model within an ice-sheet scale ice-sheet
model, and the results are encouraging.  As such, I would like to recommend publication.  However, I have a few issues that I think need to be clarified or thought about first.\\
The major comments are here, followed by some specific but more minor points.}
{}

\reply{1. The first term of (33), involving the pressure derivative and which represents
changes in englacial water content, ought to appear in (34a) also, since this term
derives from the mass flux into/out of the englacial system, and it is the addition
of this term to the mass conservation equation (34a) that gives rise to its appearance in (33).  As it stands in (34), subtraction of the first and third equations
puts the $\partial P/\partial t$ term into the opening/closure equation $\partial W/\partial t$, which I don't see justification for.}
{}

\reply{2. p4738, l11, and this section generally---is it clear that these arguments prove
stability for the \emph{system} of equations in this model (in which the coefficients in (60),
say are varying at each timestep due to the pressure evolution)? The analysis
here seems to be for a standard advection-diffusion equation on its own, but it
is not immediately clear to me that standard results can be used here. I have
no doubt that the method is stable, but I think if the stability properties are to be
discussed in this much detail, it needs to be done for the whole system together,
and not for the individual components of the operator splitting separately. Or if
there is an argument as to why this is sufficient, that should be included.}
{}

\reply{3. The boundary conditions should really be described in more detail.  It'd be help-
ful to state mathematically what boundary conditions are imposed (in section 5
say), rather than having it algorithmically described in section 7.  In particular,
the diffusive nature of the $W$ equation suggests that one should apply some sort
of conditions on $W$ at all boundaries, but these are rather hidden, and in sec-
tion 9.1 it is claimed that there are convergence issues associated with a jump
in $W$, which seems at odds with the diffusive term.  I suspect the boundary con-
ditions are mostly imposed by step (vi) on p4742, but I was not entirely clear on
what is meant by 'not computing' the divided difference contribution to the flux
divergence.}
{}

\reply{Finally, I felt the paper might be shortened without losing detail; there are a number
of places where the discussion of relatively simple points is laboured. Sections that
might be reduced include section 2.4, section 4.3, section 6.2, section 7.1, section
9.2.1, (could just reference Aschwanden et al for much of this?), section 9.2.4, and the
appendix.}
{}

\reply{Specific comments\\
1. p4708, l3, also throughout - I do not see why the parabolic equation is always
described as a 'regularization', which suggests some element of artifice. For the
physical system described, the equation is parabolic, and there is no need to
treat it as a regularization.}
{}

\reply{2. p4708, l7 - I’d temper this by saying that till is 'sometimes' observed, as I don’t
think it is true that it is always observed.}
{}

\reply{3. p4708, l20 - it is not the inclusion of wall melt in the mass conservation equation
that leads to the instability but rather then inclusion of wall melt in the kinematic
opening-closure equation.}
{}

\reply{4. p4711, l9 - given the coupling with PISM, it seems a bit odd to say that you
‘accept’ the hydrostatic approximation, since you should be calculating $P_o$ consistently with the ice flow. As I understand it $P_o$ is always hydrostatic for the level
of approximation in PISM, so this would seem a better justification.}
{}

\reply{5. p4713, l11, also throughout - I find the repeated reference to the 'advection-diffusion equation' a bit misleading as although it has advection and diffusion
terms, it is rather different from what is normally associated with that term, as the
velocity depends on the pressure which is evolving simultaneously.  Perhaps this
is my own connotation of advection-diffusion, but I think it should be emphasized
that (12) is not stand-alone and is inherently coupled to more equations.}
{}

\reply{6. p4717 - the prescription of a minimum value for $N$ seems a bit arbitrary---could
it be explained briefly what this physically represents? (e.g.~this is the level at which the till becomes sufficiently deformable that a cavity system is developed and that effectively caps the water pressure?) I would have thought a critical pressure, rather than a critical fraction of overburden, might be more reasonable? That aside, I found the prescription of $W_{til}^{max}$, and subsequent derivation of till thickness $\eta$ (22) rather odd, since it seems more natural to prescribe the thickness of till $\eta$ and have $W_{til}^{max}$ derived from that (and $\delta$ and $P_o$).  As it is, $\eta$ varies as the overburden varies (when coupled with ice flow), so that there is implicit redistribution of sediment.}
{}

\reply{7. p4721, (30) and following sentences - it is a bit confusing to write $P = P_{FC}(W)$
here (and in (29), and similarly in the appendix), as the formula depends upon $P_o$
and therefore space, as well as on $W$. It’d be clearer to include $x$ as an additional
argument here ((30) is not then a clean porous-medium equation).}
{}

\reply{8. p4723, l5 - this sentence reads rather strangely.  Aren't most of the parameters 'user-adjustable'?  What is meant by temporal 'detail' in the pressure evolution - is it suggesting that $\phi_0=0$ is 'correct'?  Later that paragraph, what is meant by diffusive 'range', and would it not scale as $\phi_0^{1/2}$?}
{}

\reply{9. p4723, l16-22 - this algorithm is certainly a lot more computationally efficient
than the method used to solve the elliptic variational problem of Schoof et al
(2012), but it should be noted that the schemes are not solving exactly the same
problem (at least, for non-steady states, which is where the computational cost
lies).  Difficulties of Schoof et al’s method stemmed notably from discontinuities
in W associated with unfilled cavities, which are absent in the current problem.}
{}

\reply{10. p4727, l6 - I'm not sure how much we know that the system is close to steady
state 'much of the time', so I'd recommend removing this; justification for looking
at steady states is probably not required.}
{}

\reply{11. p4728, l1 - clarify that this statement is for a given discharge?}
{}

\reply{12. p4729, l11 - I am confused by the 'solution' $W = W_r$ to (45). This would only be
a solution if the ice surface were a very particular shape?}
{}

\reply{13. Section 6.2 - the discussion of the boundary conditions here seems unnecessar-
ily confusing and it could be much clearer just to state the shape, sliding velocity,
and boundary conditions that are used, rather than explaining in generality how
the solution works. Note that $W_c$ has only been defined in the appendix so comes
out of the blue here. Since $r = L$ is the edge of the domain, the distinction be-
tween $L_-$ and $L$ seems pedantic (the definition of variables outside of the domain
has not yet been given, and is more of an algorithmic issue).}
{}

\reply{14. p4731, (48) - $\varphi_0$ is $\omega_0$ ?}
{}

\reply{15. p4731, l7 - presumably the numerical value for $W^*$ given here corresponds to a
particular parameter set? It must depend upon $k$, $H_0$ etc?}
{}

\reply{16. p4736, l20 - the right hand column here seems unnecessary?}
{}

\reply{17. p4739, l25 - The numerical values of timesteps here and on p4732 could be
brought together to save space and avoid repetition. The value of $\phi_0$ used seems
rather large; if a smaller value were used (going towards the elliptic limit) might
the timestep restriction become restrictive?}
{}

\reply{18. p4748, l15, and figure 11 - I was a bit confused by the comparison of $W$ and $P/P_o$ ;
what significance is $P/P_o$ believed to have? Doesn't a lot of this information come
just from the steady state relationship between $W$ and $P$ in (A4)?, The caption is
a bit confusing when it refers to 'pairs' $(W,P)$, but what is plotted is really $P/P_o$.}
{}

\reply{19. p4749, l9 - what is the 'actual diffusivity of the advective flux'?  'diffusive nature of
the advective flux' might be clearer.}
{}

\reply{20. p4749, l15 - this statement is rather vague, and I'm not sure what it's trying to
say.}
{}

\reply{21. p4751, l17 - something missing from this sentence?}
{}
\end{itemize}



\subsection*{Comments by Anonymous Referee \#3}\begin{itemize}
\reply{\textbf{Summary of the manuscript}\\
The manuscript (MS) describes a novel subglacial hydrology model implemented as
part of the PISM ice sheet model. To my knowledge, this model together with the
model of de Fleurian et al. (2014) (Elmer/Ice) are the currently most complex hydrology
models included in large scale ice sheet models. The hydrology model consists of a
cavity-like layer which can conduct the water horizontally, and two storage components:
a till layer and an englacial aquifer. The coupling to ice flow would be through the yield
strength of the till, which in terms depends on the amount of water stored (although no
two-way coupled runs are demonstrated). The model performs well on test cases with
analytic solutions and on an application to the Greenland ice sheet.\\
The MS is very detailed and describes the mathematical model, some analytic solutions, the numerical implementation and some test applications. The MS is suitable for
publication in GMD after the comments below are addressed.}
{}

\reply{\textbf{Mathematical model}\\
My main comments are that water in englacial storage is not accounted for, that the
statements $0 \le P \le P_o$ and $W = Y$ are inconsistent, and that boundary conditions are
omitted. Further, in the mathematical sections it is never explained how in detail the
bounds on P and also W till are enforced, although it can be deciphered from the later
sections on numerical implementation. Also the authors mention that their pressure
regularisation is necessary to allow enforcing $0 \le P \le P_o$ (by projection).  Why is this
so? Why could this not be done using the elliptic pressure equation?}
{}

\reply{Mass conservation (Eq. 1, 34a) should also take into account $W_{eng}$, the equivalent
layer of water stored englacially:
   $$\frac{\partial W}{\partial t} + \frac{\partial W_{till}}{\partial t} + \frac{\partial W_{eng}}{\partial t} + \nabla \cdot \mathbf{q} = m /\rho_w. \qquad (1)$$
In particular, for the void ratios ($\phi_0 = 0.01$) considered in this MS the $W_{eff}$ term is
important.  For instance, a relatively small pressure difference of 10 m water head leads
to a change in $W_{eff}$ of $0.1 m$ which is on the order of $W$. In fact, having $\phi_0 = 0.01$ is
probably beyond what may be considered a regularisation (i.e. having negligible effect
on the solution), and the MS should be updated accordingly.}
{}

\reply{If possible, it would be nice to state the bounds on the various state equations more
explicitly, e.g.:
  $$\frac{\partial W_{till}}{\partial t} = \begin{cases} m /\rho_w - C_d & \text{if \dots} \\
  0 & \text{otherwise}  \end{cases} \qquad (2)$$
Or if that is not possible, state the bounds next to the equations.}
{}

\reply{For the pressure, according to the numerics outlined in section 7.6, the authors solve
Eq.33 on the whole domain for $P$ and then project/update P such that $0 \le P \le P_o$
(except where $W = 0$ also $P = P_o$ ). Therefore $Y = P$ is only true in the so-called
''normal-pressure'' regions, which should be stated. In the overpressure or underpres-
sure regions the authors instead use the mathematical closures $P = P_o$ and $P = 0$,
which should also be stated. Also, it seems that the pressure equation is solved for
the whole domain using boundary conditions at the edge of the domain, which is in
contrast to Schoof et al. (2012). This difference needs to be discussed in a section
about boundary conditions.}
{}

\reply{Even apart from the storage term (which the authors acknowledge), the presented
scheme is not quite equivalent to the one in Schoof et al. (2012): To determine the
regions where pressure equation needs to be solved (Eq.34c in this MS) Schoof et al.
(2012) uses constraints on $W$ and not on $P$ (see their equations 4.1, 4.7 and 4.11). In
the region where the pressure equation is solved, Schoof et al. (2012) uses appropri-
ate boundary condition to link to the adjacent regions.  Also in underpressure regions
Schoof et al. (2012) solve both for $Y$ and $W$ (their $h$ and $h_w$).}
{}

\reply{To illustrate the impact of the different models, here a pathological case which (I think)
the mathematical model of Schoof et al. (2012) handles fine but the one in this MS less
so:}
{}

\reply{Starting with an initial, steady state with a region where $W > W_r$ and $P = P_o$.  Decrease input into that region until $P < P_o$, i.e.~something like a draining subglacial lake.
Now (as far as I understand the equations in the MS) $W$ in that region would evolve
according to Eq.13, i.e.~shrink by viscous creep (unless $P < 0$ at which point it would
again evolve according to Eq.34a). This contrasts to Schoof et al. (2012) which keeps
$P = P_o$ until $W \le W_r$.}
{}

\reply{Not getting this and other corner cases right is not bad and still results in a great
subglacial hydrology model, in particular for the application intended here. However,
Schoof et al. (2012) gets them right(er) (as far as I understand) and thus the authors'
claims that they successfully solve that problem should be a bit more qualified (see
line-comments below).}
{}

\reply{Other comments\\
The manuscript is quite lengthy and could do with some streamlining. Among others,
Section 4.1 and 5.2 should be merged, Section 9.2.1 should be shortened and Fig.~6
removed.}
{}

\reply{It would help if the authors would state the unknown variables at the beginning of the
mathematical description.}
{}

\reply{The authors mention frozen conditions but never go into details about them.  What
happens to the cavity sheet and till layer when input is negative?  What does the water
pressure do?  What do the cavities and thus $W$ do?  In fact, the evolution equation for
$Y$ does not contain a melt/freeze term so $Y > 0$ even when frozen.  How does this link
to setting $P = P_o$ when $W = 0$ (p.4742 l.4).  This should warrant at least a paragraph.}
{}

\reply{\textbf{Comments by line}\\
Comments by page and line number (add 4700 to the page number):\\
p.6 l.8 State how many parameters are used}
{}

\reply{p.6 l.8 Instead of ``We use englacial porosity as a regularization, and we preserve
physical bounds on the pressure.'' write ``We use englacial porosity as a regularization to impose physical bounds on the pressure.''  But in fact, I am not sure this statement is right, as bounds on the pressure are enforced by projecting it onto $0 \le P \le P_o$.}
{}

\reply{p.6 l.21 reword ``reasonable''}
{}

\reply{p.8 l.4-6 This is not quite right, see my Section above.}
{}

\reply{p.8 l.19-24 Maybe this paragraph should be moved to start at line 7.}
{}

\reply{p.8 l.29 Whilst no mathematical proven of convergence of grid-based models is available,  they do seem to converge under grid refinement in a statistical sense (see appendix of Werder et al. (2013)).  Also, their parameters are independent of the grid.  Thus automatic grid-resolution determination should be possible.}
{}

\reply{p.9 l.8 ``closures'' here and elsewhere can be confused with ``creep closure,'' reword.}
{}

\reply{p.9 l.19 It would help to briefly introduce which processes will be described and in
particular which are the unknown variables (or major variables as the authors call
them later).}
{}

\reply{p.10 Eq.1 add a term $\partial W_{eng}/\partial t$}
{}

\reply{p.10 l.9 it is not quite clear what ``the two-dimensional subglacial layer'' is. Presumably
it is the layer which has thickness $W$.}
{}

\reply{p.10 l.18 Specify that the pressure $P$ is at the top of the water layer too.}
{}

\reply{p.16 l.8 write ``and $N_{til} = P_o-P_{til}$ is the effective pressure of the overlying ice on the saturated till \dots''}
{}

\reply{p.16 l.10 Should be ``previous section'' but specify section number instead.}
{}

\reply{p.16 l.19 I find $N_0$ confusing. The very similar looking subscript ``o'' in $P_o$ refers overburden but the ``$0$'' is something else. Maybe $N_r$ or $N_{ref}$?}
{}

\reply{p.17 l.8-16 What follows in this part is unclear. Reformulate of this introductory sentence ``On the other hand we will describe the maximum capacity of the till by specifying \dots'' to prepare the reader that instead of working with $\delta$ you change to $W_{til}^{max}$.}
{}

\reply{p.17 l.10 Should this not just be $W_{til} < W_{til}^{max}$.  The lower bound is never used, or is it?}
{}

\reply{p.19 l.10 For this section the $Y$ equation is not needed/decoupled. That should be
mentioned.}
{}

\reply{p.20 l.22 comma after ``consider.''}
{}

\reply{p.22 l.10 Expand here (or maybe elsewhere) on how $P \le P_o$ is enforced.}
{}

\reply{p.23 l.15-22 This paragraph is a bit misplaced in this section.  Maybe the enforcement
of the various constraints, including $0 \le P \le P_o$, warrants its own section.  Which
is where this paragraph would belong.}
{}

\reply{p.23 l.19-22 These two sentences suggest that the authors have solved the ``prohibitively expensive'' problem of Werder et al. (2013).  But as discussed above, they only solve a simplified version of Schoof et al. (2012) without channels.  Reword.}
{}

\reply{p.24 Sec.5.1 I like this summary. One suggestion: write the equations in Eq.34 all as
``time derivative of unknown = something.'' Add the boundary conditions.}
{}

\reply{p.25 l.1-8 Either be specific about which functions are what type or leave the paragraph away.}
{}

\reply{p.25 l.1-8 Either be specific about which functions are what type or leave the paragraph away.}
{}

\reply{25 Sec.5.2 This section should be merged with section 4.1, probably at this location in
the MS.}
{}

\reply{p.25 l.11-19 as stated above, I don’t think this is quite the Schoof et al. (2012) model.}
{}

\reply{p.27 l.23 write ``layer thickness'' instead of ``amount''}
{}

\reply{p.28 l.1 write ``layer thickness'' instead of ``amount'' (and other places in the MS)}
{}

\reply{p.37 l.15 What happens when $W < 0$ should probably be discussed in the mathematical section too.}
{}

\reply{p.39 l.14 For a mountain glacier porosity seems to be around $0.01$ (Bartholomaus
et al., 2011).  Porosity for an ice sheet may be more on the order of $10^{-4}$.}
{}

\reply{p.40 l.13 What is the ``active subglacial layer''?}
{}

\reply{p.42 l.17 is this connected to the statement on p. 37, l.15? How?}
{}

\reply{p.45 l.20 write ``The spin-up grid sequence...''}
{}

\reply{45 Sec.9.2.1 This section is too long and detailed considering this is not about ice flow modelling.  Is this spin-up different from others used before?  Also in a similar
vein, Fig.~6~could be removed.}
{}

\reply{p.48 l.5-7 The till is either completely full or empty.  If I understand the dependence of sliding on the till hydrology correctly, this means either fully slippery on not at all.
So, is there no dependence of sliding on hydrology?  Maybe this point could be
briefly discussed.}
{}

\reply{\textbf{Comments for tables and figures}\\
Tab.~3 Why is $W_r$ so much higher here?}
{}

\reply{Fig.~2 Label $R_1$ and $L$.}
{}

\reply{Fig 2 \& 3 they could be combined.}
{}

\reply{Fig.~6 could be left away}
{}

\reply{Fig.~8 \& 11 mention what model run this is for}
{}

\reply{Fig.~11 Add a label to the colour-scale. Also, I think there is a inconsistency between
the caption and the text (p.48, l.18), one says ice thickness one says sliding
speed.}
{}
\end{itemize}



\subsection*{Comments by Editor Goldberg}\begin{itemize}
\reply{There are now 3 very helpful and insightful reviews from 3 very qualified and industrious
referees. I think that all their major concerns have merit, and I ask that you make efforts
to address these concerns. There maybe a couple of typos in reviewer 3's review, and
I disagree that $W_{eng}$ is unaccounted for (it just may need to be added to some early
equations)---but there are some very good points made about the difference between
your model and the Schoof 2012 model with respect to the regions where pressure is
either overburden or zero.}
{}

\reply{I want to highlight something that Dr Bartholomaus mentioned, offhand that the coupling is essentially one-way, because melt rate affects $N_til$, and thus yield stress, locally and $P/W$ do not in any way feed back on it. This is why I asked initially if there was
some way of allowing conduit pressure to influence till storage. I don't remember this
being emphasized anywhere in the text, and that it should be. (this also bears on Dr
Bartholomaus's comment on the mixing it is indeed odd for the ice flow to be opening
up cavities, and yet the normal stress of the asperities not affecting basal velocity.)
I hope that you can address all of these concerns, as I expect this to be a very valuable
addition to GMD.}
{}

\end{itemize}


\begin{thebibliography}{}
\providecommand{\natexlab}[1]{#1}
\providecommand{\url}[1]{{\tt #1}}
\providecommand{\urlprefix}{URL }
\expandafter\ifx\csname urlstyle\endcsname\relax
  \providecommand{\doi}[1]{doi:\discretionary{}{}{}#1}\else
  \providecommand{\doi}{doi:\discretionary{}{}{}\begingroup
  \urlstyle{rm}\Url}\fi

\bibitem{AschwandenAdalgeirsdottirKhroulev}
Aschwanden, A., Adalgeirsd{\'o}ttir, G., and Khroulev, C.: Hindcasting to
  measure ice sheet model sensitivity to initial states, The Cryosphere, 7,
  1083--1093, \doi{10.5194/tc-7-1083-2013}, 2013.

\bibitem{Bartholomausetal2011}
Bartholomaus, T.~C., Anderson, R.~S., and Anderson, S.~P.: Growth and collapse
  of the distributed subglacial hydrologic system of {K}ennicott {G}lacier,
  {A}laska, {USA}, and its effects on basal motion, J. Glaciol., 57, 985--1002,
  2011.

\bibitem{BBssasliding}
Bueler, E. and Brown, J.: Shallow shelf approximation as a ``sliding law'' in a
  thermodynamically coupled ice sheet model, J. Geophys. Res., 114, f03008,
  doi:10.1029/2008JF001179, 2009.

\bibitem{FlowersClarke2002}
Flowers, G.~E. and Clarke, G. K.~C.: A multicomponent coupled model of glacier
  hydrology 1. {T}heory and synthetic examples, J. Geophys. Res., 107, 2287,
  \doi{10.1029/2001JB001122}, 2002{\natexlab{a}}.

\bibitem{Hewitt2011}
Hewitt, I.~J.: Modelling distributed and channelized subglacial drainage: the
  spacing of channels, J. Glaciol., 57, 302--314, 2011.

\bibitem{HoffmanPrice2014}
Hoffman, M. J. and Price, S.: Feedbacks between coupled subglacial hydrology and glacier dynamics, J. Geophys. Res. Earth Surf., 119, \doi{10.1002/2013JF002943}, 2014.

\bibitem{JohnsonFastook}
Johnson, J. and Fastook, J.~L.: Northern {H}emisphere glaciation and its
  sensitivity to basal melt water, Quat. Int., 95, 65--74, 2002.
  
\bibitem{LeBrocqetal2009}
Le~Brocq, A., Payne, A., Siegert, M., and Alley, R.: A subglacial water-flow
  model for {W}est {A}ntarctica, J. Glaciol., 55, 879--888,
  \doi{10.3189/002214309790152564}, 2009.

\bibitem{SchoofCoulombBlatter}
Schoof, C.: Coulomb friction and other sliding laws in a higher order glacier
  flow model, Math. Models Methods Appl. Sci. (M3AS), 20, 157--189,
  \doi{10.1142/S0218202510004180}, 2010{\natexlab{a}}.

\bibitem{Schoof2007deformable}
Schoof, C.: Cavitation on deformable glacier beds, SIAM J. Appl. Math., 67,
  1633--1653, 2007.

\bibitem{Schoofetal2012}
Schoof, C., Hewitt, I.~J., and Werder, M.~A.: Flotation and free surface flow
  in a model for subglacial drainage. {P}art {I}: {D}istributed drainage, J.
  Fluid Mech., 702, 126--156, 2012.

\bibitem{Siegertetal2009}
Siegert, M., Le~Brocq, A., and Payne, A.: Hydrological connections between
  Antarctic subglacial lakes, the flow of water beneath the East Antarctic Ice
  Sheet and implications for sedimentary processes, pp. 3--10, Wiley-Blackwell,
  2009.

\bibitem{Tulaczyketal2000}
Tulaczyk, S., Kamb, W.~B., and Engelhardt, H.~F.: Basal mechanics of {I}ce
  {S}tream {B}, {W}est {A}ntarctica 2.~{U}ndrained plastic bed model, J.
  Geophys. Res., 105, 483--494, 2000{\natexlab{b}}.

\bibitem{vanderWeletal2013}
van~der Wel, N., Christoffersen, P., and Bougamont, M.: The influence of
  subglacial hydrology on the flow of {K}amb {I}ce {S}tream, {W}est
  {A}ntarctica, J. Geophys. Res.: Earth Surface, 118, 1--14,
  \doi{10.1029/2012JF002570}, 2013.

\bibitem{Werderetal2013}
Werder, M., Hewitt, I., Schoof, C., and Flowers, G.: Modeling channelized and
  distributed subglacial drainage in two dimensions, J Geophys. Res.: Earth
  Surface, 118, 2140--2158, 2013.
\end{thebibliography}


\end{document}